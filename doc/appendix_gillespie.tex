\section{Gillespie simulation of master equation}\label{supp_gillespie}

(Note: The Python code used for the calculations presented in this section can
be found in the
\href{https://www.rpgroup.caltech.edu//chann_cap/software/gillespie_simulation.html}{following
link} as an anotated Jupyter notebook)

So far we have generated a way to compute an approximated form of the joint
distribution of protein and mRNA $P(m, p)$ as a function of the moments of the
distribution $\ee{m^x p^y}$. This is a non-conventional form to work with the
resulting distribution of the master equation. A more conventional approach to
work with master equations whose closed-form solutions are not known or not
computable is to use stochastic simulations commonly known as Gillespie
simulations. To benchmark the performance of our approach based on distribution
moments and maximum entropy we implemented the Gillespie algorithm. Our
implementation as detailed in the corresponding Jupyter notebook makes use of
just-in-time compilation as implemented with the Python package
\href{http://numba.pydata.org}{numba}.

\subsection{mRNA distribution with Gillespie simulations}

To confirm that the implementation of the Gillespie simulation was correct we
perform the simulation at the mRNA level for which the closed-form solution of
the steady-state distribution is known as detailed in
\siref{supp_param_inference}. \fref{sfig_gillespie_mRNA} shows example
trajectories of mRNA counts. Each of these trajectories were computed over
several cell cyles, where the cell division was implemented generating a
binomially distributed random variable that depended on the last mRNA count
before the division event.

\begin{figure}[h!]
	\centering \includegraphics
  {../fig/si/figS20.pdf}
	\caption{\textbf{Stochastic trajectories of mRNA counts.} 100 stochastic
	trajectories generated with the Gillespie algorithm for mRNA counts over
	time for a two-state unregulated promoter. Cells spend a fraction of the
	cell cycle with a single copy of the promoter (light brown) and the rest of
	the cell cycle with two copies (light yellow). When trajectories reach a
	new cell cycle, the mRNA counts undergo a binomial partitioning to simulate
	the cell division.}
  \label{sfig_gillespie_mRNA}
\end{figure}

To check the implementation of our stochastic algorithm we generated several of
these stochastic trajectories in order to reconstruct the mRNA steady-state
distribution. These reconstructed distributions for a single- and double-copy
of the promoter can be compared with \eref{seq_two_state_mRNA} - the
steady-state distribution for the two-state promoter.
\fref{sfig_gillespie_mRNA_dist} shows the great agreement between the
stochastic simulation and the analytical result, confirming that our
implementation of the Gillespie simulation is correct.

\begin{figure}[h!]
	\centering \includegraphics
  {../fig/si/figS21.pdf}
	\caption{\textbf{Comparison of analytical and simulated mRNA distribution.}
	Solid lines show the steady-state mRNA distributions for one copy (light
	blue) and two copies of the promoter (dark blue) as defined by
    \eref{seq_two_state_mRNA}. Shaded regions represent the corresponding 
    distribution obtained using 2500 stochastic mRNA trajectories and taking 
    the last cell-cyle to approximate the distribution.}
  \label{sfig_gillespie_mRNA_dist}
\end{figure}

\subsection{Protein distribution with Gillespie simulations}

Having confirmed that our implementation of the Gillespie algorithm that
includes the binomial partitioning of molecules reproduces analytical results
we extended the implementation to include protein counts.
\fref{sfig_gillespie_proteins} shows representative trajectories for both mRNA
and protein counts over several cell cycles. Specially for the protein we can
see that it takes several cell cycles for counts to converge to the dynamical
steady-state observed with the deterministic moment equations. Once this
steady-state is reached, the ensemble of trajectories between cell cycles look
very similar.

\begin{figure}[h!]
	\centering \includegraphics
  {../fig/si/figS22.pdf}
	\caption{\textbf{Stochastic trajectories of mRNA and protein counts.} 2500
	protein counts over time for a two-state unregulated promoter. Cells spend
	a fraction of the cell cycle with a single copy of the promoter (light
	brown) and the rest of the cell cycle with two copies (light yellow). When
	trajectories reach a new cell cycle, the molecule counts undergo a binomial
	partitioning to simulate the cell division.}
  \label{sfig_gillespie_proteins}
\end{figure}

From these trajectories we can compute the protein steady-state distribution,
taking into account the cell-age distribution as detailed in
\siref{supp_maxent}. \fref{sfig_gillespie_proteins_dist} shows the comparison
between this distribution and the one generated using the maximum entropy
algorithm. Despite the notorious differences between the distributions, the
Gillespie simulation and the maximum entropy results are indistinguishable in
terms of the mean, variance, and skewness of the distribution. We remind the
reader that the maximum entropy is an approximation of the distribution that
gets better the more moments we add. We therefore claim that the approximation
works sufficiently well for our purpose. The enormous advantage of the maximum
entropy approach comes from the computation time. for the number of
distributions that were needed for our calculations the Gillespie algorithm
proved to be a very inefficient method given the large sample space. Our
maximum entropy approach reduces the computation time by several orders of
magnitude, allowing us to extensively explore different parameters of the
regulatory model.

\begin{figure}[h!]
	\centering \includegraphics
  {../fig/si/figS23.pdf}
	\caption{\textbf{Comparison of protein distributions.} Comparison of the
	protein distribution generated with Gillespie stochastic simulations (blue
	curve) and the maximum entropy approach presented in \siref{supp_maxent}
	(orange curve). The upper panel shows the probability mass function. The
	lower panel compares the cumulative distribution functions.}
  \label{sfig_gillespie_proteins_dist}
\end{figure}