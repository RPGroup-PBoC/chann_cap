% Preamble
\documentclass [11pt]{article}

\usepackage{setspace}
\usepackage{amssymb}
\usepackage{amsmath}
\usepackage{amsfonts}
\usepackage{amssymb}
\usepackage{setspace}
\usepackage{amsthm}
\usepackage{textcomp}
\usepackage{graphicx}
\usepackage{url}
\usepackage{color}
\usepackage[dvipsnames]{xcolor}
\definecolor{cuteBlue}{rgb}{0.258, 0.387, 0.574}
\definecolor{cuteGreen}{rgb}{0, 0.3, 0}
\usepackage{cancel}
\usepackage{comment}
\usepackage[framemethod=TikZ]{mdframed}
\usepackage{enumitem}
\usepackage{wasysym}
\usepackage{listings}
\usepackage{float}
\usepackage{booktabs}
\usepackage{fixltx2e}
\usepackage{threeparttable}
\usepackage{titling}
\usepackage{zref-base}
\usepackage{makecell}
\usepackage{array}
\usepackage{hhline}
\usepackage{titlesec}

% To make jumping between equation, figure, citation references easier
\usepackage[colorlinks=true, urlcolor=cuteBlue, citecolor=cuteGreen, linkcolor=black]{hyperref}

% To make caption labels (i.e. Figure 1, Figure 2...) bold and
% make all caption text small
\usepackage[labelfont=bf, font=small]{caption}

% For strike-out text during editing
\usepackage[normalem]{ulem}

%Latin accents
\usepackage[utf8]{inputenc}

%subfigures
\usepackage{caption}

% Add author affiliations
\usepackage{authblk}

% %%%%%%%%%%%%%%%%%%%%%%%%%%%%%%%%%%%%%%%%%%%%%%%%%%%%%%%%%%%%%%%%%%%%%%%%%%%%%%
% %%%%%%%%%%%%%%%%%%%%%%%%%%%%%%%%%%%%%%%%%%%%%%%%%%%%%%%%%%%%%%%%%%%%%%%%%%%%%%

% Margins and spacings
\setlength{\evensidemargin}{0.0cm}
\setlength{\oddsidemargin}{0.0cm}
\setlength{\topmargin}{-1.0cm}
\setlength{\textwidth}{17cm}
\setlength{\textheight}{22cm}
\setlength{\parskip}{2.5mm}
\reversemarginpar
\marginparsep  0.1in
\marginparwidth 0.7in

% Give more spacing in equation arrays
\setlength{\jot}{10pt}

% Allow page breaks in multiline equations
\allowdisplaybreaks

% Set up title spacing so we don't waste so much space
\setlength{\droptitle}{-8em}
\date{\vspace{-5em}}  % No date will appear in title.

% Spacing between section headings and text
\titlespacing\section{0pt}{12pt plus 4pt minus 2pt}{-2pt plus 2pt minus 2pt}
\titlespacing\subsection{0pt}{12pt plus 4pt minus 2pt}{-2pt plus 2pt minus 2pt}
\titlespacing\subsubsection{0pt}{12pt plus 4pt minus 2pt}{-2pt plus 2pt minus 2pt}

% Convenient micron symbol
\newcommand{\micron}{{\textmu}m}

% No excess spacing for lists
\setlist{itemsep=0pt, topsep=0pt}

% Allow paragraph indentations in lists
\setitemize{listparindent=\parindent}
\setenumerate{listparindent=\parindent}

% Column type for tables with nice spacing
\newcolumntype{M}[1]{>{\centering\arraybackslash}m{#1}}
\newcolumntype{N}{@{}m{0pt}@{}}


% %%%%%%%%%%%%%%%%%%%%%%%%%%%%%%%%%%%%%%%%%%%%%%%%%%
% Document settings
% %%%%%%%%%%%%%%%%%%%%%%%%%%%%%%%%%%%%%%%%%%%%%%%%%%

%%%%%%%%%%%%%%%%%%%%%%%%%%%%%%%%%%%%%%%%%%%%%%%%%%%%
%% Begin document
%%%%%%%%%%%%%%%%%%%%%%%%%%%%%%%%%%%%%%%%%%%%%%%%%%%%


% date
\date{\today}

\begin{document}

Dear Editors,

It is our great pleasure to submit for your consideration for publication in
Nature Physics our paper entitled ``First-principles prediction of the
information processing capacity of a simple genetic circuit.'' One of the
defining features of living organisms is their ability to extract and process
information from the environment, allowing them to respond accordingly in order
to maximize their chances of survival. With a growing interest in the limits of
information gathering, the community has come to realize that evolution has
frequently pushed living organisms to perform at the limit of what physics
allows. This has led to the suggestive hypothesis that the constant pressure on
organisms to gather as much information of their surroundings as possible is an
organizing principle in biology. However a pervasive question in the field is
how can organisms achieve such precision in their measurements given noisy
signals and signal transduction systems operating at the molecular scale. This
is a formidable challenge to tackle given the complexity of biological signaling
systems. In this work, we make progress towards understanding the interplay
between biophysical parameters such as protein copy numbers and the affinity of
interactions between molecular components to allow cells to gather information
from the environment.

Technological developments such as single cell microscopy and quantitative image
processing has allowed the physics community to turn biological systems into a
rich playground for exploring how out of equilibrium systems operate. In
particular, understanding how cells control the expression of their genes has
attracted a lot of attention given that this question is at the core of modern
biology, with several physicists making significant contributions. My group over
the last decade has worked hard in developing a simple toy model of a genetic
circuit that is tractable both theoretically and experimentally. This simple
circuit that we consider to be the ``hydrogen atom'' of gene regulation has
served as a fertile garden in which to test physical models of how cells decide
to turn on and off their genes.

In this work, we compile all data accumulated in my lab over years of careful
quantitative experiments in this system in order to inform all the parameters of
our simple kinetic model, given our understanding of the physics of the gene
regulatory machinery. The model describes how cells respond to an external
signaling molecule by expressing a protein. As we vary parameters that we can
control experimentally such as number of regulatory proteins in the cell, and
their affinity for a specific binding site in the DNA, the model predicts
changes to this input-output function that can be directly compared with new
experimental measurements. With no free parameters, we show that our model
captures the cell-to-cell variability in this response for different
combinations of parameters. We then contextualize these input-output functions
into what cells harboring this signaling system can do by computing the channel
capacity, i.e. the maximum amount of information that cells can process with
this genetic circuit. In this context, the information that cells can process
refers to the resolution with which cells can distinguish different signaling
molecule concentrations. Again, our model is able to capture experimental
inferences of this information processing capacity, demonstrating that our
simple model is able to predict the amount of bits that cells can gather from
the environment using only a priori knowledge of the parameters.

As it stands now the paper is over the word limit that the journal requires. For
this first pass we wanted to be complete and thorough to present to you and the
potential reviewers every piece of information. But we have a clear strategy on
how to summarize the findings in order for them to fit within the Nature Physics
guidelines.

Though you surely have an outstanding list of reviewers to appeal to, we suggest
other excellent candidates:

\begin{itemize}
	\item Aleksandra Walczak (Ecole Normale Superieure) - Walczak has done
	incredibly beautiful work on the role of information processing and
	evolution. We consider her one of the leading thinkers in quantitative
	biology, and more specifically in predicting evolution from first principles.
	\item Jeremy Gunawardena (Harvard University) - We consider Gunawardena to be
	one of the deepest thinkers when it comes to the role of theory in biology.
	His constant innovative thinking has kept him at the forefront of the
	interface between physics and biology.
	\item Peter Swain (University of Edinburgh) - Swain is a dean of the field of
	information processing in living organisms. Over the years he has shown how
	clear physical thinking can sharpen the types of hypothesis that can be tested
	in the context of how cells make decisions.
	\item William Bialek (Princeton University) - Bialek's work has greatly
	inspired our efforts presented here. For several years he has explored the
	physical limits of the performance that biological organisms harboring simple
	molecular signal-transduction systems can have.
	\item Andre Levchenko (Yale university) - Levchenko has shown in the past the
	application of information theory to biological signal transduction systems.
	Part of his analysis pipeline was adapted here to extract useful information
	from our experimental data.
\end{itemize}

We also request that Prof. Terrence Hwa and Prof. Jose Vilar be excluded as
possible reviewers. Though we deeply admire their work, we think they tend to be
unduly harsh when it comes to the peer-review process.

To ensure transparency with every piece of information presented in this paper,
we have made all image data publicly available on the CaltechDATA repository and
can be accessed via the DOI:10.22002/D1.1184. All code used in the processing,
analysis, and figure generation for this work is also publicly available on this
paper's GitHub repository accessible at
\url{https://github.com/mrazomej/chann_cap}. We have complemented this
information with a website solely designed for this paper
\url{https://mrazomej.github.io/chann_cap/} where the data and analysis files
are more easily accessible for the general public.

Thank you for your consideration and I look forward to hearing your decision.

\noindent
With regards,

\noindent
Rob Phillips

\noindent
Fred and Nancy Morris Professor of Biophysics and Biology

\noindent
California Institute of Technology

\end{document}
