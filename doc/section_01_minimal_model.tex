\section{Results}

\subsection{Minimal model of transcriptional regulation}\label{sec_model}

As a tractable circuit for which we have control over the parameters both
theoretically and experimentally, we chose the so-called simple repression
motif, a common regulatory scheme among prokaryotes \cite{Rydenfelt2014}. This
circuit consists of a single promoter with an RNA-polymerase (RNAP) binding
site and a single binding site for a transcriptional repressor
\cite{Garcia2011c}. The regulation due to the repressor occurs via exclusion of
the RNAP from its binding site when the repressor is bound, decreasing the
likelihood of having a transcription event. As with many important
macromolecules, we consider the repressor to be allosteric, meaning that it can
exist in two conformations, one in which the repressor is able to bind to the
specific binding site (active state) and one in which it cannot bind the
specific binding site (inactive state). The environmental signaling occurs via
passive import of an extracellular inducer that binds the repressor, shifting
the equilibrium between the two conformations of the repressor
\cite{Razo-Mejia2018}. In previous work we have extensively characterized the
mean response of this circuit under different conditions using equilibrium
based models \cite{Phillips2019}. Here we build upon these models to
characterize the full distribution of gene expression with parameters such as
repressor copy number and its affinity for the DNA being systematically varied.

As the copy number of molecular species is a discrete quantity, chemical master
equations have emerged as a useful tool to model their inherent probability
distribution \cite{Sanchez2013}. In \fref{fig2_minimal_model}(A) we show the
minimal model and the necessary set of parameters needed to compute the full
distribution of mRNA and its protein gene product. Specifically, we assume a
three-state model where the promoter can be found in a 1) transcriptionally
active state  ($A$ state), 2) a transcriptionally inactive state without the
repressor bound ($I$ state) and 3) a transcriptionally inactive state with the
repressor bound ($R$ state). We do not assume that the transition between the
active state $A$ and the inactive state $I$ happens due to RNAP binding to the
promoter as the transcriptional initiation kinetics involve several more steps
than simple binding \cite{Browning2004}. We coarse-grain all these steps into
an effective ``on" and ``off" states for the promoter consistent with
experiments demonstrating the bursty nature of gene expression in {\it E. coli}
\cite{Golding2005}. These three states generate a system of coupled
differential equations for each of the three state distributions $P_A(m, p;
t)$, $P_I(m, p; t)$ and $P_R(m, p; t)$, where $m$ and $p$ are the mRNA and
protein count per cell, respectively and $t$ is time. Given the rates depicted
in \fref{fig2_minimal_model}(A) we define the system of ODEs for a specific $m$
and $p$. For the transcriptionally active state, we have
\begin{equation}
  \begin{aligned}
    \dt{P_A(m, p)} &=
    - \overbrace{\kpoff P_A(m, p)}^{A \rightarrow I} % A -> I
    + \overbrace{\kpon P_I(m, p)}^{I \rightarrow A}\\ % I -> A
    &+ \overbrace{r_m P_A(m-1, p)}^{m-1 \rightarrow m} % m-1 -> m
    - \overbrace{r_m P_A(m, p)}^{m \rightarrow m+1}% m -> m+1
    + \overbrace{\gm (m + 1) P_A(m+1 , p)}^{m+1 \rightarrow m} % m+1 -> m
    - \overbrace{\gm m P_A(m , p)}^{m \rightarrow m-1}\\ % m -> m-1
    &+ \overbrace{r_p m P_A(m, p - 1)}^{p-1 \rightarrow p} % p-1 -> p
    - \overbrace{r_p m P_A(m, p)}^{p \rightarrow p+1} % p -> p+1
    + \overbrace{\gp (p + 1) P_A(m, p + 1)}^{p + 1 \rightarrow p} % p+1 -> p
    - \overbrace{\gp p P_A(m, p)}^{p \rightarrow p-1}, % p -> p-1
  \end{aligned}
\end{equation}
where the state transitions for each term are labeled by overbraces. For the
transcriptionally inactive state $I$, we have
\begin{equation}
  \begin{aligned}
    \dt{P_I(m, p)} &=
    \overbrace{\kpoff P_A(m, p)}^{A \rightarrow I} % A -> I
    - \overbrace{\kpon P_I(m, p)}^{I \rightarrow A} % I -> A
    + \overbrace{\kroff P_R(m, p)}^{R \rightarrow I} % R -> I
    - \overbrace{\kron P_I(m, p)}^{I \rightarrow R}\\ % I -> R
    &+ \overbrace{\gm (m + 1) P_I(m+1 , p)}^{m+1 \rightarrow m} % m+1 -> m
    - \overbrace{\gm m P_I(m , p)}^{m \rightarrow m-1}\\ % m -> m-1
    &+ \overbrace{r_p m P_I(m, p - 1)}^{p-1 \rightarrow p} % p-1 -> p
    - \overbrace{r_p m P_I(m, p)}^{p \rightarrow p+1} % p -> p+1
    + \overbrace{\gp (p + 1) P_I(m, p + 1)}^{p + 1 \rightarrow p} % p+1 -> p
    - \overbrace{\gp p P_I(m, p)}^{p \rightarrow p-1}. % p -> p-1
  \end{aligned}
\end{equation}
And finally, for the repressor bound state $R$,
\begin{equation}
  \begin{aligned}
    \dt{P_R(m, p)} &=
    - \overbrace{\kroff P_R(m, p)}^{R \rightarrow I} % R -> I
    + \overbrace{\kron P_I(m, p)}^{I \rightarrow R}\\ % I -> R
    &+ \overbrace{\gm (m + 1) P_R(m+1 , p)}^{m+1 \rightarrow m} % m+1 -> m
    - \overbrace{\gm m P_R(m , p)}^{m \rightarrow m-1}\\ % m -> m-1
    &+ \overbrace{r_p m P_R(m, p - 1)}^{p-1 \rightarrow p} % p-1 -> p
    - \overbrace{r_p m P_R(m, p)}^{p \rightarrow p+1} % p -> p+1
    + \overbrace{\gp (p + 1) P_R(m, p + 1)}^{p + 1 \rightarrow p} % p+1 -> p
    - \overbrace{\gp p P_R(m, p)}^{p \rightarrow p-1}. % p -> p-1
  \end{aligned}
\end{equation}
As we will discuss later in \secref{sec_cell_cycle} the protein degradation
term $\gp$ is approximated to be zero since we do not consider protein
degradation as a Poisson process, but rather we explicitly implement binomial
partitioning of the proteins into daughter cells upon division.

It is convenient to rewrite these equations in a compact matrix notation
\cite{Sanchez2013}. For this we define the vector $\PP(m, p)$ as
\begin{equation}
  \PP(m, p) = (P_A(m, p), P_I(m, p), P_R(m, p))^T,
\end{equation}
where $^T$ is the transpose. By defining the matrices $\Km$ to contain the
promoter state transitions, $\Rm$ and $\Gm$ to contain the mRNA production and
degradation terms, respectively, and $\Rp$ and $\Gp$ to contain the protein
production and degradation terms, respectively, the system of ODEs can then be
written as (See \siref{supp_model} for full definition of these matrices)
\begin{equation}
  \begin{aligned}
    \dt{\PP(m, p)} &= \left( \Km -\Rm -m\Gm -m\Rp -p\Gp \right) \PP(m, p)\\
    &+ \Rm \PP(m-1, p)
    + (m + 1) \Gm \PP(m + 1, p)\\
    &+ m \Rp \PP(m, p - 1)
    + (p + 1) \Gp \PP(m, p + 1).
  \end{aligned}
  \label{eq_cme_matrix}
\end{equation}
Having defined the gene expression dynamics we now proceed to determine all 
rate parameters in \eref{eq_cme_matrix}.

\subsection{Inferring parameters from published data sets}
\label{sec_param}

A decade of research in our group has characterized the simple repression motif
with an ever expanding array of predictions and corresponding experiments to
uncover the physics of this genetic circuit \cite{Phillips2019}. In doing so we
have come to understand the mean response of a single promoter in the presence
of varying levels of repressor copy numbers and repressor-DNA affinities
\cite{Garcia2011c}, due to the effect that competing binding sites and multiple
promoter copies impose \cite{Brewster2014}, and in recent work, assisted by the
Monod-Wyman-Changeux (MWC) model, we expanded the scope to the allosteric
nature of the repressor \cite{Razo-Mejia2018}. All of these studies have
exploited the simplicity and predictive power of equilibrium approximations to
these non-equilibrium systems \cite{Buchler2003}. We have also used a similar
kinetic model to that depicted in \fref{fig2_minimal_model}(A) to study the
noise in mRNA copy number \cite{Jones2014a}. As a test case of the depth of our
theoretical understanding of this simple transcriptional regulation system we
combine all of the studies mentioned above to inform the parameter values of
the model presented in \fref{fig2_minimal_model}(A).
\fref{fig2_minimal_model}(B) schematizes the data sets and experimental
techniques used to measure gene expression along with the parameters that can
be inferred from them.

\siref{supp_param_inference} expands on the details of how the inference was
performed for each of the parameters. Briefly the promoter activation and
inactivation rates $\kpon$ and $\kpoff$, as well as the transcription rate
$r_m$ were obtained in units of the mRNA degradation rate $\gm$ by fitting a
two-state promoter model (no state $R$ from \fref{fig2_minimal_model}(A))
\cite{Peccoud1995} to mRNA FISH data of an unregulated promoter (no  repressor
present in the cell) \cite{Jones2014a}. The repressor on rate is assumed to be
of the form $\kron = k_o [R]$ where $k_o$ is a diffusion-limited on rate and
$[R]$ is the concentration of active repressor in the cell \cite{Jones2014a}.
This concentration of active repressor is at the same time determined by the
repressor copy number in the cell, and the fraction of these repressors  that
are in the active state, i.e. able to bind DNA. Existing estimates of the
transition rates between conformations of allosteric molecules set them at the
microsecond scale \cite{Cui2008}. By considering this to be representative for
our repressor of interest, the separation of time-scales between the rapid
conformational changes of the repressor and the slower downstream processes
such as the open-complex formation processes allow us to model the probability
of the repressor being in the active state as an equilibrium MWC process. The
parameters of the MWC model $K_A$, $K_I$ and $\eAI$ were previously
characterized from video-microscopy and flow-cytometry data
\cite{Razo-Mejia2018}. For the repressor off rate $\kroff$ we take advantage of
the fact that the mean mRNA copy number as derived from the model in
\fref{fig2_minimal_model}(A) cast in the language of rates is of the same
functional form as the equilibrium model cast in the language of binding
energies \cite{Phillips2015}. Therefore the value of the repressor-DNA binding
energy $\eR$ constrains the value of the repressor off rate $\kroff$. These
constraints on the rates allow us to make self-consistent predictions under
both, the equilibrium and the kinetic framework. Having all parameters in hand
we can now proceed to solving the gene expression dynamics.

\begin{figure}[h!]
	\centering \includegraphics
  {./fig/main/fig2_parameter_inference.pdf}
	\caption{\textbf{Minimal kinetic model of transcriptional regulation for a
	simple repression architecture.} (A) Three-state promoter stochastic model of
	transcriptional regulation by a repressor. The regulation by the repressor
	occurs via exclusion of the transcription initiation machinery, not allowing
	the promoter to transition to the transcriptionally active state. All
	parameters highlighted with colored boxes were determined from published
	datasets based on the same genetic circuit. Parameters in dashed boxes were
	taken directly from values reported in the literature or adjusted to satisfy
	known biological restrictions (B) Data sets used to infer the parameter
	values. From left to right Garcia \& Phillips \cite{Garcia2011c} is used to
	determine $\kroff$ and $\kron$, Brewster et al. \cite{Brewster2014} is used
	to determine $\eAI$ and $\kron$, Razo-Mejia et al. \cite{Razo-Mejia2018} is
	used to determine $K_A$, $K_I$, and $\kron$ and Jones et al.
	\cite{Jones2014a} is used to determine $r_m$, $\kpon$, and $\kpoff$.}
  \label{fig2_minimal_model}
\end{figure}
