\section{Introduction}

\textcolor{magenta}{
\begin{itemize}
	\item Information is a very useful and important quantity that people are very interested in
	\item People have shown in many contexts (fruit fly, Levchenko) that mutual information is related to fitness
	\item However, it remains unclear how to compute mutual information form first principles in many settings. For example, who would tuning physical parameters (such as binding strengths of protein copy numbers) influence mutual information
	\item On the other hand, thermodynamic models related key molecular details of a system to key physical parameters. For example, equilibrium models, MWC, chemical master equations.
	\item We want to combine these two realms to create a theory that relates mutual information to the fundamental physical parameters governing a system.
\end{itemize}
}

As living organisms thrive in the environment, they are faced with constant
changes in their surroundings. From abiotic conditions to biological
interactions, living organisms at all organization levels sense and respond to
external signals. At the molecular level, where signal transduction components
exist, there are physical constraints on the accuracy and precision of these
responses given by the intrinsic stochastic fluctuations  \cite{Nemenman2010}.
This means that two genetically identical cells exposed to the same stimulus
will not have an identical response \cite{Eldar2010}.

The question then becomes: how to analyze this probabilistic rather than
deterministic relationship between inputs and outputs? This problem was worked
out in 1948 by Claude Shannon who, in his seminal work, founded the field of
information theory \cite{Shannon1948}. Shannon developed a general framework for
how to analyze information transmission through noisy communication channels. In
his work, Shannon showed that the only quantity that satisfies simple
conditions of what a metric for information should be, was of the same
functional form as the thermodynamic entropy -- thereby naming it the same
\cite{MacKay2003}. He also gave a definition, based on this information entropy,
for the relationship between inputs and outputs known as the mutual information
(see Appendix XX for details on these metrics). \mrm{This would be a useful
appendix regardless of the type of journal.}

In recent years there has been a growing interest in understanding the
theoretical limits on cellular information processing \cite{Bialek2005,
Gregor2007}, and in quantifying how close evolution has pushed cellular
signaling pathways to these theoretical limits \cite{Tkacik2008, Dubuis2013,
Petkova2016}. While these studies have treated the signaling pathway as a
``black box'' explicitly ignoring all the molecular interactions taking place in
them, other studies have explored the role that molecular players and regulatory
architectures have in these information processing tasks \cite{Rieckh2014,
Ziv2007, Voliotis2014, Tostevin2009, Tkacik2011, Tkacik2008a, Tabbaa2014}.

Over the last decade the dialogue between theory and experiments in gene
regulation has lead to predictive power not only over the mean, but the noise in
gene expression as a function of relevant parameters such as regulatory protein
copy numbers, affinity of these proteins to the DNA promoter, as well as the
extracellular concentrations of inducer molecules\cite{Jones2014a, Brewster2014,
Garcia2011ck, Razo-Mejia2018} \mrm{Too self referential so far. Include Ido,
maybe Al Sanchez. It must be experiment-theory contrasting though!}

In this work we follow the same strategy of theory-experiment contrast to
predict from first principles the effect of biophysical parameters such as
transcription factor copy number and protein-DNA affinity on the information
processing capacity of of a simple repression genetic circuit. We use a master
equation-based model for the protein copy number distribution in combination
with equilibrium model for the effect of an extracellular inducer concentration
to predict the channel capacity, i.e. the maximum information that can be
processed by this gene regulatory architecture. We then compare these
theoretical predictions with our experimental determinations of the channel
capacity, \mrm{finding a good agreement between theory and experiment}.
