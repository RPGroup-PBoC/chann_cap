\section{Introduction}

One of the key characteristics of living organisms is their unique ability to
process an environmental signal into a robust and adequate response
\cite{Nemenman2010}. Such signaling pathways are commonly characterized using
dose-response curves, which measure the mean response for varying concentrations
of the stimulus. Yet from this perspective, it is easy to overlook the inherent
variability present within these cellular processes. For example, given the
small number of molecules involved in most chemical transformations in the cell
and the fact that, at the length scale of cellular components, thermal noise is
in the same regime as other deterministic energies \cite{Phillips2006}, two
genetically identical cells exposed to the same stimulus will not have an
identical response \cite{Eldar2010}. Even in clonal populations of bacteria,
various mechanisms give rise to fluctuations in the number of protein produced
in a cell including the intrinsic noise in gene expression \cite{Elowitz2002}
and uneven partitioning during cell division \cite{Huh2011}, which ultimately
constrains how tightly protein copy numbers can be regulated \cite{Lestas2010}.
Such fluctuations are not necessarily harmful to biological systems, and in
several systems such noise has been harnessed to aid fitness. For example,
fluctuations in the chemotaxis machinery of clonal \textit{E. coli} enables the
colony to better adapt to changing environmental conditions \cite{Frankel2014}.

Within the past decade, significant strides have been made both experimentally
and theoretically that enable us to probe beyond the mean and noise of cellular
responses and explore the full probability distribution (see \fref[figExpSetup]).
In doing so, we must immediately confront a nuance that is hidden while
only considering the average response, namely, that multiple inputs can generate
the same system response. For example, the input marked by the dashed line in
\fref[figExpSetup]\letter{C} could be generated from either a small or medium
concentration of the input molecule. Subsequent cellular processes that infer
the amount of input signal in the environment based on the level of gene
expression would therefore have a large uncertainty over whether the
concentration of this molecule outside the cell is in the small or medium range.
If the cell would optimally respond in two different ways to small and medium
ranges of signal, this will necessarily result in sub-optimal cellular behavior.
In contrast, if the level of gene expression matches the orange line in
\fref[figExpSetup]\letter{C}, it was almost certainly cause by the large input
signal and subsequent cellular processes could confidently enact the appropriate
response.

In this sense, a cell can best react to its surroundings when its gene
expression profile under different ranges of stimuli are all well separated, a
notion which is quantified by calculating the mutual information between the
input signal and output of a cellular process. Mutual information has been shown
to be the natural currency of evolution in many cellular contexts [\textit{cite
	Bialek}]. For example, it has been shown that transcription factor binding
maximizes the information transfer between binding energy and the information
gain per energy \cite{Savir2016}. In this work, we explore the the mutual
information in the context of transcriptional regulation.

In the past few years, equilibrium models have been developed which capture the
mean \cite{Garcia2011c} and noise \cite{Jones2014a} of transcriptional
regulation. This work was then extended to capture the full probability
distribution of mRNA \cite{Sanchez2013} and proteins \cite{Shahrezaei2008, Swain2016} using
master equations. By applying mutual information into these settings
\cite{Tkacik2008a}, we can compute the maximum information between an
environmental input and gene expression over all possible distribution of
inputs, which is believed to be proportional to the fitness of an organism. To
test this model, we construct a synthetic circuit using the inducible
\textit{lac} operon in \textit{Escherichia coli} and verify both the gene
expression profiles and the corresponding mutual information between the input
and output. We then tune two physical parameters in the system -- the Lac
repressor copy number and the DNA-repressor binding affinity -- and demonstrate
how the mutual information matches our theoretical predictions.

We next theoretically the implications of our model on the \textit{lac} system.
For example, we explore what combinations of parameters yield the maximum
possible mutual information between the input and output. We find that the
natural \textit{lac} system's mutual information is \talComment{extremely
	close???} to this maximum value. In doing so, we can transform our deep
understanding of transcriptional regulation into the context of an evolutionary
landscape.



%But even more interesting than being able to predict shapes of gene expression
%distributions as different inducer concentrations are titrated in, we can use
%ideas from information theory to quantify the full relationship between inducer
%inputs and gene expression outputs \cite{Tkacik2008a}. In particular mutual
%information measures how well a cell can distinguish between different input
%concentration \cite{Bowsher2014}.
%
%The \textit{lac} operon in \textit{Escherichia coli} has served as a canonical
%example of transcriptional regulation. Precise, quantitative measurements have
%been made on both natural and synthetic constructs \cite{Garcia2011} that
%capture average gene expression based on the number of Lac repressors and the
%repressor-DNA binding affinity. This work was later extended to include cell to
%cell variability by measuring the Fano factor \cite{Jones2014}, with the results
%once again well matching the corresponding thermodynamic models. In this work,
%we take the next step by measuring the full distribution of gene expression as
%repressor copy number and DNA affinity are varied. In doing so, we can transform
%our deep understanding of transcriptional regulation into the context of an
%evolutionary landscape.

%It has been shown in many contexts that the natural currency of evolution comes
%in the form of a cell's signal processing capability, more specifically in the
%mutual information between an environmental cue and a cell's response
%[\textit{cite Bialek}]. \talComment{Need some background sentences on what
%	mutual information is and some history on how it has been used effectively.} In
%this paper, we combine experiment and theory to investigate how the channel
%capacity, the maximum mutual information between an environmental input and gene
%expression over all possible distribution of inputs, varies with Lac repressor
%copy number and DNA affinity. 

%Using the same data used to generate \fref[fig:fit], \fref[fig:gene_dist]
%highlights the non-deterministic input-output relationship for the simple
%repression circuit. The differences at the level of mean gene expression shown
%in \fref[fig:fit] are blurred out when considering the full distribution of gene
%expression. This implies that, at the single cell level, cells cannot uniquely
%resolve the extracellular concentration of inducer. For a cell to properly
%resolve an environment, any point on the x-axis of \fref[fig:gene_dist] should
%map to a unique concentration of inducer. The more overlap there is between
%distributions the less accurate the inference of the environmental state will
%be.
%
%Information theory provides a useful metric for quantifying biological
%phenomenon. For example, it has been shown that transcription factor binding
%maximizes the information transfer between binding energy and the information
%gain per energy \cite{Savir2016} \talComment{Some more examples here would be
%	nice.}. For our purposes, extending the analysis of simple repression to
%incorporate not only the mean gene expression but the full distribution of
%single-cell measurements enables us to compute the channel capacity and verify
%whether \talComment{what is our theoretical prediction for this?}.

%Even in clonal populations of bacteria, various mechanisms give rise to
%fluctuations in the number of protein produced in a cell including the intrinsic
%noise in gene expression \cite{Elowitz2002} and uneven partitioning during cell
%division \cite{Huh2011}, which ultimately constrains how tightly protein copy
%numbers can be regulated \cite{Lestas2010}. Such fluctuations are not
%necessarily harmful to biological systems, and in several systems such noise has
%been harnessed to aid fitness. For example, fluctuations in the chemotaxis
%machinery of clonal \textit{E. coli} enables the colony to better adapt to
%changing environmental conditions \cite{Frankel2014}.

%Measuring the channel requires the full gene expression probability
%distribution, and hence is a new metric that has not been explored for the Lac
%system. We show that experimental measurements are well characterized by a simple two-state thermodynamic model where the Lac promoter is either occupied by RNA polymerase (RNAP) or Lac repressor. We ignore the the possibility that the promoter is empty state, and show that the results are still extremely accurate \talComment{Do we want to have the full version in the SI? This allows us to get a closed analytic form for the distribution, but can we get a closed form solution for the mutual information? If not, perhaps we should just use a full 3-state model since we will be doing thing numerically anyways.}. \textit{Talk about the allosteric nature of the Lac repressor and its relationship to the gratuitous inducer IPTG...}

By working on the well studied Lac system, we can theoretically predict the channel capacity as a function of the biophysical parameters of repressor copy number, DNA affinity, and the mRNA production and degradation rates. This glimpse into the evolutionary landscape of the Lac repressor allows us to place the wild type Lac repressor in the context of the pressures acting upon it.

\begin{figure}[h!]
	\centering \includegraphics[scale=0.8]{experiment_setupV3} \caption{Transcription
		regulation under different environmental conditions. As environmental
		concentrations of the small signaling molecule IPTG increase, more Lac
		repressor molecules within the cell will transition into their allosterically
		inactivate state, where they are much less likely to bind to DNA. Consequently,
		RNAP is more likely to bind to the operator which increases the mRNA copy
		number. This mRNA is both translated into a reporter protein and degraded by
		the cell. Variability in this process will create a probability distribution
		for gene expression in the presence of different levels of IPTG, with the mean
		gene expression increasing with IPTG concentration. Note that we only consider
		a two state system where the Lac operator is either bound to RNAP or Lac
		repressor, and we ignore the case where the promoter is unbound. \talComment{More ideas: (1) Can emphasize that the middle box permits us to theoretically predict the variability in gene expression. Maybe have a knob of some kind, or a graph showing variance as a function of mean gene expression or a formula or something}; (2) We could have two arrows on the right, one leading to high mutual information where the peaks are clearly distinct and the other leading to low mutual information where they all overlap strongly}
	\label{figExpSetup}
\end{figure}

