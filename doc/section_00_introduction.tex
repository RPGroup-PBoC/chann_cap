\section{Introduction}

\textcolor{magenta}{
\begin{itemize}
	\item Information is a very useful and important quantity that people are very interested in
	\item People have shown in many contexts (fruit fly, Levchenko) that mutual information is related to fitness
	\item However, it remains unclear how to compute mutual information form first principles in many settings. For example, who would tuning physical parameters (such as binding strengths of protein copy numbers) influence mutual information
	\item On the other hand, thermodynamic models related key molecular details of a system to key physical parameters. For example, equilibrium models, MWC, chemical master equations.
	\item We want to combine these two realms to create a theory that relates mutual information to the fundamental physical parameters governing a system.
\end{itemize}
}

As living organisms thrive in the environment, they are faced with constant
changes in their surroundings. From abiotic conditions to biological
interactions, living organisms at all organization levels sense and respond to
external signals. At the molecular level, where signal transduction components
exist, there are physical constraints on the accuracy and precision of these
responses given by the intrinsic stochastic fluctuations  \cite{Nemenman2010}.
This means that two genetically identical cells exposed to the same stimulus
will not have an identical response \cite{Eldar2010}.

The implications of this biological noise is that cells do not have an infinite
resolution to distinguish signals and as a consequence there is a one-to-many
mapping between inputs and outputs. The question then becomes how to analyze
this probabilistic rather than deterministic relationship between inputs and
outputs? The abstract answer to this question was worked out in 1948 by Claude
Shannon who, in his seminal work, founded the field of information theory
\cite{Shannon1948}. Shannon developed a general framework for how to analyze
information transmission through noisy communication channels. In his work,
Shannon showed that the only quantity that satisfies simple conditions of what a
metric for information should be, was of the same functional form as the
thermodynamic entropy -- thereby naming it the same \cite{MacKay2003}. He also
gave a definition, based on this information entropy, for the relationship
between inputs and outputs known as the mutual information (see Appendix XX for
details on these metrics). \mrm{This would be a useful appendix regardless of
the type of journal.}

It is natural to think that under certain scenarios living organisms that can
better resolve signals might have an evolutionary advantage, making it more
likely that their offspring will have a higher fitness value \cite{Taylor2007a}.
In recent years there has been a growing interest in understanding the
theoretical limits on cellular information processing \cite{Bialek2005,
Gregor2007}, and in quantifying how close evolution has pushed cellular
signaling pathways to these theoretical limits \cite{Tkacik2008, Dubuis2013,
Petkova2016}. While these studies have treated the signaling pathway as a
``black box'' explicitly ignoring all the molecular interactions taking place in
them, other studies have explored the role that molecular players and regulatory
architectures have on these information processing tasks \cite{Rieckh2014,
Ziv2007, Voliotis2014, Tostevin2009, Tkacik2011, Tkacik2008a, Tabbaa2014}. These
studies on the other hand have assumed a specific functional form for the noise
level rather than deriving it from a discrete stochastic model \mrm{double check
this statement!}. This has the limitation that at small numbers the functional
form of the noise might not be well approximated by a continuous function.

On the other hand, over the last decade the dialogue between theory and
experiments in gene regulation has led to predictive power not only over the
mean, but the noise in gene expression as a function of relevant parameters such
as regulatory protein copy numbers, affinity of these proteins to the DNA
promoter, as well as the extracellular concentrations of inducer
molecules \cite{Garcia2011c, Jones2014a, Brewster2014, Razo-Mejia2018} \mrm{Too
self referential so far. Include Ido, maybe Al Sanchez. It must be
experiment-theory contrasting though!}. These models based on equilibrium and
non-equilibrium statistical physics have reached a predictive accuracy level
such that for simple cases it is now possible to design input-output functions
\cite{Brewster2012, Barnes2018}. This opens the possibility to exploit the
advancements on these predictive models to tackle the question of how much
information can genetic circuits process.

In this work we follow the same philosophy of theory-experiment dialogue to
predict from first principles the effect that biophysical parameters such as
transcription factor copy number and protein-DNA affinity have on the
information processing capacity of a simple genetic circuit. Specifically we use
a master-equation-based model to construct the protein copy number distribution
(output) as a function of an extracellular inducer concentration (input) for
different combinations of transcription factor copy numbers and binding sites.
We then compute the channel capacity, i.e. the maximum information that can be
processed by this gene regulatory architecture. All parameters used for our
model were inferred from a series of studies that span several experimental
techniques \cite{Garcia2011c, Brewster2012, Jones2014a, Brewster2014,
Razo-Mejia2018} \mrm{see Appendix XX Parameter inference}, allowing us to
perform zero-parameter fit predictions of this non-trivial quantity.
\mrm{Aztec pyramid reference}

These predictions are then contrasted with experimental data, where the channel
capacity is inferred from single-cell fluorescence distributions taken at
different concentrations of inducer for cells with previously characterized
biophysical parameters \cite{Garcia2011c, Razo-Mejia2018}. We find that our
parameter free predictions closely match the experiments.
\mrm{need to work on a more impactful last paragraph.} 
