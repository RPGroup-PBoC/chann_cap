As living organisms thrive in a given environment, they are faced with
constant changes in their surroundings. From abiotic conditions such as
temperature fluctuations or changes in osmotic pressure, to biological
interactions such as cell-to-cell communication in a tissue or in a bacterial
biofilm, living organisms of all types sense and respond to external signals.
\fref{fig1_intro}(A) shows a schematic of this process for a bacterial cell
sensing a concentration of an extracellular chemical. At the molecular level
where signal transduction unfolds mechanistically, there are physical
constraints on the accuracy and precision of these responses given by intrinsic
stochastic fluctuations \cite{Nemenman2010}. This means that two genetically
identical cells exposed to the same stimulus will not have identical responses
\cite{Eldar2010}.

One implication of this noise in biological systems is that cells do not have
an infinite resolution to distinguish signals and, as a consequence, there is a
one-to-many mapping between inputs and outputs. Furthermore, given the limited
number of possible outputs, there are overlapping responses between different
inputs. This scenario can be map to a Bayesian inference problem where cells
try to infer the state of the environment from their phenotypic response, as
schematized in \fref{fig1_intro}(B). The question then becomes this: how can
one analyze this probabilistic, rather than deterministic, relationship between
inputs and outputs? The abstract answer to this question was worked out in 1948
by Claude Shannon who, in his seminal work, founded the field of information
theory \cite{Shannon1948}. Shannon developed a general framework for how to
analyze information transmission through noisy communication channels. In his
work, Shannon showed that the only quantity that satisfies three reasonable
axioms for a measure of uncertainty was of the same functional form as the
thermodynamic entropy -- thereby christening his metric the information entropy
\cite{MacKay2003}. He also gave a definition, based on this information
entropy, for the relationship between inputs and outputs known as the mutual
information. The mutual information $I$ between input $c$ and output $p$, given
by
\begin{equation}
  I = \sum_c P(c) \sum_p P(p \mid c) \log_2 {P(p \mid c) \over P(p)},
	\label{eq_mutual_info}
\end{equation}
quantifies how much we learn about the state of the input $c$ given that we get
to observe the output $p$. In other words, the mutual information can be
thought of as a generalized correlation coefficient that quantifies the degree
to which the uncertainty about a random event decreases given the knowledge of
the average outcome of another random event \cite{Kinney2010}.

It is natural to conceive of scenarios in which living organisms that can
better resolve signals might have an evolutionary benefit, making it more
likely that their offspring will have a fitness advantage \cite{Taylor2007}. In
recent years there has been a growing interest in understanding the theoretical
limits on cellular information processing \cite{Bialek2005, Gregor2007}, and in
quantifying how close evolution has pushed cellular signaling pathways to these
theoretical limits \cite{Tkacik2008, Dubuis2013, Petkova2019}. While these
studies have treated the signaling pathway as a ``black box,'' explicitly
ignoring all the molecular interactions taking place in them, other studies
have explored the role that molecular players and regulatory architectures have
on these information processing tasks \cite{Rieckh2014, Ziv2007, Voliotis2014a,
Tostevin2009, Tkacik2011, Tkacik2008a, Tabbaa2014}. Despite the great advances
in our understanding of the information processing capabilities of molecular
mechanisms, the field still lacks a rigorous experimental test of these
detailed models with precision measurements on a simple system in which
physical parameters can be perturbed. In this work we approach this task with a
system that is both theoretically and experimentally tractable in which
molecular parameters can be varied in a controlled manner.

Over the last decade the dialogue between theory and experiments in gene
regulation has led to predictive power of models not only over the mean level
of gene expression, but the noise as a function of relevant parameters such as
regulatory protein copy numbers, affinity of these proteins to the DNA
promoter, as well as the extracellular concentrations of inducer molecules
\cite{Golding2005, Garcia2011c, Vilar2013, Xu2015}. These models based on
equilibrium and non-equilibrium statistical physics have reached a predictive
accuracy level such that, for simple cases, it is now possible to design
input-output functions \cite{Brewster2012, Barnes2019}. This opens the
opportunity to exploit these predictive models to tackle the question of how
much information genetic circuits can process. This question lies at the heart
of understanding the precision of the cellular response to environmental
signals. \fref{fig1_intro}(C) schematizes a scenario in which two bacterial
strains respond with different levels of precision to three possible
environmental states, i.e., inducer concentrations. The overlap between the
three different responses is what precisely determines the resolution with
which cells can distinguish different inputs. This is analogous to how the
point spread function limits the ability to resolve two light emitting point
sources.

In this work we follow the same philosophy of theory-experiment dialogue used
to determine model parameters to predict from first principles the effect that
biophysical parameters such as transcription factor copy number and protein-DNA
affinity have on the information processing capacity of a simple genetic
circuit. Specifically, to predict the mutual information between an
extracellular chemical signal (input $c$) and the corresponding cellular
response in the form of protein expression (output $p$)
(\eref{eq_mutual_info}), we must compute the input-output function $P(p \mid
c)$. To do so, we use a master-equation-based model to construct the protein
copy number distribution as a function of an extracellular inducer
concentration for different combinations of transcription factor copy numbers
and binding sites. Having these input-output distributions allows us to compute
the mutual information $I$ between inputs and outputs for any arbitrary input
distribution $P(c)$. We opt to compute the channel capacity, i.e., the maximum
information that can be processed by this gene regulatory architecture, defined
as \eref{eq_mutual_info} maximized over all possible input distributions
$P(c)$. By doing so we examine the physical limits of what cells can do in
terms of information processing by harboring these genetic circuits.
Nevertheless, given the generality of the input-output function $P(p \mid c)$
we derive, the model presented here can be used to compute the mutual
information for any arbitrary input distribution $P(c)$. All parameters used
for our model were inferred from a series of studies that span several
experimental techniques \cite{Garcia2011c, Jones2014a, Brewster2014,
Razo-Mejia2018}, allowing us to make parameter-free predictions of this
information processing capacity \cite{Phillips2019}.

These predictions are then contrasted with experimental data, where the channel
capacity is inferred from single-cell fluorescence distributions taken at
different concentrations of inducer for cells with previously characterized
biophysical parameters \cite{Garcia2011c, Razo-Mejia2018}. We find that our
parameter-free predictions quantitatively track the experimental data up to a
systematic deviation. The lack of numerical agreement between our model and the
experimental data poses new challenges towards having a foundational,
first-principles understanding of the physics of cellular decision-making.

The reminder of the paper is organized as follows. In \secref{sec_model} we
define the minimal theoretical model and parameter inference for a simple
repression genetic circuit. \secref{sec_param} discusses how all parameters for
the minimal model are determined from published datasets that explore different
aspects of the simple repression motif. \secref{sec_moments} computes the
moments of the mRNA and protein distributions from this minimal model. In
\secref{sec_cell_cycle} we explore the consequences of variability in gene copy
number during the cell cycle. In this section we compare experimental and
theoretical quantities related to the moments of the distribution, specifically
the predictions for the fold-change in gene expression (mean expression
relative to an unregulated promoter) and the gene expression noise (standard
deviation over mean). \secref{sec_maxent} follows with reconstruction of the
full mRNA and protein distribution from the moments using the maximum entropy
principle. Finally \secref{sec_channcap} uses the distributions from
\secref{sec_maxent} to compute the maximum amount of information that the
genetic circuit can process. Here we again contrast our zero-parameter fit
predictions with experimental inferences of the channel capacity.

\begin{figure}[h!]
	\centering \includegraphics
  {./fig/main/fig1_intro.pdf}
	\caption{\textbf{Cellular signaling systems sense the environment with
	different degrees of precision}. (A) Schematic representation of a cell as
	a noisy communication channel. From an environmental input (inducer
	molecule concentration) to a phenotypic output (protein expression level),
	cellular signaling systems can be modeled as noisy communication channels.
	(B) We treat cellular response to an external stimulus as a Bayesian
	inference  of the state of the environment. As the phenotype (protein
	level) serves as the internal representation of the environmental state
	(inducer concentration), the probability of a cell being in a specific
	environment given this internal representation $P(c \mid p)$ is a function
	of the probability of the response given that environmental state $P(p \mid
	c)$. (C) The precision of the inference of the environmental state depends
	on how well can cells resolve different inputs. For three different levels
	of input (left panel) the green strain responds more precisely than the
	purple strain since the output distributions overlap less (middle panel).
	This allows the green strain to make a more precise inference of the
	environmental state given a phenotypic response (right panel).}
  \label{fig1_intro}
\end{figure}
