\section{Alvaro Sanchez}

\begin{tcolorbox}
The description of the rates are not consistent with the level of coarse
graining in the model. It is perfectly fine if you guys lump all of the
complicated kinetic steps into an effective ``on'' and ``off'' state. But these
has to be explicitly stated in the paper not to make us believe that your
$\kpon$ and $\kpoff$ rates are actual binding and unbinding rates of the RNAP.
\end{tcolorbox}

We greatly appreciate Al's input towards making more explicit the assumptions
that go into the model. All these details, while subtle are extremely important
to highlight the shortcomings of our approach. To make sure that we do not
misguide the reader to believe that the rates we show for the transition between
the transcriptionally active and inactive states we have added the following
text in the results section

``In Fig. 2(A) we show the minimal model and the necessary set of parameters
needed to predict mRNA and protein distributions. Specifically, we assume a
three-state model where the promoter can be found 1) In a transcriptionally
active state ($A$ state), 2) In a transcriptionally inactive state without the
repressor bound ($I$ state) and 3) with the repressor bound ($R$ state). We
highlight that we are not assuming that the transition between the active state
$A$ and the inactive state $I$ happens due to RNAP binding to the promoter. The
transcriptional initiation kinetics involve several more steps than simple
binding \cite{Browning2004}. We coarse-grain all these steps into an effective
``on" and ``off" states for the promoter consistent with experiments
demonstrating the bursty nature of gene expression in {\it E. coli}
\cite{Golding2005}.''

Fig. 2(A) has been accordingly updated to be more consistent with this
assumption in the model. Also in the discussion we added the following based on
your input:

``Also the minimal model in Fig 2(A), despite being widely used, is an
oversimplification of the physical picture of how the transcriptional machinery
works. The coarse-graining of all the kinetic steps involved in the
transcription initiation into two effective promoter states - active and
inactive - ignores potential kinetic regulatory mechanisms of intermediary
states \cite{Scholes2017}. Furthermore it has been argued that despite the fact
that the mRNA count distribution does not follow a Poisson distribution, this
effect could be caused by unknown factors not at the level of transcriptional
regulation \cite{Choubey2018}.''
