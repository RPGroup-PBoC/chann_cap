\section{Rachel Taubman}

This paper is very interesting. The use of the moments of the protein
distribution to get to channel capacity seems like an exciting metric for future
channel capacity experiments. I mostly have questions about the timescale of the
experiments conducted for the data in this paper and the ways that the E. Coli
might have interacted with their environments over the course of the
experiments. I also noticed a couple of typos in the draft; let me know if you
want me to point them out. These are two papers Sarah sent me that influenced a
lot of my questions:

Paper 1: “Noise, Information and Fitness in Changing Environments,” Pedraza,
Garcia and Pérez-Ortiz
Paper 2: “Optimality and evolutionary tuning of the expression level of a
protein”, Dekel and Alon

\subsection{Timescale questions}
\begin{tcolorbox}
Could there be slightly different cell cycle timescales for cells with different
channel capacities?
\end{tcolorbox}

\begin{tcolorbox}
What experiment timescale are you envisioning for fitness experiments? Would a
different timescale make it difficult to connect the results of a fitness
experiment to the findings in this paper?
\end{tcolorbox}

\subsection{Experimental conditions, evolution, fitness}

\begin{tcolorbox}
Do the channel capacity parameters account for the fact that after ~150
generations, E. coli would probably select against LacZ activity in the presence
of only IPTG, as opposed to lactose, which would be conferring a growth benefit?
(Fig. 4 (A), paper 2). Based on paper 1 results, I’m wondering if it would be
good to conduct separate trials with just IPTG, and some with varying lactose
levels to compare results
\end{tcolorbox}

\begin{tcolorbox}
Do the calculations consider how cells could be changing their environment
during growth? (Fig 4. (A), Paper 1)
\end{tcolorbox}

\begin{tcolorbox}
How do you think the optimal entropy distribution would be different, if at all,
if you periodically varied the environmental conditions, like in paper 1?
\end{tcolorbox}

\begin{tcolorbox}
Would entropy distributions evolve in a predictable way? If variability is
better for fitness in those environments, would the mutual information depend
more on the overall variability of environment or on the conditions of the
environment at given points in time? Could E. coli evolve to have channel
capacity mostly attuned to a variability parameter?
\end{tcolorbox}

\begin{tcolorbox}
Would it be possible to come up with a fitness function for the experimental
data for this paper? Do you have growth and division data for the studies used
for this paper? If so, could a simple fitness function predict the channel
capacities you predicted and experimentally determined?
\end{tcolorbox}

\begin{tcolorbox}
Also would “optimal protein expression for max growth” definitely correspond to
maximum channel capacity? 
\end{tcolorbox}
