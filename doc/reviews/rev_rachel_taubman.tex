\section{Rachel Taubman}

This paper is very interesting. The use of the moments of the protein
distribution to get to channel capacity seems like an exciting metric for future
channel capacity experiments. I mostly have questions about the timescale of the
experiments conducted for the data in this paper and the ways that the E. Coli
might have interacted with their environments over the course of the
experiments. I also noticed a couple of typos in the draft; let me know if you
want me to point them out. These are two papers Sarah sent me that influenced a
lot of my questions:

Paper 1: “Noise, Information and Fitness in Changing Environments,” Pedraza,
Garcia and Pérez-Ortiz

Paper 2: “Optimality and evolutionary tuning of the expression level of a
protein”, Dekel and Alon

\subsection{Timescale questions}
\begin{tcolorbox}
Could there be slightly different cell cycle timescales for cells with different
channel capacities?
\end{tcolorbox}
We thank Rachel for this super fun comment. It is an interesting question to
ask whether cells with different channel capacities would have different
response times. In the present paper we imagine that any environmental state
change that cells could experience would happen at a time scale much longer than
the time it takes them to reach a steady-state gene expression. This time scale
to reach steady-state is set by the protein degradation time, which in our case
the only source of protein degradation is via dilution due to cell growth and
division. In general this is the time scale that it would take for cells to
reach a steady-state expression, and there are studies (such as the one
mentioned at the end of this response) that argue this could serve as a form of
short-term memory for bacteria when exposed to environments that fluctuate much
faster. But for the purpose of our work we are explicitly ignoring the time-
scales having to do with the environment itself.

Lambert, G., & Kussel, E. (2014). Memory and Fitness Optimization of Bacteria
under Fluctuating Environments. PLoS Genetics, 10(9), e1004556.
http://doi.org/10.1371/journal.pgen.1004556

\begin{tcolorbox}
What experiment timescale are you envisioning for fitness experiments? Would a
different timescale make it difficult to connect the results of a fitness
experiment to the findings in this paper?
\end{tcolorbox}
We again thank Rachel for this excellent comment. The issue of the time scale
for our follow-up experiments is a great source of concern for us. As our model
explicitly works at a steady state under a constant environment, we need to
work under conditions where these assumptions are approximately met. One
alternative that we have thought would be to let cells reach steady state -
which we have shown to be in the order of 8 hours - without any selection
pressure, and then only when they have reached this point we include the same
environmental signal (this is the inducer concentration), but this time
accompanied with the challenge (an antibiotic for example). This strategy of
letting cells reach gene expression steady-state before they face the challenges
has been done in this fantastic paper from which we have gotten lots of ideas
for the follow-up experiments.

Poelwijk, F., de Vos, M. G. J., & Tans, S. (2011). Tradeoffs and Optimality in
the Evolution of Gene Regulation. Cell, 146(3), 462–470.
http://doi.org/10.1016/j.cell.2011.06.035

\subsection{Experimental conditions, evolution, fitness}

\begin{tcolorbox}
Do the channel capacity parameters account for the fact that after ~150
generations, E. coli would probably select against LacZ activity in the presence
of only IPTG, as opposed to lactose, which would be conferring a growth benefit?
(Fig. 4 (A), paper 2). Based on paper 1 results, I’m wondering if it would be
good to conduct separate trials with just IPTG, and some with varying lactose
levels to compare results
\end{tcolorbox}
We again thank Rachel for the great food for thought. In Alon's paper (which we
are big fans of) they grew cells over several generations under a constant
environment. In the case for cells growing in the absence (or very low
concentration) of lactose there was an advantage on loosing the expression of
the lac operon. Our study deals with ever-changing environments on which
expressing the gene (which is not LacZ for our synthetic system) is sometimes
beneficial and other times detrimental. That is the basis for the argument on
why natural selection would act on the amount of information that cells can
process. If cells can infer the environmental state and respond accordingly by
up downregulating their gene expression, then they could grow faster than either
not expressing the gene at all or expressing it constitutively.

\begin{tcolorbox}
Do the calculations consider how cells could be changing their environment
during growth? (Fig 4. (A), Paper 1)
\end{tcolorbox}
While the question itself is very interesting, it falls out of the scope for
this piece of work. In this case we assume that the environment is fixed. There
is no interaction between cells and environment that can affect the environment
itself. This is a simplification rather than a reflection of the reality of how
cells interact with their environment. It is true that in future iterations of
this type of questions the feedback between enviromnenttal conditions and the
organisms themselves should be taken into account.

\begin{tcolorbox}
Would it be possible to come up with a fitness function for the experimental
data for this paper? Do you have growth and division data for the studies used
for this paper? If so, could a simple fitness function predict the channel
capacities you predicted and experimentally determined?
\end{tcolorbox}
This is definitely something we are working on. We have in mind several schemes
for how to implement a simple fitness function both in theory and in experiment,
but this is definitely work in progress.
