\section{Nature Communications Reviewer 02}

This clearly written paper combines a wealth of experimental data to define and
parameterize a stochastic model of regulated gene expression, predict protein
distributions from this model, and then estimate the mutual information between
these protein distributions and the concentration of an inducer molecule. The
predicted mutual information agrees favorably with the mutual information
estimated from experiments.

\begin{tcolorbox}
As I understand things, the authors have demonstrated that it is possible to
build a stochastic model of gene expression that can accurately predict protein
distributions after being parameterized by careful fitting to a collection of
experiments. The agreements with the experimental estimates of capacity come
from these accurate predictions of the protein distributions, and the authors
even show in the SI that their predictions need not be so accurate to
successfully predict capacities: a symmetric, Gaussian approximation of the
protein distributions does almost as well.

Although it is important (and reassuring) to know that these predictions are
possible, the take-home messages of the paper could be clearer. If the authors
were to do everything again, are all the experiments necessary? As progress in
the end becomes numerical, is it better to use Gillespie simulations rather than
go through numerical integration of the moment equations and the max-ent
approach? Would the linear noise approximation correctly predict the capacity?
There are hints of the answers to these questions in the SI, but I think these
or similar points should be made a focus in the discussion.
\end{tcolorbox}
We are glad that the reviewer found the paper easy to follow.


\begin{tcolorbox}
Similarly, I do not have a clear picture of the biological conclusions. Is the
value of the capacity found surprising? It is much lower than log2(12). The
non-monotonic behavior in Fig 5 is interesting, but is it because of high
repressors permanently repressing the promoter (page 13) or from the allostery
of the repressors (page 15) or both? Why do extrinsic fluctuations not matter? I
would expect extrinsic fluctuations to generate much of the spread of the
empirical distributions. Similarly, in the model, I believe that even the number
of repressors is fixed for each strain. Why does that work? By making a Gaussian
approximation to the distributions, perhaps more progress can be made
analytically and so give an understanding of the value of the capacity and ways
to increase it (the techniques in Swain et al., PNAS 2002, might circumvent the
numerical integration).
\end{tcolorbox}

That said, the presentation is exceptionally clear and the SI, which is written
pedagogically, should become a valuable resource.

I have only two technical questions. In Eq. S22, are you assuming that all
bacteria replicate their DNA at the same point in the cell cycle? If so, that
should be stated. I enjoyed reading Sec S6. Is it correct that the number of
bins in the plateau region of Fig S23 can be used in a loose way to determine
the fluorescence per molecule in your data so that you can interpret Eq. S121 as
an equality?
