\section{Sarah Marzen}

So while I like the idea of the paper, I do have some comments that might occur
to the referees:

\begin{tcolorbox}
1)  Why is channel capacity the right thing to calculate?  Are these cells such
good engineers of their environment that they can tune P(c)?  Or are you just
saying that this is the maximum resolution these circuits could possibly have?
Either way, might want to justify channel capacity as the thing to calculate
earlier in the paper.  And if you're going to imply a fitness benefit to
additional bits, I think it's worth noting that maximum channel capacity
transducers are deterministic, though that seems to go against how the weak
binding gives higher channel capacity than strong binding (Fig 5)-- in fact,
that counterintuitive behavior with weakness of the binding site could use some
explanation.
\end{tcolorbox}

\begin{tcolorbox}
2)  To get a seemingly static distribution $P(p|c)$, you numerically integrate
$P_t(p|c)$ over time $t$.  If you're in steady state, then this procedure is
fine, and time $t$ shouldn't matter.  If you're dealing with a nonstationary
problem by averaging and getting a channel capacity from that average, then
well... it seems a little fishy to me, and I feel like you should be dealing
with the mutual information between input and output trajectories.
\end{tcolorbox}

\begin{tcolorbox}
3)  I still think that when you say your theoretically predicted channel
capacity is the same as the experimentally measured channel capacity, you're
really saying that the theoretically predicted $P(p|c)$ is equivalent to the
measured $P(p|c)$.  Is that true?  And I have a bone to pick about the word
"measured"-- I wouldn't say that you've measured the channel capacity since
you've really measured $P(p|c)$ and calculated channel capacity from that (as in
you didn't directly measure channel capacity, but had to infer it from what you
did measure).
\end{tcolorbox}
