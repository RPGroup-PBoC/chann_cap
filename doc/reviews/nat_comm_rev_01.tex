\section{Nature Communications Reviewer 01}

The paper by Razo-Mejia and Phillips show that a simple “first principle” model
with experimentally estimated parameters agrees with the measurement of
experimentally measured channel capacity statistics. The work is of very high
technical quality and I don’t have any comments or concerns related to the
theoretical, computational and experimental steps used here. I do however have a
few, more conceptual, disagreements with the approaches that the authors used.
The first is a bit more general and the last two might reflect differences in
scientific cultures between disciplines and communities. Therefore, I am not
making any recommendation whether this paper should be accepted and I leave it
to the editor to judge how these differences in opinion should impact
publication.

\begin{tcolorbox}
Going beyond mean field behavior for a stochastic model is important and
demonstrating agreement of higher moments or full distributions is an important
step. However, the authors used to compare not the full distributions but a
complex statistic calculated from these distributions, the channel capacity. It
is not clear to me if the agreement of this complex statistic is as strong as
directly testing the underlying distributions that generated it. For example,
let's say that the optimization of input distributions between theory and
experiment is different while overall channel capacity is the same. Does that
still support the notion that theory and experiments agree? I would have been
more satisfied if the model can predict full distributions or at least the first
few moments calculated based on stochastic simulation and experimentally
measured expression distributions would have agreed.
\end{tcolorbox}

We thank Reviewer 1 for this comment. We completely agree that channel capacity,
being a complex statistic that averages over a family of protein distributions
for different inducer concentrations might not directly reflect whether or not
the theoretical input-output distributions $P(p \mid c)$ agree with the
experimental ones. Figures S17-S19 show the comparisons that the reviewer
suggest. From these collection of plots we can see that to a large degree the
theoretical predictions agree with the experimental data for the tested
combinations of repressor copy number, repressor-DNA binding energy, and
inducer concentration. The reason we decided not to focus on this agreement in
the main text was because we believe that the information-theoretic perspective
has important connections with ``what the cell cares about.'' Several pieces of
work cited in the paper such as \cite{Bergstrom2004, Taylor2007, Tkacik2008,
Nemenman2010} motivate the idea that natural selection can act o the number of
bits that living systems can process.

Having said that, in section 1.5 and in the Discussion we point the reader to
the Figures in the SI showing the comparison of the distributions since we
completely agree that seeing this agreement increases the confidence on the
predictive power of our theoretical model.

\begin{tcolorbox}
The authors use a simple kinetic model of expression from a gene in the presence
of a repressor and write it using chemical master equation formulation. This
model was used in the past by the group and they even refer to it as the
“hydrogen atom” of gene expression. To derive the values that they are
interested in predicting, the authors go through a complex series of analytical
derivation. Due to their complexity at a few steps along the way the authors
resort to numerical approximations. In the end, I am completely unclear what is
the value of all the complex derivation. The channel capacity could have been
directly estimated using standard Gillespie stochastic simulation without all
the trouble and numerical estimation of six distributions moments that are then
used to feed another numerical estimation procedure that uses maximum entropy,
etc. I am all for analytical derivation when in the end there is a clear and
interesting relationship that emerges and enhances our understanding of the
system. This is not the case here and direct stochastic simulations would have
been easier to follow and provide the same amount (or even higher) degree of
accuracy.
\end{tcolorbox}

We thank again the reviewer for this comment. To clarify this point, the
derivations done in the paper to infer the parameter values are based on our
insight that if the equilibrium model and the kinetic model are describing the
same process, the predictions should be self-consistent between them. Therefore
all kinetic rates for our model are inferred under such constraint. For these
inferences there were no numerical approximations done. All we did was use our
knowledge from working with equilibrium models, along with the data that came
with it, and use Bayesian inference to learn the parameter values for our
kinetic model. So every parameter for the model presented in Fig. 2(A) was
inferred without referring to numerical approximations.

The numerics entered our calculations when we needed to solve the dynamics of
the master equation over several cell cycles. These numerics are necessary since
most master equations do not have closed form analytical solutions, and if they
do they usually involve complicated functions such as confluent hypergeometric
functions \cite{Peccoud1995, Shahrezaei2008}. While we agree that a Gillespie
simulation would  give an equivalent result, computationally this is an
expensive task. To put this in perspective, for Fig. 5(A) where we compute the
channel capacity as a function of repressor copy number for 3 different
operators, we have a sample space of $\approx 50$ possible values for the mRNA
copy number times $\approx 20,000$ possible values of the protein copy number.
That is a sample space of size $\approx 10^6$. To sample this reliably we would
need to perform on the order of $N^2$ number of Gillespie runs over
several cell cycles, where $N$ is the sample space size. Running each trajectory
for a moderate time of $\approx 10^4$ time steps takes us to a total of
$\approx 10^{16}$. If each step takes on the order of 1 $\mu$s for an efficient
vectorized implementation of Gillespie, the simulation will be running up to



\begin{tcolorbox}
Finally, and this is perhaps where the cultural differences I mentioned above
are most in effect, I found the paper to lack any major findings. The only new
biological insights I got from reading this paper is that the model shown in
Figure 2 is very likely a correct one. That is exactly what I was expecting, as
this model has been tested in different forms for a few decades now. Showing yet
another higher degree of “correctness” of an already tested theory seems
somewhat marginal to me. I appreciate that this is not the same in other
scientific communities and therefore I specifically emphasize that this point
should be considered by the editor who is the best judge of interest of the
readership of this journal. The other few points made by the authors related to
mutual information and the cases where cells are or aren’t able to sense
environmental changes were already addressed by the large body of work that
utilized information theory to analyze cell response distributions.
\end{tcolorbox}

We thank the reviewer for giving us his/her vision of our findings from the a
different perspective to ours. While it is impossible to argue in favor or
against a personal opinion on the scientific merit of our work, we stand by our
position that the work presented here represents an important advancement for
the field. We agree that similar kinetic models have been proposed in the past.
The difference is that these models have either been fit to the data they were
directly predicting \cite{Golding2005}, or they remained as theoretical work
with no connection to actual single-cell data \cite{Shahrezaei2008}. These are
not criticisms to these outstanding pieces of work, but we view the value of our
contribution as a further advancement in this field by having a predictive
model with experiments specifically design to test the model.

To our knowledge our group was the first one to propose a molecularly detailed
three-state model rather than the usual coarse-grained two-state model
\cite{Phillips2015}. The difference being that in the classic two-state model
the regulation comes from transitioning from the trancriptionally active state
$A$ to the inactive state $I$. So, upon removal of the regulation we are left
only with the $A$ state, and the strong prediction that the steady-state mRNA
distribution from this model has to be Poisson. For our three-state case if we
remove the regulation given by the repressor we are left with a two-state model
that still transitions between states $A$ and $I$. That means we are not
constrained by the case that the mRNA distribution has to strictly be
Poissonian, but we have the flexibility to account for the bursty nature of gene
expression of constitutive promoters. We show makes a significant difference
with respect to the model being able to predict the steady state mRNA
distribution of a constitutive promoter as well as the regulated case.

Furthermore, we see this paper as a culmination of more than a decade of work
in our lab where we have extensively dissected this simple system with
increasing levels of complexity in our predictions. From a physics-oriented
perspective we find it compelling that we can take a molecular system, use an
extremely simple theoretical model that accounts for the interactions we think
are relevant, determine all parameters from different data sets spanning several
methods to measure gene expression, and still be able to predict full
input-output distributions as well as complex statistics such as the mutual
information on newly acquired data with no further fit parameters.

From a more biological-oriented perspective, we see the information-theoretic
frame of our analysis as extremely relevant to the question of what are the
limits to the performance of these molecular signaling systems. In addition, as
the reviewer alluded in the comment, there is a rich literature that explores
the relevance of mutual information as a trait on which natural selection can
act \cite{Bergstrom2004, Taylor2007, Tkacik2008, Nemenman2010}. So the
theory-experiment dialogue presented in this work opens the possibility of
having predictive power over evolutionary trajectories based on the capacity of
a cells signaling network to infer the state of the environment with high
accuracy.
