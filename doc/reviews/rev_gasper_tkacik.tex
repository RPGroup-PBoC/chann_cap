\section{Gasper Tkacik}

This is a beautiful connection between theory and the experiment!

I’m not commenting much on the bulk of the paper that constructs a biophysical
model except in so far it links to questions below.  I have some questions that
I feel could make the paper even stronger:

\begin{tcolorbox}
1) What are the optimal distributions recovered by BA algorithm? Don’t worry if
they are spiky: it is interesting to ask what is the total weight at low vs high
vs middle expression or so. Is there real evidence for a meaningful
“intermediate” state of induction, since you go over 1 bit? Maybe some playing
around needs to be done to figure out what are the key properties and not
numerical artefacts of these optimal solutions, or parametrize them in some
low-D space and optimize there. Our previous experiences have been that these
distributions are not very finely tuned, and only a few key statistics are
really important.
\end{tcolorbox}

\begin{tcolorbox}
2) How does the capacity you report depend on various assumptions: (i) cell
cycle; (ii) non-gaussian noise distributions (order-by-order); (iii) multi-state
promoter; (iv) optimal vs log-flat distributions (= is Blahut-Arimoto really
important)? It would be great to have some sort of quantitative summary, all in
terms of the information, of how these (and other?) factors impact transmission,
like a table or a pie chart or something. This would also give us a metric on
how much various assumptions / mechanisms functionally matter, which many people
ask themselves about. You guys are in a perfect situation since you have a real
biophysical model that is properly calibrated. So something like: it is really
important to get the three state model, but not so important to take into
account the multiple copies of the DNA while replicating.
\end{tcolorbox}

\begin{tcolorbox}
3) The fact that high repressor copy number turns down the transmission is
interesting -> we had to put in a “cost” to regulation explicitly to get an
optimum in transmission, otherwise in our theory with Bill and Alex, more TF is
always good. Maybe it is an interesting comment to see more precisely why this
happens. Do you titrate out the inducer with excess TFs? Or is it that there is
some residual binding of the TF even in the inactive form, and it thus represses
still, which gets important at high repressor copy number? In general, it is an
interesting point since we have been speculating about how excess TFs can mess
up regulation (https://www.nature.com/articles/ncomms12307), and have been doing
some followup work on this.
\end{tcolorbox}

\begin{tcolorbox}
4) What I find extremely cool is that you have “experimentally instantiated” the
kind of optimization w.r.t. molecular parameters that we studied with Bill and
Aleks and Sokolowski in the PRE series of papers (Part I, II, III, IV). For
instance, Figure 4 clearly suggests that there is an optimal combination of
repressors/cell and the binding energy (at a given copy number, this would give
you an optimum for the binding energy). This is an ab-initio prediction for the
energy, which could be contrasted in the WT system to the actual value. I’m not
sure that in this paper there is scope to do this, but maybe it is nice to
connect it to optimization approaches and show why also “mutual information”
might be a relevant quantity: it could have high predictive power if it
correctly predicts the molecular properties of regulation. Sometimes, even the
existence of the optimum of information w.r.t. some molecular parameter is
surprising (at least to us it was in the first years) since it arises from the
non-trivial interaction between the shape of mean induction, dynamic range, and
the noise response.
\end{tcolorbox}

\begin{tcolorbox}
5) Crazy question: if you set up an experiment where the bugs have an interest
to express tightly in response to the environment, could you see evolutionary
dynamics towards information maximum? Or examine how they evolve towards higher
information? Long time ago we wanted to take this approach with Sergey
Kryazhimskiy (at UCSD). I think it is still a thrilling prospect for an
evolutionary experiment… this is something you seem to suggest in the last
sentence, and I would be very curious on how you’d plan to do it (we had some
more concrete ideas, in that case with yeast and temperature, but this would be
great to maybe think about together).
\end{tcolorbox}

\begin{tcolorbox}
In general, I think points (2) and (4) can be highlighted in the Discussion as a
motivation for mutual information as an actually interesting / relevant metric.
First, it can “measure” how important each actual physical effect, or our
approximation in the computations, is to the function of a regulatory circuit
(2). Second, it can predict optimal molecular / systems parameters that can be
compared to real measurements (4). I know these are obvious in hindsight, but
for a broader field, one could highlight again.
\end{tcolorbox}
