\documentclass [11pt]{article}

\usepackage{setspace}
\usepackage{amssymb}
\usepackage{amsmath} 
\usepackage{amsfonts} 
\usepackage{amssymb}
\usepackage{setspace}
\usepackage{amsthm}
\usepackage{textcomp}
\usepackage{graphicx}
\usepackage{url}
\usepackage{color}
\usepackage[table]{xcolor}
\usepackage{cancel}
\usepackage{comment}
\usepackage[framemethod=TikZ]{mdframed}
\usepackage{enumitem}
\usepackage{wasysym}
\usepackage{listings}
\usepackage[colorlinks=true, urlcolor=blue, hyperfootnotes=false, hyperfigures=false, hyperindex=false, linkcolor=black]{hyperref}
\usepackage[font=small, margin=1cm]{caption}
\usepackage{float}
\usepackage{booktabs}
\usepackage{fixltx2e}
\usepackage{threeparttable}
\usepackage{titling}
\usepackage{zref-base}
\usepackage{makecell}
\usepackage{array}
\usepackage{hhline}
\usepackage{titlesec}

%Latin accents
\usepackage[utf8]{inputenc}

%subfigures
\usepackage{caption}
\usepackage{subcaption}
% Margins and spacings
\setlength{\evensidemargin}{0.0cm}
\setlength{\oddsidemargin}{0.0cm}
\setlength{\topmargin}{-1.0cm}
\setlength{\textwidth}{17cm}
\setlength{\textheight}{22cm}
\setlength{\parskip}{2.5mm}
\reversemarginpar
\marginparsep  0.1in
\marginparwidth 0.7in

% Give more spacing in equation arrays
\setlength{\jot}{10pt}

% Allow page breaks in multiline equations
\allowdisplaybreaks

% Set up title spacing so we don't waste so much space
\setlength{\droptitle}{-8em}
\date{\vspace{-5em}}  % No date will appear in title.

% Spacing between section headings and text
\titlespacing\section{0pt}{12pt plus 4pt minus 2pt}{-2pt plus 2pt minus 2pt}
\titlespacing\subsection{0pt}{12pt plus 4pt minus 2pt}{-2pt plus 2pt minus 2pt}
\titlespacing\subsubsection{0pt}{12pt plus 4pt minus 2pt}{-2pt plus 2pt minus 2pt}

% Set up problem structure
\newtheoremstyle{break}% name
  {}%         Space above, empty = `usual value'
  {2em}%         Space below
  {}%         Body font
  {}%         Indent amount (empty = no indent, \parindent = para indent)
  {\bfseries}% Thm head font
  {.}%        Punctuation after thm head
  {\newline}% Space after thm head: \newline = linebreak
  {}%         Thm head spec

% Problems and solutions with counter
\newcounter{solution}
\renewcommand{\thesolution}{\arabic{solution}}


% %%%%%%%%%%%%%%%%%%%%%%%%%%%%%%%%%%%%%%%%%%%%%%%%%%%%%%%%%%%%%%%%%%%
% Set up boxes for solutions
% %%%%%%%%%%%%%%%%%%%%%%%%%%%%%%%%%%%%%%%%%%%%%%%%%%%%%%%%%%%%%%%%%%%
%% setup the solution environment
\mdfdefinestyle{solution}{%
    outerlinewidth=2pt,
    bottomline=false,
    leftline=false,rightline=false,
    topline=false,
    font=\small,%
    backgroundcolor=gray!20,%
    skipabove=-5em,
    skipbelow=\baselineskip,
    roundcorner=10pt,%
    innertopmargin=1em,%
    innerbottommargin=1em,%
    splittopskip=1em,%
    splitbottomskip=1em,%
    frametitle=\mbox{},
}
\newmdenv[%
    style=solution,
    settings={\global\refstepcounter{solution}},
    frametitlefont={\bfseries Solution~\thesolution\quad},
]{solution}

%% Shaded box
\newmdenv[style=solution]{shaded}

%% setup the solution environment
\mdfdefinestyle{unshaded}{%
    outerlinewidth=1pt,
    bottomline=true,
    leftline=true,rightline=true,
    topline=true,
    font=\small,%
    backgroundcolor=white,%
    skipabove=-5em,
    skipbelow=\baselineskip,
    roundcorner=10pt,%
    innertopmargin=1em,%
    innerbottommargin=1em,%
    splittopskip=1em,%
    splitbottomskip=1em,%
    frametitle=\mbox{},
}

%% Unshaded box
\newmdenv[style=unshaded]{unshaded}
% %%%%%%%%%%%%%%%%%%%%%%%%%%%%%%%%%%%%%%%%%%%%%%%%%%%%%%%%%%%%%%%%%%%
% %%%%%%%%%%%%%%%%%%%%%%%%%%%%%%%%%%%%%%%%%%%%%%%%%%%%%%%%%%%%%%%%%%%


% Convenient micron symbol
\newcommand{\micron}{{\textmu}m}

\theoremstyle{break}
\newtheorem{problem}{Problem}

%% \theoremstyle{break2}
%% \newtheorem{solution}{Solution}
%% \numberwithin{solution}{section}

% No excess spacing for lists
\setlist{itemsep=0pt, topsep=0pt}

% Allow paragraph indentations in lists
\setitemize{listparindent=\parindent}
\setenumerate{listparindent=\parindent}

% Column type for tables with nice spacing
\newcolumntype{M}[1]{>{\centering\arraybackslash}m{#1}}
\newcolumntype{N}{@{}m{0pt}@{}}

% %%%%%%%%%%%%%%%%%%%%%%%%%%%%%%%%%%%%%%%%%%%%%%%%%%%%%%%%%%%%%%%%%%%
% Settings for display of code
% %%%%%%%%%%%%%%%%%%%%%%%%%%%%%%%%%%%%%%%%%%%%%%%%%%%%%%%%%%%%%%%%%%%
\definecolor{mygreen}{rgb}{0,0.6,0}
\definecolor{mygray}{rgb}{0.5,0.5,0.5}
\definecolor{mymauve}{rgb}{0.58,0,0.82}

\lstset{ %
  backgroundcolor=\color{gray!20},   % choose the background color; you must add \usepackage{color} or \usepackage{xcolor}
  basicstyle=\footnotesize\ttfamily, % the size of the fonts that are used for the code
  breakatwhitespace=false,         % sets if automatic breaks should only happen at whitespace
  breaklines=true,                 % sets automatic line breaking
  captionpos=b,                    % sets the caption-position to bottom
  commentstyle=\color{mygreen},    % comment style
  deletekeywords={...},            % if you want to delete keywords from the given language
  escapeinside={\%*}{*)},          % if you want to add LaTeX within your code
  extendedchars=true,              % lets you use non-ASCII characters; for 8-bits encodings only, does not work with UTF-8
  frame=single,                    % adds a frame around the code
  keepspaces=true,                 % keeps spaces in text, useful for keeping indentation of code (possibly needs columns=flexible)
  keywordstyle=\color{blue},       % keyword style
  morekeywords={*,...},            % if you want to add more keywords to the set
  numbers=left,                    % where to put the line-numbers; possible values are (none, left, right)
  numbersep=5pt,                   % how far the line-numbers are from the code
  numberstyle=\tiny\color{mygray}, % the style that is used for the line-numbers
  rulecolor=\color{black},         % if not set, the frame-color may be changed on line-breaks within not-black text (e.g. comments (green here))
  showspaces=false,                % show spaces everywhere adding particular underscores; it overrides 'showstringspaces'
  showstringspaces=false,          % underline spaces within strings only
  showtabs=false,                  % show tabs within strings adding particular underscores
  stepnumber=2,                    % the step between two line-numbers. If it's 1, each line will be numbered
  stringstyle=\color{mymauve},     % string literal style
  tabsize=2,                       % sets default tabsize to 2 spaces
  title=\lstname,                   % show the filename of files included with \lstinputlisting; also try caption instead of title
  xleftmargin=2em,                 % Indent the listings
  belowskip=-2em                      % No extra space below listing
}

%% Code below removes line numbering for listings that have less than 12 lines
\makeatletter
\newcounter{mylstlisting}
\newcounter{mylstlines}
\lst@AddToHook{PreSet}{%
  \stepcounter{mylstlisting}%
  \ifnum\mylstlines<12\relax
    \lstset{numbers=none}
  \else
    \lstset{numbers=left}
  \fi
  \setcounter{mylstlines}{0}%
}
\lst@AddToHook{EveryPar}{%
  \stepcounter{mylstlines}%
}
\lst@AddToHook{ExitVars}{%
  \begingroup
    \zref@wrapper@immediate{%
      \zref@setcurrent{default}{\the\value{mylstlines}}%
      \zref@labelbyprops{mylstlines\the\value{mylstlisting}}{default}%
    }%
  \endgroup
}

% \mylstlines print number of lines inside listing caption
\newcommand*{\mylstlines}{%
  \zref@extractdefault{mylstlines\the\value{mylstlisting}}{default}{0}%
}
\makeatother

% %%%%%%%%%%%%%%%%%%%%%%%%%%%%%%%%%%%%%%%%%%%%%%%%%%%%%%%%%%%%%%%%%%%
% %%%%%%%%%%%%%%%%%%%%%%%%%%%%%%%%%%%%%%%%%%%%%%%%%%%%%%%%%%%%%%%%%%%

%Add a printed pdf page
\usepackage{pdfpages}
\usepackage{mathtools}
\usepackage{wrapfig}