\begin{abstract}
  Given the stochastic nature of gene expression, genetically identical cells
  exposed to the same environmental inputs will produce different outputs. This
  heterogeneity has been hypothesized to have consequences for how cells are
  able to survive in changing environments. Recent work has explored the use of
  information theory as a framework to understand the accuracy with which cells
  can ascertain the state of their surroundings. Yet the predictive power of
  these approaches is limited and has not been rigorously tested using
  precision measurements. To that end, we generate a minimal model for a simple
  genetic circuit in which all parameter values for the model come from
  independently published data sets. We then predict the information processing
  capacity of the genetic circuit for a suite of biophysical parameters such as
  protein copy number and protein-DNA affinity. We compare these parameter-free
  predictions with an experimental determination of protein expression
  distributions and the information processing capacity of {\it E. coli} cells,
  and find that our minimal model describes the experimental data within a
  systematic multiplicative deviation. This work advances our understanding of
  the precision with which simple molecular circuits process information.
\end{abstract}
