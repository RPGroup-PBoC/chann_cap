\section{Three-state promoter model for simple repression}\label{supp_model}

In order to tackle the question of how much information the simple repression
motif can process we require the joint probability distribution of mRNA and
protein $P(m, p)$. To obtain this distribution we use the chemical master
equation formalism as described in \secref{sec_model}. Specifically we assume a
three-state model where the promoter can be found 1) with RNAP bound ($B$
state), 2) empty ($E$ state) and 3) with the repressor bound ($R$ state) (See
\fref{fig2_minimal_model}(A)). These three states generate a system of coupled
differential equations for each of the three state distributions $P_B(m, p)$,
$P_E(m, p)$ and $P_R(m, p)$. Given the rates shown in
\fref{fig2_minimal_model}(A) let us define the system of ODEs. For the RNAP
bound state we have
\begin{equation}
  \begin{aligned}
    \dt{P_B(m, p)} &=
    - \overbrace{\kpoff P_B(m, p)}^{P \rightarrow E} % P -> E
    + \overbrace{\kpon P_E(m, p)}^{E \rightarrow P}\\ % E -> P
    &+ \overbrace{r_m P_p(m-1, p)}^{m-1 \rightarrow m} % m-1 -> m
    - \overbrace{r_m P_p(m, p)}^{m \rightarrow m+1}% m -> m+1
    + \overbrace{\gm (m + 1) P_B(m+1 , p)}^{m+1 \rightarrow m} % m+1 -> m
    - \overbrace{\gm m P_B(m , p)}^{m \rightarrow m-1}\\ % m -> m-1
    &+ \overbrace{r_p m P_B(m, p - 1)}^{p-1 \rightarrow p} % p-1 -> p
    - \overbrace{r_p m P_B(m, p)}^{p \rightarrow p+1} % p -> p+1
    + \overbrace{\gp (p + 1) P_B(m, p + 1)}^{p + 1 \rightarrow p} % p+1 -> p
    - \overbrace{\gp p P_B(m, p)}^{p \rightarrow p-1}. % p -> p-1
  \end{aligned}
\end{equation}
For the empty state $E$ we have
\begin{equation}
  \begin{aligned}
    \dt{P_E(m, p)} &=
    \overbrace{\kpoff P_B(m, p)}^{P \rightarrow E} % P -> E
    - \overbrace{\kpon P_E(m, p)}^{E \rightarrow P} % E -> P
    + \overbrace{\kroff P_R(m, p)}^{R \rightarrow E} % R -> E
    - \overbrace{\kron P_E(m, p)}^{E \rightarrow R}\\ % E -> R
    &+ \overbrace{\gm (m + 1) P_E(m+1 , p)}^{m+1 \rightarrow m} % m+1 -> m
    - \overbrace{\gm m P_E(m , p)}^{m \rightarrow m-1}\\ % m -> m-1
    &+ \overbrace{r_p m P_E(m, p - 1)}^{p-1 \rightarrow p} % p-1 -> p
    - \overbrace{r_p m P_E(m, p)}^{p \rightarrow p+1} % p -> p+1
    + \overbrace{\gp (p + 1) P_E(m, p + 1)}^{p + 1 \rightarrow p} % p+1 -> p
    - \overbrace{\gp p P_E(m, p)}^{p \rightarrow p-1}. % p -> p-1
  \end{aligned}
\end{equation}
And finally for the repressor bound state $R$ we have
\begin{equation}
  \begin{aligned}
    \dt{P_R(m, p)} &=
    - \overbrace{\kroff P_R(m, p)}^{R \rightarrow E} % R -> E
    + \overbrace{\kron P_E(m, p)}^{E \rightarrow R}\\ % E -> R
    &+ \overbrace{\gm (m + 1) P_R(m+1 , p)}^{m+1 \rightarrow m} % m+1 -> m
    - \overbrace{\gm m P_R(m , p)}^{m \rightarrow m-1}\\ % m -> m-1
    &+ \overbrace{r_p m P_R(m, p - 1)}^{p-1 \rightarrow p} % p-1 -> p
    - \overbrace{r_p m P_R(m, p)}^{p \rightarrow p+1} % p -> p+1
    + \overbrace{\gp (p + 1) P_R(m, p + 1)}^{p + 1 \rightarrow p} % p+1 -> p
    - \overbrace{\gp p P_R(m, p)}^{p \rightarrow p-1}. % p -> p-1
  \end{aligned}
\end{equation}

For an unregulated promoter, i.e. a promoter within a cell that has no
repressors present, and therefore constitutively expresses the gene, we use a
two-state model in which the state $R$ is not allowed. All the terms in the
system of ODEs containing $\kron$ or $\kroff$ are then set to zero.

As detailed in \secref{sec_model} it is convenient to express this system using
matrix notation \cite{Sanchez2013}. For this we define $\PP(m, p) = (P_B(m, p),
P_E(m, p), P_R(m, p))^T$. Then the system of ODEs can be expressed as
\begin{equation}
  \begin{aligned}
    \dt{\PP(m, p)} &= \Km \PP(m, p)
    - \Rm \PP(m, p) + \Rm \PP(m-1, p)
    - m \Gm \PP(m, p) + (m + 1) \Gm \PP(m + 1, p)\\
    &- m \Rp \PP(m, p) + m \Rp \PP(m, p - 1)
    - p \Gp \PP(m, p) + (p + 1) \Gp \PP(m, p + 1),
  \end{aligned}
\end{equation}
where we defined the following matrices: The promoter state transition matrix
$\Km$
\begin{align}
  \Km \equiv
  \begin{bmatrix}
    -\kpoff   & \kpon         & 0\\
    \kpoff    & -\kpon -\kron  & \kroff\\
    0         & \kron         & -\kroff
  \end{bmatrix},
\end{align}
The mRNA production $\Rm$ and degradation $\Gm$ matrices
\begin{equation}
  \Rm \equiv
  \begin{bmatrix}
    r_m   & 0 & 0\\
    0     & 0 & 0\\
    0     & 0 & 0\\
  \end{bmatrix},
\end{equation}
and
\begin{equation}
  \Gm \equiv
  \begin{bmatrix}
    \gm   & 0   & 0\\
    0     & \gm & 0\\
    0     & 0   & \gm\\
  \end{bmatrix}.
\end{equation}
For the protein we also define production $\Rp$ and degradation $\Gp$ matrices
as
\begin{equation}
  \Rp \equiv
  \begin{bmatrix}
    r_m   & 0   & 0\\
    0     & r_m & 0\\
    0     & 0   & r_m\\
  \end{bmatrix},
\end{equation}
and
\begin{equation}
  \Gp \equiv
  \begin{bmatrix}
    \gp   & 0   & 0\\
    0     & \gp & 0\\
    0     & 0   & \gp\\
  \end{bmatrix}.
\end{equation}

The corresponding equation for the unregulated two-state promoter takes the
exact same form with the definition of the matrices following the same scheme
without including the third row and third column, and setting $\kron$ and
$\kroff$ to zero.

\section{Parameter inference}\label{supp_param_inference}

With the objective of generating falsifiable predictions with meaningful
parameters we infer the kinetic rates for this three-state promoter model using
different data sets generated in our lab over the last decade concerning
different aspects of the regulation of the simple repression motif. For example,
for the RNAP rates $\kpon$ and $\kpoff$, and the mRNA production rate $r_m$ we
use single-molecule mRNA FISH counts from an unregulated promoter
\cite{Jones2014a}. Once these parameters are fixed, we use the values to
constrain the repressor rates $\kron$ and $\kroff$. These repressor rates are
obtained using information from mean gene expression measurements from bulk LacZ
colorimetric assays \cite{Garcia2011c}. We also expand our model to include the
allosteric nature of the repressor protein, taking advantage of video microscopy
measurements done in the context of multiple promoter copies \cite{Brewster2014}
and flow-cytometry measurements of the mean response of the system to different
levels of induction \cite{Razo-Mejia2018}. In what follows of this section we
detail the steps taken to infer the parameter values. At each step the values of
the parameters inferred in previous sections constrain the values of the
following parameters, building in this way a self-consistent theory informed by
work that spans several experimental techniques.

\subsection{RNAP rates from unregulated two-state promoter}

We begin our parameter inference problem with the RNAP rates $\kpon$ and
$\kpoff$, as well as the mRNA production rate $r_m$. In this case there are only
two states available to the promoter -- the empty state $E$ and the RNAP bound
state $P$. That means that the third ODE for $P_R(m, p)$ is removed from the
system. The mRNA steady state distribution for this particular two-state
promoter model was solved analytically by Peccoud and Ycart \cite{Peccoud1995}.
The mRNA distribution $P(m) \equiv P_E(m) + P_B(m)$ is of the form
\begin{equation}
  \small
  P(m \mid \kpon, \kpoff, r_m, \gm) =
  {\Gamma \left( \frac{\kpon}{\gm} + m \right) \over
  \Gamma(m + 1) \Gamma\left( \frac{\kpoff+\kpon}{\gm} + m \right)}
  {\Gamma\left( \frac{\kpoff+\kpon}{\gm} \right) \over
  \Gamma\left( \frac{\kpon}{\gm} \right)}
  \left( {r_m \over \gm} \right)^m
  F_1^1 \left( {\kpon \over \gm} + m,
  {\kpoff + \kpon \over \gm} + m,
  -{r_m \over \gm} \right),
  \label{seq_two_state_mRNA}
\end{equation}
where $\Gamma(\cdot)$ is the gamma function, and $F_1^1$ is the confluent
hypergeometric function of the first kind. This rather complicated expression
will aid us to find parameter values for the rates. The inferred rates $\kpon$,
$\kpoff$ and $r_m$ will be expressed in units of the mRNA degradation rate
$\gm$. This is because the model in \eref{seq_two_state_mRNA} is homogeneous in
time, meaning that if we divide all rates by a constant it would be equivalent
to multiplying the time scale of the problem by the same constant. As we will
discuss in the next section, \eref{seq_two_state_mRNA} has degeneracy in the
parameter values. What this means is that a change in one of the parameters,
specifically $r_m$, can be compensated by a change in another parameter,
specifically $\kpoff$ to obtain the exact same distribution. To work around this
intrinsic limitation of the model we will include in our inference prior
information from what we know from equilibrium-based models.

\subsubsection*{Bayesian parameter inference of RNAP rates}

In order to make progress at inferring the RNAP rates we make use of the
single-molecule mRNA FISH data from Jones et al. \cite{Jones2014a}.
\fref{sfig_lacUV5_FISH} shows the mRNA per cell distribution for the
\textit{lacUV5} promoter used throughout this work. This promoter being very
strong has a mean copy number of $\ee{m} \approx 18$ mRNA/cell.
\rp{maybe mention data we show in CBN to explain what's going on?}

\begin{figure}[h!]
	\centering \includegraphics
  {../fig/chemical_master_mRNA_FISH/lacUV5_smFISH_data.pdf}
	\caption{\textbf{\textit{lacUV5} mRNA per cell distribution.} Data from
	\cite{Jones2014a} of the unregulated \textit{lacUV5} promoter as inferred
	from single molecule mRNA FISH.}
  \label{sfig_lacUV5_FISH}
\end{figure}

Having this data in hand we now use Bayesian parameter inference. Writing Bayes
theorem we have
\begin{equation}
  P(\kpon, \kpoff, r_m \mid D) = {P(D \mid \kpon, \kpoff, r_m)
  P(\kpon, \kpoff, r_m) \over P(D)},
  \label{seq_bayes_rnap_rates}
\end{equation}
where $D$ represents the data. For this case the data consists of single-cell
mRNA counts $D = \{ m_1, m_2, \ldots, m_N \}$, where $N$ is the number of cells.
We assume that each cell's measurement is independent of each other such that we
can rewrite \eref{seq_bayes_rnap_rates} as
\begin{equation}
  P(\kpon, \kpoff, r_m \mid \{m_i\}) \propto
  \prod_{i=1}^N P(m_i \mid \kpon, \kpoff, r_m)
  P(\kpon, \kpoff, r_m),
  \label{seq_bayes_sample}
\end{equation}
where we ignore the normalization constant $P(D)$. The likelihood term $P(m_i
\mid \kpon, \kpoff, r_m)$ is exactly given by \eref{seq_two_state_mRNA} with
$\gm = 1$. Given that we have this functional form for the distribution we can
use Markov Chain Monte Carlo (MCMC) sampling to explore the 3D parameter space
in order to fit \eref{seq_two_state_mRNA} to the mRNA-FISH data.

\subsubsection*{Constraining the rates given prior thermodynamic knowledge.}

One of the strengths of the Bayesian approach is that we can include all the
prior knowledge on the parameters when performing an inference
\cite{MacKay2003}. Basic features such as the fact that the rates have to be
strictly positive must constrain the values that these parameters can take. For
the specific rates analyzed in this section we know more than the simple
constraint of non-negative values. The expression of an unregulated promoter has
been studied from a thermodynamic perspective \cite{Brewster2012}. Since these
equilibrium models work in the thermodynamic limit of large particle number they
are not useful to inform us about large deviations from the expected value
\mrm{I need to discuss this point in person with Rob}.
Nevertheless at the mean of the distribution both, the kinetic language and the
equilibrium language must agree. That means that we can use what we know about
the mean gene expression, and how this is related to parameters such as molecule
copy numbers and binding affinities, to constrain the values that the rates in
question can take.

In the case of this two-state promoter it can be shown that the mean number of
mRNA is given by \cite{Phillips2015} (See \siref{supp_moments} for moment
computation)
\begin{equation}
  \ee{m} = {r_m \over \gm} {\kpon \over \kpon + \kpoff}.
  \label{seq_mean_kinetic}
\end{equation}
Another way of expressing this is as ${r_m \over \gm} \times
p_{\text{bound}}^{(p)}$, where $p_{\text{bound}}^{(p)}$ is the probability of
the RNAP being bound at the promoter. The thermodynamic picture has an
equivalent result where the mean number of mRNA is given by \cite{Brewster2012,
Bintu2005a}
\begin{equation}
  \left\langle m \right\rangle = {r_m \over \gm}
  {{P \over \Nns} e^{-\beta\eP}
  \over 1 + {P \over \Nns} e^{-\beta\eP}},
  \label{seq_mean_thermo}
\end{equation}
where $P$ is the number of RNAP
per cell, $\Nns$ is the number of non-specific binding sites,
$\eP$ is the RNAP binding energy in $k_BT$ units and
$\beta\equiv {(k_BT)}^{-1}$.

Using \eref{seq_mean_kinetic} and \eref{seq_mean_thermo} we can easily see that
if these frameworks are to be equivalent, then it must be true that
\begin{equation}
  {\kpon \over \kpoff} = {P \over \Nns} e^{-\beta\eP},
\end{equation}
or equivalently
\begin{equation}
  \ln \left({\kpon \over \kpoff}\right) =
  -\beta\eP + \ln P - \ln \Nns.
\end{equation}
To put numerical values into these variables we can use information from the
literature. The RNAP copy number is order $P \approx 1000-3000$ RNAP/cell for a
1 hour doubling time \cite{Klumpp2008}.
\rp{Can Heinemann help us do better?}
\mrm{I need to talk with Nathan to see if I can get the data from him.}
As for the number of non-specific
binding sites and the binding energy we have that $\Nns = 4.6\times 10^6$
\cite{Bintu2005a}, and $-\beta\eP \approx 5 - 7$
\cite{Brewster2012}. Given these values we define a Gaussian prior for the ratio
of these two quantities of the form
\begin{equation}
  P\left({\kpon \over \kpoff} \right) \propto
  \exp \left\{ - {\left(\ln \left({\kpon \over \kpoff}\right) -
  \left(-\beta\eP + \ln P - \ln \Nns \right) \right)^2
  \over 2 \sigma^2} \right\},
  \label{seq_prior_single}
\end{equation}
where $\sigma$ is the variance that accounts for the uncertainty in these
parameters. We include this prior as part of the prior term $P(\kpon, \kpoff,
r_m)$ of \eref{seq_bayes_sample}. We then use MCMC to sample out of the
posterior distribution given by \eref{seq_bayes_sample}. \fref{sfig_mcmc_rnap}
shows the MCMC samples of the posterior distribution. For the case of the
$\kpon$ parameter there is a single symmetric peak. $\kpoff$ and $r_m$ have a
rather long tail towards large values. In fact, the 2D projection of $\kpoff$ vs
$r_m$ shows that the model is sloppy, meaning that the two parameters are highly
correlated. This feature is a common problem for many non-linear systems used in
biophysics and systems biology \cite{Transtrum2015}. What this implies is that
we can change the value of $\kpoff$, and then compensate by a change in $r_m$ in
order to maintain the shape of the mRNA distribution. Therefore it is impossible
from the data and the model themselves to narrow down a single value for the
parameters. Nevertheless since we included the prior information on the rates as
given by the analogous form between the equilibrium and non-equilibrium
expressions for the mean mRNA level, we obtained a more constrained parameter
value for the RNAP rates and the transcription rate that we will take as the
peak of this long-tailed distribution.

\begin{figure}[h!]
	\centering \includegraphics
  {../fig/chemical_master_mRNA_FISH/lacUV5_mRNA_prior_corner_plot.pdf}
	\caption{\textbf{MCMC posterior distribution.} Sampling out of
	\eref{seq_bayes_sample} the plot shows 2D and 1D projections of the 3D
	parameter space. The parameter values are (in units of the mRNA degradation
	rate $\gm$) $\kpon = 4.4^{+0.8}_{-0.3}$, $\kpoff = 20.4^{+52.1}_{-8.4}$ and
	$r_m = 106.1^{+184.8}_{-31.2}$ which are the modes of their respective
	distributions, where the superscripts and subscripts represent the upper and
	lower bounds of the 95$^\text{th}$ percentile of the parameter value
  distributions}
  \label{sfig_mcmc_rnap}
\end{figure}

The inferred values $\kpon = 4.4^{+0.8}_{-0.3}$, $\kpoff = 20.4^{+52.1}_{-8.4}$
and $r_m = 106.1^{+184.8}_{-31.2}$ are given in units of the mRNA degradation
rate $\gm$. Given the asymmetry of the parameter distributions we report the
upper and lower bound of the 95$^\text{th}$ percentile of the posterior
distributions. Assuming a mean life-time for mRNA of $\approx$ 3 min (from
this
\href{http://bionumbers.hms.harvard.edu/bionumber.aspx?&id=107514&ver=1&trm=mRNA%20mean%20lifetime}{link})
we have an mRNA degradation rate of $\gm \approx 2.84 \times 10^{-3} s^{-1}$.
Using this value gives the following values for the inferred rates: $\kpon =
0.012_{-0.001}^{+0.002} s^{-1}$, $\kpoff = {0.06}_ {-0.02}^{+0.15} s^{-1}$, and
$r_m = 0.3_{-0.09}^{+0.5} s^{-1}$.
\mrm{This is where a statement should be done
with respect to known values in the literature}
\rp{Absolutely}

\fref{sfig_lacUV5_theory_data} compares the experimental data from
\fref{sfig_lacUV5_FISH} with the resulting distribution obtained by substituting
the most likely parameter values into \eref{seq_two_state_mRNA}. As we can see
this two-state model fits the data adequately.

\begin{figure}[h!]
	\centering \includegraphics[width=0.5\columnwidth]
  {../fig/chemical_master_mRNA_FISH/lacUV5_two_state_mcmc_fit.pdf}
	\caption{\textbf{Experimental vs. theoretical distribution of mRNA per cell
  using parameters from Bayesian inference.} Dotted line shows the result of
  using \eref{seq_two_state_mRNA} along with the parameters inferred for the
  rates. Blue bars are the same data as \fref{fig_lacUV5_FISH} obtained from
  \cite{Jones2014a}}
  \label{sfig_lacUV5_theory_data}
\end{figure}

\subsubsection*{Accounting for variability in the number of promoters}

As discussed in ref. \cite{Jones2014a} and further expanded in
\cite{Peterson2015} an important source of cell-to-cell variability in gene
expression in bacteria is the fact that depending on the position relative to
the chromosome replication origin, and the growth rate, cells can have multiple
copies of any given gene. Genes closer to the replication origin have on average
higher gene copy number compared to genes at the opposite end. For the locus in
which our reporter construct is located (\textit{galK}) and the doubling time of
the mRNA FISH experiments we expect to have $\approx$ 1.66 copies of the gene
\cite{Jones2014a, Bremer1996}. This implies that the cells spend 2/3 of the cell
cycle with two copies of the promoter and the rest with a single copy.

To account for this variability in gene copy we extend the model assuming that
when cells have two copies of the promoter the mRNA production rate is $2 r_m$
compared to the rate $r_m$ for a single promoter copy. The probability of
observing certain mRNA copy $m$ is therefore given by
\begin{equation}
  P(m) = f \cdot P(m \mid \text{one promoter}) +
  (1 - f) \cdot P(m \mid \text{two promoters}),
  \label{seq_prob_multipromoter}
\end{equation}
where $f = 1/3$ is the fraction of the cell cycle that cells spend with a single
copy of the promoter. Both terms $P(m \mid \text{promoter copy})$ are given by
\eref{seq_two_state_mRNA} with the only difference being the rate $r_m$. It is
important to acknowledge that \eref{seq_prob_multipromoter} assumes that once
the gene is replicated the time scale in which the mRNA count relaxes to the new
steady state is much shorter than the time that the cells spend in this two
promoter copies state. This approximation should be valid for a short lived mRNA
molecule, but the assumption is not applicable for proteins whose degradation
rate is comparable to the cell cycle length as explored in
\secref{sec_cell_cycle}.

In order to repeat the Bayesian inference including this variability in gene
copy number we must split the mRNA count data in two sets -- cells with a single
copy of the promoter and cells with two copies of the promoter. For the single
molecule mRNA FISH data there is no labeling of the locus, making it impossible
to determine the number of copies of the promoter for any given cell. We
therefore follow Jones et al. \cite{Jones2014a} in using the cell area as a
proxy for stage in the cell cycle. In their approach they sorted cells by area,
considering the low 33$\th$ percentile as cells with a single promoter copy,
with the rest being cells with two copies of the promoter. This approach ignores
that cells are not uniformly distributed along the cell cycle. As first
discussed in \cite{Powell1956} populations of cells in a log-phase are
exponentially distributed along the cell cycle. This distribution is of the form
\begin{equation}
P(a) = (\ln 2) \cdot 2^{1 - a},
\label{seq_cell_cycle_dist}
\end{equation}
where $a \in [0, 1]$ is the stage of the cell cycle, with $a = 0$ being the
start of the cycle and $a = 1$ being the cell division. \fref{sfig_cell_area}
shows the separation of the two groups based on area where
\eref{seq_cell_cycle_dist} was used to weigh the distribution along the cell
cycle.

\begin{figure}[h!]
	\centering \includegraphics
  {../fig/chemical_master_mRNA_FISH/area_division_expo.pdf}
	\caption{\textbf{Separation of cells based on cell size.} Using the area as
  a proxy for state on the cell cycle, cells can be sorted into two groups --
  small cells (with one promoter copy) and large cells (with two promoter
  copies). The vertical black line delimits the threshold that divides both
  groups as weighted by \eref{seq_cell_cycle_dist}.}
  \label{sfig_cell_area}
\end{figure}

\fref{sfig_mRNA_by_size} shows the distribution of both groups. As expected
larger cells have a higher mRNA copy number on average.

\begin{figure}[h!]
	\centering \includegraphics
  {../fig/chemical_master_mRNA_FISH/lacUV5_mRNA_size_PMF_CDF_expo.pdf}
	\caption{\textbf{mRNA distribution for small and large cells.} (A) histogram
	and (B) cumulative distribution function of the small and large cells as
	determined in \fref{sfig_cell_area}. The triangles above histograms in (A)
	indicate the mean mRNA copy number for each group.}
  \label{sfig_mRNA_by_size}
\end{figure}

We modify \eref{seq_bayes_sample} to account for the two separate groups of
cells. Let $N_s$ be the number of cells in the small size group and $N_l$ the
number of cells in the large size group. Then the posterior distribution for the
parameters is of the form
\begin{equation}
  \small
P(\kpon, \kpoff, r_m \mid \{m_i\}) \propto
  \prod_{i=1}^{N_s} \cdot P(m_i \mid \kpon, \kpoff, r_m)
  \prod_{j=1}^{N_l} \cdot P(m_j \mid \kpon, \kpoff, 2 r_m)
  P(\kpon, \kpoff, r_m),
  \label{seq_bayes_sample_double}
\end{equation}
where we split the product of small and large cells.

For the two-promoter model the prior shown in \eref{seq_prior_single} requires
a small modification. The two-promoter model has a mean mRNA copy number of
the form
\begin{equation}
  \ee{m} = f \ee{m}_1 + (1 - f) \ee{m}_2,
  \label{seq_mean_m_double}
\end{equation}
where $\ee{m}_1$ and $\ee{m}_2$ are the mean number of mRNA for a single
promoter and two promoters respectively. Given that we assume that the only
difference between these two cases is the change in transcription rate from
$r_m$ in the single promoter case to $2 r_m$ in the two-promoter case we can
write \eref{seq_mean_m_double} as
\begin{equation}
  \ee{m} = f \cdot {r_m \over \gm} {\kpon \over \kpon + \kpoff} +
      (1 -f) \cdot {2 r_m \over \gm} {\kpon \over \kpon + \kpoff}.
\end{equation}
This can be simplified to
\begin{equation}
  \ee{m} = (2 - f) {r_m \over \gm} {\kpon \over \kpon + \kpoff}.
  \label{seq_mean_m_double_rates}
\end{equation}

Equating \eref{seq_mean_m_double_rates} to \eref{seq_mean_thermo} to again make
both predictions from the equilibrium and kinetic model be self-consistent at
the mean gives
\begin{equation}
  (2 - f) {\kpon \over \kpon + \kpoff} =
  {{P \over \Nns} e^{-\beta\eP}
  \over 1 + {P \over \Nns} e^{-\beta\eP}}.
\end{equation}
Solving for $\kpon \over \kpoff$ results in
\begin{equation}
  \left[ (1 + \rho)(2 - f) - \rho \right] \left({\kpon \over \kpoff}\right) =
  \rho,
  \label{seq_kinetic_thermo_equiv}
\end{equation}
where we define $\rho \equiv {P \over \Nns} e^{-\beta\eP}$. To simplify things
further we notice that for the specified values of $P = 1000 - 3000$ per cell,
$\Nns = 4.6 \times 10^6$ bp, and $-\beta\eP = 5 - 7$, we can safely assume that
$\rho \ll 1$. This simplifying assumption has been previously called the weak
promoter approximation \cite{Garcia2011c}. Given this we can simplify
\eref{seq_kinetic_thermo_equiv} as
\begin{equation}
  {\kpon \over \kpoff} = {1 \over 2 - f} {P \over \Nns} e^{-\beta\eP}.
\end{equation}
Taking the log from both sides gives
\begin{equation}
  \ln\left({\kpon \over \kpoff}\right) = -\ln (2 - f) + \ln P - \ln\Nns
  - \beta\eP.
\end{equation}
With this we can set as before a Gaussian prior to constain the ratio of the
RNAP rates as
\begin{equation}
  P\left({\kpon \over \kpoff} \right) \propto
  \exp \left\{ - {\left(\ln \left({\kpon \over \kpoff}\right) -
  \left[-\ln(2 - f) -\beta\eP + \ln P - \ln \Nns \right) \right]^2
  \over 2 \sigma^2} \right\},
  \label{seq_prior_double}
\end{equation}

\fref{sfig_mcmc_rnap_double} shows the result of sampling out of
\eref{seq_bayes_sample_double}. Again we see that the model is highly sloppy
with large credible regions obtained for $\kpoff$ and $r_m$. Nevertheless again
the use of the prior information allow us to get a parameter value consistent
with the equilibrium picture.

\begin{figure}[h!]
	\centering \includegraphics
  {../fig/chemical_master_mRNA_FISH/lacUV5_mRNA_double_expo_corner_plot.pdf}
	\caption{\textbf{MCMC posterior distribution for a multi-promoter model.}
	Sampling out of \eref{seq_bayes_sample_double} the plot shows 2D and 1D
	projections of the 3D parameter space. The parameter values are (in units of
	the mRNA degradation rate $\gm$) $\kpon = 6^{+0.8}_{-0.3}$, $\kpoff =
	85.7^{+437}_{-47.6}$ and $r_m = 154.8^{+684}_{-72.7}$ which are the modes of
	their respective distributions, where the superscripts and subscripts
	represent the upper and lower bounds of the 95$\th$ percentile of the
	parameter value distributions. The sampling was bounded to values < 1000 for
  numerical stability when computing the confluent hypergeometric function.}
  \label{sfig_mcmc_rnap_double}
\end{figure}

Using again the mRNA mean lifetime of gives the following values for the
parameters: $\kpon = 0.035_{-0.002}^{+0.004} s^{-1}$, $\kpoff = {0.47}_
{-0.26}^{+2.43} s^{-1}$, and $r_m = 0.860_{-0.4}^{+3.8} s^{-1}$. \mrm{again need
to compare with what is known about these rates.}.
\fref{sfig_lacUV5_theory_data_double} shows the result of applying
\eref{seq_prob_multipromoter} using these parameter values. Specifically
\fref{sfig_lacUV5_theory_data_double}(A) shows the global distribution including
cells with one and two promoters and \fref{sfig_lacUV5_theory_data_double}(B)
splits the distributions within the two populations. Given that the model
adequately describes both populations independently and pooled together we
confirm that using the cell area as a proxy for stage in the cell cycle and the
doubling of the transcription rate once cells have two promoters are reasonable
approximations.

\begin{figure}[h!]
	\centering \includegraphics
  {../fig/chemical_master_mRNA_FISH/lacUV5_multi_prom_fit.pdf}
  \caption{\textbf{Experimental vs. theoretical distribution of mRNA per cell
  using parameters for multi-promoter model.} (A) Solid curve shows the result
  of using \eref{seq_prob_multipromoter} with the parameters inferred by
  sampling \eref{seq_bayes_sample_double}. Blue bars are the same data as
  \fref{sfig_lacUV5_FISH} from \cite{Jones2014a}. (B) Split distributions of
  small cells (light blue bars) and large cells (dark blue) with the
  corresponding theoretical predictions with transcription rate $r_m$ (light
  blue line) and transcription rate $2 r_m$ (dark blue line)}
	\label{sfig_lacUV5_theory_data_double}
\end{figure}

\section{Repressor rates from three-state regulated promoter.}

Having determined the RNAP rates we now proceed to determine the repressor rates
$\kron$ and $\kroff$. The values of these rates are constrained again by the
correspondence between our kinetic picture and what we know from equilibrium
models \cite{Phillips2015}. For this we again exploit the feature that only at
the mean both, the kinetic language and the thermodynamic language should have
equivalent predictions. Over the last decade there has been a lot of effort in
developing equilibrium models for gene expression regulation \cite{Buchler2003,
Vilar2011, Bintu2005a}. In particular our group has extensively characterized
the simple repression motif using this formalism \cite{Garcia2011c,
Brewster2014, Razo-Mejia2018}.

The dialogue between theory and experiments has led to simplified expressions
that capture the phenomenology of the gene expression response as a function of
natural variables such as molecule count and affinities between molecular
players. A particularly interesting quantity for the simple repression motif
used by Garcia \& Phillips \cite{Garcia2011c} as the fold-change in gene
expression is given by
\begin{equation}
  \foldchange = {\ee{\text{gene expression}(R > 0)} \over
                 \ee{\text{gene expression}(R = 0)}},
\end{equation}
where $R$ is the number of repressors per cell. The fold-change is simply the
mean expression level in the presence of the repressor relative to the mean
expression level in the absence of regulation. In the language of statistical
mechanics this quantity takes the form \cite{Garcia2011c}
\begin{equation}
  \foldchange = \left( 1 + {R \over \Nns} e^{-\beta\eR} \right)^{-1},
  \label{seq_fc_thermo}
\end{equation}
where $\eR$ is the repressor-DNA binding energy.

To compute the fold-change in the chemical master equation language we compute
the first moment of the steady sate mRNA distribution $\ee{m}$ for both, the
three-state promoter ($R>0$) and the two-state promoter case ($R=0$) (See
\siref{supp_moments} for moment derivation). The unregulated (two-state)
promoter mean mRNA copy number is given by \eref{seq_mean_m_double_rates}. For
the regulated (three-state) promoter we have an equivalent expression of the
form
\begin{equation}
  \ee{m (R > 0)} = (2 - f){r_m \over \gm} {\kroff\kron
  \over \kpoff\kroff + \kpoff\kron + \kroff\kpon}.
\end{equation}
Computing the fold-change then gives
\begin{equation}
  \foldchange = {\ee{m (R > 0)} \over \ee{m (R = 0)}} =
  {\kroff \left( \kpoff + \kpon \right) \over
  \kpoff\kron + \kroff \left( \kpoff + \kpon \right)},
  \label{seq_fold_change_cme}
\end{equation}
where the factor $(2 -f)$ due to the multiple promoter copies cancels out.

Given that the number of repressors per cell $R$ is an experimental variable
that we can control, we assume that the rate at which the promoter transitions
form the empty state to the repressor bound state $\kron$ is given by the
concentration of repressors $[R]$ times a diffusion limited rate on $k_o$
\cite{Jones2014a}.  For the diffusion limited constant $k_o$ we use the value
used by Jones et al. \cite{Jones2014a} \mrm{Find real reference for this value
that Brewster never gave me.}. With this in hand we can rewrite
\eref{seq_fold_change_cme} as
\begin{equation}
  \foldchange = \left( 1 + {k_0 [R] \over \kroff}
                {\kpon \over \kpon + \kpoff} \right)^{-1}.
  \label{seq_fc_kinetic}
\end{equation}

We note that both \eref{seq_fc_thermo} and \eref{seq_fc_kinetic} have the same
functional form. Therefore if both languages predict the same output for the
mean gene expression level, it must be true that
\begin{equation}
  {k_o [R] \over \kroff}{\kpon \over \kpon + \kpoff} =
  {R \over \Nns} e^{-\beta\eR}.
\end{equation}
Solving for $\kroff$ gives
\begin{equation}
  \kroff = {k_o [R] \Nns e^{\beta\eR} \over R}{\kpon \over \kpon + \kpoff}.
  \label{seq_kroff_complete}
\end{equation}

Since the reported value of $k_o$ is given in units of nM$^{-1}$s$^{-1}$ in
order for the units to cancel properly the repressor concentration has to be
given in nM rather than absolute count. If we consider that the repressor
concentration is equal to
\begin{equation}
[R] = \frac{R}{V_{cell}}\cdot \frac{1}{N_A},
\end{equation}
where $R$ is the absolute repressor copy number per cell, $V_{cell}$ is the cell
volume and $N_A$ is Avogadro's number. The \textit{E. coli} cell volume is in
the order of 2.1 fL = $10^{-15}$ L \mrm{get reference from Nathan}, and
Avogadro's number is $6.022 \times 10^{23}$. If we further include the
conversion factor to turn M into nM we find that
\rp{Look at Sukjon Jun's data in CBN. Maybe do better on cell volume.}
\mrm{I need to double check with Nathan, but I think that this volume comes
from that data set.}
\begin{equation}
[R] = {R \over 2.1 \times 10^{-15} L} \cdot {1 \over 6.022 \times 10^{23}}
\cdot {10^9 \text{ nmol} \over 1 \text{ mol}} \approx 1.66 \times R.
\end{equation}
Using this we simplify \eref{seq_kroff_complete} as
\begin{equation}
  \kroff \approx 1.66 \cdot k_o \cdot \Nns e^{\beta\eR}
   \cdot {\kpon \over \kpon + \kpoff}.
  \label{seq_kroff}
\end{equation}
What \eref{seq_kroff} shows is the direct relationship that must be satisfied if
the equilibrium model is set to be consistent with the non-equilibrium kinetic
picture.
