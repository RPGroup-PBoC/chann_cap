\section{Three-state promoter model for simple repression}\label{supp_model}

In order to tackle the question of how much information the simple repression
motif can process we require the joint probability distribution of mRNA and
protein $P(m, p; t)$. To obtain this distribution we use the chemical master
equation formalism as described in \secref{sec_model}. Specifically, we assume
a three-state model where the promoter can be found 1) in a transcriptionally
active state  ($A$ state), 2) in a transcriptionally inactive state without the
repressor bound ($I$ state) and 3) with the repressor bound ($R$ state). (See
\fref{fig2_minimal_model}(A)). These three states generate a system of coupled
differential equations for each of the three state distributions $P_A(m, p)$,
$P_I(m, p)$ and $P_R(m, p)$. Given the rates shown in
\fref{fig2_minimal_model}(A) let us define the system of ODEs. For the
transcriptionally active state we have
\begin{equation}
  \begin{aligned}
    \dt{P_A(m, p)} &=
    - \overbrace{\kpoff P_A(m, p)}^{A \rightarrow I} % A -> I
    + \overbrace{\kpon P_I(m, p)}^{I \rightarrow A}\\ % I -> A
    &+ \overbrace{r_m P_A(m-1, p)}^{m-1 \rightarrow m} % m-1 -> m
    - \overbrace{r_m P_A(m, p)}^{m \rightarrow m+1}% m -> m+1
    + \overbrace{\gm (m + 1) P_A(m+1 , p)}^{m+1 \rightarrow m} % m+1 -> m
    - \overbrace{\gm m P_A(m , p)}^{m \rightarrow m-1}\\ % m -> m-1
    &+ \overbrace{r_p m P_A(m, p - 1)}^{p-1 \rightarrow p} % p-1 -> p
    - \overbrace{r_p m P_A(m, p)}^{p \rightarrow p+1} % p -> p+1
    + \overbrace{\gp (p + 1) P_A(m, p + 1)}^{p + 1 \rightarrow p} % p+1 -> p
    - \overbrace{\gp p P_A(m, p)}^{p \rightarrow p-1}. % p -> p-1
  \end{aligned}
\end{equation}
For the inactive promoter state $I$ we have
\begin{equation}
  \begin{aligned}
    \dt{P_I(m, p)} &=
    \overbrace{\kpoff P_A(m, p)}^{A \rightarrow I} % A -> I
    - \overbrace{\kpon P_I(m, p)}^{I \rightarrow A} % I -> A
    + \overbrace{\kroff P_R(m, p)}^{R \rightarrow I} % R -> I
    - \overbrace{\kron P_I(m, p)}^{I \rightarrow R}\\ % I -> R
    &+ \overbrace{\gm (m + 1) P_I(m+1 , p)}^{m+1 \rightarrow m} % m+1 -> m
    - \overbrace{\gm m P_I(m , p)}^{m \rightarrow m-1}\\ % m -> m-1
    &+ \overbrace{r_p m P_I(m, p - 1)}^{p-1 \rightarrow p} % p-1 -> p
    - \overbrace{r_p m P_I(m, p)}^{p \rightarrow p+1} % p -> p+1
    + \overbrace{\gp (p + 1) P_I(m, p + 1)}^{p + 1 \rightarrow p} % p+1 -> p
    - \overbrace{\gp p P_I(m, p)}^{p \rightarrow p-1}. % p -> p-1
  \end{aligned}
\end{equation}
And finally for the repressor bound state $R$ we have
\begin{equation}
  \begin{aligned}
    \dt{P_R(m, p)} &=
    - \overbrace{\kroff P_R(m, p)}^{R \rightarrow I} % R -> I
    + \overbrace{\kron P_I(m, p)}^{I \rightarrow R}\\ % I -> R
    &+ \overbrace{\gm (m + 1) P_R(m+1 , p)}^{m+1 \rightarrow m} % m+1 -> m
    - \overbrace{\gm m P_R(m , p)}^{m \rightarrow m-1}\\ % m -> m-1
    &+ \overbrace{r_p m P_R(m, p - 1)}^{p-1 \rightarrow p} % p-1 -> p
    - \overbrace{r_p m P_R(m, p)}^{p \rightarrow p+1} % p -> p+1
    + \overbrace{\gp (p + 1) P_R(m, p + 1)}^{p + 1 \rightarrow p} % p+1 -> p
    - \overbrace{\gp p P_R(m, p)}^{p \rightarrow p-1}. % p -> p-1
  \end{aligned}
\end{equation}

For an unregulated promoter, i.e. a promoter in a cell that has no repressors
present, and therefore constitutively expresses the gene, we use a two-state
model in which the state $R$ is not allowed. All the terms in the system of ODEs
containing $\kron$ or $\kroff$ are then set to zero.

As detailed in \secref{sec_model} it is convenient to express this system using
matrix notation \cite{Sanchez2013}. For this we define $\PP(m, p) = (P_A(m, p),
P_I(m, p), P_R(m, p))^T$. Then the system of ODEs can be expressed as
\begin{equation}
  \begin{aligned}
    \dt{\PP(m, p)} &= \Km \PP(m, p)
    - \Rm \PP(m, p) + \Rm \PP(m-1, p)
    - m \Gm \PP(m, p) + (m + 1) \Gm \PP(m + 1, p)\\
    &- m \Rp \PP(m, p) + m \Rp \PP(m, p - 1)
    - p \Gp \PP(m, p) + (p + 1) \Gp \PP(m, p + 1),
  \end{aligned}
\end{equation}
where we defined matrices representing the promoter state transition $\Km$,
\begin{align}
  \Km \equiv
  \begin{bmatrix}
    -\kpoff   & \kpon         & 0\\
    \kpoff    & -\kpon -\kron  & \kroff\\
    0         & \kron         & -\kroff
  \end{bmatrix},
\end{align}
mRNA production, $\Rm$, and degradation, $\Gm$, as
\begin{equation}
  \Rm \equiv
  \begin{bmatrix}
    r_m   & 0 & 0\\
    0     & 0 & 0\\
    0     & 0 & 0\\
  \end{bmatrix},
\end{equation}
and
\begin{equation}
  \Gm \equiv
  \begin{bmatrix}
    \gm   & 0   & 0\\
    0     & \gm & 0\\
    0     & 0   & \gm\\
  \end{bmatrix}.
\end{equation}
For the protein we also define production $\Rp$ and degradation $\Gp$ matrices
as
\begin{equation}
  \Rp \equiv
  \begin{bmatrix}
    r_p   & 0   & 0\\
    0     & r_p & 0\\
    0     & 0   & r_p\\
  \end{bmatrix}
\end{equation}
and
\begin{equation}
  \Gp \equiv
  \begin{bmatrix}
    \gp   & 0   & 0\\
    0     & \gp & 0\\
    0     & 0   & \gp\\
  \end{bmatrix}.
\end{equation}

The corresponding equation for the unregulated two-state promoter takes the
exact same form with the definition of the matrices following the same scheme
without including the third row and third column, and setting $\kron$ and
$\kroff$ to zero.

A closed-form solution for this master equation might not even exist. The
approximate solution of chemical master equations of this kind is an active
area of research. As we will see in \siref{supp_param_inference} the two-state
promoter master equation has been analytically solved for the mRNA
\cite{Peccoud1995} and protein distributions \cite{Shahrezaei2008}. For our
purposes, in \siref{supp_maxent} we will detail how to use the Maximum Entropy
principle to approximate the full distribution for the two- and three-state
promoter.

\section{Parameter inference}\label{supp_param_inference}

(Note: The Python code used for the calculations presented in this section can
be found in the
\href{https://www.rpgroup.caltech.edu//chann_cap/software/chemical_master_mRNA_FISH_mcmc.html}{following
link} as an anotated Jupter notebook)

With the objective of generating falsifiable predictions with meaningful
parameters, we infer the kinetic rates for this three-state promoter model
using different data sets generated in our lab over the last decade concerning
different aspects of the regulation of the simple repression motif. For
example, for the unregulated promoter transition rates $\kpon$ and $\kpoff$ and
the mRNA production rate $r_m$, we use single-molecule mRNA FISH counts from an
unregulated promoter \cite{Jones2014a}. Once these parameters are fixed, we use
the values to constrain the repressor rates $\kron$ and $\kroff$. These
repressor rates are obtained using information from mean gene expression
measurements from bulk LacZ colorimetric assays \cite{Garcia2011c}. We also
expand our model to include the allosteric nature of the repressor protein,
taking advantage of video microscopy measurements done in the context of
multiple promoter copies \cite{Brewster2014} and flow-cytometry measurements of
the mean response of the system to different levels of induction
\cite{Razo-Mejia2018}. In what follows of this section we detail the steps
taken to infer the parameter values. At each step the values of the parameters
inferred in previous steps constrain the values of the parameters that are not
yet determined, building in this way a self-consistent model informed by work
that spans several experimental techniques.

\subsection{Unregulated promoter rates}

We begin our parameter inference problem with the promoter on and off rates
$\kpon$ and $\kpoff$, as well as the mRNA production rate $r_m$. In this case
there are only two states available to the promoter -- the inactive state $I$
and the transcriptionally active  state $A$. That means that the third ODE for
$P_R(m, p)$ is removed from the system. The mRNA steady state distribution for
this particular two-state promoter model was solved analytically by Peccoud and
Ycart \cite{Peccoud1995}. This distribution $P(m) \equiv P_I(m) + P_A(m)$ is of
the form
\begin{equation}
  \small
  P(m \mid \kpon, \kpoff, r_m, \gm) =
  {\Gamma \left( \frac{\kpon}{\gm} + m \right) \over
  \Gamma(m + 1) \Gamma\left( \frac{\kpoff+\kpon}{\gm} + m \right)}
  {\Gamma\left( \frac{\kpoff+\kpon}{\gm} \right) \over
  \Gamma\left( \frac{\kpon}{\gm} \right)}
  \left( {r_m \over \gm} \right)^m
  F_1^1 \left( {\kpon \over \gm} + m,
  {\kpoff + \kpon \over \gm} + m,
  -{r_m \over \gm} \right),
  \label{seq_two_state_mRNA}
\end{equation}
where $\Gamma(\cdot)$ is the gamma function, and $F_1^1$ is the confluent
hypergeometric function of the first kind. This rather complicated expression
will aid us to find parameter values for the rates. The inferred rates $\kpon$,
$\kpoff$ and $r_m$ will be expressed in units of the mRNA degradation rate
$\gm$. This is because the model in \eref{seq_two_state_mRNA} is homogeneous in
time, meaning that if we divide all rates by a constant it would be equivalent
to multiplying the characteristic time scale by the same constant. As we will
discuss in the next section, \eref{seq_two_state_mRNA} has degeneracy in the
parameter values. What this means is that a change in one of the parameters,
specifically $r_m$, can be compensated by a change in another parameter,
specifically $\kpoff$, to obtain the exact same distribution. To work around
this intrinsic limitation of the model we will include in our inference prior
information from what we know from equilibrium-based models.

\subsubsection*{Bayesian parameter inference of RNAP rates}

In order to make progress at inferring the unregulated promoter state
transition rates, we make use of the single-molecule mRNA FISH data from Jones
et al. \cite{Jones2014a}. \fref{sfig_lacUV5_FISH} shows the distribution of mRNA per cell
for the \textit{lacUV5} promoter used for our inference. This
promoter, being very strong, has a mean copy number of $\ee{m} \approx 18$
mRNA/cell.

\begin{figure}[h!]
	\centering \includegraphics
  {../fig/si/figS01.pdf}
	\caption{\textbf{\textit{lacUV5} mRNA per cell distribution.} Data from
	\cite{Jones2014a} of the unregulated \textit{lacUV5} promoter as inferred
	from single molecule mRNA FISH.}
  \label{sfig_lacUV5_FISH}
\end{figure}

Having this data in hand we now turn to Bayesian parameter inference. Writing
Bayes theorem we have
\begin{equation}
  P(\kpon, \kpoff, r_m \mid D) = {P(D \mid \kpon, \kpoff, r_m)
  P(\kpon, \kpoff, r_m) \over P(D)},
  \label{seq_bayes_rnap_rates}
\end{equation}
where $D$ represents the data. For this case the data consists of single-cell
mRNA counts $D = \{ m_1, m_2, \ldots, m_N \}$, where $N$ is the number of
cells. We assume that each cell's measurement is independent of the others such
that we can rewrite \eref{seq_bayes_rnap_rates} as
\begin{equation}
  P(\kpon, \kpoff, r_m \mid \{m_i\}) \propto
  \left[\prod_{i=1}^N P(m_i \mid \kpon, \kpoff, r_m) \right]
  P(\kpon, \kpoff, r_m),
  \label{seq_bayes_sample}
\end{equation}
where we ignore the normalization constant $P(D)$. The likelihood term $P(m_i
\mid \kpon, \kpoff, r_m)$ is exactly given by \eref{seq_two_state_mRNA} with
$\gm = 1$. Given that we have this functional form for the distribution, we can
use Markov Chain Monte Carlo (MCMC) sampling to explore the 3D parameter space
in order to fit \eref{seq_two_state_mRNA} to the mRNA-FISH data.

\subsubsection*{Constraining the rates given prior thermodynamic knowledge.}

One of the strengths of the Bayesian approach is that we can include all the
prior knowledge on the parameters when performing an inference
\cite{MacKay2003}. Basic features such as the fact that the rates have to be
strictly positive constrain the values that these parameters can take. For
the specific rates analyzed in this section we know more than the simple
constraint of non-negative values. The expression of an unregulated promoter has
been studied from a thermodynamic perspective \cite{Brewster2012}. Given the
underlying assumptions of these equilibrium models, in which the probability of
finding the RNAP bound to the promoter is proportional to the transcription
rate \cite{Bintu2005a}, they can only make statements about the mean expression
level. Nevertheless if both the thermodynamic and the kinetic model describe
the same process, the predictions for the mean gene expression level must
agree. That means that we can use what we know about the mean gene expression,
and how this is related to parameters such as molecule copy numbers and binding
affinities, to constrain the values that the rates in question can take.

In the case of this two-state promoter it can be shown that the mean number of
mRNA is given by \cite{Sanchez2013} (See \siref{supp_moments} for moment
computation)
\begin{equation}
  \ee{m} = {r_m \over \gm} {\kpon \over \kpon + \kpoff}.
  \label{seq_mean_kinetic}
\end{equation}
Another way of expressing this is as ${r_m \over \gm} \times
p_{\text{active}}^{(p)}$, where $p_{\text{active}}^{(p)}$ is the probability of
the promoter being in the transcriptionally active state. The thermodynamic
picture has an equivalent result where the mean number of mRNA is given by
\cite{Brewster2012, Bintu2005a}
\begin{equation}
  \left\langle m \right\rangle = {r_m \over \gm}
  {{P \over \Nns} e^{-\beta\eP}
  \over 1 + {P \over \Nns} e^{-\beta\eP}},
  \label{seq_mean_thermo}
\end{equation}
where $P$ is the number of RNAP per cell, $\Nns$ is the number of non-specific
binding sites, $\eP$ is the RNAP binding energy in $k_BT$ units and $\beta\equiv
{(k_BT)}^{-1}$. Using \eref{seq_mean_kinetic} and \eref{seq_mean_thermo} we can
easily see that if these frameworks are to be equivalent, then it must be true
that
\begin{equation}
  {\kpon \over \kpoff} = {P \over \Nns} e^{-\beta\eP},
\end{equation}
or equivalently
\begin{equation}
  \ln \left({\kpon \over \kpoff}\right) =
  -\beta\eP + \ln P - \ln \Nns.
\end{equation}
To put numerical values into these variables we can use information from the
literature. The RNAP copy number is order $P \approx 1000-3000$ RNAP/cell for a
1 hour doubling time \cite{Klumpp2008}. As for the number of non-specific
binding sites and the binding energy, we have that $\Nns = 4.6\times 10^6$
\cite{Bintu2005a} and $-\beta\eP \approx 5 - 7$ \cite{Brewster2012}. Given
these values we define a Gaussian prior for the log ratio of these two
quantities of the form
\begin{equation}
  P\left(\ln \left({\kpon \over \kpoff}\right) \right) \propto
  \exp \left\{ - {\left(\ln \left({\kpon \over \kpoff}\right) -
  \left(-\beta\eP + \ln P - \ln \Nns \right) \right)^2
  \over 2 \sigma^2} \right\},
  \label{seq_prior_single}
\end{equation}
where $\sigma$ is the variance that accounts for the uncertainty in these
parameters. We include this prior as part of the prior term $P(\kpon, \kpoff,
r_m)$ of \eref{seq_bayes_sample}. We then use MCMC to sample out of the
posterior distribution given by \eref{seq_bayes_sample}. \fref{sfig_mcmc_rnap}
shows the MCMC samples of the posterior distribution. For the case of the
$\kpon$ parameter there is a single symmetric peak. $\kpoff$ and $r_m$ have a
rather long tail towards large values. In fact, the 2D projection of $\kpoff$ vs
$r_m$ shows that the model is sloppy, meaning that the two parameters are highly
correlated. This feature is a common problem for many non-linear systems used in
biophysics and systems biology \cite{Transtrum2015}. What this implies is that
we can change the value of $\kpoff$, and then compensate by a change in $r_m$ in
order to maintain the shape of the mRNA distribution. Therefore it is impossible
from the data and the model themselves to narrow down a single value for the
parameters. Nevertheless since we included the prior information on the rates as
given by the analogous form between the equilibrium and non-equilibrium
expressions for the mean mRNA level, we obtained a more constrained parameter
value for the RNAP rates and the transcription rate that we will take as the
peak of this long-tailed distribution.

\begin{figure}[h!]
	\centering \includegraphics
  {../fig/si/figS02.pdf}
	\caption{\textbf{MCMC posterior distribution.} Sampling out of
	\eref{seq_bayes_sample} the plot shows 2D and 1D projections of the 3D
	parameter space. The parameter values are (in units of the mRNA degradation
	rate $\gm$) $\kpon = 4.3^{+1}_{-0.3}$, $\kpoff = 18.8^{+120}_{-10}$ and $r_m =
	103.8^{+423}_{-37}$ which are the modes of their respective distributions,
	where the superscripts and subscripts represent the upper and lower bounds of
	the 95$^\text{th}$ percentile of the parameter value distributions}
  \label{sfig_mcmc_rnap}
\end{figure}

The inferred values $\kpon = 4.3^{+1}_{-0.3}$, $\kpoff = 18.8^{+120}_{-10}$
and $r_m = 103.8^{+423}_{-37}$ are given in units of the mRNA degradation
rate $\gm$. Given the asymmetry of the parameter distributions we report the
upper and lower bound of the 95$^\text{th}$ percentile of the posterior
distributions. Assuming a mean life-time for mRNA of $\approx$ 3 min (from
this
\href{http://bionumbers.hms.harvard.edu/bionumber.aspx?&id=107514&ver=1&trm=mRNA%20mean%20lifetime}{link})
we have an mRNA degradation rate of $\gm \approx 2.84 \times 10^{-3} s^{-1}$.
Using this value gives the following values for the inferred rates: $\kpon =
0.024_{-0.002}^{+0.005} s^{-1}$, $\kpoff = {0.11}_ {-0.05}^{+0.66} s^{-1}$, and
$r_m = 0.3_{-0.2}^{+2.3} s^{-1}$.

\fref{sfig_lacUV5_theory_data} compares the experimental data from
\fref{sfig_lacUV5_FISH} with the resulting distribution obtained by substituting
the most likely parameter values into \eref{seq_two_state_mRNA}. As we can see
this two-state model fits the data adequately.

\begin{figure}[h!]
	\centering \includegraphics[width=0.5\columnwidth]
  {../fig/si/figS03.pdf}
	\caption{\textbf{Experimental vs. theoretical distribution of mRNA per cell
  using parameters from Bayesian inference.} Dotted line shows the result of
  using \eref{seq_two_state_mRNA} along with the parameters inferred for the
  rates. Blue bars are the same data as \fref{sfig_lacUV5_FISH} obtained from
  \cite{Jones2014a}.}
  \label{sfig_lacUV5_theory_data}
\end{figure}

\subsection{Accounting for variability in the number of promoters}

As discussed in ref. \cite{Jones2014a} and further expanded in
\cite{Peterson2015} an important source of cell-to-cell variability in gene
expression in bacteria is the fact that, depending on the growth rate and the
position relative to the chromosome replication origin, cells can have multiple
copies of any given gene. Genes closer to the replication origin have on
average higher gene copy number compared to genes at the opposite end. For the
locus in which our reporter construct is located (\textit{galK}) and the
doubling time of the mRNA FISH experiments we expect to have $\approx$ 1.66
copies of the gene \cite{Jones2014a, Bremer1996}. This implies that the cells
spend 2/3 of the cell cycle with two copies of the promoter and the rest with a
single copy.

To account for this variability in gene copy we extend the model assuming that
when cells have two copies of the promoter the mRNA production rate is $2 r_m$
compared to the rate $r_m$ for a single promoter copy. The probability of
observing a certain mRNA copy $m$ is therefore given by
\begin{equation}
  P(m) = P(m \mid \text{one promoter}) \cdot P(\text{one promoter}) +
  P(m \mid \text{two promoters}) \cdot P(\text{two promoters}).
  \label{seq_prob_multipromoter}
\end{equation}
Both terms $P(m \mid \text{promoter copy})$ are given by
\eref{seq_two_state_mRNA} with the only difference being the rate $r_m$. It is
important to acknowledge that \eref{seq_prob_multipromoter} assumes that once
the gene is replicated the time scale in which the mRNA count relaxes to the
new steady state is much shorter than the time that the cells spend in this two
promoter copies state. This approximation should be valid for a short lived
mRNA molecule, but the assumption is not applicable for proteins whose
degradation rate is comparable to the cell cycle length as explored in
\secref{sec_cell_cycle}.

In order to repeat the Bayesian inference including this variability in gene
copy number we must split the mRNA count data into two sets -- cells with a
single copy of the promoter and cells with two copies of the promoter. For the
single molecule mRNA FISH data there is no labeling of the locus, making it
impossible to determine the number of copies of the promoter for any given
cell. We therefore follow Jones et al. \cite{Jones2014a} in using the cell area
as a proxy for stage in the cell cycle. In their approach they sorted cells by
area, considering cells below the 33$\th$ percentile having a single promoter
copy and the rest as having two copies. This approach ignores that cells are
not uniformly distributed along the cell cycle. As first derived in
\cite{Powell1956} populations of cells in a log-phase are exponentially
distributed along the cell cycle. This distribution is of the form
\begin{equation}
P(a) = (\ln 2) \cdot 2^{1 - a},
\label{seq_cell_cycle_dist}
\end{equation}
where $a \in [0, 1]$ is the stage of the cell cycle, with $a = 0$ being the
start of the cycle and $a = 1$ being the cell division (See
\siref{supp_cell_age_dist} for a derivation of \eref{seq_cell_cycle_dist}).
\fref{sfig_cell_area} shows the separation of the two groups based on area
where \eref{seq_cell_cycle_dist} was used to weight the distribution along the
cell cycle.

\begin{figure}[h!]
	\centering \includegraphics
  {../fig/si/figS04.pdf}
	\caption{\textbf{Separation of cells based on cell size.} Using the area as
  a proxy for position in the cell cycle, cells can be sorted into two groups --
  small cells (with one promoter copy) and large cells (with two promoter
  copies). The vertical black line delimits the threshold that divides both
  groups as weighted by \eref{seq_cell_cycle_dist}.}
  \label{sfig_cell_area}
\end{figure}

A subtle, but important consequence of \eref{seq_cell_cycle_dist} is that
computing any quantity for a single cell is not equivalent to computing the same
quantity for a population of cells. For example, let us assume that we want to
compute the mean mRNA copy number $\ee{m}$. For a single cell this would be of
the form
\begin{equation}
  \ee{m}_{\text{cell}} = \ee{m}_1 \cdot f + \ee{m}_2 \cdot (1 - f),
\end{equation}
where $\ee{m}_i$ is the mean mRNA copy number with $i$ promoter copies in the
cell, and $f$ is the fraction of the cell cycle that cells spend with a single
copy of the promoter. For a single cell the probability of having a single
promoter copy is equivalent to this fraction $f$. But \eref{seq_cell_cycle_dist}
tells us that if we sample unsynchronized cells we are not sampling uniformly
across the cell cycle. Therefore for a population of cells the mean mRNA is
given by
\begin{equation}
  \ee{m}_{\text{population}} = \ee{m}_1 \cdot \phi + \ee{m}_2 \cdot (1 - \phi)
  \label{seq_mean_m_pop}
\end{equation}
where the probability of sampling a cell with one promoter $\phi$ is given by
\begin{equation}
  \phi = \int_0^f P(a) da,
\end{equation}
where $P(a)$ is given by \eref{seq_age_prob}. What this equation computes is
the probability of sampling a cell during a stage of the cell cycle $< f$ where
the reporter gene hasn't been replicated yet. \fref{sfig_mRNA_by_size} shows the
distribution of both groups. As expected larger cells have a higher mRNA copy
number on average.

\begin{figure}[h!]
	\centering \includegraphics
  {../fig/si/figS05.pdf}
	\caption{\textbf{mRNA distribution for small and large cells.} (A) histogram
	and (B) cumulative distribution function of the small and large cells as
	determined in \fref{sfig_cell_area}. The triangles above histograms in (A)
	indicate the mean mRNA copy number for each group.}
  \label{sfig_mRNA_by_size}
\end{figure}

We modify \eref{seq_bayes_sample} to account for the two separate groups of
cells. Let $N_s$ be the number of cells in the small size group and $N_l$ the
number of cells in the large size group. Then the posterior distribution for the
parameters is of the form
\begin{equation}
  \small
P(\kpon, \kpoff, r_m \mid \{m_i\}) \propto
  \left[\prod_{i=1}^{N_s} P(m_i \mid \kpon, \kpoff, r_m)\right]
  \left[\prod_{j=1}^{N_l} P(m_j \mid \kpon, \kpoff, 2 r_m)\right]
  P(\kpon, \kpoff, r_m),
  \label{seq_bayes_sample_double}
\end{equation}
where we split the product of small and large cells.

For the two-promoter model the prior shown in \eref{seq_prior_single} requires a
small modification. \eref{seq_mean_m_pop} gives the mean mRNA copy number of a
population of asynchronous cells growing at steady state. Given that we assume
that the only difference between having one vs. two promoter copies is the
change in transcription rate from $r_m$ in the single promoter case to $2 r_m$
in the two-promoter case we can
write \eref{seq_mean_m_pop} as
\begin{equation}
  \ee{m} = \phi \cdot {r_m \over \gm} {\kpon \over \kpon + \kpoff} +
      (1 -\phi) \cdot {2 r_m \over \gm} {\kpon \over \kpon + \kpoff}.
\end{equation}
This can be simplified to
\begin{equation}
  \ee{m} = (2 - \phi) {r_m \over \gm} {\kpon \over \kpon + \kpoff}.
  \label{seq_mean_m_double_rates}
\end{equation}

Equating \eref{seq_mean_m_double_rates} and \eref{seq_mean_thermo} to again
require self-consistent predictions of the mean mRNA from the equilibrium and
kinetic models gives
\begin{equation}
  (2 - \phi) {\kpon \over \kpon + \kpoff} =
  {{P \over \Nns} e^{-\beta\eP}
  \over 1 + {P \over \Nns} e^{-\beta\eP}}.
\end{equation}
Solving for $\kpon \over \kpoff$ results in
\begin{equation}
  \left({\kpon \over \kpoff}\right) =
  {\rho \over \left[ (1 + \rho)(2 - \phi) - \rho \right]},
  \label{seq_kinetic_thermo_equiv}
\end{equation}
where we define $\rho \equiv {P \over \Nns} e^{-\beta\eP}$. To simplify things
further we notice that for the specified values of $P = 1000 - 3000$ per cell,
$\Nns = 4.6 \times 10^6$ bp, and $-\beta\eP = 5 - 7$, we can safely assume that
$\rho \ll 1$. This simplifying assumption has been previously called the weak
promoter approximation \cite{Garcia2011c}. Given this we can simplify
\eref{seq_kinetic_thermo_equiv} as
\begin{equation}
  {\kpon \over \kpoff} = {1 \over 2 - \phi} {P \over \Nns} e^{-\beta\eP}.
\end{equation}
Taking the log of both sides gives
\begin{equation}
  \ln\left({\kpon \over \kpoff}\right) = -\ln (2 - \phi) + \ln P - \ln\Nns
  - \beta\eP.
\end{equation}
With this we can set as before a Gaussian prior to constain the ratio of the
RNAP rates as
\begin{equation}
  P\left(\ln \left({\kpon \over \kpoff}\right) \right)  \propto
  \exp \left\{ - {\left(\ln \left({\kpon \over \kpoff}\right) -
  \left[-\ln(2 - \phi) -\beta\eP + \ln P - \ln \Nns \right) \right]^2
  \over 2 \sigma^2} \right\}.
  \label{seq_prior_double}
\end{equation}

\fref{sfig_mcmc_rnap_double} shows the result of sampling out of
\eref{seq_bayes_sample_double}. Again we see that the model is highly sloppy
with large credible regions obtained for $\kpoff$ and $r_m$. Nevertheless,
again the use of the prior information allows us to get a parameter values
consistent with the equilibrium picture.

\begin{figure}[h!]
	\centering \includegraphics
  {../fig/si/figS06.pdf}
	\caption{\textbf{MCMC posterior distribution for a multi-promoter model.}
	Sampling out of \eref{seq_bayes_sample_double} the plot shows 2D and 1D
	projections of the 3D parameter space. The parameter values are (in units of
	the mRNA degradation rate $\gm$) $\kpon = 6.4^{+0.8}_{-0.4}$, $\kpoff =
	132^{+737}_{-75}$ and $r_m = 257^{+1307}_{-132}$ which are the modes of
	their respective distributions, where the superscripts and subscripts
	represent the upper and lower bounds of the 95$\th$ percentile of the
	parameter value distributions. The sampling was bounded to values $<$ 1000 for
  numerical stability when computing the confluent hypergeometric function.}
  \label{sfig_mcmc_rnap_double}
\end{figure}

Using again a mRNA mean lifetime of $\approx 3$ min gives the following
values for the parameters: $\kpon = {0.03}_{-0.002}^{+0.004} s^{-1}$, $\kpoff =
{0.7} _{-0.4}^{+4.1} s^{-1}$, and $r_m = {1.4}_{-0.7}^{+7.3} s^{-1}$.
\fref{sfig_lacUV5_theory_data_double} shows the result of applying
\eref{seq_prob_multipromoter} using these parameter values. Specifically
\fref{sfig_lacUV5_theory_data_double}(A) shows the global distribution
including cells with one and two promoters and
\fref{sfig_lacUV5_theory_data_double}(B) splits the distributions within the
two populations. Given that the model adequately describes both populations
independently and pooled together we confirm that using the cell area as a
proxy for stage in the cell cycle and the doubling of the transcription rate
once cells have two promoters are reasonable approximations.

\begin{figure}[h!]
	\centering \includegraphics
  {../fig/si/figS07.pdf}
  \caption{\textbf{Experimental vs. theoretical distribution of mRNA per cell
  using parameters for multi-promoter model.} (A) Solid line shows the result
  of using \eref{seq_prob_multipromoter} with the parameters inferred by
  sampling \eref{seq_bayes_sample_double}. Blue bars are the same data as
  \fref{sfig_lacUV5_FISH} from \cite{Jones2014a}. (B) Split distributions of
  small cells (light blue bars) and large cells (dark blue) with the
  corresponding theoretical predictions with transcription rate $r_m$ (light
  blue line) and transcription rate $2 r_m$ (dark blue line)}
	\label{sfig_lacUV5_theory_data_double}
\end{figure}

It is hard to make comparisons with literature reported values because these
kinetic rates are effective parameters hiding a lot of the complexity of
transcription initiation \cite{Browning2004}. Also the non-identifiability of
the parameters restricts our explicit comparison of the actual numerical values
of the inferred rates. Nevertheless from the model we can see that the mean
burst  size for each transcription event is given by $r_m / \kpoff$. From our
inferred values we obtain then a mean burst size of $\approx 1.9$ transcripts
per cell. This is similar to the reported burst size of 1.15 on a similar
system on \textit{E. coli} \cite{Yu2006}.

\subsection{Repressor rates from three-state regulated promoter.}

Having determined the unregulated promoter transition rates we now proceed to
determine the repressor rates $\kron$ and $\kroff$. The values of these rates
are constrained again by the correspondence between our kinetic picture and
what we know from equilibrium models \cite{Phillips2015}. For this analysis we
again exploit the feature that, at the mean, both the kinetic language and the
thermodynamic language should have equivalent predictions. Over the last decade
there has been great effort in developing equilibrium models for gene
expression regulation \cite{Buchler2003, Vilar2011, Bintu2005a}. In particular
our group has extensively characterized the simple repression motif using this
formalism \cite{Garcia2011c, Brewster2014, Razo-Mejia2018}.

The dialogue between theory and experiments has led to simplified expressions
that capture the phenomenology of the gene expression response as a function of
natural variables such as molecule count and affinities between molecular
players. A particularly interesting quantity for the simple repression motif
used by Garcia \& Phillips \cite{Garcia2011c} is the fold-change in gene
expression, defined as
\begin{equation}
  \foldchange = {\ee{\text{gene expression}(R \neq 0)} \over
                 \ee{\text{gene expression}(R = 0)}},
\end{equation}
where $R$ is the number of repressors per cell and $\ee{\cdot}$ is the
population average. The fold-change is simply the mean expression level in the
presence of the repressor relative to the mean expression level in the absence
of regulation. In the language of statistical mechanics this quantity takes the
form
\begin{equation}
  \foldchange = \left( 1 + {R \over \Nns} e^{-\beta\eR} \right)^{-1},
  \label{seq_fc_thermo}
\end{equation}
where $\eR$ is the repressor-DNA binding energy, and as before $\Nns$ is the
number of non-specific binding sites where the repressor can bind
\cite{Garcia2011c}.

To compute the fold-change in the chemical master equation language we compute
the first moment of the steady sate mRNA distribution $\ee{m}$ for both the
three-state promoter ($R \neq 0$) and the two-state promoter case ($R=0$) (See
\siref{supp_moments} for moment derivation). The unregulated (two-state)
promoter mean mRNA copy number is given by \eref{seq_mean_m_double_rates}. For
the regulated (three-state) promoter we have an equivalent expression of the
form
\begin{equation}
  \ee{m (R \neq 0)} = (2 - \phi){r_m \over \gm} {\kroff\kpon
  \over \kpoff\kroff + \kpoff\kron + \kroff\kpon}.
\end{equation}
Computing the fold-change then gives
\begin{equation}
  \foldchange = {\ee{m (R \neq 0)} \over \ee{m (R = 0)}} =
  {\kroff \left( \kpoff + \kpon \right) \over
  \kpoff\kron + \kroff \left( \kpoff + \kpon \right)},
  \label{seq_fold_change_cme}
\end{equation}
where the factor $(2 - \phi)$ due to the multiple promoter copies, the
transcription rate $r_m$ and the mRNA degradation rate $\gm$ cancel out.

Given that the number of repressors per cell $R$ is an experimental variable
that we can control, we assume that the rate at which the promoter transitions
from the transcriptionally inactive state to the repressor bound state,
$\kron$, is given by the concentration of repressors $[R]$ times a diffusion
limited on rate $k_o$. For the diffusion limited constant $k_o$ we use the
value used by Jones et al. \cite{Jones2014a}. With this in hand we can rewrite
\eref{seq_fold_change_cme} as
\begin{equation}
  \foldchange = \left( 1 + {k_0 [R] \over \kroff}
                {\kpoff \over \kpon + \kpoff} \right)^{-1}.
  \label{seq_fc_kinetic}
\end{equation}

We note that both \eref{seq_fc_thermo} and \eref{seq_fc_kinetic} have the same
functional form. Therefore if both languages predict the same output for the
mean gene expression level, it must be true that
\begin{equation}
  {k_o [R] \over \kroff}{\kpoff \over \kpon + \kpoff} =
  {R \over \Nns} e^{-\beta\eR}.
\end{equation}
Solving for $\kroff$ gives
\begin{equation}
  \kroff = {k_o [R] \Nns e^{\beta\eR} \over R}{\kpoff \over \kpon + \kpoff}.
  \label{seq_kroff_complete}
\end{equation}

Since the reported value of $k_o$ is given in units of nM$^{-1}$s$^{-1}$ in
order for the units to cancel properly the repressor concentration has to be
given in nM rather than absolute count. If we consider that the repressor
concentration is equal to
\begin{equation}
[R] = \frac{R}{V_{cell}}\cdot \frac{1}{N_A},
\end{equation}
where $R$ is the absolute repressor copy number per cell, $V_{cell}$ is the cell
volume and $N_A$ is Avogadro's number. The \textit{E. coli} cell volume is 2.1
fL \cite{Radzikowski2016}, and Avogadro's number is $6.022 \times 10^{23}$. If
we further include the conversion factor to turn M into nM we find that
\begin{equation}
[R] = {R \over 2.1 \times 10^{-15} L} \cdot {1 \over 6.022 \times 10^{23}}
\cdot {10^9 \text{ nmol} \over 1 \text{ mol}} \approx 0.8 \times R.
\end{equation}
Using this we simplify \eref{seq_kroff_complete} as
\begin{equation}
  \kroff \approx 0.8 \cdot k_o \cdot \Nns e^{\beta\eR}
   \cdot {\kpoff \over \kpon + \kpoff}.
  \label{seq_kroff}
\end{equation}
What \eref{seq_kroff} shows is the direct relationship that must be satisfied if
the equilibrium model is set to be consistent with the non-equilibrium kinetic
picture. \tref{stab_koff} summarizes the values obtained for the three operator
sequences used throughout this work. To compute these numbers the number of
non-specific binding sites $\Nns$ was taken to be $4.6 \times 10^6$ bp, i.e. the
size of the {\it E. coli} K12 genome.

\begin{table}[]
  \caption{\textbf{Binding sites and corresponding parameters.}}
\begin{tabular}{|c|c|c|}
\hline
 Operator & $\eR\; (k_BT)$ & $\kroff \; (s^{-1})$  \\ \hline
 O1 & -15.3 & 0.002  \\ \hline
 O2 & -13.9 & 0.008  \\ \hline
 O3 & -9.7  & 0.55   \\ \hline
\end{tabular}
\label{stab_koff}
\end{table}

{\it In-vivo} measurements of the Lac repressor off rate have been done with
single-molecule resolution \cite{Hammar2014}. The authors report a mean
residence time of $5.3 \pm 0.2$ minutes for the repressor on an O1 operator. The
corresponding rate is $\kroff \approx 0.003$ $(s^{-1})$, very similar value to
what we inferred from our model. In this same reference the authors determined
that on average the repressor takes $30.9 \pm 0.5$ seconds to bind to the
operator \cite{Hammar2014}. Given the kinetic model presented in
\fref{fig2_minimal_model}(A) this time can be converted to the probability of
not being on the repressor bound state $P_{\text{not }R}$. This is computed as
\begin{equation}
  P_{\text{not }R} = {\tau_{\text{not }R} \over
                      \tau_{\text{not }R} + \tau_{R}},
\end{equation}
where $\tau_{\text{not }R}$ is the average time that the operator is not
occupied by the repressor and $\tau_{R}$ is the average time that the repressor
spends bound to the operator. Substituting the numbers from \cite{Hammar2014}
gives $P_{\text{not }R} \approx 0.088$. From our model we can compute the zeroth
moment $\ee{m^0 p^0}$ for each of the three promoter states. This moment is
equivalent to the probability of being on each of the promoter states. Upon
substitution of our inferred rate parameters we can compute $P_{\text{not }R}$
as
\begin{equation}
  P_{\text{not }R} = 1 - P_R \approx 0.046,
\end{equation}
where $P_R$ is the probability of the promoter being bound by the repressor. The
value we obtained is within a factor of two from the one reported in
\cite{Hammar2014}.
