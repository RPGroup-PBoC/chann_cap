\section{Three-state promoter model for simple repression}

In order to tackle the question of how to compute the full joint distribution of
mRNA and protein $P(m, p)$ we use the chemical master equation formalism as
described in \secref{sec_model}. Specifically we assume a three-state model
where the promoter can be found 1) with RNAP bound ($P$ state), 2) empty ($E$
state) and 3) with the repressor bound ($R$ state) (See
\fref{fig2_minimal_model}(A). These three states generate a system of three
coupled differential equations for each of the three state distributions $P_P(m,
p)$, $P_E(m, p)$ and $P_R(m, p)$. Given the rates shown in
\fref{fig2_minimal_model}(A) let us define the system of ODEs. For the RNAP
bound state we have
\begin{equation}
  \begin{aligned}
    \dt{P_P(m, p)} &=
    - \overbrace{\kpoff P_P(m, p)}^{P \rightarrow E} % P -> E
    + \overbrace{\kpon P_E(m, p)}^{E \rightarrow P}\\ % E -> P
    &+ \overbrace{r_m P_p(m-1, p)}^{m-1 \rightarrow m} % m-1 -> m
    - \overbrace{r_m P_p(m, p)}^{m \rightarrow m+1}% m -> m+1
    + \overbrace{\gm (m + 1) P_P(m+1 , p)}^{m+1 \rightarrow m} % m+1 -> m
    - \overbrace{\gm m P_P(m , p)}^{m \rightarrow m-1}\\ % m -> m-1
    &+ \overbrace{r_p m P_P(m, p - 1)}^{p-1 \rightarrow p} % p-1 -> p
    - \overbrace{r_p m P_P(m, p)}^{p \rightarrow p+1} % p -> p+1
    + \overbrace{\gp (p + 1) P_P(m, p + 1)}^{p + 1 \rightarrow p} % p+1 -> p
    - \overbrace{\gp p P_P(m, p)}^{p \rightarrow p-1}. % p -> p-1
  \end{aligned}
\end{equation}
For the empty state $E$ we have
\begin{equation}
  \begin{aligned}
    \dt{P_E(m, p)} &=
    \overbrace{\kpoff P_P(m, p)}^{P \rightarrow E} % P -> E
    - \overbrace{\kpon P_E(m, p)}^{E \rightarrow P} % E -> P
    + \overbrace{\kroff P_R(m, p)}^{R \rightarrow E} % R -> E
    - \overbrace{\kron P_E(m, p)}^{E \rightarrow R}\\ % E -> R
    &+ \overbrace{\gm (m + 1) P_E(m+1 , p)}^{m+1 \rightarrow m} % m+1 -> m
    - \overbrace{\gm m P_E(m , p)}^{m \rightarrow m-1}\\ % m -> m-1
    &+ \overbrace{r_p m P_E(m, p - 1)}^{p-1 \rightarrow p} % p-1 -> p
    - \overbrace{r_p m P_E(m, p)}^{p \rightarrow p+1} % p -> p+1
    + \overbrace{\gp (p + 1) P_E(m, p + 1)}^{p + 1 \rightarrow p} % p+1 -> p
    - \overbrace{\gp p P_E(m, p)}^{p \rightarrow p-1}. % p -> p-1
  \end{aligned}
\end{equation}
And finally for the repressor bound state $R$ we have
\begin{equation}
  \begin{aligned}
    \dt{P_R(m, p)} &=
    - \overbrace{\kroff P_R(m, p)}^{R \rightarrow E} % R -> E
    + \overbrace{\kron P_E(m, p)}^{E \rightarrow R}\\ % E -> R
    &+ \overbrace{\gm (m + 1) P_R(m+1 , p)}^{m+1 \rightarrow m} % m+1 -> m
    - \overbrace{\gm m P_R(m , p)}^{m \rightarrow m-1}\\ % m -> m-1
    &+ \overbrace{r_p m P_R(m, p - 1)}^{p-1 \rightarrow p} % p-1 -> p
    - \overbrace{r_p m P_R(m, p)}^{p \rightarrow p+1} % p -> p+1
    + \overbrace{\gp (p + 1) P_R(m, p + 1)}^{p + 1 \rightarrow p} % p+1 -> p
    - \overbrace{\gp p P_R(m, p)}^{p \rightarrow p-1}. % p -> p-1
  \end{aligned}
\end{equation}

For an unregulated promoter, i.e. a promoter within a cell that has no
repressors present, and therefore constitutively expresses the gene, we use a
two-state model in which the state $R$ is not allowed. All the terms in the
system of ODEs containing $\kron$ or $\kroff$ are then set to zero.

As detailed in \secref{sec_model} it is convenient to express this system using
matrix notation \cite{Sanchez2013}. For this we define $\PP(m, p) = (P_P(m, p),
P_E(m, p), P_R(m, p))$. Then the system of PDEs can be expressed as
\begin{equation}
  \begin{aligned}
    \dt{\PP(m, p)} &= \Km \PP(m, p)
    - \Rm \PP(m, p) + \Rm \PP(m-1, p)
    - m \Gm \PP(m, p) + (m + 1) \Gm \PP(m + 1, p)\\
    &- m \Rp \PP(m, p) + m \Rp \PP(m, p - 1)
    - p \Gp \PP(m, p) + (p + 1) \Gp \PP(m, p + 1),
  \end{aligned}
\end{equation}
where we defined the following matrices: The promoter state transition matrix
$\Km$
\begin{align}
  \Km \equiv
  \begin{bmatrix}
    -\kpoff   & \kpon         & 0\\
    \kpoff    & -\kpon -\kron  & \kroff\\
    0         & \kron         & -\kroff
  \end{bmatrix},
\end{align}
The mRNA production $\Rm$ and degradation $\Gm$ matrices
\begin{equation}
  \Rm \equiv
  \begin{bmatrix}
    r_m   & 0 & 0\\
    0     & 0 & 0\\
    0     & 0 & 0\\
  \end{bmatrix},
\end{equation}
and
\begin{equation}
  \Gm \equiv
  \begin{bmatrix}
    \gm   & 0   & 0\\
    0     & \gm & 0\\
    0     & 0   & \gm\\
  \end{bmatrix}.
\end{equation}
For the protein we also define a production $\Rp$ and degradation $\Gp$ matrices
as
\begin{equation}
  \Rp \equiv
  \begin{bmatrix}
    r_m   & 0   & 0\\
    0     & r_m & 0\\
    0     & 0   & r_m\\
  \end{bmatrix},
\end{equation}
and
\begin{equation}
  \Gp \equiv
  \begin{bmatrix}
    \gp   & 0   & 0\\
    0     & \gp & 0\\
    0     & 0   & \gp\\
  \end{bmatrix}.
\end{equation}

The corresponding equation for the unregulated two-state promoter takes the
exact same form with the definition of the matrices following the same scheme
without including the third row and third column, and setting $\kron$ and
$\kroff$ to zero.

\section{Parameter inference}

With the objective of generating falsifiable predictions with meaningful
parameters we infer the kinetic rates from this three-state promoter model using
different data sets generated in our lab over the last decade concerning
different aspects of the regulation of this simple genetic circuit. The path
used to systematically find parameter values was constrained by the nature of
the theoretical and experimental relevance of each of the available data sets.
For example, for the RNAP rates $\kpon$ and $\kpoff$ and the mRNA production
rate $r_m$ we used single-molecule mRNA FISH counts from an unregulated promoter
\cite{Jones2014a}. Once these parameters are fixed, we use these values to
constraint the repressor rates $\kron$ and $\kroff$. These repressor rates are
obtained using information from mean gene expression measurements from bulk LacZ
colorimetric assays \cite{Garcia2011c}. We also expand our model to include the
allosteric nature of the repressor protein, taking advantage of video microscopy
measurements done in the context of multiple promoter copies \cite{Brewster2014}
and flow-cytometry measurements of the mean response of the system to different
levels of induction \cite{Razo-Mejia2018}.

\subsection{RNAP rates from unregulated two-state promoter}

We begin our parameter inference problem with the RNAP rates $\kpon$ and
$\kpoff$, as well as the mRNA production rate $r_m$. In this case there are only
two states  available to the promoter -- the empty state $E$ and the RNAP bound
state $P$. That means that the third PDE for $P_R(m)$ is removed from the
system. The steady state distribution of mRNA for this particular two-state
promoter model has been analytically solved by Peccoud and Ycart
\cite{Peccoud1995}. The mRNA distribution $P(m) \equiv P_E(m) + P_P(m)$ is of
the form
\begin{equation}
  P(m) = {\Gamma \left( \frac{\kpon}{\gm} + m \right) \over
  \Gamma(m + 1) \Gamma\left( \frac{\kpoff+\kpon}{\gm} + m \right)}
  {\Gamma\left( \frac{\kpoff+\kpon}{\gm} \right) \over
  \Gamma\left( \frac{\kpon}{\gm} \right)}
  \left( {r_m \over \gm} \right)^m
  F_1^1 \left( {\kpon \over \gm} + m,
  {\kpoff + \kpon \over \gm} + m,
  -{r_m \over \gm} \right),
  \label{seq_two_state_mRNA}
\end{equation}
where $\Gamma(\cdot)$ is the gamma function, and $F_1^1$ is the confluent
hypergeometric function of the first kind. This rather complicated expression
will aid us to find parameter values for the rates. The inferred rates $\kpon$,
$\kpoff$ and $r_m$ are expressed in units of the mRNA degradation rate $\gm$.
This is because the model in \eref{seq_two_state_mRNA} is homogeneous in time,
meaning that if we divide all rates by a constant it would be equivalent to
multiplying the time scale of the problem by the same constant.

\subsubsection*{Bayesian parameter inference of RNAP rates}

In order to make progress at inferring the RNAP rates from experimental data we
make use of the single-molecule mRNA FISH data from Jones et al.
\cite{Jones2014a}. \fref{sfig_lacUV5_FISH} shows the mRNA per cell distribution
for the \textit{lacUV5} promoter. This promoter, being rather strong has a mean
copy number of $\ee{m} \approx 18$ mRNA/cell.

\begin{figure}[h!]
	\centering \includegraphics
  {../fig/chemical_master_mRNA_FISH/lacUV5_smFISH_data.pdf}
	\caption{\textbf{\textit{lacUV5} mRNA per cell distribution.} Data from
	\cite{Jones2014a} of the unregulated \textit{lacUV5} promoter as inferred
	from single molecule mRNA FISH.}
  \label{sfig_lacUV5_FISH}
\end{figure}

Having this data in hand we now use Bayesian parameter inference to infer the
parameter values of our rates. Writing Bayes theorem we have
\begin{equation}
  P(\kpon, \kpoff, r_m \mid D) = {P(D \mid \kpon, \kpoff, r_m)
  P(\kpon, \kpoff, r_m) \over P(D)},
  \label{seq_bayes_rnap_rates}
\end{equation}
where $D$ represents our data. For our particular case our data is conformed by
single-cell mRNA counts $D = \{ m_1, m_2, \ldots, m_N \}$, where $N$ is the
number of cells. We assume that each cell's measurement is independent of each
other such that we can rewrite
\eref{seq_bayes_rnap_rates} as
\begin{equation}
  P(\kpon, \kpoff, r_m \mid \{m_i\}) \propto
  \prod_{i=1}^N P(m_i \mid \kpon, \kpoff, r_m)
  P(\kpon, \kpoff, r_m).
  \label{seq_bayes_sample}
\end{equation}
Where the likelihood term $P(m_i \mid \kpon, \kpoff, r_m)$ is exactly given by
\eref{seq_two_state_mRNA} with $\gm = 1$.

\subsubsection*{Constraining the rates given prior thermodynamic knowledge.}

One of the strengths of the Bayesian approach is that we can include all the
prior knowledge on the parameters when performing the
inference\cite{MacKay2003}. Basic features such as the fact that the rates have
to be strictly positive will constraint the values that these parameters can
take. For these specific rates we know more than the simple constraint of
non-negative values. The expression of an unregulated promoter has been studied
from a thermodynamic perspective \cite{Brewster2012}. Since these equilibrium
models work in the thermodynamic limit of large particle number they are not
useful to inform us about large deviations from the expected value \mrm{My
understanding was that this is the reason we couldn't use the thermodynamic
approach, but I'm not sure anymore given Rob's recent comments}. Nevertheless at
the mean of the distribution both, the kinetic language and the equilibrium
language must agree. That means that we can use what we know about the mean gene
expression, and how this is related to parameters such as molecule copy numbers
and binding affinities, to constraint the values that these rates can take.

In the case of this two-state promoter it can be shown that the mean number of
mRNA is given by \cite{Phillips2015}
\begin{equation}
  \ee{m} = {r_m \over \gm} {\kpon \over \kpon + \kpoff}.
\end{equation}
Another way of expressing this is as ${r_m \over \gm} \times
p_{\text{bound}}^{(p)}$, where $p_{\text{bound}}^{(p)}$ is the probability of
the RNAP being bound at the promoter. The thermodynamic picture has an
equivalent result where the mean number of mRNA is given by \cite{Brewster2012,
Bintu2005a} \begin{equation} \left\langle m \right\rangle = {r_m \over \gm} {{P
\over N_{NS}} e^{-\beta\Delta\varepsilon_p} \over 1 + {P \over N_{NS}}
e^{-\beta\Delta\varepsilon_p}}, \end{equation} where $P$ is the number of RNAP
per cell, $N_{NS}$ is the number of non-specific binding sites,
$\Delta\varepsilon_p$ is the RNAP binding energy in $k_BT$ units and
$\beta\equiv {k_BT}^{-1}$.

Using these two equations we can easily see that if these frameworks are to be
equivalent, then it must be true that
\begin{equation}
  {\kpon \over \kpoff} = {P \over N_{NS}} e^{-\beta\Delta\varepsilon_p},
\end{equation}
or equivalently
\begin{equation}
  \ln \left({\kpon \over \kpoff}\right) =
  -\beta\Delta\varepsilon_p + \ln P - \ln N_{NS}.
\end{equation}
To put numerical values into these variables we can use information from the
literature. The RNAP copy number is order $P \approx 1000-3000$
RNAP/cell for a 1 hour doubling time \cite{Klumpp2008}. As for the number of
non-specific binding sites and the binding energy we have that $N_{NS} =
4.6\times 10^6$ \cite{Bintu2005a}, and $-\beta\Delta\varepsilon_p \approx 5 - 7$
\cite{Brewster2012}. Given these values we define a Gaussian prior for the ratio
of these two quantities of the form
\begin{equation}
  P\left({\kpon \over \kpoff} \right) \propto
  \exp \left\{ - {\left(\ln \left({\kpon \over \kpoff}\right) -
  \left(-\beta\Delta\varepsilon_p + \ln P - \ln N_{NS} \right) \right)^2
  \over 2 \sigma^2} \right\},
\end{equation}
where $\sigma$ is the variance that accounts for the uncertainty on these
parameters. We include this prior as part of the prior term $P(\kpon, \kpoff,
r_m)$ of \eref{seq_bayes_sample}. We then use Markov Chain Monte Carlo to sample
out of the posterior distribution given by \eref{seq_bayes_sample}.
\fref{sfig_mcmc_rnap} shows the MCMC samples of the posterior distribution. For
the case of the $\kpon$ parameter there is a single symmetric peak. $\kpoff$ and
$r_m$ have a rather long tail towards large values. As a matter of fact, the 2D
projection of $\kpoff$ vs $r_m$ shows that the model is sloppy, meaning that the
two parameters are highly correlated, a common problem in highly non-linear
systems used in biophysics and systems biology \cite{Transtrum2015}. What this
implies is that we can change the value of $\kpoff$, and then compensate by a
change on $r_m$ in order to maintain the shape of the mRNA distribution.
Therefore it is impossible from the data and the model themselves to narrow down
to a single value for the parameters. Since we used the prior information on the
rates as given by the analogous form between the equilibrium and non-equilibrium
expressions for the mean mRNA level, this allowed us to obtain at a more
constrained parameter value for the RNAP rates and the transcription rate.

\begin{figure}[h!]
	\centering \includegraphics
  {../fig/chemical_master_mRNA_FISH/lacUV5_mRNA_prior_corner_plot.pdf}
	\caption{\textbf{MCMC posterior distribution.} Sampling out of
	\eref{seq_bayes_sample} the plot shows 2D and 1D projections of the 3D
	parameter space. The parameter values are (in units of the mRNA degradation
	rate $\gm$) $\kpon = 4.4^{+0.8}_{-0.3}$, $\kpoff = 20.4^{+52.1}_{-8.4}$ and
	$r_m = 106.1^{+184.8}_{-31.2}$ which are the modes of their respective
	distributions, where the superscripts and subscripts represent the upper and
	lower bounds of the 95$^\text{th}$ percentile of the parameter value
  distributions}
  \label{sfig_mcmc_rnap}
\end{figure}

The inferred values $\kpon = 4.4^{+0.8}_{-0.3}$, $\kpoff = 20.4^{+52.1}_{-8.4}$
and $r_m = 106.1^{+184.8}_{-31.2}$ are given in units of the mRNA degradation
rate $\gm$. Given the asymmetry of the parameter distributions we report the
upper and lower bound of the 95$^\text{th}$ percentile of the posterior
distributions. Assuming a mean life-time for mRNA of $\approx$ 3 min (from
this
\href{http://bionumbers.hms.harvard.edu/bionumber.aspx?&id=107514&ver=1&trm=mRNA%20mean%20lifetime}{link})
we have an mRNA degradation rate of $\gm \approx 2.84 \times 10^{-3} s^{-1}$.
Using this value gives the following values for the inferred rates: $\kpon =
0.012_{-0.001}^{+0.002} s^{-1}$, $\kpoff = {0.06}_ {-0.02}^{+0.15} s^{-1}$, and
$r_m = 0.3_{-0.09}^{+0.5} s^{-1}$. \mrm{This is where a statement should be done
with respect to known values in the literature}

\fref{sfig_lacUV5_theory_data} shows the result of substituting these parameter
values onto \eref{seq_two_state_mRNA}. As we can see this two-state model fits
the data adequately.

\begin{figure}[h!]
	\centering \includegraphics[width=0.5\columnwidth]
  {../fig/chemical_master_mRNA_FISH/lacUV5_two_state_mcmc_fit.pdf}
	\caption{\textbf{Experimental vs. theoretical distribution of mRNA per cell
  using parameters from Bayesian inference.} Dotted line shows the result of
  using \eref{seq_two_state_mRNA} along with the parameters inferred for the
  rates. Blue bars are the same data as \fref{fig_lacUV5_FISH} from
  \cite{Jones2014a}}
  \label{sfig_lacUV5_theory_data}
\end{figure}

\subsubsection*{Accounting for variability in the number of promoters}

As discussed in ref. \cite{Jones2014a} and further expanded in
\cite{Peterson2015} an important source of noise in gene expression level in
bacteria is the fact that depending on the position relative to the chromosome
replication origin, and the growth rate, cells can have multiple copies of a
given gene. Genes closer to the replication origin have on average higher gene
copy number compare to genes at the opposite end. For the locus in which our
reporter construct is located (\textit{galK}) and the doubling time of the mRNA
FISH experiments we expect to have $\approx$ 1.66 copies of the gene
\cite{Bremer1996}. This implies that the cells spend 2/3 of the cell cycle with
two copies of the promoter and the rest with a single copy.

To account for this variability in gene copy we extend the model assuming that
when cells have two copies of the promoter the production rate is $2 r_m$
compared to the rate $r_m$ for a single copy. Then the probability of observing
certain mRNA copy $m$ is given by
\begin{equation}
  P(m) = f \cdot P(m \mid \text{one promoter}) +
  (1 - f) \cdot P(m \mid \text{two promoters}),
  \label{seq_prob_multipromoter}
\end{equation}
where $f = 1/3$ is the fraction of the cell cycle that cells spend with a single
copy of the promoter. Both terms $P(m \mid \text{promoter copy})$ are given by
\eref{seq_two_state_mRNA} with the only difference being the rate $r_m$. It is
important to acknowledge that \eref{seq_prob_multipromoter} assumes that once
the cell replicates the promoter the time scale in which the mRNA count relaxes
to the new steady state is shorter than the time that the cells spend in this
two-promoter state. This approximation should be valid for a short lived mRNA
molecule, but the assumption is not applicable for proteins whose degradation
rate is comparable to the cell cycle length as explored in
\secref{sec_cell_cycle}.

In order to repeat the Bayesian inference including this variability in gene
copy number we must split the mRNA count data in two sets -- cells with a single
copy of the promoter and cells with two copies of the promoter. For the single
molecule mRNA FISH data there is no labeling of the locus, making it impossible
to determine the number of copies of the promoter for any given cell. We
therefore follow Jones et al. \cite{Jones2014a} and use the cell area as a proxy
for stage in the cell cycle. In their approach they sorted cells by area,
considering the low 33th percentile as cells with a single promoter copy, with
the rest being cells with two copies of the promoter. This approach ignores that
cells are not uniformly distributed along the cell cycle. As first discussed in
\cite{Powell1956} populations of cells in a log-phase are exponentially
distributed along the cell cycle. This distribution is of the form
\begin{equation}
P(a) = (\ln 2) \cdot 2^{1 - a},
\label{seq_cell_cycle_dist}
\end{equation}
where $a \in [0, 1]$ is the stage of the cell cycle, with $a = 0$ being the
start of the cycle and $a = 1$ being the division. \fref{sfig_cell_area} shows
the separation of the two groups based on area where \eref{seq_cell_cycle_dist}
was used to weigh the distribution along the cell cycle.

\begin{figure}[h!]
	\centering \includegraphics
  {../fig/chemical_master_mRNA_FISH/area_division_expo.pdf}
	\caption{\textbf{Separation of cells based on cell size.} Using the area as
  a proxy for state on the cell cycle, cells can be sorted into two groups --
  small cells (with one promoter copy) and large cells (with two promoter
  copies). The vertical black line delimits the threshold that divides both
  groups as weighted by \eref{seq_cell_cycle_dist}.}
  \label{sfig_cell_area}
\end{figure}

\fref{sfig_mRNA_by_size} shows the distribution of both groups. As expected
larger cells have a higher mRNA copy number on average.

\begin{figure}[h!]
	\centering \includegraphics
  {../fig/chemical_master_mRNA_FISH/lacUV5_mRNA_size_PMF_CDF_expo.pdf}
	\caption{\textbf{mRNA distribution for small and large cells.} (A)
  probability mass function and (B) cumulative distribution function of the
  small and large cells as determined in \fref{sfig_cell_area}. The triangles
  above histograms in (A) indicate the mean mRNA copy number for each group.}
  \label{sfig_mRNA_by_size}
\end{figure}

We modify \eref{seq_bayes_sample} to account for the two separate groups of
cells. Let $N_s$ be the number of cells in the small size group and $N_l$ the
number of cells in the large size group. Then the posterior distribution for the
parameters is of the form
\begin{equation}
  \small
P(\kpon, \kpoff, r_m \mid \{m_i\}) \propto
  \prod_{i=1}^{N_s} f \cdot P(m_i \mid \kpon, \kpoff, r_m)
  \prod_{j=1}^{N_l} (1 - f) \cdot P(m_j \mid \kpon, \kpoff, 2 r_m)
  P(\kpon, \kpoff, r_m),
  \label{seq_bayes_sample_double}
\end{equation}
where we split the product of small and large cells.
\fref{sfig_mcmc_rnap_double} shows the result of sampling out of
\eref{seq_bayes_sample_double}. Again we see that the model is highly sloppy
with large credible regions obtained for $\kpoff$ and $r_m$.

\begin{figure}[h!]
	\centering \includegraphics
  {../fig/chemical_master_mRNA_FISH/lacUV5_mRNA_double_expo_corner_plot.pdf}
	\caption{\textbf{MCMC posterior distribution for a multi-promoter model.}
	Sampling out of \eref{seq_bayes_sample_double} the plot shows 2D and 1D
	projections of the 3D parameter space. The parameter values are (in units of
	the mRNA degradation rate $\gm$) $\kpon = 6^{+0.8}_{-0.4}$, $\kpoff =
	52.9^{+124}_{-28.6}$ and $r_m = 102.9^{+195.5}_{-44.1}$ which are the modes of
	their respective distributions, where the superscripts and subscripts
	represent the upper and lower bounds of the 95$^\text{th}$ percentile of the
	parameter value distributions.}
  \label{sfig_mcmc_rnap_double}
\end{figure}

and using again the mRNA mean lifetime
of 350 seconds gives the following values for the parameters: $\kpon =
0.017_{-0.001}^{+0.002} s^{-1}$, $\kpoff = {0.24}_ {-0.11}^{+0.46} s^{-1}$, and
$r_m = 0.5_{-0.2}^{+0.8} s^{-1}$. \mrm{again need to compare with what is known
about these rates.}. \fref{fig_lacUV5_theory_data_double} shows the result of
applying \eref{seq_prob_multipromoter} with these parameters.

\begin{figure}[h!]
	\centering \includegraphics[width=0.5\columnwidth]
  {../fig/chemical_master_mRNA_FISH/lacUV5_two_state_mcmc_multi_copy.pdf}
	\caption{\textbf{Experimental vs. theoretical distribution of mRNA per cell
  using parameters for single and multi promoter model} Purple dotted curve
  shows the result of using \eref{seq_prob_multipromoter} witht the parameters
  inferred by sampling \eref{seq_bayes_sample_double}. For comparison orange
  dotted line shows the model from \fref{fig_lacUV5_theory_data}. Blue bars are
  the same data as \fref{fig_lacUV5_FISH} from \cite{Jones2014a}.}
  \label{fig_lacUV5_theory_data_double}
\end{figure}

\section{Repressor rates from three-state regulated promoter.}

Having determined the RNAP rates we now proceed to determine the repressor rates
$\kron$ and $\kroff$. The value of these rates is constrained by what we know
from equilibrium models \cite{Phillips2015}. For this we again exploit the
feature that only at the mean both, the kinetic language and the thermodynamic
language should have equivalent predictions. Over the last decade there has been
a lot of effort in developing equilibrium models for gene expression regulation
\cite{Buchler2003,Vilar2011,Bintu2005a}. In particular our group has extensively
characterized the simple repression motif using this formalism
\cite{Garcia2011c,Brewster2014,Razo-Mejia2018}.

The dialogue between theory and experiments has lead to simplified expressions
that capture the phenomenology of the gene expression response as a function of
natural variables such as molecule count and affinities between molecular
players. A particularly interesting quantity defined by Garcia \& Phillips
\cite{Garcia2011c} as the fold-change in gene expression is given by
\begin{equation}
  \foldchange = {\ee{\text{gene expression}(R > 0)} \over
                 \ee{\text{gene expression}(R = 0)}},
\end{equation}
where $R$ is the number of transcriptional repressors per cell. Basically the
fold-change is the mean expression level in the presence of the repressor
divided by the expression level in the absence of regulation. In the language of
statistical mechanics this quantity is of the form \cite{Garcia2011c}
\begin{equation}
  \foldchange = \left( 1 + {R \over \Nns} e^{-\beta\eR} \right)^{-1},
  \label{seq_fc_thermo}
\end{equation}
where $\Nns$ is the number of non-specific binding sites in the genome (taken
as the size of the \textit{E. coli} genome $4.6\times 10^6$), $\eR$ is the
repressor-DNA binding energy and $\beta \equiv {1 \over k_BT}$.

To compute the fold-change in the chemical master equation language we compute
the first moment of the steady sate mRNA distribution $\ee{m}$ for both, the
three-state promoter ($R>0$) and the two-state promoter case ($R=0$)
\mrm{See section XX for moment derivation}. The latter gives
\begin{equation}
  \ee{m (R = 0)} = {r_m \over \gm} {\kpon \over \kpon + \kpoff}.
\end{equation}
The three-state promoter has a steady-state mean mRNA copy number of the form
\begin{equation}
  \ee{m (R > 0)} = {r_m \over \gm} {\kroff\kron
  \over \kpoff\kroff + \kpoff\kron + \kroff\kpon}.
\end{equation}
Computing the fold-change then gives
\begin{equation}
  \foldchange = {\ee{m (R > 0)} \over \ee{m (R = 0)}} =
  {\kroff \left( \kpoff + \kpon \right) \over
  \kpoff\kron + \kroff \left( \kpoff + \kpon \right)}.
  \label{seq_fold_change_cme}
\end{equation}

Given that the number of repressors per cell $R$ is an experimental variable
that we can control, we assume that the rate at which the promoter transitions
form the empty state to the repressor bound state $\kron$ is given by the
concentration of repressors $[R]$ times a diffusion limited rate $k_o$
\cite{Jones2014a}.  For the diffusion limited constant $k_o$ we use the value
used by Jones et al. \cite{Jones2014a} \mrm{Find real reference for this value
that Brewster never gave me.}. With this in hand we can rewrite
\eref{seq_fold_change_cme} as
\begin{equation}
  \foldchange = \left( 1 + {k_0 [R] \over \kroff}
                {\kpon \over \kpon + \kpoff} \right)^{-1}.
  \label{seq_fc_kinetic}
\end{equation}

We note that both \eref{seq_fc_thermo} and \eref{seq_fc_kinetic} have the same
functional form. Therefore if both languages predict the same output for the
mean gene expression level, it must be true that
\begin{equation}
  {k_o [R] \over \kroff}{\kpon \over \kpon + \kpoff} =
  {R \over \Nns} e^{-\beta\eR}.
\end{equation}
Solving for $\kroff$ gives
\begin{equation}
  \kroff = {k_o [R] \Nns e^{\beta\eR} \over R}{\kpon \over \kpon + \kpoff}.
  \label{seq_kroff_complete}
\end{equation}

In order for the units to cancel properly the repressor concentration has to be
given in nM rather than absolute count. If we consider that the repressor
concentration is equal to
\begin{equation}
[R] = \frac{R}{V_{cell}}\cdot \frac{1}{Av},
\end{equation}
where $R$ is the absolute repressor copy number per cell, $V_{cell}$ is the cell
volume and $Av$ is Avogadro's number. The \textit{E. coli} cell volume is in the
order of 2.1 fL = $10^{-15}$ L \mrm{get reference from Nathan}, and Avogadro's
number is $6.022 \times 10^{23}$. If we further include the conversion factor to
turn M into nM we find that
\begin{equation}
[R] = {R \over 2.1 \times 10^{-15} L} \cdot {1 \over 6.022 \times 10^{23}}
\cdot {10^9 \text{ nmol} \over 1 \text{ mol}} \approx 1.66 \times R.
\end{equation}
Using this we simplify \eref{seq_kroff_complete} as
\begin{equation}
  \kroff = 0.8 \cdot k_o \cdot \Nns e^{\beta\eR}
   \cdot {\kpon \over \kpon + \kpoff}.
  \label{seq_kroff}
\end{equation}
What \eref{seq_kroff} shows is the direct relationship that must be true if the
equilibrium model must be self consistent with the non-equilibrium kinetic
picture.

Putting all these parameters together we can generate zero-parameter fit
predictions for the full mRNA and protein distributions.
