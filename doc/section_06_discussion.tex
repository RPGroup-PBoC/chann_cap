\section*{Discussion}

Building on Shannon's formulation of information theory there have been
significant efforts using this theoretical framework to understand the
information processing capabilities of biological systems and the evolutionary
consequences for organisms harboring signal transduction systems
\cite{Bergstrom2004, Taylor2007a, Tkacik2008, Polani2009, Nemenman2010,
Rivoire2011}. More recently with the mechanistic dissection of molecular
signaling pathways significant progress has been made on the question of the
physical limits of cellular detection and the role that features such as
feedback loops play in this task \cite{Bialek2005, Libby2007, Tkacik2011,
Rhee2012a, Voliotis2014a}. But the field still lacks a rigorous experimental
test of these ideas with precision measurements on a system that is tractable
both experimentally and theoretically.

In this paper we take advantage of the recent progress on the quantitative
modeling of input-output functions of genetic circuits to build a minimal model
of the so-called simple repression motif \mrm{Aztec pyramid ref}. By combining a
series of studies on this circuit spanning diverse experimental methods for
measuring gene expression we infer all parameter values of our model - allowing
us to generate parameter-free predictions for processes related to information
processing. Some of the model parameters for our kinetic formulation of the
input-output function are informed by inferences made from equilibrium
assumptions. We use the fact that if both languages describe the same system the
predictions must be self-consistent. In other words, if the equilibrium model
can only make statements about the mean mRNA and mean protein copy number, those
predictions must be equivalent to what the kinetic model has to say about these
same quantities. This condition therefore constrains the value that the kinetic
rates in the model can take. To test whether or not the kinetic picture can
reproduce the predictions made by the equilibrium model we compare the
experimental and theoretical fold-change in protein copy number for a suite of
biophysical parameters and environmental conditions. The agreement between
theory and experiment demonstrates that both languages can indeed make
consistent predictions.

The kinetic treatment of the system brings with it increasing predictive power
compared to the equilibrium picture. Under the kinetic formulation the
predictions are not limited only to the mean but to any moment of the mRNA and
protein distribution. We first tested these novel predictions by comparing the
noise in protein copy number (standard deviation / mean) with experimental data.
Since the model is able to accurately predict the noise in protein count we
extended our analysis to infer entire protein distributions at different input
signal concentrations by using the maximum entropy principle. What this means is
that we computed moments of the protein distribution, and then used these
moments to build an approximation of the full distribution. Finally we use these
protein distributions to compute the maximum amount of information that the
genetic circuit is able to process.

By maximizing the mutual information between input signal concentration and
output protein distribution over all possible input distributions we predicted
the channel capacity for a suite of biophysical parameters. We compared the
theoretical  channel capacity - a metric of how precise is the inference that
cells can make about the state of the environment given their response - with
experimental data, finding that our minimal model is able to predict with no
free parameters this non-trivial quantity. To our knowledge, this is the first
example in which the ongoing theory-experiment dialogue on gene regulation gives
insights into the  information processing capabilities of a genetic circuit.

The findings of this work open the opportunity to accurately test intriguing
ideas that connect Shannon's metric of how accurately a signaling system can
infer the state of the signal, with Darwinian fitness \cite{Taylor2007a}.
Beautiful work has been done in the context of the developmental program of the
early {\it Drosophila} embryo \cite{Tkacik2008, Petkova2016}. These studies
demonstrated that the input-output function of the pair-rule genes works at
channel capacity, suggesting that natural selection has acted on these signaling
pathways, pushing them to operate at the limit of what the physics of these
systems allows. Our system differs from the early embryo in the sense that we
have a tunable circuit with variable amounts of information processing
capabilities. Furthermore, compared with the fly embryo in which the organism
tunes both the input and output distributions over evolutionary time, we have
experimental control of the distribution of inputs that the cells are exposed
to.
