\subsection{Computing the moments of the mRNA and protein distributions}
\label{sec_moments}

Finding analytical solutions to chemical master equations is often fraught with
difficulty. An alternative approach is to to approximate the distribution. One
such scheme of approximation, the maximum entropy principle, makes use of the
moments of the distribution to approximate the full distribution. In this
section we will demonstrate an iterative algorithm to compute the mRNA and
protein distribution moments.

The kinetic model for the simple repression motif depicted in
\fref{fig2_minimal_model}(A) consists of an infinite system of ODEs for each
possible pair of mRNA and protein copy number, $(m, p)$. To compute any moment
of the distribution, we define a vector
\begin{equation}
	\ee{\bb{m^x p^y}} \equiv (\ee{m^x p^y}_A, \ee{m^x p^y}_I, \ee{m^x p^y}_R)^T,
\end{equation}
where $\ee{m^x p^y}_S$ is the expected value of $m^x p^y$ in state $S \in \{A,
I, R\}$ for $x, y \in \mathbb{N}$. In other words, just as we defined the
vector $\PP(m, p)$, here we define a vector to collect the expected value of
each of the promoter states. By definition, any of these moments $\ee{m^x
p^y}_S$ can be computed as
\begin{equation}
  \ee{m^x p^y}_S \equiv \sum_{m=0}^\infty \sum_{p=0}^\infty m^x p^y P_S(m, p).
  \label{eq_mom_def}
\end{equation}
Summing over all possible values for $m$ and $p$ in \eref{eq_cme_matrix}
results in an ODE for any moment of the distribution of the form (See
\siref{supp_moments} for full derivation)
\begin{equation}
  \begin{aligned}
    \dt{\bb{\ee{m^x p^y}}} &=
    \Km \bb{\ee{m^x p^y}}\\
    &+ \Rm \bb{\ee{p^y \left[ (m + 1)^x -m^x \right]}}
     + \Gm \bb{\ee{m p^y \left[ (m - 1)^x - m^x \right]}}\\
    &+ \Rp \bb{\ee{m^{(x + 1)} \left[ (p + 1)^y - p^y \right]}}
     + \Gp \bb{\ee{m^x p \left[ (p - 1)^y - p^y \right]}}.
    \label{eq_gral_mom}
  \end{aligned}
\end{equation}

Given that all transitions in our stochastic model are first order reactions,
\eref{eq_gral_mom} has no moment-closure problem \cite{Voliotis2014a}. This
means that the dynamical equation for a given moment only depends on lower
moments (See \siref{supp_moments} for full proof). This feature of our model
implies, for example, that the second moment of the protein distribution
$\ee{p^2}$ depends only on the first two moments of the mRNA distribution
$\ee{m}$ and $\ee{m^2}$, the first protein moment $\ee{p}$, and the
cross-correlation term $\ee{mp}$. We can therefore define $\bmu$ to be a vector
containing all moments up to $\bb{\ee{m^x p^y}}$ for all promoter states,
\begin{equation}
\bmu = \left[ \bb{\ee{m^0 p^0}},
							\bb{\ee{m^1 p^0}},
							\ldots, \bb{\ee{m^x p^y}} \right]^T.
\end{equation}
Explicitly for the three-state promoter model depicted in
\fref{fig2_minimal_model}(A) this vector takes the form
\begin{equation}
	\bmu = \left[ \ee{m^0 p^0}_A, \ee{m^0 p^0}_I, \ee{m^0 p^0}_R,
	\ldots, \ee{m^x p^y}_A, \ee{m^x p^y}_I, \ee{m^x p^y}_R \right]^T.
\end{equation}

Given this definition we can compute the general moment dynamics as
\begin{equation}
\dt{\bmu} = \bb{A} \bmu,
\label{eq_mom_dynamics}
\end{equation}
where $\bb{A}$ is a square matrix that contains all the numerical coefficients
that relate each of the moments. We can then use \eref{eq_gral_mom} to build
matrix $\bb{A}$ by iteratively substituting values for the exponents $x$ and
$y$ up to a specified value. In the next section, we will use
\eref{eq_mom_dynamics} to numerically integrate the dynamical equations for our
moments of interest as cells progress through the cell cycle. We will then use
the value of the moments of the distribution to approximate the full gene
expression distribution. This method is computationally more efficient than
trying to numerically integrate the infinite set of equations describing the
full probability distribution $\bb{P}(m, p)$, or using a stochastic algorithm 
to sample from the distribution.