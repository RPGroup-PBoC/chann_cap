\subsection*{Computing the moments of the mRNA and protein distributions}
\label{sec_moments}

Solving chemical master equations represent a challenge that is still an active
area of mathematical research \cite{Dinh2016}. An alternative approach is to
find schemes to approximate the distribution. One such scheme, the maximum
entropy (MaxEnt) approach, makes use of the moments of the distribution to
approximate the full distribution. In this section we will demonstrate an
iterative algorithm to compute the mRNA and protein distribution moments.

\mrm{Note that in the previous section I will have already written the master
equation in matrix notation.}
Our simple repression kinetic model depicted in \fref{fig2_minimal_model}(A)
consists on an infinite system of ODEs for each possible pair $m, p$. To
compute any moment of our chemical master equation we define a vector
\begin{equation}
	\ee{\bb{m^x p^y}} \equiv (\ee{m^x p^y}_E, \ee{m^x p^y}_P, \ee{m^x p^y}_R)^T,
\end{equation}
where $\ee{m^x p^y}_S$ is the expected value of $m^x p^y$ in state $S \in \{E,
P, R\}$ for $x, y \in \mathbb{N}$. In other words, just as we defined the vector
$\PP(m, p)$, here we define a vector to collect the expected value of each of
the promoter states. By definition any of these moments $\ee{m^x p^y}_S$ are
computed as
\begin{equation}
  \ee{m^x p^y}_S \equiv \sum_{m=0}^\infty \sum_{p=0}^\infty m^x p^y P_S(m, p).
  \label{eq_mom_def}
\end{equation}

Summing over all possible $m$ and $p$ values in \mrm{ref chem eq in matrix
notation} results in a ODE for any moment of the form \mrm{See appendix XX for
full derivation}
\begin{equation}
  \begin{aligned}
    \dt{\bb{\ee{m^x p^y}}} &=
    \Km \bb{\ee{m^x p^y}}\\
    &+ \Rm \bb{\ee{p^y \left[ (m + 1)^x -m^x \right]}}\\
    &+ \Gm \bb{\ee{m p^y \left[ (m - 1)^x - m^x \right]}}\\
    &+ \Rp \bb{\ee{m^{(x + 1)} \left[ (p + 1)^y - p^y \right]}}\\
    &+ \Gp \bb{\ee{m^x p \left[ (p - 1)^y - p^y \right]}}.
    \label{eq_gral_mom}
  \end{aligned}
\end{equation}

Given that all transitions in our stochastic model are first order reactions,
\eref{eq_gral_mom} has no moment-closure problem. What this means is that the
dynamical equation for a given moment only depend on lower moments \mrm{See
appendix XX for full proof}. This feature of our model implies for example that
the second moment of the protein distribution $\ee{p^2}$ depends only on the
first two moments of the mRNA distribution $\ee{m}$, and $\ee{m^2}$, the first
protein moment $\ee{p}$  and the cross-correlation term $\ee{mp}$. We can
therefore define $\bb{\mu}$ to be a vector containing all moments up to
$\bb{\ee{m^x p^y}}$ for all promoter states. This is
\begin{equation}
\bb{\mu} = \left[ \bb{\ee{m^0 p^0}},
								  \bb{\ee{m^1 p^0}},
									\ldots \bb{\ee{m^x p^y}} \right]^T.
\end{equation}
Explicitly for the three-state promoter model depicted in
\fref{fig2_minimal_model}(A) this vector looks like
\begin{equation}
	\bb{\mu} = \left[ \ee{m^0 p^0}_E, \ee{m^0 p^0}_P, \ee{m^0 p^0}_R, \ldots
                 \ee{m^x p^y}_E, \ee{m^x p^y}_P, \ee{m^x p^y}_R \right]^T.
\end{equation}

Given this definition we can define the general moment dynamics as
\begin{equation}
\dt{\mu} = \bb{A \mu},
\label{eq_mom_dynamics}
\end{equation}
where $\bb{A}$ is a square matrix that contains all the connections in the
network, i.e. the numeric coefficients that relate each of the moments. We can
then use \eref{eq_gral_mom} to build matrix $\bb{A}$ by iteratively substituting
values for the exponents $x$ and $y$ up to a specified value. In the next
section we will use \eref{eq_mom_dynamics} to numerically integrate the
dynamical equations for our moments of interest as cells progress through the
cell cycle.
