% Preamble
\documentclass [11pt]{article}

\usepackage{setspace}
\usepackage{amssymb}
\usepackage{amsmath}
\usepackage{amsfonts}
\usepackage{amssymb}
\usepackage{setspace}
\usepackage{amsthm}
\usepackage{textcomp}
\usepackage{graphicx}
\usepackage{url}
\usepackage{color}
\usepackage[dvipsnames]{xcolor}
\definecolor{cuteBlue}{rgb}{0.258, 0.387, 0.574}
\definecolor{cuteGreen}{rgb}{0, 0.3, 0}
\usepackage{cancel}
\usepackage{comment}
\usepackage[framemethod=TikZ]{mdframed}
\usepackage{enumitem}
\usepackage{wasysym}
\usepackage{listings}
\usepackage{float}
\usepackage{booktabs}
\usepackage{fixltx2e}
\usepackage{threeparttable}
\usepackage{titling}
\usepackage{zref-base}
\usepackage{makecell}
\usepackage{array}
\usepackage{hhline}
\usepackage{titlesec}

% To make jumping between equation, figure, citation references easier
\usepackage[colorlinks=true, urlcolor=cuteBlue, citecolor=cuteGreen, linkcolor=black]{hyperref}

% To make caption labels (i.e. Figure 1, Figure 2...) bold and
% make all caption text small
\usepackage[labelfont=bf, font=small]{caption}

% For strike-out text during editing
\usepackage[normalem]{ulem}

%Latin accents
\usepackage[utf8]{inputenc}

%subfigures
\usepackage{caption}

% Add author affiliations
\usepackage{authblk}

% %%%%%%%%%%%%%%%%%%%%%%%%%%%%%%%%%%%%%%%%%%%%%%%%%%%%%%%%%%%%%%%%%%%%%%%%%%%%%%
% %%%%%%%%%%%%%%%%%%%%%%%%%%%%%%%%%%%%%%%%%%%%%%%%%%%%%%%%%%%%%%%%%%%%%%%%%%%%%%

% Margins and spacings
\setlength{\evensidemargin}{0.0cm}
\setlength{\oddsidemargin}{0.0cm}
\setlength{\topmargin}{-1.0cm}
\setlength{\textwidth}{17cm}
\setlength{\textheight}{22cm}
\setlength{\parskip}{2.5mm}
\reversemarginpar
\marginparsep  0.1in
\marginparwidth 0.7in

% Give more spacing in equation arrays
\setlength{\jot}{10pt}

% Allow page breaks in multiline equations
\allowdisplaybreaks

% Set up title spacing so we don't waste so much space
\setlength{\droptitle}{-8em}
\date{\vspace{-5em}}  % No date will appear in title.

% Spacing between section headings and text
\titlespacing\section{0pt}{12pt plus 4pt minus 2pt}{-2pt plus 2pt minus 2pt}
\titlespacing\subsection{0pt}{12pt plus 4pt minus 2pt}{-2pt plus 2pt minus 2pt}
\titlespacing\subsubsection{0pt}{12pt plus 4pt minus 2pt}{-2pt plus 2pt minus 2pt}

% Convenient micron symbol
\newcommand{\micron}{{\textmu}m}

% No excess spacing for lists
\setlist{itemsep=0pt, topsep=0pt}

% Allow paragraph indentations in lists
\setitemize{listparindent=\parindent}
\setenumerate{listparindent=\parindent}

% Column type for tables with nice spacing
\newcolumntype{M}[1]{>{\centering\arraybackslash}m{#1}}
\newcolumntype{N}{@{}m{0pt}@{}}


%%%%%%%%%%%%%%%%%%%%%%%%%%%%%%%%%%%%%%%%%%%%%%%%%%%%
%% Begin packages
%%%%%%%%%%%%%%%%%%%%%%%%%%%%%%%%%%%%%%%%%%%%%%%%%%%%
% Commenting
\newcommand{\mrm}[1]{\textcolor{ForestGreen}{(MR:~#1)}} % Commenting
\newcommand{\rp}[1]{\textcolor{red}{(RP:~#1)}} % Commenting

% To bold the first sentence of every figure caption
\newcommand{\captionStroke}[1]{\textbf{#1}}

% To define more useful LaTeX commands
\usepackage{xparse}

% Equation referencing
\newcommand{\eref}[1]{Eq.~\ref{#1}}
% \DeclareDocumentCommand \eref{oooo}{\IfNoValueTF{#2}{Eq.~(\ref{#1})}
% {\IfNoValueTF{#3}{Eqs.~(\ref{#1}) and (\ref{#2})}
% {\IfNoValueTF{#4} {Eqs.~(\ref{#1})-(\ref{#3})}
% {Eqs.~(\ref{#1})-(\ref{#4})}}}}
%
% Figure referencing
\newcommand{\fref}[1]{Fig.~\ref{#1}}
% \DeclareDocumentCommand \fref{ooo}
% {\IfNoValueTF{#2}{Fig.~\ref{#1}}{\IfNoValueTF{#3}{Figs.~\ref{#1} and
% \ref{#2}}{Figs.~\ref{#1}-\ref{#3}}}}

% Section referencing
\newcommand{\secref}[1]{Section~\ref{#1}}

% Letters for figure sub-parts
\newcommand{\letter}[1]{#1} % For main text. Ex: As shown in Fig. 12A
\newcommand{\letterParen}[1]{(#1)} % For captions. Ex: (A) Low limit and (B) high limit

%%%%%%%%%%%%%%%%%%%%%%%%%%%%%%%%%%%%%%%%%%%%%%%%%%%%
% Personalized functions
%%%%%%%%%%%%%%%%%%%%%%%%%%%%%%%%%%%%%%%%%%%%%%%%%%%%
% Rates
\newcommand{\kpon}{k^{(p)}_{\text{on}}}
\newcommand{\kpoff}{k^{(p)}_{\text{off}}}
\newcommand{\kron}{k^{(r)}_{\text{on}}}
\newcommand{\kroff}{k^{(r)}_{\text{off}}}
\newcommand{\gm}{\gamma _m}
\newcommand{\gp}{\gamma _p}
% Matrices
\newcommand{\Km}{\bb{K}}
\newcommand{\Rm}{\bb{R}_m}
\newcommand{\Gm}{\bb{\Gamma}_m}
\newcommand{\Rp}{\bb{R}_p}
\newcommand{\Gp}{\bb{\Gamma}_p}
% Expected value
\newcommand{\ee}[1]{\left\langle #1 \right\rangle}
% Bold math
\newcommand{\bb}[1]{\mathbf{#1}}
% Time derivative
\newcommand{\dt}[1]{{d{#1} \over dt}}
\newcommand{\ddt}[1]{{\partial{#1} \over \partial t}}
% Sum over m and p
\newcommand{\smp}{\sum_m \sum_p}
% Bold P (for distribution vector)
\newcommand{\PP}{\bb{P}}
% Thermodynamic variables
\newcommand{\foldchange}{\text{fold-change}}
\newcommand{\Nns}{N_\text{NS}}
\newcommand{\eR}{\Delta\varepsilon_r}
\newcommand{\eAI}{\Delta\varepsilon_{AI}}
\newcommand{\pbound}{p_{_\text{bound}}}

\let\oldth\th
\renewcommand\th{\text{th}}


%%%%%%%%%%%%%%%%%%%%%%%%%%%%%%%%%%%%%%%%%%%%%%%%%%%%
%% Begin document
%%%%%%%%%%%%%%%%%%%%%%%%%%%%%%%%%%%%%%%%%%%%%%%%%%%%
\title{\textbf{First-principles prediction of the information processing
capacity of a simple genetic circuit}}
\author[1]{Manuel Razo-Mejia}
\author[1, 2, 3, *]{Rob Phillips}
\affil[1]{Division of Biology and Biological Engineering, California Institute
of Technology, Pasadena, CA 91125, USA}
\affil[2]{Department of Physics, California Institute of Technology, Pasadena,
CA 91125, USA}
\affil[3]{Department of Applied Physics, California Institute of Technology,
Pasadena, CA 91125, USA}
\affil[*]{Correspondence: phillips@pboc.caltech.edu}

\date{\today}

\setcounter{Maxaffil}{0}
% Set affiliations in small font
\renewcommand\Affilfont{\itshape\small}

\begin{document}
\maketitle
	% \begin{abstract}
  Given the stochastic nature of gene expression, genetically identical cells
  exposed to the same environmental inputs will produce different outputs. This
  heterogeneity has consequences on how cells are able to survive in changing
  environments. Recent work has explored the use of information theory as a
  framework to understand the accuracy with which cells can ascertain the state
  of their surroundings. Yet the predictive power of these approaches is
  limited and has not been rigorously tested using precision measurements. To
  that end, we generate a minimal model for a simple genetic circuit in which
  all parameter values for the model come from independently published data
  sets. We then predict the information processing capacity of the genetic
  circuit for a sweep of biophysical parameters such as protein copy number and
  protein-DNA affinity. We compare these parameter-free predictions with
  experimental inferences of the information processing capacity of {\it E.
  coli} cells to find that our minimal model accurately reproduces the
  experimental data.
\end{abstract}

	% As living organisms thrive in some given environment, they are faced with
constant changes in their surroundings. From abiotic conditions such as
temperature fluctuations or changes in osmotic pressure to biological
interactions such as cell-to-cell communication in a tissue or in a bacterial
biofilm, living organisms of all types sense and respond to external signals as
depicted in \fref{fig1_intro}(A) for a bacterial cell sensing a concentration of
a chemical. At the molecular level where signal transduction unfolds
mechanistically, there are physical constraints on the accuracy and precision of
these responses given by the intrinsic stochastic fluctuations
\cite{Nemenman2010}. This means that two genetically identical cells exposed to
the same stimulus will not have an identical response \cite{Eldar2010}.

The implications of this biological noise is that cells do not have an infinite
resolution to distinguish signals and, as a consequence, there is a one-to-many
mapping between inputs and outputs. In that sense, one could think of cells
performing a Bayesian inference of the state of the environment given their
phenotypic response as schematized in \fref{fig1_intro}(B). The question then
becomes how to analyze this probabilistic rather than deterministic relationship
between inputs and outputs? The abstract answer to this question was worked out
in 1948 by Claude Shannon who, in his seminal work, founded the field of
information theory \cite{Shannon1948}. Shannon developed a general framework for
how to analyze information transmission through noisy communication channels. In
his work, Shannon showed that the only quantity that satisfies simple conditions
of what a metric for information should be, was of the same functional form as
the thermodynamic entropy -- thereby christening his metric the information
entropy \cite{MacKay2003}. He also gave a definition, based on this information
entropy, for the relationship between inputs and outputs known as the mutual
information \mrm{see Appendix XX for details on these metrics}.

It is natural to conceive of certain scenarios in which living organisms that
can better resolve signals might have an evolutionary advantage, making it more
likely that their offspring will have a fitness advantage \cite{Taylor2007a}. In
recent years there has been a growing interest in understanding the theoretical
limits on cellular information processing \cite{Bialek2005, Gregor2007}, and in
quantifying how close evolution has pushed cellular signaling pathways to these
theoretical limits \cite{Tkacik2008, Dubuis2013, Petkova2016}. While these
studies have treated the signaling pathway as a ``black box'' explicitly
ignoring all the molecular interactions taking place in them, other studies have
explored the role that molecular players and regulatory architectures have on
these information processing tasks \cite{Rieckh2014, Ziv2007, Voliotis2014,
Tostevin2009, Tkacik2011, Tkacik2008a, Tabbaa2014}. Despite the great
advancement in our understanding of the information processing capabilities of
molecular mechanisms, the field still lacks a rigorous experimental test of
these ideas with precision measurements on a simple system tractable both
theoretically and experimentally.

On the other hand, over the last decade the dialogue between theory and
experiments in gene regulation has led to predictive power not only over the
mean, but the noise in gene expression as a function of relevant parameters such
as regulatory protein copy numbers, affinity of these proteins to the DNA
promoter, as well as the extracellular concentrations of inducer
molecules \cite{Garcia2011c, Jones2014a, Brewster2014, Razo-Mejia2018} \mrm{Too
self referential so far. Include Ido, maybe Al Sanchez. It must be
experiment-theory contrasting though!}. These models based on equilibrium and
non-equilibrium statistical physics have reached a predictive accuracy level
such that for simple cases it is now possible to design input-output functions
\cite{Brewster2012, Barnes2018}. This opens the possibility to exploit these
predictive models to tackle the question of how much information genetic
circuits can process. The question lays at the heart of understanding the
precision of the cellular response to environmental signals.
\fref{fig1_intro}(C) schematizes a scenario in which for three different
stimuli, i.e. inducer concentrations, two bacterial strains respond with
different levels of precision given the degrees of overlap between outputs. It
is precisely this overlap between responses that determines the resolution with
which cells can distinguish different inputs.
\rp{You could mention overlap of point spread function and rayleigh diffraction
limit.}

In this work we follow the same philosophy of theory-experiment dialogue to
predict from first principles the effect that biophysical parameters such as
transcription factor copy number and protein-DNA affinity have on the
information processing capacity of a simple genetic circuit. Specifically we use
a master-equation-based model to construct the protein copy number distribution
(output) as a function of an extracellular inducer concentration (input) for
different combinations of transcription factor copy numbers and binding sites.
We then compute the channel capacity, i.e. the maximum information that can be
processed by this gene regulatory architecture. All parameters used for our
model were inferred from a series of studies that span several experimental
techniques \cite{Garcia2011c, Brewster2012, Jones2014a, Brewster2014,
Razo-Mejia2018}, allowing us to perform parameter free predictions of this
non-trivial quantity. \mrm{Aztec pyramid reference}

These predictions are then contrasted with experimental data, where the channel
capacity is inferred from single-cell fluorescence distributions taken at
different concentrations of inducer for cells with previously characterized
biophysical parameters \cite{Garcia2011c, Razo-Mejia2018}. We find that our
parameter free predictions closely match the experiments. \mrm{In this sense we
demonstrate how our minimal model can be used to quantify the resolution with
which cells can resolve the environmental state with no free parameters.}

The reminder of the paper is organized as follows. In \secref{sec_model} we
define the minimal theoretical model and parameter inference for a simple
repression genetic circuit. \secref{sec_moments} computes the moments of the
mRNA and protein distributions from this minimal model. In
\secref{sec_cell_cycle} we explore the consequences of variability in gene copy
number along the cell cycle. In this section we compare experimental and
theoretical quantities related to the moments of the distribution. Specifically
the predictions for the fold-change in gene expression (mean relative expression
with respect to an unregulated promoter) and the gene expression noise (standard
deviation over mean). \secref{sec_maxent} follows with reconstruction of the
full mRNA and protein distribution from the moments using the maximum entropy
principle. Finally \secref{sec_channcap} uses the distributions from
\secref{sec_maxent} to compute the maximum amount of information that the
genetic circuit can process. Here we again contrast our zero-parameter fit
predictions with experimental inferences of the channel capacity.


\begin{figure}[h!]
	\centering \includegraphics
  {./fig/main/intro_v02.pdf}
	\caption{\textbf{Cellular signaling systems sense the environment with
	different degrees of precision}. (A) Schematic representation of cells as a
	noisy communication channel. From an environmental input (inducer molecule
	concentration) to a phenotypic output (protein expression level), cellular
	signaling systems can be modeled as noisy communication channels. (B) We treat
	cellular response to an external stimuli as a Bayesian inference  of the state
	of the environment. As the phenotype (protein level) serves as the internal
	representation of the environmental state (inducer concentration), the
	probability of a cell being on a specific environment is a function of the
	probability of the response given that environmental state. (C) The precision
	of the inference of the environmental state depends on how well can cells
	resolve different inputs. For three different levels of input (left panel) the
	green strain responds more precisely than the purple strain since the
	output distributions overlap less (middle panel). This allows the green strain
	to make a more precise inference of the environmental state given a
	phenotypic response.}
  \label{fig1_intro}
\end{figure}

	% \section{Results}

\subsection{Minimal model of transcriptional regulation}\label{sec_model}

We begin by defining the simple repression genetic circuit to be used throughout
this work. As a tractable circuit for which we have control over the parameters
both theoretically and experimentally we chose the so-called simple  repression
motif, a common regulatory scheme among prokaryotes \cite{Rydenfelt2014}. This
circuit consists of a single promoter with an RNA-polymerase (RNAP) binding site
and a single binding site for a transcriptional repressor \cite{Garcia2011c}.
The regulation due to the repressor occurs via exclusion of the RNAP from its
binding site when the repressor is bound, decreasing the likelihood of having a
transcription event. As with many important macromolecules, we consider the
repressor to be allosteric, meaning that it can exist in two conformations, one
in which the repressor is able to bind to the specific binding site (active
state) and one in which it cannot bind the specific binding site (inactive
state). The environmental signaling occurs via passive import of an
extracellular inducer that binds the repressor, shifting the equilibrium between
the two conformations of the repressor \cite{Razo-Mejia2018}. In previous
publications we have extensively characterized the mean response of this circuit
under different conditions using equilibrium based models \cite{Phillips2019}.
In this work we build upon these models to characterize the full distribution of
gene expression with parameters such as repressor copy number and its affinity
for the DNA are systematically varied.

Given the discrete nature of molecular species copy numbers inside cells,
chemical master equations have emerged as a useful tool to model the inherent
probability distribution of these counts \cite{Sanchez2013}. In
\fref{fig2_minimal_model}(A) we show the minimal model and the necessary set of
parameters needed to predict mRNA and protein distributions. Specifically, we
assume a three-state model where the promoter can be found 1) with RNAP bound
($B$ state), 2) empty ($E$ state) and 3) with the repressor bound ($R$ state).
These three states generate a system of coupled differential equations for each
of the three state distributions $P_B(m, p; t)$, $P_E(m, p; t)$ and $P_R(m, p;
t)$, where $m$ and $p$ are the mRNA and protein count per cell, respectively and
$t$ is the time. Given the rates shown in \fref{fig2_minimal_model}(A) we define
the system of ODEs for a specific $m$ and $p$. For the RNAP bound state we have
\begin{equation}
  \begin{aligned}
    \dt{P_B(m, p)} &=
    - \overbrace{\kpoff P_B(m, p)}^{B \rightarrow E} % P -> E
    + \overbrace{\kpon P_E(m, p)}^{E \rightarrow B}\\ % E -> P
    &+ \overbrace{r_m P_B(m-1, p)}^{m-1 \rightarrow m} % m-1 -> m
    - \overbrace{r_m P_B(m, p)}^{m \rightarrow m+1}% m -> m+1
    + \overbrace{\gm (m + 1) P_B(m+1 , p)}^{m+1 \rightarrow m} % m+1 -> m
    - \overbrace{\gm m P_B(m , p)}^{m \rightarrow m-1}\\ % m -> m-1
    &+ \overbrace{r_p m P_B(m, p - 1)}^{p-1 \rightarrow p} % p-1 -> p
    - \overbrace{r_p m P_B(m, p)}^{p \rightarrow p+1} % p -> p+1
    + \overbrace{\gp (p + 1) P_B(m, p + 1)}^{p + 1 \rightarrow p} % p+1 -> p
    - \overbrace{\gp p P_B(m, p)}^{p \rightarrow p-1}. % p -> p-1
  \end{aligned}
\end{equation}
For the empty state $E$ we have
\begin{equation}
  \begin{aligned}
    \dt{P_E(m, p)} &=
    \overbrace{\kpoff P_B(m, p)}^{B \rightarrow E} % P -> E
    - \overbrace{\kpon P_E(m, p)}^{E \rightarrow B} % E -> P
    + \overbrace{\kroff P_R(m, p)}^{R \rightarrow E} % R -> E
    - \overbrace{\kron P_E(m, p)}^{E \rightarrow R}\\ % E -> R
    &+ \overbrace{\gm (m + 1) P_E(m+1 , p)}^{m+1 \rightarrow m} % m+1 -> m
    - \overbrace{\gm m P_E(m , p)}^{m \rightarrow m-1}\\ % m -> m-1
    &+ \overbrace{r_p m P_E(m, p - 1)}^{p-1 \rightarrow p} % p-1 -> p
    - \overbrace{r_p m P_E(m, p)}^{p \rightarrow p+1} % p -> p+1
    + \overbrace{\gp (p + 1) P_E(m, p + 1)}^{p + 1 \rightarrow p} % p+1 -> p
    - \overbrace{\gp p P_E(m, p)}^{p \rightarrow p-1}. % p -> p-1
  \end{aligned}
\end{equation}
And finally, for the repressor bound state $R$ we have
\begin{equation}
  \begin{aligned}
    \dt{P_R(m, p)} &=
    - \overbrace{\kroff P_R(m, p)}^{R \rightarrow E} % R -> E
    + \overbrace{\kron P_E(m, p)}^{E \rightarrow R}\\ % E -> R
    &+ \overbrace{\gm (m + 1) P_R(m+1 , p)}^{m+1 \rightarrow m} % m+1 -> m
    - \overbrace{\gm m P_R(m , p)}^{m \rightarrow m-1}\\ % m -> m-1
    &+ \overbrace{r_p m P_R(m, p - 1)}^{p-1 \rightarrow p} % p-1 -> p
    - \overbrace{r_p m P_R(m, p)}^{p \rightarrow p+1} % p -> p+1
    + \overbrace{\gp (p + 1) P_R(m, p + 1)}^{p + 1 \rightarrow p} % p+1 -> p
    - \overbrace{\gp p P_R(m, p)}^{p \rightarrow p-1}. % p -> p-1
  \end{aligned}
\end{equation}
As we will discuss later in \secref{sec_cell_cycle} the protein degradation term
$\gp$ is set to zero since we do not consider protein degradation as a Poission
process, but rather we explicitly implement binomial partitioning as the cells
grow and divide.

It is convenient to rewrite these equations in a compact matrix notation
\cite{Sanchez2013}. For this we define the vector $\PP(m, p)$ as
\begin{equation}
  \PP(m, p) = (P_B(m, p), P_E(m, p), P_R(m, p))^T,
\end{equation}
where $T$ is the transpose. By defining the matrices $\Km$ to contain the
promoter state transitions, $\Rm$ and $\Gm$ to contain the mRNA production and
degradation terms, respectively, and $\Rp$ and $\Gp$ to contain the protein
production and degradation terms, respectively, the system of ODEs can then be
written as (See \siref{supp_model} for full definition of these matrices)
\begin{equation}
  \begin{aligned}
    \dt{\PP(m, p)} &= \left( \Km -\Rm -m\Gm -m\Rp -p\Gp \right) \PP(m, p)\\
    &+ \Rm \PP(m-1, p)
    + (m + 1) \Gm \PP(m + 1, p)\\
    &+ m \Rp \PP(m, p - 1)
    + (p + 1) \Gp \PP(m, p + 1).
  \end{aligned}
  \label{eq_cme_matrix}
\end{equation}

\subsection{Inferring parameters from published data sets}
\label{sec_param}

A decade of research in our group has characterized the simple repression motif
with an ever expanding array of predictions and corresponding experiments to
uncover the physics of this genetic circuit \cite{Phillips2019}. In doing so
we have come to understand the mean response of a single promoter in the
presence of varying levels of repressor copy numbers and repressor-DNA
affinities \cite{Garcia2011c}, due to the effect that competing binding sites
and multiple promoter copies impose \cite{Brewster2014}, and in recent work,
assisted by the Monod-Wyman-Changeux (MWC) model, we expanded the scope to the
allosteric nature of the repressor \cite{Razo-Mejia2018}. All of these studies
have exploited the simplicity and predictive power of equilibrium approximations
to these non-equilibrium systems \cite{Buchler2003}. We have also used a similar
kinetic model to the one depicted in \fref{fig2_minimal_model}(A) to study the
noise in mRNA copy number \cite{Jones2014a}. As a test case of the depth of our
theoretical understanding of the so-called ``hydrogen atom'' of transcriptional
regulation we combine all of the studies mentioned above to inform the parameter
values of the model presented in \fref{fig2_minimal_model}(A).
\fref{fig2_minimal_model}(B) schematizes the data sets and experimental
techniques used to measure gene expression along with the parameters that can be
inferred from them.

\siref{supp_param_inference} expands on the details of how the inference was
performed for each of the parameters. Briefly the RNAP rates $\kpon$ and
$\kpoff$, as well as the transcription rate $r_m$ were obtained in units of the
mRNA degradation rate $\gm$ by fitting a two-state promoter model (no state $R$
from \fref{fig2_minimal_model}(A)) \cite{Peccoud1995} to mRNA FISH data of an
unregulated promoter (no  repressor present in the cell) \cite{Jones2014a}. The
repressor on rate is assumed to be of the form $\kron = k_o [R]$ where $k_o$ is
a diffusion-limited on rate and $[R]$ is the concentration of active repressor
in the cell \cite{Jones2014a}. This concentration of active repressor is at the
same time determined by the mean repressor copy number in the cell, and the
fraction of repressors in the active state. Existing estimates of the transition
rates between conformations of allosteric molecules set them at the microsecond
scale \cite{Cui2008}. By considering this to be representative for our repressor
of interest, the separation of time-scales between the rapid conformational
changes of the repressor and the slower downstream processes such as the
open-complex formation processes allow us to model the probability of the
repressor being in the active state as an equilibrium MWC process. The
parameters of the MWC model $K_A$, $K_I$ and $\eAI$ were previously
characterized from video-microscopy and flow-cytometry data
\cite{Razo-Mejia2018}. For the repressor off rate $\kroff$ we take advantage of
the fact that the mean mRNA copy number as derived from the model in
\fref{fig2_minimal_model}(A) cast in the language of rates is of the same
functional form as the equilibrium model cast in the language of binding
energies \cite{Phillips2015}. Therefore the value of the repressor-DNA binding
energy $\eR$ constrains the value of the repressor off rate $\kroff$. These
constraints on the rates allow us to make self-consistent predictions under
both, the equilibrium and the kinetic framework.

\begin{figure}[h!]
	\centering \includegraphics
  {./fig/main/parameter_inference_v03.pdf}
	\caption{\textbf{Minimal kinetic model of transcriptional regulation for a
	simple repression architecture.} (A) Three-state promoter stochastic model of
	transcriptional regulation by a repressor. The regulation by the repressor
	occurs via exclusion of the RNAP, not allowing a promoter state in which both,
	the repressor and the RNAP are bound simultaneously. All parameters
	highlighted with colored boxes were determined with published datasets based
	on the same genetic circuit. (B) Data sets used to infer the parameter values.
	From left to right Garcia \& Phillips \cite{Garcia2011c} is used to determine
	$\kroff$ and $\kron$, Brewster et al. \cite{Brewster2014} is used to determine
	$\eAI$ and $\kron$, Razo-Mejia et al. \cite{Razo-Mejia2018} is used to
	determine $K_A$, $K_I$, and $\kron$ and Jones et al. is used to determine
	$r_m$, $\kpon$, and $\kpoff$.}
  \label{fig2_minimal_model}
\end{figure}

	% \subsection{Computing the moments of the mRNA and protein distributions}

Solving chemical master equations represent a challenge that is still an active
area of mathematical research \cite{Dinh2016}. One approach is to find schemes
to approximate the distribution. One such scheme, the maximum entropy (MaxEnt)
approach, makes use of the moments of the distribution to approximate the full
distribution. In this section we will demonstrate an iterative algorithm to
compute the mRNA and protein joint distribution moments.

\mrm{Note that in the previous section I will have already written the master
equation in matrix notation.}
Our simple repression kinetic model depicted in \fref{fig2_minimal_model}(A)
consists on an infinite system of ODEs for each possible pair $m, p$. To
compute any moment of our chemical master equation we define a vector
\begin{equation}
	\ee{\bb{m^x p^y}} \equiv (\ee{m^x p^y}_E, \ee{m^x p^y}_P, \ee{m^x p^y}_R)^T,
\end{equation}
where $\ee{m^x p^y}_S$ is the expected value of $m^x p^y$ in state $S \in \{E,
P, R\}$ for $x, y \in \mathbb{N}$. In other words, just as we defined the vector
$\PP(m, p)$, here we define a vector to collect the expected value of each of
the promoter states. By definition any of these moments $\ee{m^x p^y}_S$ are
computed as
\begin{equation}
  \ee{m^x p^y}_S \equiv \sum_{m=0}^\infty \sum_{p=0}^\infty m^x p^y P_S(m, p).
  \label{eq_mom_def}
\end{equation}

Summing over all possible $m$ and $p$ values in \mrm{ref chem eq in matrix
notation} results in a ODE for any moment of the form \mrm{See appendix XX for
full derivation}
\begin{equation}
  \begin{aligned}
    \dt{\bb{\ee{m^x p^y}}} &=
    \Km \bb{\ee{m^x p^y}}\\
    &+ \Rm \bb{\ee{p^y \left[ (m + 1)^x -m^x \right]}}\\
    &+ \Gm \bb{\ee{m p^y \left[ (m - 1)^x - m^x \right]}}\\
    &+ \Rp \bb{\ee{m^{(x + 1)} \left[ (p + 1)^y - p^y \right]}}\\
    &+ \Gp \bb{\ee{m^x p \left[ (p - 1)^y - p^y \right]}}.
    \label{eq_gral_mom}
  \end{aligned}
\end{equation}

Given that all transitions in our stochastic model are first order reactions,
\eref{eq_gral_mom} has no moment-closure problem. What this means is that the
dynamical equation for a given moment only depend on lower moments \mrm{See
appendix XX for full proof}. This feature of our model implies for example that
the second moment of the protein distribution $\ee{p^2}$ depends only on the
first two moments of the mRNA distribution $\ee{m}$, and $\ee{m^2}$, the first
protein moment $\ee{p}$  and the cross-correlation term $\ee{mp}$. We can
therefore define $\bb{\mu}$ to be a vector containing all moments up to
$\bb{\ee{m^x p^y}}$ for all promoter states. This is
\begin{equation}
\bb{\mu} = \left[ \bb{\ee{m^0 p^0}},
								  \bb{\ee{m^1 p^0}},
									\ldots \bb{\ee{m^x p^y}} \right]^T.
\end{equation}
Explicitly for the three-state promoter model depicted in
\fref{fig2_minimal_model}(A) this vector looks like
\begin{equation}
	\bb{\mu} = \left[ \ee{m^0 p^0}_E, \ee{m^0 p^0}_P, \ee{m^0 p^0}_R, \ldots
                 \ee{m^x p^y}_E, \ee{m^x p^y}_P, \ee{m^x p^y}_R \right]^T.
\end{equation}

Given this definition we can define the general moment dynamics as
\begin{equation}
\dt{\mu} = \bb{A \mu},
\label{eq_mom_dynamics}
\end{equation}
where $\bb{A}$ is a square matrix that contains all the connections in the
network, i.e. the numeric coefficients that relate each of the moments. We can
then use \eref{eq_gral_mom} to build matrix $\bb{A}$ by iteratively substituting
values for the exponents $x$ and $y$ up to a specified value. In the next
section we will use \eref{eq_mom_dynamics} to numerically integrate the
dynamical equations for our moments of interest as cells progress through the
cell cycle.

	% \section{Accounting for cell-cycle dependent variability in gene dosage}

As cells progress through the cell cycle the genome has to be replicated to
guarantee that each daughter cell to receives a copy of the genetic material.
This replication of the genome implies that cells spend parts of the cell cycle
with multiple copies of each gene depending on the cellular growth rate and the
relative position of the gene with respect to the replication origin
\cite{Bremer1996}. Genes closer to the replication origin spend larger fractions
of the cell cycle with multiple copies compared to genes closer to the
replication termination site. \fref{fig3_cell_cycle}(A) depicts a schematic of
this process where the replication origin ({\it oriC}) and the relevant locus
for our experimental measurements ({\it galK}) are highlighted.

Since this change in gene copy number has been shown to have an effect on
cell-to-cell variability in gene expression \cite{Jones2014a, Peterson2015}, we
now extend our minimal model to account for these changes in gene copy number
during the cell cycle.  We assume that the only difference between the
single-copy state and the two-copies states of the cell is a doubling of the
mRNA production rate $r_m$. In particular the RNAP rates $\kpon$ and $\kpoff$
and the mRNA production rate $r_m$ inferred in section \mrm{param section}
assume that cells spend a fraction $f$ of the cell cycle  with one copy of the
promoter (mRNA production rate $r_m$) and a fraction $(1-f)$ of the cell cycle
with two copies of the promoter (mRNA production rate $2 r_m$). This inference
was performed assuming that at each cell state the mRNA level immediately
reaches the steady state value for the corresponding mRNA production rate. The
steady state assumption is justified since the timescale to reach this steady
state depends only on the degradation rate $\gm$, which for the mRNA  is much
shorter ($\approx 3$ min) than the length of the cell cycle (100 min for our
experimental conditions). Appendix \mrm{ref appendix for parameter inference}
shows that by using area as a proxy for stage in the cell cycle a two-state
promoter model is able to capture the experimental data from single molecule
mRNA counts of an unregulated promoter.

Given that we assume that the protein degradation rate $\gp$ is set by the cell
division time, we cannot assume that the protein count reaches the corresponding
steady state value for each stage along the cell cycle. In other words, cells do
not spend long enough time with two copies of the promoter for the protein to
reach the steady state value corresponding to a transcription rate of $2 r_m$.
We therefore use the dynamical equations developed in \mrm{ref moments section}
to numerically integrate the moments of the distribution dynamics with the
corresponding parameters for each phase of the cell cycle.
\fref{fig3_cell_cycle}(B) shows an example corresponding to the mean mRNA level
(upper panel) and the mean protein level (lower panel) for the case of the
two-state unregulated promoter. Since we inferred the RNAP rate parameters
assuming that mRNA reaches steady state at each stage, we see that the numerical
integration of the equations are consistent with the assumption of having the
mRNA reach a stable value at each stage (See \fref{fig3_cell_cycle}(B) upper
panel). On the other hand the mean protein level does not reach a stable value
at each of the cellular stages. Nevertheless it is interesting to observe that
after a couple of cell cycles the trajectory from cycle to cycle follows a
repetitive pattern (See \fref{fig3_cell_cycle}(B) lower panel).

\begin{figure}[h!]
	\centering \includegraphics
  {./fig/main/cell_cycle_moments_v01.pdf}
	\caption{\textbf{Accounting for gene copy number variability during the cell
	cycle.} (A) Schematic of a replicating genome. As cells progress through the
	cell cycle the genome is replicated, duplicating gene copies for a fraction of
	the cell cycle. {\it oriC} indicates the replication origin, and {\it galK}
	indicates the locus at which the experimental construct was integrated. (B)
	mean mRNA (upper panel) and mean protein (lower panel) dynamics. Cells spend a
	fraction of the cell cycle with a single copy of the promoter (light brown)
	and the rest of the cell cycle with two copies (light yellow). Black arrows
	indicate time of cell division. (C) Zero parameter-fit predictions (solid
	lines) and experimental data (circles) of the fold-change (upper row) and
	noise $\nu$ (lower row) for different repressor binding sites with different
	affinities and different repressor copy numbers per cell. White dots on the
	lower row are plotted on a different scale for visual clarity. \mrm{I need to
	discuss this with you Rob. It has to do with being 100\% honest by showing all
	of the data, but not letting some outliers with known pathologies distract the
	reader from the main point.}}
  \label{fig3_cell_cycle}
\end{figure}

	% \subsection{Maximum Entropy approximation}\label{sec_maxent}

Having numerically computed the moments of the mRNA and protein distributions as
cells progress through the cell cycle we now proceed to make an approximating
reconstruction of the full distributions given this limited information. As
hinted in \secref{sec_moments} the maximum entropy (MaxEnt) principle, first
proposed by E.T. Jaynes in 1957, approximates the entire distribution by
maximizing the Shannon entropy subject to constraints given by, among other
quantities, the values of the moments of the distribution \cite{Jaynes1957}.
This procedure leads to a distribution of the form
\mrm{See appendix XX for full derivation.}
\begin{equation}
  P_H(m, p) = {1 \over \mathcal{Z}}
              \exp \left( - \sum_{(x,y)} \lambda_{(x,y)} m^x p^y \right),
  \label{eq_maxEnt_joint}
\end{equation}
where $\lambda_{(x,y)}$ is the Lagrange multiplier associated with the
constraint set by the moment $\ee{m^x p^y}$, and $\mathcal{Z}$ is a
normalization constant. The more moments $\ee{m^x p^y}$ included as constraints,
the more accurate the approximation resulting from \eref{eq_maxEnt_joint}
becomes.

The computational challenge then becomes a minimization routine in which the
values for the Lagrange multipliers $\lambda_{(x,y)}$ that are consistent with
the constraints set by the moments values $\ee{m^x p^y}$ need to be found.
\mrm{Appendix XX} details our implementation of a robust algorithm to find such
values. \fref{fig4_maxent} shows example predicted protein distributions
reconstructed using six moments of the protein distribution for a suite of
different biophysical parameters and environmental inducer concentrations. As
repressor-DNA binding affinity (columns in \fref{fig4_maxent}) and repressor
copy number (rows in \fref{fig4_maxent}) are varied, the responses to different
signals (i.e. inducer concentrations) overlap to varying degrees. For
example the upper right corner frame with a weak binding site ($\eR = -9.7 \;
k_BT$) and a low repressor copy number (22 repressors per cell) has virtually
identical distributions regardless of the input inducer concentration. This
means that cells with this set of parameters cannot resolve a difference in any
concentration of the signal. As the number of repressors is increased along the
rows, the degree of overlap between distributions changes, allowing cells to
better resolve the value of the signal input. On the opposite extreme the lower
left panel shows a strong binding site ($\eR = -15.3 \; k_BT$) and a high
repressor copy number (1740 repressors per cell) shows overlap between
distributions due to the lack of ability of the system to respond to the inducer
given the high degree of repression, giving again little ability for the cells
to resolve the inputs. In the following section we formalize the notion of how
well cells can resolve different inputs from an information theoretic
perspective via the channel capacity.

\begin{figure}[h!]
	\centering \includegraphics
  {./fig/main/PMF_grid_joyplot_protein.pdf}
	\caption{\textbf{MaxEnt protein distributions for varying physical
	parameters.} Predicted protein distributions under different inducer (IPTG)
	concentrations for different combinations of repressor-DNA affinities
	(columns) and repressor copy numbers (rows). The first six moments of the
	protein distribution used to constrain the MaxEnt approximation were computed
	by integrating \eref{eq_gral_mom} as cells progressed through the cell cycle
	as described in \secref{sec_cell_cycle}.}
  \label{fig4_maxent}
\end{figure}

	\subsection{Theoretical prediction of the channel capacity}
\label{sec_channcap}

As a useful measure of the ability of the genetic circuit to allow the cell to
infer the environmental state, i.e. the inducer concentration, we turn to the
channel capacity. The channel capacity is defined as the mutual information
between input and output, maximized over all possible input distributions.
Putting this into mathematical terms we define $c$ as the inducer concentration.
$P(c)$ represents the distribution of inducer and $P(p \mid c)$ the distribution
of protein counts given a fixed inducer concentration - effectively the
distributions shown in \fref{fig4_maxent}. The channel capacity is then given by
\begin{equation}
  C \equiv \max_{P(c)} I(p; c),
  \label{eq_chann_cap}
\end{equation}
where $I(p; c)$, the mutual information between protein count and inducer
concentration is given by \eref{eq_mutual_info}.

If used as a metric of how reliably a signaling system can infer the state of
the external signal, the channel capacity, when measured in bits, is commonly
interpreted as the logarithm of the number of states that the signaling system
can properly resolve. For example, a signaling system with a channel capacity of
$C$ bits is interpreted as being able to resolve $2^C$ states, though channel
capacities with fractional values are allowed. As a result, we prefer the
Bayesian interpretation that the mutual information, and as a consequence the
channel capacity, quantifies the improvement in the inference of the input when
considering the output compared to just using the prior distribution of the
input by itself for prediction \cite{Voliotis2014a, Bowsher2014}. Under this
interpretation a channel capacity of a fractional bit still quantifies an
improvement of the ability of the signaling system to infer the value of the
extracellular signal compared to having no sensing system at all.

Computing the channel capacity as defined in \eref{eq_chann_cap} implies
optimizing over an infinite space of possible distributions $P(c)$. For special
cases in which the noise is small compared to the dynamic range, approximate
analytical equations have been derived \cite{Tkacik2008a}. But given the high
cell-to-cell variability that our model predicts, the conditions of the
so-called small noise approximation are not satisfied. We therefore appeal to a
numerical solution known as the Blahut-Arimoto algorithm \cite{Blahut1972}. This
algorithm, starting on any (discrete) distribution $P(c)$, converges to the
distribution at channel capacity. \fref{fig5_channcap}(A) shows zero-parameter
fit predictions of the channel capacity as a function of the number of
repressors for different repressor-DNA affinities (solid lines). These
predictions are contrasted with experimental determinations of the channel
capacity as inferred from single-cell fluorescence intensity distributions taken
over 12 different concentrations of inducer. Briefly, from single-cell
fluorescent measurements we can approximate the input-output distribution $P(p
\mid c)$. Once these conditional distributions are fixed, the task of finding
the input distribution at channel capacity become a computational minimization
routine apt for gradient descent or similar algorithms. For the particular case
of the channel capacity on a system with a discrete number of inputs and
outputs the Blahut-Arimoto algorithm is built in such a way that it guarantees
the convergence towards the optimal input distribution (See
\siref{supp_channcap} for further details). \fref{fig5_channcap}(B) shows
example input-output functions for different values of the channel capacity.
This illustrates that having access to no information (zero channel capacity) is
a consequence of having overlapping input-output functions (lower panel). On the
other hand, the more separated the input-output distributions are (upper panel)
the higher the channel capacity can be.

\fref{fig5_channcap}(A) has interesting features that are worth highlighting. On
one extreme for cells with no transcription factors there is no information
processing potential as this simple genetic circuit would be constitutively
expressed regardless of the environmental state. As cells increase the
transcription factor copy number, the channel capacity increases until it
reaches a maximum to then fall back down at high repressor copy number since the
promoter would be permanently repressed. The steepness of the increment in
channel capacity as well as the height of the maximum expression highly depend
on the repressor-DNA affinity. For strong binding sites (blue curve in
\fref{fig5_channcap}(A)) there is a rapid increment in the channel capacity, but
the maximum value reached is smaller compared to a weaker binding site (orange
curve in \fref{fig5_channcap}(A)).

\begin{figure}[h!]
	\centering \includegraphics
  {./fig/main/fig5_channcap.pdf}
	\caption{\textbf{Comparison of theoretical and experimental channel capacity.}
	(A) Channel capacity as inferred using the Blahut-Arimoto algorithm
	\cite{Blahut1972} for varying number of repressors and repressor-DNA
	affinities. All inferences were performed using 12 IPTG concentrations as
	detailed in the Methods. Lines represent zero-parameter fit predictions done
	with the maximum entropy distributions as those shown in \fref{fig4_maxent}.
	Points represent inferences made from single cell fluorescence distributions
	(See \siref{supp_channcap} for further details). Solid lines indicate plot in
	logarithmic scale, while dashed line indicates linear scale. (B) Example
  input-output functions of opposite limits of channel capacity. Lower panel
  illustrates that zero channel capacity indicates that all distributions
  overlap. Upper panel illustrates that as the channel capacity increases, the
  separation between distributions increases as well.}
  \label{fig5_channcap}
\end{figure}

	% \section{Three-state promoter model for simple repression}

One of the simplest and most common regulatory architectures in \textit{E. coli}
is the so-called simple repression motif \cite{Rydenfelt2014}. This consists of
a single binding site for the RNA polymerase (RNAP) and another binding site for
a transcriptional repressor \mrm{See Fig. XX}. We imagine that once the
repressor is bound to the promoter, it occludes the RNAP from binding,
effectively decreasing the transcriptional activity of the promoter.

In order to tackle the question of how to compute the full joint distribution of
mRNA and protein $P(m, p)$ we use the chemical master equation formalism.
Specifically we assume a three-state model where the promoter can be found 1)
with RNAP bound ($P$ state), 2) empty ($E$ state) and 3) with the repressor
bound ($R$ state) \mrm{See Fig. XX}. These three states generate a system of
three coupled partial differential equations for each of the three state
distributions $P_P(m, p)$, $P_E(m, p)$ and $P_R(m, p)$. Given the rates shown in
\mrm{Fig. XX} let us define the system of PDEs. For the RNAP bound state we have
\begin{equation}
  \begin{aligned}
    \dt{P_P(m, p)} &=
    - \overbrace{\kpoff P_P(m, p)}^{P \rightarrow E} % P -> E
    + \overbrace{\kpon P_E(m, p)}^{E \rightarrow P}\\ % E -> P
    &+ \overbrace{r_m P_p(m-1, p)}^{m-1 \rightarrow m} % m-1 -> m
    - \overbrace{r_m P_p(m, p)}^{m \rightarrow m+1}% m -> m+1
    + \overbrace{\gm (m + 1) P_P(m+1 , p)}^{m+1 \rightarrow m} % m+1 -> m
    - \overbrace{\gm m P_P(m , p)}^{m \rightarrow m-1}\\ % m -> m-1
    &+ \overbrace{r_p m P_P(m, p - 1)}^{p-1 \rightarrow p} % p-1 -> p
    - \overbrace{r_p m P_P(m, p)}^{p \rightarrow p+1} % p -> p+1
    + \overbrace{\gp (p + 1) P_P(m, p + 1)}^{p + 1 \rightarrow p} % p+1 -> p
    - \overbrace{\gp p P_P(m, p)}^{p \rightarrow p-1}. % p -> p-1
  \end{aligned}
\end{equation}
For the empty state $E$ we have
\begin{equation}
  \begin{aligned}
    \dt{P_E(m, p)} &=
    \overbrace{\kpoff P_P(m, p)}^{P \rightarrow E} % P -> E
    - \overbrace{\kpon P_E(m, p)}^{E \rightarrow P} % E -> P
    + \overbrace{\kroff P_R(m, p)}^{R \rightarrow E} % R -> E
    - \overbrace{\kron P_E(m, p)}^{E \rightarrow R}\\ % E -> R
    &+ \overbrace{\gm (m + 1) P_E(m+1 , p)}^{m+1 \rightarrow m} % m+1 -> m
    - \overbrace{\gm m P_E(m , p)}^{m \rightarrow m-1}\\ % m -> m-1
    &+ \overbrace{r_p m P_E(m, p - 1)}^{p-1 \rightarrow p} % p-1 -> p
    - \overbrace{r_p m P_E(m, p)}^{p \rightarrow p+1} % p -> p+1
    + \overbrace{\gp (p + 1) P_E(m, p + 1)}^{p + 1 \rightarrow p} % p+1 -> p
    - \overbrace{\gp p P_E(m, p)}^{p \rightarrow p-1}. % p -> p-1
  \end{aligned}
\end{equation}
And finally for the represor bound state $R$ we have
\begin{equation}
  \begin{aligned}
    \dt{P_R(m, p)} &=
    - \overbrace{\kroff P_R(m, p)}^{R \rightarrow E} % R -> E
    + \overbrace{\kron P_E(m, p)}^{E \rightarrow R}\\ % E -> R
    &+ \overbrace{\gm (m + 1) P_R(m+1 , p)}^{m+1 \rightarrow m} % m+1 -> m
    - \overbrace{\gm m P_R(m , p)}^{m \rightarrow m-1}\\ % m -> m-1
    &+ \overbrace{r_p m P_R(m, p - 1)}^{p-1 \rightarrow p} % p-1 -> p
    - \overbrace{r_p m P_R(m, p)}^{p \rightarrow p+1} % p -> p+1
    + \overbrace{\gp (p + 1) P_R(m, p + 1)}^{p + 1 \rightarrow p} % p+1 -> p
    - \overbrace{\gp p P_R(m, p)}^{p \rightarrow p-1}. % p -> p-1
  \end{aligned}
\end{equation}

It is convenient to express this system using matrix notation. For this we
define $\PP(m, p) = (P_P(m, p), P_E(m, p), P_R(m, p))$. Then the system of PDEs
can be expressed as
\begin{equation}
  \begin{aligned}
    \dt{\PP(m, p)} &= \Km \PP(m, p)
    - \Rm \PP(m, p) + \Rm \PP(m-1, p)
    - m \Gm \PP(m, p) + (m + 1) \Gm \PP(m + 1, p)\\
    &- m \Rp \PP(m, p) + m \Rp \PP(m, p)
    - p \Gp \PP(m, p) + (p + 1) \Gp \PP(m, p + 1),
  \end{aligned}
\end{equation}
where we defined the following matrices: The promoter state transition matrix
$\Km$
\begin{align}
  \Km \equiv
  \begin{bmatrix}
    -\kpoff   & \kpon         & 0\\
    \kpoff    & -\kpon -\kron  & \kroff\\
    0         & \kron         & -\kroff
  \end{bmatrix},
\end{align}
The mRNA production $\Rm$ and degradation $\Gm$ matrices
\begin{equation}
  \Rm \equiv
  \begin{bmatrix}
    r_m   & 0 & 0\\
    0     & 0 & 0\\
    0     & 0 & 0\\
  \end{bmatrix},
\end{equation}
and
\begin{equation}
  \Gm \equiv
  \begin{bmatrix}
    \gm   & 0   & 0\\
    0     & \gm & 0\\
    0     & 0   & \gm\\
  \end{bmatrix}.
\end{equation}
For the protein we also define a production $\Rp$ and degradation $\Gp$ matrices
as
\begin{equation}
  \Rp \equiv
  \begin{bmatrix}
    r_m   & 0   & 0\\
    0     & r_m & 0\\
    0     & 0   & r_m\\
  \end{bmatrix},
\end{equation}
and
\begin{equation}
  \Gp \equiv
  \begin{bmatrix}
    \gp   & 0   & 0\\
    0     & \gp & 0\\
    0     & 0   & \gp\\
  \end{bmatrix}.
\end{equation}

\section{Parameter inference}

With the objective of generating falsifiable predictions with meaningful
parameters we infer the kinetic rates from this three-state model using
different data sets generated over the last decade concerning different aspects
of the regulation of this simple genetic circuit. The path used to
systematically find parameter values was constrained by the nature of the
theoretical and experimental relevance of each of the available data sets. For
example, for the RNAP rates $\kpon$ and $\kpoff$ and the mRNA production rate
$r_m$ we used single-molecule mRNA FISH counts from an unregulated promoter
\cite{Jones2014a}. Once these parameters are fixed, we use these values to
constraint the repressor rates $\kron$ and $\kroff$. These repressor rates are
obtained using information from mean gene expression measuremnts from bulk LacZ
colorimetric assays \cite{Garcia2011c}, and single molecule microscopy
\cite{Elf2007}. We also expand our model to include the allosteric nature of the
repressor protein, taking advantage of video microscopy measurements done in the
context of multiple promoter copies \cite{Brewster2014} and flow-cytometry
measurements of the mean response of the system to different levels of induction
\cite{Razo-Mejia2018}.

\subsection{RNAP rates from unregulated two-state promoter}

We begin our parameter inference problem with the RNAP rates $\kpon$ and
$\kpoff$, as well as the mRNA production rate $r_m$. In this case there
are only two states  available to the promoter -- the empty state $E$ and the
RNAP bound state $P$. That means that the third PDE for $P_R(m)$ is removed from
the system. This particular two-state promoter system at this mRNA level has
been analytically solved by Peccoud and Ycart \cite{Peccoud1995}. The steady
state mRNA distribution $P(m) \equiv P_E(m) + P_P(m)$ is of the form
\begin{equation}
  P(m) = {\Gamma \left( \frac{\kpon}{\gm} + m \right) \over
  \Gamma(m + 1) \Gamma\left( \frac{\kpoff+\kpon}{\gm} + m \right)}
  {\Gamma\left( \frac{\kpoff+\kpon}{\gm} \right) \over
  \Gamma\left( \frac{\kpon}{\gm} \right)}
  \left( {r_m \over \gm} \right)^m
  F_1^1 \left( {\kpon \over \gm} + m,
  {\kpoff + \kpon \over \gm} + m,
  -{r_m \over \gm} \right),
  \label{eq_two_state_mRNA}
\end{equation}
where $\Gamma(\cdot)$ is the gamma function, and $F_1^1$ is the confluent
hypergeometric function of the first kind. This rather convoluted expression
will aid us to find parameter values for the rates. The inferred rates $\kpon$,
$\kpoff$ and $r_m$ are expressed in units of the mRNA degradation rate $\gm$.
This is because the model in \eref{eq_two_state_mRNA} is homogeneous in time,
meaning that if one divided all rates by a constant it would be equivalent to
multiplying the time scale of the problem by the same constant.

\subsubsection{Bayesian parameter inference of RNAP rates}

In order to make progress at inferring these parameters from experimental data
we use the single-molecule mRNA FISH data from Jones et al. \cite{Jones2014a}.
\fref{fig_lacUV5_FISH} shows the mRNA per cell distribution for the
\textit{lacUV5} promoter. This promoter, being rather strong has a mean copy
number of $\ee{m} \approx 18$ mRNA/cell.

\begin{figure}[h!]
	\centering \includegraphics[width=0.5\columnwidth]
  {../fig/chemical_master_mRNA_FISH/lacUV5_smFISH_data.pdf}
	\caption{\textbf{\textit{lacUV5} mRNA per cell distribution.} Data from
	\cite{Jones2014a} of the unregulated \textit{lacUV5} promoter as inferred
	from single molecule mRNA FISH.}
  \label{fig_lacUV5_FISH}
\end{figure}

Having this data in hand we now use Bayesian parameter inference to infer the
parameter values of our rates. Writing Bayes theorem we have
\begin{equation}
  P(\kpon, \kpoff, r_m \mid D) = {P(D \mid \kpon, \kpoff, r_m)
  P(\kpon, \kpoff, r_m) \over P(D)},
  \label{eq_bayes_rnap_rates}
\end{equation}
where $D$ represents our data. In this case our data is conformed by single-cell
mRNA counts $D = \{ m_1, m_2, \ldots, m_N \}$, where $N$ is the number of cells.
We assume that each cell is independent of each other such that we can rewrite
\eref{eq_bayes_rnap_rates} as
\begin{equation}
  P(\kpon, \kpoff, r_m \mid \{m_i\}) \propto
  \prod_{i=1}^N P(m_i \mid \kpon, \kpoff, r_m)
  P(\kpon, \kpoff, r_m).
  \label{eq_bayes_sample}
\end{equation}
Where the likelihood term $P(m_i \mid \kpon, \kpoff, r_m)$ is exactly given by
\eref{eq_two_state_mRNA} with $\gm = 1$.

\paragraph{Constraining the rates given prior thermodynamic knowledge.}

One of the advantages of Bayesian analysis is that we can include all the prior
knowledge on the parameters when inferring the rates. Basic features such as the
fact that the rates have to be strictly positive will constraint the values that
these parameters can take. In this particular case we know more than the
simple constraint of non-negative values. The expression of an unregulated
promoter has been studied from a thermodynamic perspective \cite{Brewster2012}.
Since these equilibrium models work in the thermodynamic limit of large particle
number they are not useful to inform us about large deviations from the expected
value. Nevertheless at the mean value both, the kinetic language and the
equilibrium language must agree. That means that we can use what we know about
the mean gene expression, and how this is related to parameters such as molecule
copy numbers and binding affinities, to constraint the values that these rates
can take.

In the case of this two-state promoter it can be shown that the mean number of
mRNA is given by \cite{Phillips2015}
\begin{equation}
  \ee{m} = {r_m \over \gm} {\kpon \over \kpon + \kpoff},
\end{equation}
which is basically ${r_m \over \gm} \times p_{\text{bound}}^{(p)}$, where
$p_{\text{bound}}^{(p)}$ is the probability of the RNAP being bound at the
promoter.

The thermodynamic picture has an equivalent result where the mean number
of mRNA is given by \cite{Brewster2012, Bintu2005a}
\begin{equation}
  \left\langle m \right\rangle = {r_m \over \gm}
  {{P \over N_{NS}} e^{-\beta\Delta\varepsilon_p} \over
  1 + {P \over N_{NS}} e^{-\beta\Delta\varepsilon_p}},
\end{equation}
where $P$ is the number of RNAP per cell, $N_{NS}$ is the number of non-specific
binding sites, $\Delta\varepsilon_p$ is the RNAP binding energy in $k_BT$ units
and $\beta\equiv {k_BT}^{-1}$ .

Using these two equations we can easily see that if these frameworks are to be
equivalent, then it must be true that
$$
{\kpon \over \kpoff} = {P \over N_{NS}} e^{-\beta\Delta\varepsilon_p},
$$
or
$$
\ln \left({\kpon \over \kpoff}\right) =
-\beta\Delta\varepsilon_p + \ln P - \ln N_{NS}.
$$

We know that the RNAP copy number is order $P \approx 1000-3000$ RNAP/cell for a
1 hour doubling time \cite{Klumpp2008}, we also know that $N_{NS} = 4.6\times
10^6$ \cite{Bintu2005a}, and $-\beta\Delta\varepsilon_p \approx 5 - 7$
\cite{Brewster2012}. Given these values we define a Gaussian prior for the ratio
of these two quantities of the form
$$
P(\kpon / \kpoff) \propto \exp
\left\{ - {\left(\ln \left({\kpon \over \kpoff}\right) -
\left(-\beta\Delta\varepsilon_p + \ln P - \ln N_{NS} \right) \right)^2
\over 2 \sigma^2} \right\},
$$
where $\sigma$ is the variance that accounts for the uncertainty on these
parameters. We include this prior as part of the prior term $P(\kpon, \kpoff,
r_m)$ of \eref{eq_bayes_sample}. We then use Markov Chain Monte Carlo to sample
out of the posterior distribution in \eref{eq_bayes_sample}.
\fref{fig_mcmc_rnap} shows the MCMC samples of the posterior distribution. We
see that for the case of the $\kpon$ parameter there is a single symmetric peak.
$\kpoff$ and $r_m$ have a rather long tail towards large values. As a matter of
fact, the 2D projection of $\kpoff$ vs $r_m$ shows that the model is sloppy,
meaning that the two parameters are highly correlated \cite{Transtrum2015}. What
this implies is that one could change the value of $\kpoff$, and then compensate
by a change on $r_m$ in order to maintain the shape of the mRNA distribution.
Having used the prior knowledge on the equilibrium picture of the RNAP binding
allowed us to get at a more constrained parameter value for these rates.

\begin{figure}[h!]
	\centering \includegraphics[width=0.5\columnwidth]
  {../fig/chemical_master_mRNA_FISH/lacUV5_mRNA_prior_corner_plot.pdf}
	\caption{\textbf{MCMC posterior distribution.} Sampling out of
	\eref{eq_bayes_sample} the plot shows 2D and 1D projections of the 3D
	parameter space. The parameter values are (in units of the mRNA degradation
	rate $\gm$) $\kpon = 4.4^{+0.8}_{-0.3}$, $\kpoff = 20.4^{+52.1}_{-8.4}$ and
	$r_m = 106.1^{+184.8}_{-31.2}$ which are the modes of their respective
	distributions, where the superscripts and subscripts represent the upper and
	lower bounds of the 95$^\text{th}$ percentile of the parameter value
  distributions}
  \label{fig_mcmc_rnap}
\end{figure}

The inferred values $\kpon = 4.4^{+0.8}_{-0.3}$, $\kpoff = 20.4^{+52.1}_{-8.4}$
and $r_m = 106.1^{+184.8}_{-31.2}$ are given in units of the mRNA degradation
rate $\gm$. Given the asymmetry of the parameter distributions we report the
upper and lower bound of the 95$^\text{th}$ percentile of these distributions.
Assuming a mean life-time for mRNA of $\approx$ 5 min (from this
\href{http://bionumbers.hms.harvard.edu/bionumber.aspx?&id=107514&ver=1&trm=mRNA%20mean%20lifetime}{link})
we have an mRNA degradation rate of $\gm \approx 2.84 \times 10^{-3} s^{-1}$.
Using this value gives the following values for the inferred rates: $\kpon =
0.012_{-0.001}^{+0.002} s^{-1}$, $\kpoff = {0.06}_ {-0.02}^{+0.15} s^{-1}$, and
$r_m = 0.3_{-0.09}^{+0.5} s^{-1}$. \mrm{This is where a statement should be
done with respect to known values in the literature}

\fref{fig_lacUV5_theory_data} shows the result of substituting these parameter
values onto \eref{eq_two_state_mRNA}. As we can see this two-state model fits
the data adequately.

\begin{figure}[h!]
	\centering \includegraphics[width=0.5\columnwidth]
  {../fig/chemical_master_mRNA_FISH/lacUV5_two_state_mcmc_fit.pdf}
	\caption{\textbf{Experimental vs. theoretical distribution of mRNA per cell
  using parameters from Bayesian inference.} Dotted line shows the result of
  using \eref{eq_two_state_mRNA} along with the parameters inferred for the
  rates. Blue bars are the same data as \fref{fig_lacUV5_FISH} from
  \cite{Jones2014a}}
  \label{fig_lacUV5_theory_data}
\end{figure}

\subsection{Accounting for variability in the number of promoters}

As discussed in ref. \cite{Jones2014a} and further expanded in
\cite{Peterson2015} an important source of noise in gene expression level in
bacteria is the fact that depending on the position relative to the chromosome
replication origin, and the growth rate, cells can have multiple copies of a
gene. Genes closer to the replication origin will have on average higher gene
copy number compare to genes at the opposite end. For the locus in which our
reporter construct is located (\textit{galK}) we expect to have $\approx$ 1.66
copies of the gene at a doubling time of 1 hour \mrm{this should cite Rob's
reference for the transcription book figure I got this from}. This implies that
the cells spend 2/3 of the cell cycle with two copies of the promoter and the
rest with a single copy.

To account for this variability in gene copy we extend the model to account for
this difference assuming that when cells have two copies of the promoter the
production rate is $2 r_m$ compared to the rate $r_m$ for a single copy. Then
the probability of observing certain mRNA copy $m$ is given by
\begin{equation}
  P(m) = f \cdot P(m \mid \text{one promoter}) +
  (1 - f) \cdot P(m \mid \text{two promoters}),
  \label{eq_prob_multipromoter}
\end{equation}
where $f = 1/3$ is the fraction of the cell cycle that cells spend with a single
copy of the promoter. Both terms $P(m \mid \text{promoter copy})$ are given by
\eref{eq_two_state_mRNA} witht the only difference being the rate $r_m$. It is
important to acknowledge that \eref{eq_prob_multipromoter} assumes that once the
cell replicates the promoter the time scale in which the mRNA count relaxes to
the new equilibrium is shorter than the time that the cells spend in this
two-promoter state. This approximation should be valid for a short lived mRNA
molecule, but the same will not be applicable for proteins whose degradation
rate is comparable to the cell cycle length.

In order to repeat the Bayesian inference assuming this model we need to split
our data in two sets -- cells with a single copy of the promoter and cells with
two copies of the promoter. Since for the single molecule mRNA FISH data there
is no labeling of the locus Jones et al. used area as a proxy for stage in the
cell cycle \cite{Jones2014a}. What that means is that by sorting cells by size
they considered the low 33th percentile as cells with a single promoter copy,
with  the rest being cells with two copies of the promoter. This approach
ignores that cells are not uniformly distributed along the cell cycle. As first
discussed in \cite{Powell1956} populations of cells in a log-phase are
exponentially distributed along the cell cycle. This distribution is of the form
\begin{equation}
P(a) = (\ln 2) \cdot 2^{1 - a},
\label{eq_cell_cycle_dist}
\end{equation}
where $a \in [0, 1]$ is the stage of the cell cycle, with $a = 0$ being the
start of the cycle and $a = 1$ being the division. \fref{fig_cell_area} shows
the separation of the two groups based on area where \eref{eq_cell_cycle_dist}
was used to weigh the distribution along the cell cycle.

\begin{figure}[h!]
	\centering \includegraphics[width=0.5\columnwidth]
  {../fig/chemical_master_mRNA_FISH/area_division_expo.pdf}
	\caption{\textbf{Separation of cells based on cell size.} Using the area as
  a proxy for state on the cell cycle, cells can be sorted into two groups --
  small cells (with one promoter copy) and large cells (with two promoter
  copies). The vertical black line delimits the threshold that divides both
  groups as weighted by \eref{eq_cell_cycle_dist}.}
  \label{fig_cell_area}
\end{figure}

To confirm that this sorting of cells by area gives a distinction in the mRNA
count \fref{fig_mRNA_by_size} shows the distribution of both groups. As expected
larger cells have a higher mRNA copy number on average.

\begin{figure}[h!]
	\centering \includegraphics[width=0.5\columnwidth]
  {../fig/chemical_master_mRNA_FISH/lacUV5_mRNA_size_PMF_CDF.pdf}
	\caption{\textbf{mRNA distribution for small and large cells.} (A)
  probability mass function and (B) cumulative distribution function of the
  small and large cells as determined in \fref{fig_cell_area}. The triangles
  above histograms in (A) indicate the mean mRNA copy number for each group.}
  \label{fig_mRNA_by_size}
\end{figure}

We modify \eref{eq_bayes_sample} to account for the two separate groups of
cells. Let $N_s$ be the number of cells in the small size group and $N_l$ the
number of cells in the large size group. Then the posterior distribution for the
parameters is of the form
\begin{equation}
  \small
P(\kpon, \kpoff, r_m \mid \{m_i\}) \propto
  \prod_{i=1}^{N_s} f \cdot P(m_i \mid \kpon, \kpoff, r_m)
  \prod_{j=1}^{N_l} (1 - f) \cdot P(m_j \mid \kpon, \kpoff, 2 \cdot r_m)
  P(\kpon, \kpoff, r_m).
  \label{eq_bayes_sample_double}
\end{equation}

Sampling \eref{eq_bayes_sample_double} with MCMC and using again the mRNA mean
lifetime of 350 seconds gives the following values for the parameters: $\kpon =
0.017_{-0.001}^{+0.002} s^{-1}$, $\kpoff = {0.24}_ {-0.11}^{+0.46} s^{-1}$, and
$r_m = 0.5_{-0.2}^{+0.8} s^{-1}$. \mrm{again need to compare with what is
known about these rates.}. \fref{fig_lacUV5_theory_data_double} shows the result
of applying \eref{eq_prob_multipromoter} with these parameters.

\begin{figure}[h!]
	\centering \includegraphics[width=0.5\columnwidth]
  {../fig/chemical_master_mRNA_FISH/lacUV5_two_state_mcmc_multi_copy.pdf}
	\caption{\textbf{Experimental vs. theoretical distribution of mRNA per cell
  using parameters for single and multi promoter model} Purple dotted curve
  shows the result of using \eref{eq_prob_multipromoter} witht the parameters
  inferred by sampling \eref{eq_bayes_sample_double}. For comparison orange
  dotted line shows the model from \fref{fig_lacUV5_theory_data}. Blue bars are
  the same data as \fref{fig_lacUV5_FISH} from \cite{Jones2014a}.}
  \label{fig_lacUV5_theory_data_double}
\end{figure}

\section{Repressor rates from three-state regulated promoter.}

Having determined the RNAP rates we now proceed to determine the repressor rates
$\kron$ and $\kroff$. The value of these rates is constrained by what we know
from equilibrium models \cite{Phillips2015}. For this we again exploit the
feature that only at the mean both, the kinetic language and the thermodynamic
language should have equivalent predictions. Over the last decade there has been
a lot of effort in developing equilibrium models for gene expression regulation
\cite{Buchler2003,Vilar2011,Bintu2005a}. In particular our group has extensively
characterized the simple repression motif using this formalism
\cite{Garcia2011c,Brewster2014,Razo-Mejia2018}.

The dialogue between theory and experiments has lead to simplified expressions
that capture the phenomenology of the gene expression response as a function of
natural variables such as molecule count and affinities between molecular
players. A particularly interesting quantity defined by Garcia \& Phillips
\cite{Garcia2011c} as the fold-change in gene expression is given by
\begin{equation}
  \foldchange = {\ee{\text{gene expression}(R > 0)} \over
                 \ee{\text{gene expression}(R = 0)}},
\end{equation}
where $R$ is the number of transcriptional repressors per cell. Basically the
fold-change is the mean expression level in the presence of the repressor
divided by the expression level in the absence of regulation. In the language of
statistical mechanics this quantity is of the form \cite{Garcia2011c}
\begin{equation}
  \foldchange = \left( 1 + {R \over \Nns} e^{-\beta\eR} \right)^{-1},
  \label{eq_fc_thermo}
\end{equation}
where $\Nns$ is the number of non-specific binding sites in the genome (taken
as the size of the \textit{E. coli} genome $4.6\times 10^6$), $\eR$ is the
repressor-DNA binding energy and $\beta \equiv {1 \over k_BT}$.

To compute the fold-change in the chemical master equation language we compute
the first moment of the steady sate mRNA distribution $\ee{m}$ for both, the
three-state promoter ($R>0$) and the two-state promoter case ($R=0$)
\mrm{See section XX for moment derivation}. The latter gives
\begin{equation}
  \ee{m (R = 0)} = {r_m \over \gm} {\kpon \over \kpon + \kpoff}.
\end{equation}
The three-state promoter has a steady-state mean mRNA copy number of the form
\begin{equation}
  \ee{m (R > 0)} = {r_m \over \gm} {\kroff\kron
  \over \kpoff\kroff + \kpoff\kron + \kroff\kpon}.
\end{equation}
Computing the fold-change then gives
\begin{equation}
  \foldchange = {\ee{m (R > 0)} \over \ee{m (R = 0)}} =
  {\kroff \left( \kpoff + \kpon \right) \over
  \kpoff\kron + \kroff \left( \kpoff + \kpon \right)}.
  \label{eq_fold_change_cme}
\end{equation}

Given that the number of repressors per cell $R$ is an experimental variable
that we can control, we assume that the rate at which the promoter transitions
form the empty state to the repressor bound state $\kron$ is given by the
concentration of repressors $[R]$ times a diffusion limited rate $k_o$
\cite{Jones2014a}.  For the diffusion limited constant $k_o$ we use the value
used by Jones et al. \cite{Jones2014a} \mrm{Find real reference for this value
that Brewster never gave me.}. With this in hand we can rewrite
\eref{eq_fold_change_cme} as
\begin{equation}
  \foldchange = \left( 1 + {k_0 [R] \over \kroff}
                {\kpon \over \kpon + \kpoff} \right)^{-1}.
  \label{eq_fc_kinetic}
\end{equation}

We note that both \eref{eq_fc_thermo} and \eref{eq_fc_kinetic} have the same
functional form. Therefore if both languages predict the same output for the
mean gene expression level, it must be true that
\begin{equation}
  {k_o [R] \over \kroff}{\kpon \over \kpon + \kpoff} =
  {R \over \Nns} e^{-\beta\eR}.
\end{equation}
Solving for $\kroff$ gives
\begin{equation}
  \kroff = {k_o [R] \Nns e^{\beta\eR} \over R}{\kpon \over \kpon + \kpoff}.
  \label{eq_kroff_complete}
\end{equation}

In order for the units to cancel properly the repressor concentration has to be
given in nM rather than absolute count. If we consider that the repressor
concentration is equal to
\begin{equation}
[R] = \frac{R}{V_{cell}}\cdot \frac{1}{Av},
\end{equation}
where $R$ is the absolute repressor copy number per cell, $V_{cell}$ is the cell
volume and $Av$ is Avogadro's number. The \textit{E. coli} cell volume is in the
order of 2.1 fL = $10^{-15}$ L \mrm{get reference from Nathan}, and Avogadro's
number is $6.022 \times 10^{23}$. If we further include the conversion factor to
turn M into nM we find that
\begin{equation}
[R] = {R \over 2.1 \times 10^{-15} L} \cdot {1 \over 6.022 \times 10^{23}}
\cdot {10^9 \text{ nmol} \over 1 \text{ mol}} \approx 1.66 \times R.
\end{equation}
Using this we simplify \eref{eq_kroff_complete} as
\begin{equation}
  \kroff = 0.8 \cdot k_o \cdot \Nns e^{\beta\eR}
   \cdot {\kpon \over \kpon + \kpoff}.
  \label{eq_kroff}
\end{equation}
What \eref{eq_kroff} shows is the direct relationship that must be true if the
equilibrium model must be self consistent with the non-equilibrium kinetic
picture.

Putting all these parameters together we can generate zero-parameter fit
predictions for the full mRNA and protein distributions.

	% \section{Computing moments from the master equation}

\subsection{Defining the chemical master equation for simple repression}

In this section we will compute the moment equations for the distribution  $P(m,
p; t)$. As defined in section \mrm{ref model section} the chemical master
equation for our simple genetic circuit written in matrix notation is of the
form
\begin{equation}
  \begin{aligned}
    \dt{\PP(m, p)} &= \Km \PP(m, p)
    - \Rm \PP(m, p) + \Rm \PP(m-1, p)
    - m \Gm \PP(m, p) + (m + 1) \Gm \PP(m + 1, p)\\
    &- m \Rp \PP(m, p) + m \Rp \PP(m, p)
    - p \Gp \PP(m, p) + (p + 1) \Gp \PP(m, p + 1),
  \end{aligned}
  \label{eq_cme_matrix}
\end{equation}
where $\PP(m, p)$ is a vector with two entries in the case of the unregulated
promoter
\begin{equation}
  \PP(m, p) = (P_P(m, p), P_E(m, p))^T,
\end{equation}
and a vector with three entries for the regulated promoter
\begin{equation}
  \PP(m, p) = (P_P(m, p), P_E(m, p), P_R(m, p))^T.
\end{equation}

Without lost of generality here we will focus on the three-state regulated
promoter. The computation of the moments for the two-state promoter follows the
exact same procedure, only that the definition of the matrices change.
For \eref{eq_cme_matrix} we have defined the following matrices:

The promoter state transition matrix
$\Km$
\begin{align}
  \Km \equiv
  \begin{bmatrix}
    -\kpoff   & \kpon         & 0\\
    \kpoff    & -\kpon -\kron  & \kroff\\
    0         & \kron         & -\kroff
  \end{bmatrix},
\end{align}
The mRNA production $\Rm$ and degradation $\Gm$ matrices
\begin{equation}
  \Rm \equiv
  \begin{bmatrix}
    r_m   & 0 & 0\\
    0     & 0 & 0\\
    0     & 0 & 0\\
  \end{bmatrix},
\end{equation}
and
\begin{equation}
  \Gm \equiv
  \begin{bmatrix}
    \gm   & 0   & 0\\
    0     & \gm & 0\\
    0     & 0   & \gm\\
  \end{bmatrix}.
\end{equation}
For the protein we also define a production $\Rp$ and degradation $\Gp$ matrices
as
\begin{equation}
  \Rp \equiv
  \begin{bmatrix}
    r_m   & 0   & 0\\
    0     & r_m & 0\\
    0     & 0   & r_m\\
  \end{bmatrix},
\end{equation}
and
\begin{equation}
  \Gp \equiv
  \begin{bmatrix}
    \gp   & 0   & 0\\
    0     & \gp & 0\\
    0     & 0   & \gp\\
  \end{bmatrix}.
\end{equation}

\subsection{Computing a distribution moment}

To compute any moment of our chemical master equation let us define a vector
\begin{equation}
  \ee{\bb{m^x p^y}} \equiv (\ee{m^x p^y}_E, \ee{m^x p^y}_P, \ee{m^x p^y}_R)^T,
\end{equation}
where $\ee{m^x p^y}_S$ is the expected value of $m^x p^y$ in state $S \in
\{E, P, R\}$. In other words, just as we defined the vector $\PP(m, x)$, here
we define a vector to collect the expected value of each of the promoter states.

	% \label{sec_multi_gene}
\section{Accounting for the variability in gene copy number during the cell
cycle}

When growing in rich media, bacteria can double every $\approx$ 20 minutes. With
a replication fork that travels at $\approx$ 1000 bp per second, and a genome of
$\approx$ 5 Mbp for {\it E. coli}\cite{Moran2010}, a cell would need $\approx$
80 minutes to replicate its genome. The apparent paradox is solved by the fact
that cells have multiple replisomes, i.e. molecular machines that replicate the
genome running in parallel. Cells can have up to 8 copies of the genome being
replicated at all time depending on the growth rate \cite{Bremer1996}.

That means that during the cell cycle gene copy number varies. This variation
depends on the relative position of the gene with respect to the replication
origin, having genes close to the origin spending more time with multiple copies
compare to genes closer to the termination. This change in gene dosage has a
direct effect on the cell-to-cell variability in gene expression
\cite{Jones2014a, Peterson2015}. Furthermore, since the time to reach steady
state is determined by the degradation rate, mRNA and protein will experience
experience these changes in gene copy number differently from each other.

\subsection{Numerical integration of moment equations}

For our specific locus ({\it galK}) and a doubling time of $\approx$ 100 min,
cells have on average 1.4 copies of the reporter gene during the cell cycle.
What this means is that cells spend 60\% of the time having one copy of the gene
and 40\% of the time with two copies. Our model needs to account for this
variability in gene copy number along the cell cycle. To do so we numerically
integrate the moment equations derived in \mrm{reference moment derivation
section} for a time $t = [0, t_s]$ with an mRNA production rate $r_m$, where
$t_s$ is the time point at which the replication fork reaches our specific
locus. For the remaining time before the cell division $t = [t_s, t_d]$ that the
cell spends with two promoters, we assume that the only parameter that changes
is the mRNA production rate from $r_m$ to $2 r_m$. This simplifying assumption
ignores potential changes in protein translation rate $r_p$ or changes in the
repressor copy number that would be reflected in changes on the repressor on
rate $\kron$.

\subsubsection{Initial conditions for numerical integration}

In order to define the initial conditions for the numerical integration for each
of the cell cycle stages we follow a simple procedure:
\begin{enumerate}
  \item Initialize the zeroth moment, i.e. the probability of the promoter of
  being on each of the two (unregulated) or three (regulated) states at any
  value, constrained to the fact that the sum of all zeroth moments should add
  up to one.
  \item Integrate moment equations for a long time using parameters
  corresponding to a single promoter. This will set the initial conditions to
  be used for the single promoter case
  \item Integrate the moment equations for a time $t = [0, t_s]$ using the
  initial conditions determined in the previous step.
  \item Update parameters to the ones corresponding to two promoter copies.
  \item Integrate the moment equations for a time $t = [t_s, t_d]$ using the
  last time point of the single promoter time as initial conditions.
  \item Divide the mRNA first moment $\ee{\bb{m}}$ and the protein first moment
  $\ee{\bb{p}}$ by two to represent the cell division.
  \item Integrate the moment equations for a long time with the constraint that
  the zeroth moment, the mRNA and protein first moments remain fixed as defined
  by halving the values from the previous step.
  \item Repeat process from step 3, using as initial conditions the last point
  of the integration in step 7.
\end{enumerate}

Step 7 in our procedure to determine initial conditions serves as a way to
determine how higher moments change as the cells divide. Only the first moments
of both mRNA and protein can simply be halved after cell division, but it is
less clear how higher moments of the distribution change as cells half their
content. To visualize why this is not a trivial step let's consider a binomially
distributed variable $X \sim \text{Bin}(N, p)$. For the first moment, given by
$\ee{X} = Np$ is easy to see that upon halving the number of trials $N$, the new
first moment would simply be $\ee{X}_{\text{new}} = {N \over 2} p$. But for the
second moment given by $\ee{X^2} = Np - Np^2 + N^2p^2$ it is obvious that
halving the number of attempts does not translate to halving the second moment.
Given this non-linear transformation between moments, we concluded that the
easiest way to converge to the value of higher moments after cell division was
to simply integrate the moment equations keeping the first moments fixed until
higher moments relaxed to a steady state given this constrained conditions.

\fref{fig_first_mom_cycles} shows how the first moment of both mRNA and protein
changes over several cell cycles. The mRNA seems to quickly relax to the steady
state corresponding to the parameters for both a single and two promoter copies.
This is expected since the parameters for the mRNA production were determined
in the first place under this assumption (See \mrm{cite mRNA MCMC section}). On
the other hand given that this relaxation time is determined by the degradation
rate the protein doesn't reach such steady state for either case. Interestingly
Once a couple of cell cycles have passed the cells seem to have a reproducible
trajectory over cell cycles. We call this a dynamic steady-state for the
protein.

\begin{figure}[h!]
	\centering \includegraphics
  {../fig/moment_dynamics_numeric/first_mom_cycles.pdf}
	\caption{\textbf{First moment dynamics over cell the cell cycle.}(A) mean
	mRNA and (B) mean protein copy number as the cell cycle progresses. The light
	shaded region delimits the fraction of the cell cycle that cells spend with  a
	single copy of the promoter. The dark shaded region delimits the fraction of
	the cell cycle that cells spend with two copies of the promoter. For a 100
  min doubling time at the {\it galK} locus cells spend 60\% of the time with
  one copy of the promoter and the rest with two copies.}
  \label{fig_first_mom_cycles}
\end{figure}

Since our experiments were not performed with synchronized cells, in principle
we sampled cells over the entirety of the cell cycle, so the moments that we
determined experimentally correspond to an average over the cell cycle.  In
order to compute these averages in the following section we discuss how to
account for the fact that cells are not uniformly distributed along the cell
cycle

\subsection{Exponentially distributed ages}

As first mentioned in Section \mrm{ref to mRNA MCMC section}, cells in a log
phase have exponentially distributed ages along the cell cycle, having more
young cells compared to old ones. Specifically the probability of a cell of
being at any time point in the cell cycle is given by \cite{Powell1956}
\begin{equation}
  P(a) = (\ln 2) \cdot 2^{1 - a},
  \label{eq_age_prob}
\end{equation}
where $a \in [0, 1]$ is the stage of the cell cycle, with $a = 0$ being the
start of the cycle and $a = 1$ being the division.

Our numerical integration of the moment equations gave us a time evolution of
the moments along the cell cycle. Without loss of generality let's focus on the
first mRNA moment $\ee{m(t)}$ (the same can be applied to all other moments).
In order to calculate the first moment along the entire cell cycle we must
average each time point by the corresponding probability that a cell is found
in such time point. This translates to computing the integral
\begin{equation}
  \ee{m} = \int_{\text{beginning cell cycle}}^{\text{end cell cycle}}
                       \ee{m(t)} P(t) dt.
\end{equation}

If we map each time point in the cell cycle into a fraction we can use
\eref{eq_age_prob} and compute instead
\begin{equation}
  \ee{m} = \int_0^1 \ee{m(a)} P(a) da.
  \label{eq_moment_avg}
\end{equation}
We perform this integral numerically for all moments using Simpson's rule.

	% \section{Maximum entropy approximation of distributions}\label{supp_maxent}

On the one hand the solution of chemical master equations like the one proposed
in \mrm{ref model section} represent a had mathematical challenge. As presented
in \mrm{ref FISH  mcmc section} Peccoud \& Ycart derived a closed-form solution
for the two-state promoter at the mRNA level \cite{Peccoud1995}. In an
impressive display of mathematical skills Shahrezaei \& Swain were able to
derive an approximate solution for the one- (not considered in this work) and
two-state promoter master equation at the protein level \cite{Shahrezaei2008}.
Nevertheless both of these solutions do not give instantaneous insights about
the distributions as they involve complicated terms such as confluent
hypergeometric functions.

On the other hand there has been a great deal of work to generate methods that
can approximate the solution of these discrete states Markovian models
\cite{Ale2013, Andreychenko2017, Frohlich2016, Schnoerr2017, Smadbeck2013,
Hasenauer2014, Dinh2016}. In particular for master equations like the one that
concerns us here whose moments can be easily computed, the moment expansion
method provides a simple method to approximate the full joint distribution of
mRNA and protein \cite{Smadbeck2013}. In this section we will explain the
principles behind this method and show the implementation for our particular
study case.

\subsection{The MaxEnt principle}

The principle of maximum entropy (MaxEnt) first proposed by E. T. Jaynes in 1957
tackles the question of given limited information what is the least biased
inference one can make about a particular probability distribution
\cite{Jaynes1957}. In particular Jaynes used this principle to show the
correspondence between statistical mechanics and information theory,
demonstrating for example that the Boltzmann distribution is the probability
distribution that maximizes Shannon's entropy subject to a constraint that the
average energy of the system is fixed.

To illustrate the principle let us focus on a univariate distribution $P_X(x)$.
The $n^{\text{th}}$ moment of the distribution for a discrete set of possible
values of $x$ is given by
\begin{equation}
  \ee{x^n} \equiv \sum_x x^n P_X(x).
  \label{eq_mom_ref}
\end{equation}

Now assume that we have knowledge of the first $m$ moments $\bb{\ee{x}}_m = (
\ee{x}, \ee{x^2}, \ldots, \ee{x^m} )$. The question is then how can we use this
information to build an estimator $P_H(x \mid \bb{\ee{x}}_m)$ of the
distribution
such that
\begin{equation}
  \lim_{m \rightarrow \infty} P_H(x \mid \bb{\ee{x}}_m) \rightarrow P_X(x),
\end{equation}
i.e. that the more moments we add to our approximation, the more the estimator
distribution converges to the real distribution.

The MaxEnt principle tells us that our best guess for this estimator is to build
it on the base of maximizing the Shannon entropy, constrained by the information
we have about these $m$ moments. The maximization of Shannon's entropy
guarantees that we are the least committed possible to information that we do
not posses. The Shannon entropy for an univariate discrete distribution is
given by
\begin{equation}
  H(x) \equiv - \sum_x P_X(x) \log P_X(x).
\end{equation}

For an optimization problem subject to constraints we make use of the method of
the Lagrange multipliers. For this we define the Lagrangian $\mathcal{L}(x)$ as
\begin{equation}
  \mathcal{L}(x) \equiv H(x) - \sum_{i=0}^m
  \left[ \lambda_i \left( \ee{x^i} - \sum_x x^i P_X(x) \right) \right],
\end{equation}
where $\lambda_i$ is the Lagrange multiplier associated with the $i^\th$
moment. The inclusion of the zeroth moment is an additional constraint to
guarantee the normalization of the resulting distribution.

Since $P_X(x)$ has a finite set of discrete values if we take the derivative of
the Lagrangian with respect to $P_X(x)$ what this implies is that we chose a
particular value of $X = x$. Therefore from the sum over all possible $x$ values
only a single term survives. With this in mind we take the derivative of the
Lagrangian obtaining
\begin{equation}
  {d\mathcal{L} \over d P_X(x)} = -\log P_X(x) - 1 -
  \sum_{i=0}^m \lambda_i x^i.
\end{equation}

Equating this derivative to zero and solving for the distribution (that we now
start calling $P_H(x)$, our MaxEnt estimator) gives
\begin{equation}
  P_H(x) = \exp \left(- 1 - \sum_{i=0}^m \lambda_i x^i \right)
         ={1 \over \mathcal{Z}}
         \exp \left( - \sum_{i=1}^m \lambda_i x^i \right),
  \label{eq_maxEnt}
\end{equation}
where $\mathcal{Z}$ is the normalization constant that can be obtained by
substituting this solution into the normalization constraint. This results in
\begin{equation}
  \mathcal{Z} \equiv \exp\left( 1 + \lambda_0 \right) =
  \sum_x \exp \left( - \sum_{i=1}^m \lambda_i x^i \right).
\end{equation}

\eref{eq_maxEnt} is the general form of the MaxEnt distribution for a univariate
distribution. The computational challenge then consists in finding numerical
values for the Lagrange multipliers $\{ \lambda_i \}$ such that $P_H(x)$
satisfies our constraints. In other words, the Lagrange multipliers weight the
contribution of each term in the exponent such that when computing any of the
moments we recover the value of our constraint. Mathematically what this means
is that $P_H(x)$ must satisfy
\begin{equation}
  \sum_x x^n P_H(x) =
  \sum_x {x^n \over \mathcal{Z}}
  \exp \left( - \sum_{i=1}^m \lambda_i x^i \right) = \ee{x^n}.
\end{equation}

As an example of how to apply the MaxEnt principle let us use the classic
problem of a six-face die. If we are only told that after a large number of die
rolls the mean value of the face is $\ee{x} = 4.5$ (note that a fair die has a
mean of $3.5$), what would the least biased guess fot the distribution look
like? The MaxEnt principle tells us that our best guess would be of the form
\begin{equation}
  P_H(x) = {1 \over \mathcal{Z}} \exp \left( \lambda x \right).
\end{equation}
Using any numerical minimization package we can easily find the numerical value
of the Lagrange multiplier $\lambda$  that satisfies our constraint.
\fref{fig_maxent_die} shows two two examples of distributions that satisfy the
constraint. Panel (A) shows a distribution consistent with the 4.5 average where
both 4 and 5 are equally likely. Nevertheless in the information we got about
the nature of the die it was never stated that some of the faces were forbidden.
In that sense the distribution is committing to information about the process
that we do not posses. Panel (B) by contrast shows the MaxEnt distribution that
satisfies this constraint. Since this distribution maximizes Shannon's entropy
it is guaranteed to be the least biased distribution given the available
information.

\begin{figure}[h!]
	\centering \includegraphics
  {../fig/MaxEnt_approx_joint/biased_die_dist.pdf}
	\caption{\textbf{Maximum entropy distribution of six-face die.} (A)biased
  distribution consistent with the constraint $\ee{x} = 4.5$. (B) MaxEnt
  distribution also consistent with the constraint.}
  \label{fig_maxent_die}
\end{figure}

\subsubsection{The mRNA and protein joint distribution}

The MaxEnt principle can easily be extended to multivariate distributions. For
our particular case we are interested in the mRNA and protein joint distribution
$P(m, p)$. The definition of a moment $\ee{m^x p^y}$ is a natural extension of
\eref{eq_mom_ref} of the form
\begin{equation}
  \ee{m^x p^y} = \sum_m \sum_p m^x p^y P(m, p).
\end{equation}

As a consequence the MaxEnt joint distribution $P_H(m, p)$ is of the form
\begin{equation}
  P_H(m, p) = {1 \over \mathcal{Z}}
              \exp \left( - \sum_{(x,y)} \lambda_{(x,y)} m^x p^y \right),
  \label{eq_maxEnt_joint}
\end{equation}
where $\lambda_{(x,y)}$ is the Lagrange multiplier associated with the moment
$\ee{m^x p^y}$, and again $\mathcal{Z}$ is the normalization constant given by
\begin{equation}
  \mathcal{Z} = \sum_m \sum_p
              \exp \left( - \sum_{(x, y)} \lambda_{(x, y)} m^x p^y \right).
\end{equation}
Note that the sum in the exponent is taken over all available $(x, y)$ pairs
that define the moment constraints for the distribution.

\subsection{The Bretthorst rescaling algorithm}

The determination of the Lagrange multipliers suffer from a numerical under and
overflow problem due to the difference in magnitude between the constraints.
This becomes a problem when higher moments are taken into account. The resulting
numerical values for the Lagrange multipliers end up being separated by several
orders of magnitude. For routines such as Newton-Raphson or other minimization
algorithms that can be used to find these Lagrange multipliers these different
scales become problematic.

To get around this problem we implemented a variation to the algorithm due to G.
Larry Bretthorst, E.T. Jaynes' last student. With a very simple argument we can
show that linearly rescaling the constraints, the Lagrange multipliers and the
``rules'' for how to compute each of the moments, i.e. each of the individual
products that go into the moment calculation, should converge to the same MaxEnt
distribution. In order to see this let's consider again an univariate
distribution $P_X(x)$ that we are trying to reconstruct given the first two
moments. The MaxEnt distribution can be written as
\begin{equation}
  P_H(x) = {1 \over \mathcal{Z}}
  \exp \left(- \lambda_1 x - \lambda_2 x^2 \right) =
  {1 \over \mathcal{Z}}
  \exp \left(- \lambda_1 x \right) \exp \left( - \lambda_2 x^2 \right).
\end{equation}
We can always rescale the terms in any way and obtain the same result. Let's say
that for some reason we want to rescale the quadratic terms by a factor $a$. We
can define a new Lagrange multiplier $\lambda_2' \equiv {\lambda_2 \over a}$
that compensates for the rescaling of the terms, obtaining
\begin{equation}
  P_H(x) = {1 \over \mathcal{Z}}
  \exp \left(- \lambda_1 x \right) \exp \left( - \lambda_2' ax^2 \right).
\end{equation}
Computationally it might be more efficient to find the numerical value of
$\lambda_2'$ rather than $\lambda_2$ maybe because it is of the same order of
magnitude as $\lambda_1$. Then we can always multiply $\lambda_2'$ by $a$ to
obtain back the constraint for our quadratic term. What this means is that that
we can always rescale the MaxEnt problem to make it numerically more stable,
then we can rescale back to obtain the value of the Lagrange multipliers.

Bretthorst algorithm goes even further by further transforming the constraints
and the variables to make the constraints orthogonal, making the computation
much more effective. We now explain the implementation of the algorithm to our
joint distribution of interest $P(m, p)$.

\subsubsection{Algorithm implementation}

Let the matrix $\bb{A}$ contain all the rules used to compute the moments that
serve as constraints, where each entry is of the form
\begin{equation}
  A_{ij} = m_i^{x_j} \cdot p_i^{y_j},
\end{equation}
i.e. the $i^\th$ entry of our sample space consisting of of the product of all
possible pairs ($m, p$) elevated to the appropriate powers $x$ and $y$
associated with the $j^{\text{th}}$ constraint. Let also $\bb{v}$ be a vector
containing all the constraints with each entry of the form
\begin{equation}
  v_j = \ee{m^{x_j} p^{y_j}}.
\end{equation}

That means that the Lagrangian $\mathcal{L}$ to be used for this constrained
maximization problem takes the form
\begin{equation}
  \mathcal{L} = -\sum_i P_i \ln P_i + \lambda_0 \left( 1 - \sum_i P_i \right)
  + \sum_{j>0} \lambda_j \left( v_j - \sum_i A_{ij} P_i \right),
\end{equation}
where $\lambda_0$ is the Lagrange multiplier associated with the normalization
constraint, and $\lambda_j$ is the Lagrange multiplier associated with the
$j^\th$ constraint.

With this notation in hand we now proceed to rescale the problem. The first
step consists of rescaling the rules to compute the entries of matrix $\bb{A}$
as
\begin{equation}
  A_{ij}' = {A_{ij} \over G_j},
\end{equation}
where $G_j$ serves to normalize the moments such that all the Lagrange
multipliers are of the same order of magnitude. This normalization satisfies
\begin{equation}
G_j^2 = \sum_i A_{ij}^2,
\end{equation}
or in terms of our particular problem
\begin{equation}
G_j^2 = \sum_m \sum_p \left( m^{x_j} p^{y_j} \right)^2.
\end{equation}

Since we rescale the rules to compute the constraints, the constraints must
also be rescaled simply as
\begin{equation}
v_j' = \ee{m^{x_j} p^{y_j}}' = {\ee{m^{x_j} p^{y_j}} \over G_j}.
\end{equation}

The Lagrange multipliers must compensate this rescaling since at the end of the
day the probability must add up to the same value. Therefore we rescale the
$\lambda_j$ terms as as
\begin{equation}
\lambda_j' = \lambda_j G_j.
\end{equation}

This rescaling by itself would already improve the algorithm convergence since
now all the Lagrange multipliers would not have drastically different values.
Bretthorst proposes another linear transformation to make the optimization
routine even more efficient. For this we generate orthogonal constraints that
make Newton-Raphson and similar routines converge faster. The transformation is
as follows
\begin{equation}
  A_{ik}'' = \sum_j {e}_{jk} A_{ij}',
\end{equation}
for the entires of matrix $\bb{A}$, and
\begin{equation}
  v_k'' = \sum_j {e}_{jk} u_j',
\end{equation}
for entires of the constraint vector $\bb{v}$, finally
\begin{equation}
  \lambda_k'' = \sum_j {e}_{jk} \beta_j,
\end{equation}
for the Lagrange multipliers. Here ${e}_{jk}$ is the $j^\th$ component
of the $k^\th$ eigenvector of the matrix $\bb{E}$ with entries
\begin{equation}
  {E}_{kj} = \sum_i {A}_{ik}' {A}_{ij}'.
\end{equation}

This transformation guarantees that the matrix $\bb{A}''$ has the property
\begin{equation}
  \sum_i A_{ij}'' A_{jk}'' = \beta_j \delta_{jk},
\end{equation}
where $\beta_j$ is the $j^\th$ eigenvalue of the matrix $\bb{E}$ and
$\delta_{jk}$ is the delta function. What this means is that, as desired, the
constraints are orthogonal to each other, improving the algorithm convergence
speed.

\subsection{Predicting distributions for simple repression constructs}

Having explained the theoretical background along with the practical
difficulties and a workaround strategy proposed by Bretthorst we implemented
the inference using the moments obtained from averaging over the variability
along the cell cycle (See \mrm{ref section no multiple gene copies}).
\fref{fig_pmf_mRNA} and \fref{fig_pmf_protein} present these inferences for
both mRNA and protein levels respectively for different values of the
repressor-DNA binding energy and repressor copy numbers per cell. From these
plots we can easily appreciate that despite the fact that the mean of each
distribution changes as the induction level changes, there is a lot of overlap
between distributions. This as a consequence means that at the single cell level
cells cannot perfectly resolve between different inputs. The influence of the
biophysical parameters is interesting to explore. For the upper right corner
where cells have a very weak binding site and a small amount of repressors, all
of the inducer concentrations basically overlap for the most part, giving cells
very little capacity of resolving different inputs. As the number of repressors
per cell or the reprssor-DNA interaction increases, the different distributions
start to overlap less, allowing cells to extract more information about the
inputs.

\begin{figure}[h!]
	\centering \includegraphics
  {../fig/MaxEnt_approx_joint/PMF_grid_joyplot_mRNA.pdf}
	\caption{\textbf{Maximum entropy mRNA distributions for simple repression
	constructs.} mRNA distributions for different biophysical parameters. From
	left to right the repressor-DNA affinity decreases as defined by the three
	lacI operators O1 ($-15.3 \; k_BT$), O2 ($ -13.9 \; k_BT$), and O3 ($-9.7 \;
	k_BT$). From top to bottom the mean repressor copy number per cell increases.
	The twelve curves on each plot represent the twelve IPTG concentrations used
	in the experiments with higher concentrations represented by lighter colors.}
  \label{fig_pmf_mRNA}
\end{figure}

\begin{figure}[h!]
	\centering \includegraphics
  {../fig/MaxEnt_approx_joint/PMF_grid_joyplot_protein.pdf}
	\caption{\textbf{Maximum entropy protein distributions for simple repression
	constructs.} Protein distributions for different biophysical parameters. From
	left to right the repressor-DNA affinity decreases as defined by the three
	lacI operators O1 ($-15.3 \; k_BT$), O2 ($ -13.9 \; k_BT$), and O3 ($-9.7 \;
	k_BT$). From top to bottom the mean repressor copy number per cell increases.
	The twelve curves on each plot represent the twelve IPTG concentrations used
	in the experiments with higher concentrations represented by lighter colors.}
  \label{fig_pmf_protein}
\end{figure}

	% \section{Derivation of the two stage promoter equation}

Shahreaei and Swain derive the full analytical protein distribution for a two
stage (i.e. an unregulated promoter) and a three stage (i.e. a promoter that
transitions between active and inactive) promoter. In this section wi will
follow the derivation augmenting the details at each step for clarity.

First let us write the chemical master equation for this system. Let $p$ and $m$
be the protein and mRNA copy numbers, respectively, and $P_{m,p}(t)$ be the
probability of having $m$ mRNA and $p$ proteins at time t. Then we can use the
``spread the butter'' approach to write the discrete difference equation
\begin{equation}
\begin{aligned}
P_{m,p}(t + \Delta t) =
P_{m,p}(t)
+ \overbrace{r_m \Delta t \left[ P_{m-1,p}(t) - P_{m,p}(t) \right]}^\text{mRNA
production}
+ \overbrace{r_p m
\Delta t \left[ P_{m, p-1}(t) - P_{m, p}(t) \right]}^\text{protein
production}\\
+ \underbrace{\gamma_m \Delta t \left[ (m + 1) P_{m+1,p}(t) - m P_{m, p}(t)
\right]}_\text{mRNA degradation}
+ \underbrace{\gamma_p \Delta t \left[ (p + 1) P_{m, p+1}(t) - p P_{m, p}(t)
\right]}_\text{protein degradation},
\end{aligned}
\end{equation}
where $r_m$ and $r_p$ are the mRNA and protein production rates respectively,
$\gamma_m$ and $\gamma_p$ are the mRNA and protein degradation rates
respectively, and $\Delta t$ is a time interval small enough so that only one
event can take place.

We rearrange the terms, divide both sides by $\Delta t$ , obtaining
\begin{equation}
  \begin{aligned}
\frac{P_{m,p}(t + \Delta t) - P_{m,p}(t)}{\Delta t} = r_m \left[ P_{m-1,p}(t) -
P_{m,p}(t) \right] + r_p m \left[ P_{m, p-1}(t) - P_{m, p}(t) \right]\\
+ \gamma_m \left[ (m + 1) P_{m+1,p}(t) - m P_{m, p}(t) \right] + \gamma_p \left[
 (p + 1) P_{m, p+1}(t) - p P_{m, p}(t) \right].
  \end{aligned}
\end{equation}
We now take the limit $\Delta t \rightarrow 0$ to obtain the final form of the
chemical master equation
\begin{equation}
\begin{aligned}
\frac{\partial P_{m,p}}{\partial t} = r_m \left[ P_{m-1,p}(t) - P_{m,p}(t)
\right] + r_p m \left[ P_{m, p-1}(t) - P_{m, p}(t) \right]\\
+ \gamma_m \left[ (m + 1) P_{m+1,p}(t) - m P_{m, p}(t) \right] + \gamma_p \left[
(p + 1) p_{m, p+1}(t) - p p_{m, p}(t) \right].
  \end{aligned}
\end{equation}

Let us now divide by the slowest rate, i.e. $\gamma_r$
\begin{equation}
\begin{aligned}
\frac{1}{\gamma_p} \frac{\partial P_{m,p}}{\partial t} =
\frac{r_m}{\gamma_p} \left[ P_{m-1,p}(t) - P_{m,p}(t) \right]
+ \frac{r_p}{\gamma_p} m  \left[ P_{m, p-1}(t) - P_{m, p}(t) \right]\\
+ \frac{\gamma_m}{\gamma_p} \left[ (m + 1) P_{m+1,p}(t) - m P_{m, p}(t)
\right]
+ \left[ (p + 1) P_{m, p+1}(t) - p P_{m, p}(t) \right].
\end{aligned}
\label{eq_cme_over_gammap}
\end{equation}
We now introduce the following variables:
\begin{align}
  a \equiv \frac{r_m}{\gamma_p}\\
  b \equiv \frac{r_p}{\gamma_m}\\
  \gamma \equiv \frac{\gamma_m}{\gamma_p}\\
  \tau \equiv \gamma_p \cdot t
\end{align}
Substituting these variables into \eref[eq_cme_over_gammap] we obtain
\begin{equation}
\begin{aligned}
\frac{\partial P_{m,p}}{\partial \tau} =
a \left[ P_{m-1,p}(t) - P_{m,p}(t) \right]
+ b \gamma m  \left[ P_{m, p-1}(t) - P_{m, p}(t) \right]\\
+ \gamma \left[ (m + 1) P_{m+1,p}(t) - m P_{m, p}(t) \right]
+ \left[ (p + 1) P_{m, p+1}(t) - p P_{m, p}(t) \right].
\end{aligned}
\label{eq_cme_tau}
\end{equation}
Note that we used $\frac{1}{\gamma_p}\frac{\partial}{\partial t} =
\frac{\partial}{\partial \tau}$.

We now define the generating function
\begin{equation}
F\left[ s, z \right] = \sum_{m=0}^{\infty} \sum_{p=0}^{\infty} s^m z^p P_{m, p},
\end{equation}
where from now on we abbreviate $P_{m, p}(t)$ as $P_{m, p}$. The generating
function will allow us to write a single PDE rather than an infinite system of
PDEs for each mRNA and protein copy number. The generating function time
derivative is given by
\begin{equation}
\frac{\partial F}{\partial \tau} =
\frac{\partial}{\partial \tau} \sum_{m=0}^{\infty} \sum_{p=0}^{\infty} s^m z^p
P_{m, p} =
\sum_{m=0}^{\infty}
\sum_{p=0}^{\infty} s^m z^p \frac{\partial}{\partial \tau} P_{m, p}.
\label{eq_dF_dtau}
\end{equation}
We now substitute \eref[eq_cme_tau] into \eref[eq_dF_dtau]
\begin{equation}
\begin{aligned}
\frac{\partial F}{\partial \tau} =
\sum_{m=0}^{\infty} \sum_{p=0}^{\infty} s^m z^p a \left[ P_{m-1,p}(t) -
P_{m,p}(t) \right]
+ b \gamma m  \left[ P_{m, p-1}(t) - P_{m, p}(t) \right]\\
+ \gamma \left[ (m + 1) P_{m+1,p}(t) - m P_{m, p}(t) \right]
+ \left[ (p + 1) P_{m, p+1}(t) - p P_{m, p}(t) \right].
\end{aligned}
\label{eq_dF_dtau_complete}
\end{equation}

Note that
\begin{equation}
\sum_{m=0}^{\infty} \sum_{p=0}^{\infty} s^m z^p \left( P_{m+k, p} - P_{m, p}
\right) =
\sum_{m=0}^{\infty} \sum_{p=0}^{\infty} s^m z^p P_{m+k, p}
- \sum_{m=0}^{\infty} \sum_{p=0}^{\infty} s^m z^p P_{m, p}.
\end{equation}
We can now factorize $s^{-k}$ from the first term on the left hand side and
redefine the variable to sum over.
\begin{equation}
\sum_{m=0}^{\infty} \sum_{p=0}^{\infty} s^m z^p \left( P_{m+k, p} - P_{m, p}
\right) =
s^{-k} \sum_{(m+k)=0}^{\infty} \sum_{p=0}^{\infty} s^{m+k} z^p P_{m+k, p}
- \sum_{m=0}^{\infty} \sum_{p=0}^{\infty} s^m z^p P_{m, p}.
\end{equation}
But since both sums on the left hand side are taken over the same ranges we can
write it as
\begin{equation}
\sum_{m=0}^{\infty} \sum_{p=0}^{\infty} s^m z^p \left( P_{m+k, p} - P_{m, p}
\right) =
\left( s^{-k} - 1 \right) \sum_{m=0}^{\infty} \sum_{p=0}^{\infty} s^m z^p P_{m,
p}
\end{equation}

With this identity we can rewrite \eref[eq_dF_dtau_complete] as
\begin{equation}
\begin{aligned}
\frac{\partial F}{\partial \tau} =
(s - 1) \sum_{m=0}^{\infty} \sum_{p=0}^{\infty} s^m z^p \left( a P_{m,p} \right)
+ (z - 1) \sum_{m=0}^{\infty} \sum_{p=0}^{\infty} s^m z^p b \gamma m P_{m,p}
\\
+ \left( s^{-1} - 1 \right) \sum_{m=0}^{\infty} \sum_{p=0}^{\infty} s^m z^p
\gamma m P_{m,p}
+ \left( z^{-1} - 1 \right) \sum_{m=0}^{\infty} \sum_{p=0}^{\infty} s^m z^p
\gamma p P_{m,p}.
\end{aligned}
\label{eq_dF_dtau_identity}
\end{equation}

Another useful identity can be derived if we note that
\begin{equation}
\sum_{m=0}^{\infty} \sum_{p=0}^{\infty} s^m z^p m P_{m, p} =
\sum_{m=0}^{\infty} \sum_{p=0}^{\infty} z^p s \frac{\partial s^m}{\partial s}
P_{m, p}.
\end{equation}
But since $z^p$ and $P_{m, p}$ do not depend on $s$ we can write
\begin{equation}
\sum_{m=0}^{\infty} \sum_{p=0}^{\infty} s^m z^p m P_{m, p} =
s \frac{\partial}{\partial s}\left( \sum_{m=0}^{\infty} \sum_{p=0}^{\infty} s^m z^p P_{m, p} \right) =
s \frac{\partial}{\partial s}F.
\end{equation}

Using this identity in \eref[eq_dF_dtau_identity] allow us to remove all the
sums, obtaining a single PDE of the form
\begin{equation}
\frac{\partial F}{\partial \tau} =
(s - 1) a F + (z - 1) b \gamma s \frac{\partial F}{\partial s}
+ \left( s^{-1} - 1 \right) \gamma s \frac{\partial F}{\partial s}
+ \left( z^{-1} - 1 \right) z \frac{\partial F}{\partial Z}.
\end{equation}
Rearranging terms we obtain
\begin{equation}
\frac{\partial F}{\partial \tau} =
a (sF - s)
+ b \gamma \left( z s \frac{\partial F}{\partial s} - s \frac{\partial
F}{\partial s} \right)
+ \gamma \left( \frac{\partial F}{\partial s} - s \frac{\partial F}{\partial s}
\right)
+ \left( \frac{\partial F}{\partial z} - z \frac{\partial F}{\partial z}
\right).
\end{equation}

	% \section{Method of the characteristics}

\manuelComment{based on this video: https://www.youtube.com/watch?v=LpHqrlrU5pM}

The method of the characteristics is a powerful method to solve PDE. This method enable us to solve a wide range of problems including:
\begin{enumerate}
  \item Linear:
  \begin{equation}
  a(x, y) \frac{\partial u}{\partial x}
  + b(x, y) \frac{\partial u}{\partial y}
  + c(x, y) u
  = f(x, y)
  \end{equation}
  \item Semi-linear:
  \begin{equation}
  a(x, y) \frac{\partial u}{\partial x}
  + b(x, y) \frac{\partial u}{\partial y}
  + c(x, y) u
  = f(x, y, u)
  \end{equation}
  \item  Quasi-linear:
  \begin{equation}
  a(x, y) \frac{\partial u}{\partial x}
  + b(x, y) \frac{\partial u}{\partial y}
  = f(x, y, u)
  \end{equation}
\end{enumerate}

In this introduction  we will focus on semi-linear problems that also include
the linear ones. Let us consider a PDE of the form
\begin{equation}
  a(x, y) \frac{\partial u}{\partial x}
  + b(x, y) \frac{\partial u}{\partial y}
  = f(x, y, u),
  \label{eq_semi_linear}
\end{equation}
which is a semi-linear PDE. Semi-linear equations include equations of the form
\begin{equation}
  a(x, y) \frac{\partial u}{\partial x}
  + b(x, y) \frac{\partial u}{\partial y}
  + c(x, y) u
  = g(x, y),
\end{equation}
with $f(x, y, u) = g(x, y) - c(x, y) u$.

This PDE must have an initial condition associated with it. This initial
condition can be of the form
\begin{equation}
  u(x, 0) = u_0(x),
\end{equation}
or
\begin{equation}
  u(0, y) = u_1(x).
\end{equation}
The pair of a PDE like \eref[eq_semi_linear] and an initial condition is
sometimes called a semi-linear Cauchy problem.

Any solution to these Cauchy problems can be represented by a surface $u \equiv
u(x, y)$ in a 3D space as schematically shown in \fref[fig_integral_surface]
\begin{figure}[h!]
	\centering \includegraphics[scale=.75]{fig/fig1_method_characteristics.png}
	\caption{\captionStroke{Schematic integral surface.}}
	\label{fig_integral_surface}
\end{figure}

A surface $u = u(x, y)$ that solves a Cauchy problem is known as an integral
surface since integration is used to solve these problems. We can write down the
equation of this integral surface as
\begin{equation}
  F(x, y, u) \equiv u(x, y) - u = 0.
\end{equation}
Note that the first $u(x, y)$ is a function, while the second $u$ is a
variable. So for example if
\begin{equation}
  u(x, y) = x - y^2,
\end{equation}
then
\begin{equation}
  F(x, y, u) = x - y^2 - u.
\end{equation}

From vector calculus we know that any vector $\mathbf{n}$ normal to the surface $F = 0$ is given by the gradient
\begin{equation}
  \mathbf{n} = \nabla F =
  \frac{\partial u}{\partial x} \mathbf{i}
  + \frac{\partial u}{\partial y} \mathbf{j}
  - \mathbf{k}.
\end{equation}

\fref[fig_normal_vector] shows a schematic representation of this vector. The
vector is represented pointing downwards since $F = u(x, y) - u$.

\begin{figure}[h!]
	\centering \includegraphics[scale=.75]{fig/fig2_method_characteristics.png}
	\caption{\captionStroke{Schematic integral surface with normal vector.}}
	\label{fig_normal_vector}
\end{figure}

We can rewrite the PDE as a dot product of the form
\begin{equation}
  \left\langle a(x, y), b(x, y), f(x, y, u) \right\rangle \cdot \mathbf{n} = 0.
\end{equation}

This implies that since the dot product is zero, the vector $\left\langle a(x,
y), b(x, y), f(x, y, u) \right\rangle$ must be normal to $\mathbf{n}$. Since we
stated that the vector $\mathbf{n}$ is normal to the surface $F(x, y, u) = 0$,
any vector $\mathbf{v}$ normal to $\mathbf{n}$ must lie in the tangential plane
to $F = 0$ at every point as depicted in \fref[fig_tangential_vector]
\begin{figure}[h!]
	\centering \includegraphics[scale=.75]{fig/fig3_method_characteristics.png}
	\caption{\captionStroke{Schematic integral surface with normal vector
  $\mathbf{n}$ and tangential vector $\mathbf{v}$.}}
	\label{fig_tangential_vector}
\end{figure}

But what do all these details mean? This means that the PDE requires for any
integral surface that solves the equation to be tangential to the vector
\begin{equation}
  \mathbf{v} = a(x,y) \mathbf{i} +
  b(x, y) \mathbf{j} +
  f(x, y, u) \mathbf{k}.
\end{equation}
That is, if we start at some point set by the initial condition and we move in
the direction of the vector $\mathbf{v}$ (which we can determine just by looking
at the PDE) then we move along a curve that lies entirely within the surface
$F = 0$. This curve, depicted in \fref[fig_characteristic_curve] is called a
\textit{characteristic curve}. And by finding the collection of these
characteristic curves we can therefore reconstruct the entire integral surface,
solving therefore the PDE.
\begin{figure}[h!]
	\centering \includegraphics[scale=.75]{fig/fig4_method_characteristics.png}
	\caption{\captionStroke{Schematic of a characteristic curve on an integral
  surface.}}
	\label{fig_characteristic_curve}
\end{figure}

In other words, so far we have shown that given an initial condition
$u(x, 0) = u_0(x)$ as long as we move along the direction given by the vector
$\mathbf{v} = \left\langle a(x,y), b(x,y), f(x,y,u) \right\rangle$ we are
guaranteed to lay on the integral surface. So the process to reconstruct the
entire surface would be to choose a value for $x$, then start at $u_0(x)$ and
move along $\mathbf{v}$; then we would choose a new value of $x$ and repeat the
process. At the end, the union of all these paths along the integral surface
will determine entirely the solution of the PDE. The advantage of the method as
we will show is that walking along the characteristic curve is equivalent to
solving an ODE, which for many cases we know how to do.

We can describe all the points on a characteristic curve by parametrizing
the curve with a vector function
\begin{equation}
  \mathbf{p}(t) = x(t) \mathbf{i} + y(t) \mathbf{j} + u(t) \mathbf{k}.
\end{equation}
Furthermore, we can obtain a vector tangential to this curve $\mathbf{p}(t)$ by
differentiating with respect to the parametrization variable $t$, i.e.
\begin{equation}
  \mathbf{p}'(t) = x'(t) \mathbf{i}+ y'(t) \mathbf{j} + u'(t) \mathbf{k}.
\end{equation}

Since the curve $\mathbf{p}(t)$ lays on the integral surface $F = 0$, and
$\mathbf{p}'(t)$ is tangential to this curve, that means that $\mathbf{p}'(t)$
also lies on the surface.

We stated earlier that if we move along the direction of $\mathbf{v}$ we will
reconstruct the path of the characteristic curve which we now parametrized as
$\mathbf{p}(t)$. That implies that $\mathbf{v}$ and $\mathbf{p}'(t)$ must be
pointing in the same direction, i.e.
\begin{equation}
  \mathbf{p}'(t) = \lambda \mathbf{v},
\end{equation}
where $lambda$ is a scalar. This can also be written as
\begin{equation}
  x'(t)\mathbf{i} + y'(t)\mathbf{j} + u'(t)\mathbf{k} =
  \lambda a(x,y) + \lambda b(x,y) + \lambda f(x,y,u).
\end{equation}
in other words
\begin{align}
  x'(t) = \lambda a(x,y),\\
  y'(t) = \lambda b(x,y), \\
  u'(t) = \lambda f(x,y,u).
\end{align}
We can now solve for $\lambda$ and equate all of the solutions, obtaining
\begin{equation}
  {{dx \over dt} \over a(x,y)}} =
  {{dy \over dt} \over b(x,y)} =
  {{du \over dt} \over f(x,y,u)},
\end{equation}
or simply in a compact differential form
\begin{equation}
  {dx \over a} = {dy \over b} = {du \over f}.
\end{equation}

%	\pagebreak
\phantom{This ensures that the next section starts on a new page}
\pagebreak
\section{To Do}

\begin{itemize}	
%	\item Get distribution of error from Jane's paper. 
%	\begin{itemize}
%		\item Use Jane's closed form solution and write this up in our paper
%		
%		\item Also consider doing (numerically, not analytically) a 3-state context
%		with an empty promoter. See if the numerical result is any different from
%		Jane's 2-state solution.
%		
%		\item Read Peter Swain's paper "Analytical distributions for stochastic gene
%		expression" to see how to get from Jane's mRNA profile to the gene expression
%		profile. Manuel thinks that we lose some information at every step of the
%		central dogma (see "Data processing inequality" in my Chrome Bookmarks) and
%		that if we can get the gene expression prediction/experiment to match then we
%		can try to measure the mRNA levels directly and see if they also match the
%		prediction. That would be a really nice match between theory and experiment.
%		Manuel and I should discuss the derivations in Peter Swain's paper go.
%	\end{itemize}
	
	\item As per Swain's last point before the Discussion, if the RNAP and repressor's binding-unbinding rates are fast, then the 3-stage model reduces to the 2-stage model with $a \to \frac{\kappa_0}{\kappa_0 + \kappa_1} a$, where $\frac{\kappa_0}{\kappa_0 + \kappa_1}$ is the fraction of the time that the system is in the RNAP bound state in equilibrium.
	\begin{itemize}
		\item Figure out experimentally what these rate constants are for our system
		
		\item Rob has previously suggested that Jeff Gelles knows what many of these rate constants are for the Lac system
		
		\item If these rate constants are fast, we can use the much simpler 2-stage model protein distribution, and we can also easily add in the empty promoter stage (i.e. a 4-stage model) if we just modify $a$ accordingly
	\end{itemize}
	
	\item Expand Swain's 3-stage model (promoter either RNAP bound or repressor bound) to a 4-stage model. I think this should be straightforward to do analytically.
	
	\item Explore the channel capacity analytically in the limit of small noise (discuss whether this is a valid assumption from our data). \textbf{As a starting point, look at Rieckh and Tkacik's ``Noise and information transmission in promoters with multiple internal states''}
	
	\item Other things that would be very interesting to explore:
		\begin{itemize}
			\item Be theorists, and consider how other variables affect the channel capacity. For example, the (active) repressor-DNA binding energy, or $K_A/K_I$, or the rates in the problem. Would be really cool to basically look at the effects of all of the variables that we have on channel capacity, and consider which yield the coolest results.
			\item Varying cross all $R$ and $k_{off}$ values, what is the maximal channel capacity you can achieve?
			\item Potentially look at Mitch Lewis's strains which will have different $K_{DNA}$ or $K_{A}$ and $K_{I}$ parameters and redo the analysis there, showing that the theory is amazingly good!
		\end{itemize}
	
	
	\item Really important reference for us is \cite{Hansen2015}.
		\begin{itemize}
			\item (Page 6, Figure 2A) Their measured channel capacity for amplitude modulated (AM) signal, was 1.3 for the natural promoter. We are also planning to measure this AM signal, but not the frequency modulated (FM) signal that they also measured.
			\item (Page 7) "Measuring mutual information in a cell population subject to extrinsic noise...underestimates the intrinsic information transduction capacity of a promoter." They claimed this was caused by cells being at different stages of the cell cycle. After correcting for this error, they measured a channel capacity of 1.5, which still means the system can only distinguish between two or three input states
			\item (Page 8) They made a mutant promoter that had a channel capacity of ~2. I assume we are not going to pursue this angle, but it shows that natural operators do not necessarily maximize channel capacity.
			\item Overall, I found it difficult to understand what change in channel capacity was "significant." They claimed that an increase of 0.15 is "small but robust," but I wondered whether this was comparable to their experimental noise. I am worried that if the Lac system also yields small channel capacities, we will end up making statements like "we saw a 0.1 increase in channel capacity between O2 and O1..."
			\item To rephrase that last point, we hope that the Lac system can distinguish multiple inputs (what they call the Rheostat Model), but if channel capacity is ~1 then you can only distinguish the OFF and ON state (i.e. their Noisy Switch Model). Based on your calculations of channel capacity so far, how many bits do you suspect we will find?
			\item I think our paper needs to go \textit{beyond} this paper in a meaningful way. One way I think we will improve upon their results is that we have a \textbf{theoretical prediction} for the form of the variability in gene expression. We should really stress that!
		\end{itemize}
	
	\item Another important reference \cite{Chevalier2015}:
		\begin{itemize}
			\item Several other groups are making similar mutual information measurements (between ligand concentration and gene expression) in their systems, and so we should really stress what is new about our paper - that we are \textit{predicting everything theoretically}! References 6-10 from this paper should be good to cite.
			\item I really like their short and sweet introduction on mutual information: ``Mutual information (MI) is a natural metric for characterizing information transmission between the inputs of a stochastic network and its nodes. MI quantifies the level of precision with which a given node(s) in a network estimates and responds to an input(s) by accounting for both the mean and variability in the response.''
			\item They also claimed that our theoretical measurements should overestimate the measured mutual information because of: (1) variability of cell's being at different stages of the cell cycle and (2) variability that is extrinsic to the pathway. We can keep that in mind if we theoretically overshoot the measured values.
			\item Finally, they point out that the measured mutual information is time-dependent, so we need to be super careful about making all of our measurements using precisely the same method. Hopefully the O1, O2, O3, Oid strains all grow at exactly the same speed, because otherwise if you always wait until OD 0.5 (or whatever) but this waiting time is different between strains, that could introduce a bias into the measurements.
		\end{itemize}
	
	\item In \href{http://www.pnas.org/content/113/11/E1470.full}{Rodrigues and Shakhnovich 2016} (PNAS) that near their Equation 1 they state that a fitness function based on an enzyme's activity has been shown for the lac operon's beta-galactosidase: see \href{http://www.genetics.org/content/genetics/115/1/25.full.pdf}{this paper}
	
%	\textcolor{brown}{
%	\item (Paper 2) How can the theory be applied to an evolutionary context? For example, compete RBS 1027 against RBS 446, and predict both which will win and exactly by how much. Then unconstrain the system and let them evolve over time on the growth rate versus mutual information graph and see that they (hopefully) move upwards. (May need to make new strains that have LacI and possibly SacB for this)
%	\begin{itemize}
%		\item Cost/benefit function. Do we use Lassig Paper (Nonlinear fitness landscape of a molecular pathway), Uri Alon paper, or our own? (At the very least, should try both cost/benefit functions and see if they give different results.)
%		\item Make a strain as close to optimal as possible. Manuel tried doing this, but got unphysical parameters. Wiggle, wiggle, wiggle them until they are reasonable. Noah is willing to help on this end, if desired (potentially as another member of the project, if desired).
%		\item We agree that there is evolutionary pressure to move up in the growth rate plot, but is there any pressure to move left or right (other than being able to access more upwards space the further you go to the right). Give readers some intuition on what it means for two points to be on the same horizontal line (i.e. what could their profiles look like); what about two vertical points? Do these points have smaller errors, or must they have different curve shapes? Can we make a phase diagram in some way of the different Mutual Info/Growth Rate plot? If you take any curve and scale its error bars up or down, what trajectory do you get on the Mutual Info/Growth Rate plot?
%		\item Keep in mind that we are trying to maximize \textit{growth rate} and not \textit{mutual information}. To be honest about this, we should talk a bit about a strain that maximizes mutual information at the cost of growth rate. Could there be some other metric that is analogous to growth rate and still uses the full distribution?
%	\end{itemize}
%	\item (Future) Does the natural Lac operon provide qualitatively different behavior than the synthetic ones? Explore this theoretically, and if it does then we could potentially make these strains as well.
%	}
\end{itemize}
	% Bibliography
	\bibliographystyle{unsrt}
	\bibliography{./chann_cap_bib}

\end{document}
