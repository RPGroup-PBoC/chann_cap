\subsection{Theoretical prediction of the channel capacity}
\label{sec_channcap}

As a useful measure of the ability of the genetic circuit to allow the cell
infer the environmental state, i.e. the inducer concentration, we turn to the
channel capacity. The channel capacity is defined as the maximum mutual
information between input and output, maximized over all possible input
distributions. Putting this into mathematical terms let us define $c$ as the
inducer concentration. $P(c)$ represents the distribution of inducer and $P(p
\mid c)$ the distribution of protein counts given a fixed inducer concentration.
The channel capacity is then given by
\begin{equation}
  C \equiv \max_{P(c)} I(p; c),
  \label{eq_chann_cap}
\end{equation}
where $I(p; c)$, the mutual information between protein count and inducer
concentration is given by \cite{Shannon1948}
\begin{equation}
  I(p; c) = \sum_c P(c) \sum_p P(p \mid c) \log_2 {P(p \mid c) \over P(p)}.
\end{equation}

If used as a metric of how reliably a signaling system can infer the state
of the external signal, the channel capacity, when measured in bits, is commonly
interpreted as the logarithm of the number of states that the signaling system
can properly resolve. So for example a signaling system with a channel capacity
of $C$ bits is interpreted as being able to resolve $2^C$ states. We caution
against the use of this interpretation as channel capacities with fractional
values are allowed. We prefer the Bayesian interpretation that the mutual
information, and as a consequence the channel capacity, quantifies the
improvement in the inference of the input when utilizing the output compared to
just using the prior distribution of the input by itself for prediction
\cite{Voliotis2014, Bowsher2014}. Under this interpretation a channel capacity
of a fractional bit still quantifies an improvement of the ability of the
signaling system to infer the value of the extracellular signal compared to
having no sensing system at all.

Computing the channel capacity as defined in \eref{eq_chann_cap} implies
optimizing over an infinite space of distributions possible $P(c)$. For special
cases in which the noise is small compared to the dynamic range, approximate
analytical equations have been derived \cite{Tkacik2008a}. But given the high
cell-to-cell variability that our model predicts, the conditions of the
so-called small noise approximation are not satisfied. We therefore appeal to
a numerical solution known as the Blahut-Arimoto algorithm \cite{Blahut1972}.
This algorithm starting on any (discrete) distribution $P(C)$ converges to the
distribution at channel capacity by iteratively applying a series of simple
operations. \fref{fig5_channcap} shows zero-parameter fit predictions of the
channel capacity as a function of the number of repressors for different
repressor-DNA affinities (solid lines). These predictions are contrasted with
experimental determinations of the channel capacity as inferred from single-cell
fluorescence intensity distributions taken over 12 different concentrations of
inducer \mrm{See Methods for further details}.

\fref{fig5_channcap} has interesting features that are worth highlighting. On
one extreme for cells with no transcription factors there is no information
processing potential as this simple genetic circuit would be constitutively
expressed regardless of the environmental state. As cells increase the
transcription factor copy number, the channel capacity increases until it
reaches a maximum to then fall back down at high repressor copy number since the
promoter would be permanently repressed \mrm{See Fig. SXX}. The steepness of the
increment in channel capacity as well as the height of the maximum expression
highly depend on the repressor-DNA affinity. For strong binding sites (blue
curve in \fref{fig5_channcap}) there is a quick increment in the channel
capacity, but the maximum value reached is smaller compared to a weak binding
site (green curve in \fref{fig5_channcap}).

\begin{figure}[h!]
	\centering \includegraphics
  {./fig/main/theory_vs_data_channel_capacity.pdf}
	\caption{\textbf{Comparison of theoretical and experimental channel capacity.}
	The channel capacity as inferred using the Blahut-Arimoto algorithm
	\cite{Blahut1972} for varying number of repressors and repressor-DNA
	affinities. All inferences were performed using 12 IPTG concentrations as
	detailed in the Methods. Solid lines represent zero-parameter fit predictions
	done with MaxEnt distributions as the ones showed in \fref{fig4_maxent}. Dots
	represent inferences made from single cell fluorescence distributions (See
	Methods for further details).}
  \label{fig5_channcap}
\end{figure}
