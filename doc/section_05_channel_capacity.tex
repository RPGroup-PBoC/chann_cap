\subsection{Theoretical prediction of the channel capacity}
\label{sec_channcap}

We now turn our focus to the channel capacity which is a metric by which we can
quantify the degree to which cells can measure the environmental state (in this
context, the inducer concentration). The channel capacity is defined as the
mutual information between input and output, maximized over all possible input
distributions. Putting this into mathematical terms we define $c$ as the
inducer concentration. $P(c)$ represents the distribution of inducer and $P(p
\mid c)$ the distribution of protein counts given a fixed inducer concentration
-- effectively the distributions shown in \fref{fig4_maxent}(A). The channel
capacity is then given by
\begin{equation}
  C \equiv \sup_{P(c)} I,
  \label{eq_chann_cap}
\end{equation}
where $I$, the mutual information between protein count and inducer
concentration is given by \eref{eq_mutual_info}.

If used as a metric of how reliably a signaling system can infer the state of
the external signal, the channel capacity, when measured in bits, is commonly
interpreted as the logarithm of the number of states that the signaling system
can properly resolve. For example, a signaling system with a channel capacity
of $C$ bits is interpreted as being able to resolve $2^C$ states, though
channel capacities with fractional values are allowed. We therefore prefer
the Bayesian interpretation that the mutual information, and as a consequence
the channel capacity, quantifies the improvement in the inference of the input
when considering the output compared to just using the prior distribution of
the input by itself for prediction \cite{Voliotis2014a, Bowsher2014}. Under
this interpretation a channel capacity of a fractional bit still quantifies an
improvement of the ability of the signaling system to infer the value of the
extracellular signal compared to having no sensing system at all.

Computing the channel capacity as defined in \eref{eq_chann_cap} implies
optimizing over an infinite space of possible distributions $P(c)$. For special
cases in which the noise is small compared to the dynamic range, approximate
analytical equations have been derived \cite{Tkacik2008a}. But given the high
cell-to-cell variability that our model predicts, the conditions of the
so-called small noise approximation are not satisfied. We therefore appeal to a
numerical solution known as the Blahut-Arimoto algorithm \cite{Blahut1972}.
This algorithm, starting on any (discrete) distribution $P(c)$, converges to
the distribution at channel capacity. \fref{fig5_channcap}(A) shows
zero-parameter fit predictions of the channel capacity as a function of the
number of repressors for different repressor-DNA affinities (solid lines).
These predictions are contrasted with experimental determinations of the
channel capacity as inferred from single-cell fluorescence intensity
distributions taken over 12 different concentrations of inducer. Briefly, from
single-cell fluorescent measurements we can approximate the input-output
distribution $P(p \mid c)$. Once these conditional distributions are fixed, the
task of finding the input distribution at channel capacity become a
computational minimization routine that can be undertaken using conjugate
gradient or similar algorithms. For the particular case of the channel capacity
on a system with a discrete number of inputs and outputs the Blahut-Arimoto
algorithm is built in such a way that it guarantees the convergence towards the
optimal input distribution (See \siref{supp_channcap} for further details).
\fref{fig5_channcap}(B) shows example input-output functions for different
values of the channel capacity. This illustrates that having access to no
information (zero channel capacity) is a consequence of having overlapping
input-output functions (lower panel). On the other hand, the more separated the
input-output distributions are (upper panel) the higher the channel capacity
can be.

All theoretical predictions in \fref{fig5_channcap}(A) are systematically above
the experimental data. In \siref{supp_channcap} we show using the small noise
approximation \cite{Tkacik2008, Tkacik2008a} that given the constant
multiplicative factor by which our model underestimates the noise in the
protein distribution, we expect the channel capacity to be off by a constant
additive factor. This factor of $\approx 0.43$ bits can recover the agreement
between the model and the experimental data. Although our theoretical
predictions in \fref{fig5_channcap}(A) do not numerically match the
experimental inference of the channel capacity, the model capture interesting
qualitative features of the data that are worth highlighting. On one extreme
for cells with no transcription factors there is no information processing
potential as this simple genetic circuit would be constitutively expressed
regardless of the environmental state. As cells increase the transcription
factor copy number, the channel capacity increases until it reaches a maximum
to then fall back down at high repressor copy number since the promoter would
be permanently repressed. The steepness of the increment in channel capacity as
well as the height of the maximum expression highly depend on the repressor-DNA
affinity. For strong binding sites (blue curve in \fref{fig5_channcap}(A))
there is a rapid increment in the channel capacity, but the maximum value
reached is smaller compared to a weaker binding site (orange curve in
\fref{fig5_channcap}(A)).

\begin{figure}[h!]
	\centering \includegraphics
  {./fig/main/fig5_channcap.pdf}
	\caption{\textbf{Comparison of theoretical and experimental channel
	capacity.} (A) Channel capacity as inferred using the Blahut-Arimoto
	algorithm \cite{Blahut1972} for varying number of repressors and
	repressor-DNA affinities. All inferences were performed using 12 IPTG
	concentrations as detailed in the Methods. Lines represent zero-parameter
	fit predictions done with the maximum entropy distributions as those shown
	in \fref{fig4_maxent}. Points represent inferences made from single cell
	fluorescence distributions (See \siref{supp_channcap} for further details).
	Theoretical curves were smoothed out to remove numerical irregularities.
	(B) Example input-output functions of opposite limits of channel capacity.
	Lower panel illustrates that zero channel capacity indicates that all
	distributions overlap. Upper panel illustrates that as the channel capacity
	increases, the separation between distributions increases as well. Arrows
	point to the corresponding channel capacity computed from the predicted distributions.}
  \label{fig5_channcap}
\end{figure}
