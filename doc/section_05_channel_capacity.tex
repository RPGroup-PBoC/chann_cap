\subsection*{Theoretical prediction of the channel capacity}

As a useful measure of the ability of the genetic circuit to infer the
environmental state, i.e. the inducer concentration, we turn to the channel
capacity. The channel capacity is defined as the maximum mutual information
between input and output, maximized over all possible input distributions.
Putting this into mathematical terms let us define $c$ as the inducer
concentration. $P(c)$ represents the distribution of inducer and $P(p \mid c)$
the distribution of protein counts given a fixed inducer concentration. The
channel capacity is then given by
\begin{equation}
  C \equiv \sup_{P(c)} I(p; c);
\end{equation}
where $I(p; c)$, the mutual information between protein count and inducer
concentration is given by \cite{Shannon1948}
\begin{equation}
  I(p; c) = \sum_c P(c) \sum_p P(p \mid c) \log_2 {P(p \mid c) \over P(p)}.
\end{equation}

If used as a metric of how reliable is a signaling system on inferring the state
of the external signal, the channel capacity, when measured in bits, is commonly
interpreted as the logarithm of the number of states that the signaling system
can properly resolve. So for example a signaling system with a channel capacity
of $C$ bits is interpreted as being able to resolve $2^C$ states. We caution
against the use of this interpretation as channel capacities with fractional
values are allowed. We prefer the Bayesian interpretation that the mutual
information, and as a consequence the channel capacity, quantifies the
improvement in the inference of the input when utilizing the output compared to
just using the prior distribution of the input by itself \cite{Voliotis2014,
Bowsher2014}. Under this interpretation a channel capacity of a fractional bit
still quantifies an improvement of the ability of the signaling system to infer
the value of the extracellular signal.
\mrm{I'll probably need to write an appendix on this very interesting point.}

\begin{figure}[h!]
	\centering \includegraphics
  {./fig/main/theory_vs_data_channel_capacity.pdf}
	\caption{\textbf{Comparison of theoretical and experimental channel
  capacity.} The channel capacity as inferred using the Blahut-Arimoto
  algorithm \cite{Blahut1972} for varying number of repressors and
  repressor-DNA affinities. All inferences were performed using 12 IPTG
  concentrations as detailed in the Methods. Solid lines represent
  zero-parameter fit predictions done with MaxEnt distributions as the ones
  showed in \fref{fig4_maxent}. Dots represent inferences made from single cell
  fluorescence distributions (See Methods for further details).}
  \label{fig5_channcap}
\end{figure}
