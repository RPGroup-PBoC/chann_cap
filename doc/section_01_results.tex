\section{Results}

\subsection{Minimal model of transcriptional regulation}

We begin by defining the simple repression genetic circuit to be used throughout
this work. As a trackable circuit for which we have control over the parameters
both theoretically and experimentally we chose the commonly found simple
repression architecture \cite{Rydenfelt2014}. This circuit consists of a single
promoter with an RNA-polymerase (RNAP) binding site and a single binding site
for a transcriptional repressor \cite{Garcia2011c}. As in many examples
throughout biology, the repressor is assumed to be allosteric in nature, meaning
that it can exist in two conformations, one in which the repressor is able to
bind to the specific binding site (active state) and one in which it cannot bind
the specific binding site (inactive state). The environmental signaling is
assumed to happen via passive import of an extracellular inducer that binds the
repressor, shifting the equilibrium between both conformations of the repressor
\cite{Razo-Mejia2018}. In previous publications we have extensively
characterized the mean response of this circuit under different conditions using
equilibrium based models \mrm{cite aztec pyramid}. In this work we build upon
these models to characterize the full distribution of gene expression as
parameters such as repressor copy number and its affinity for the DNA are
systematically varied.

Given the discrete nature of molecular species count inside cells, chemical
master equations have emerged as a useful tool to model the inherent probability
distribution of these counts \cite{Sanchez2013}. In \fref{fig2_minimal_model}(A)
we show the minimal model to and the necessary set of parameters needed to
predict mRNA and protein distribution. Specifically we assume a three-state
model where the promoter can be found 1) with RNAP bound ($P$ state), 2) empty
($E$ state) and 3) with the repressor bound ($R$ state). These three states
generate a system of coupled partial differential equations for each of the
three state distributions $P_P(m, p)$, $P_E(m, p)$ and $P_R(m, p)$, where $m$
and $p$ are the mRNA and protein count per cell, respectively. Given the rates
shown in \fref{fig2_minimal_model}(A) let us define the system of PDEs for a
specific $m$ and $p$. For the RNAP bound state we have
\mrmr{not sure if I will have space to display these extensive equations. Maybe
this is an SI thing and here I can write only the matrix notation form?}
\begin{equation}
  \begin{aligned}
    \dt{P_P(m, p)} &=
    - \overbrace{\kpoff P_P(m, p)}^{P \rightarrow E} % P -> E
    + \overbrace{\kpon P_E(m, p)}^{E \rightarrow P}\\ % E -> P
    &+ \overbrace{r_m P_p(m-1, p)}^{m-1 \rightarrow m} % m-1 -> m
    - \overbrace{r_m P_p(m, p)}^{m \rightarrow m+1}% m -> m+1
    + \overbrace{\gm (m + 1) P_P(m+1 , p)}^{m+1 \rightarrow m} % m+1 -> m
    - \overbrace{\gm m P_P(m , p)}^{m \rightarrow m-1}\\ % m -> m-1
    &+ \overbrace{r_p m P_P(m, p - 1)}^{p-1 \rightarrow p} % p-1 -> p
    - \overbrace{r_p m P_P(m, p)}^{p \rightarrow p+1} % p -> p+1
    + \overbrace{\gp (p + 1) P_P(m, p + 1)}^{p + 1 \rightarrow p} % p+1 -> p
    - \overbrace{\gp p P_P(m, p)}^{p \rightarrow p-1}. % p -> p-1
  \end{aligned}
\end{equation}
For the empty state $E$ we have
\begin{equation}
  \begin{aligned}
    \dt{P_E(m, p)} &=
    \overbrace{\kpoff P_P(m, p)}^{P \rightarrow E} % P -> E
    - \overbrace{\kpon P_E(m, p)}^{E \rightarrow P} % E -> P
    + \overbrace{\kroff P_R(m, p)}^{R \rightarrow E} % R -> E
    - \overbrace{\kron P_E(m, p)}^{E \rightarrow R}\\ % E -> R
    &+ \overbrace{\gm (m + 1) P_E(m+1 , p)}^{m+1 \rightarrow m} % m+1 -> m
    - \overbrace{\gm m P_E(m , p)}^{m \rightarrow m-1}\\ % m -> m-1
    &+ \overbrace{r_p m P_E(m, p - 1)}^{p-1 \rightarrow p} % p-1 -> p
    - \overbrace{r_p m P_E(m, p)}^{p \rightarrow p+1} % p -> p+1
    + \overbrace{\gp (p + 1) P_E(m, p + 1)}^{p + 1 \rightarrow p} % p+1 -> p
    - \overbrace{\gp p P_E(m, p)}^{p \rightarrow p-1}. % p -> p-1
  \end{aligned}
\end{equation}
And finally for the repressor bound state $R$ we have
\begin{equation}
  \begin{aligned}
    \dt{P_R(m, p)} &=
    - \overbrace{\kroff P_R(m, p)}^{R \rightarrow E} % R -> E
    + \overbrace{\kron P_E(m, p)}^{E \rightarrow R}\\ % E -> R
    &+ \overbrace{\gm (m + 1) P_R(m+1 , p)}^{m+1 \rightarrow m} % m+1 -> m
    - \overbrace{\gm m P_R(m , p)}^{m \rightarrow m-1}\\ % m -> m-1
    &+ \overbrace{r_p m P_R(m, p - 1)}^{p-1 \rightarrow p} % p-1 -> p
    - \overbrace{r_p m P_R(m, p)}^{p \rightarrow p+1} % p -> p+1
    + \overbrace{\gp (p + 1) P_R(m, p + 1)}^{p + 1 \rightarrow p} % p+1 -> p
    - \overbrace{\gp p P_R(m, p)}^{p \rightarrow p-1}. % p -> p-1
  \end{aligned}
\end{equation}

\begin{figure}[h!]
	\centering \includegraphics
  {./fig/main/parameter_inference_v03.pdf}
	\caption{\textbf{Minimal kinetic model of transcriptional regulation for a
	simple repression architecture.} (A) Three-state stochastic model of
	transcriptional regulation by a repressor. The regulation of the repressor is
	assumed to happen via steric hindrance with the RNAP, not allowing the
	promoter state in which both, the repressor and the RNAP are bound
	simultaneously. All parameters highlighted with color boxes were determined
	with published datasets based on the same genetic circuit. (B) Data sets used
	to infer the parameter values. From left to right Garcia \& Phillips
	\cite{Garcia2011c} is used to determine $\kroff$ and $\kron$, Brewster et al.
	\cite{Brewster2014} is used to determine $\eAI$ and $\kron$ Razo-Mejia et al.
	\cite{Razo-Mejia2018} is used to determine $K_A$, $K_I$, and $\kron$ and Jones
	et al. is used to determine $r_m$, $\kpon$, and $\kpoff$.}
  \label{fig2_minimal_model}
\end{figure}
