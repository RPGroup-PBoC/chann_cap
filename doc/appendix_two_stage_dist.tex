\section{Derivation of the two stage promoter equation}

Shahreaei and Swain derive the full analytical protein distribution for a two
stage (i.e. an unregulated promoter) and a three stage (i.e. a promoter that
transitions between active and inactive) promoter \cite{Shahrezaei2008}. In
this section wi will follow the derivation augmenting the details at each step
for clarity.

First let us write the chemical master equation for this system. Let $p$ and $m$
be the protein and mRNA copy numbers, respectively, and $P_{m,p}(t)$ be the
probability of having $m$ mRNA and $p$ proteins at time t. Then we can use the
``spread the butter'' approach to write the discrete difference equation
\begin{equation}
\begin{aligned}
P_{m,p}(t + \Delta t) =
P_{m,p}(t) +
\overbrace{r_m \Delta t
\left[ P_{m-1,p}(t) - P_{m,p}(t) \right]}^\text{mRNA production}
+ \overbrace{r_p m \Delta t
\left[ P_{m, p-1}(t) - P_{m, p}(t) \right]}^\text{protein production}\\
+ \underbrace{\gamma_m \Delta t
\left[ (m + 1) P_{m+1,p}(t) - m P_{m, p}(t) \right]}_\text{mRNA degradation}
+ \underbrace{\gamma_p \Delta t
\left[ (p + 1) P_{m, p+1}(t) - p P_{m, p}(t) \right]}_\text{protein
degradation},
\end{aligned}
\end{equation}
where $r_m$ and $r_p$ are the mRNA and protein production rates respectively,
$\gamma_m$ and $\gamma_p$ are the mRNA and protein degradation rates
respectively, and $\Delta t$ is a time interval small enough so that only one
event can take place.

We rearrange the terms, divide both sides by $\Delta t$ , obtaining
\begin{equation}
  \begin{aligned}
\frac{P_{m,p}(t + \Delta t) - P_{m,p}(t)}{\Delta t} =
r_m \left[ P_{m-1,p}(t) - P_{m,p}(t) \right]
+ r_p m \left[ P_{m, p-1}(t) - P_{m, p}(t) \right]\\
+ \gamma_m \left[ (m + 1) P_{m+1,p}(t) - m P_{m, p}(t) \right]
+ \gamma_p \left[ (p + 1) P_{m, p+1}(t) - p P_{m, p}(t) \right].
  \end{aligned}
\end{equation}
Note that the production of protein is taken per mRNA.
We now take the limit $\Delta t \rightarrow 0$ to obtain the final form of the
chemical master equation
\begin{equation}
  \begin{aligned}
\frac{\partial P_{m,p}}{\partial t} =
r_m \left[ P_{m-1,p}(t) - P_{m,p}(t) \right] +
r_p m \left[ P_{m, p-1}(t) - P_{m, p}(t) \right]\\
+ \gamma_m \left[ (m + 1) P_{m+1,p}(t)
- m P_{m, p}(t) \right]
+ \gamma_p \left[ (p + 1) P_{m, p+1}(t) - p P_{m, p}(t) \right].
  \end{aligned}
\end{equation}

\manuelComment{Eq. 1 of the paper.}

Let us now divide by the slowest rate, i.e. $\gamma_r$
\begin{equation}
\begin{aligned}
\frac{1}{\gamma_p} \frac{\partial P_{m,p}}{\partial t} =
\frac{r_m}{\gamma_p} \left[ P_{m-1,p}(t) - P_{m,p}(t) \right]
+ \frac{r_p}{\gamma_p} m  \left[ P_{m, p-1}(t) - P_{m, p}(t) \right]\\
+ \frac{\gamma_m}{\gamma_p} \left[ (m + 1) P_{m+1,p}(t) - m P_{m, p}(t) \right]
+ \left[ (p + 1) P_{m, p+1}(t) - p P_{m, p}(t) \right].
\end{aligned}
\label{eq_cme_over_gammap}
\end{equation}
We now introduce the following variables:
\begin{align}
  a \equiv \frac{r_m}{\gamma_p}\\
  b \equiv \frac{r_p}{\gamma_m}\\
  \gamma \equiv \frac{\gamma_m}{\gamma_p}\\
  \tau \equiv \gamma_p \cdot t
\end{align}
Substituting these variables into \eref[eq_cme_over_gammap] we obtain
\begin{equation}
\begin{aligned}
\frac{\partial P_{m,p}}{\partial \tau} =
a \left[ P_{m-1,p}(t) - P_{m,p}(t) \right]
+ b \gamma m  \left[ P_{m, p-1}(t) - P_{m, p}(t) \right]\\
+ \gamma \left[ (m + 1) P_{m+1,p}(t) - m P_{m, p}(t) \right]
+ \left[ (p + 1) P_{m, p+1}(t) - p P_{m, p}(t) \right].
\end{aligned}
\label{eq_cme_tau}
\end{equation}
Note that we used $\frac{1}{\gamma_p}\frac{\partial}{\partial t} =
\frac{\partial}{\partial \tau}$.

We now define the generating function
\begin{equation}
F\left[ s, z \right] = \sum_{m=0}^{\infty} \sum_{p=0}^{\infty} s^m z^p P_{m, p},
\end{equation}
where from now on we abbreviate $P_{m, p}(t)$ as $P_{m, p}$. The generating
function will allow us to write a single PDE rather than an infinite system of
PDEs for each mRNA and protein copy number. The generating function time
derivative is given by
\begin{equation}
\frac{\partial F}{\partial \tau} =
\frac{\partial}{\partial \tau} \sum_{m=0}^{\infty} \sum_{p=0}^{\infty} s^m z^p
P_{m, p} =
\sum_{m=0}^{\infty}
\sum_{p=0}^{\infty} s^m z^p \frac{\partial}{\partial \tau} P_{m, p}.
\label{eq_dF_dtau}
\end{equation}
We now substitute \eref[eq_cme_tau] into \eref[eq_dF_dtau]
\begin{equation}
\begin{aligned}
\frac{\partial F}{\partial \tau} =
\sum_{m=0}^{\infty} \sum_{p=0}^{\infty} s^m z^p
\left\{ a \left[ P_{m-1,p} - P_{m,p} \right]
+ b \gamma m  \left[ P_{m, p-1} - P_{m, p} \right] \right. \\
\left. + \gamma \left[ (m + 1) P_{m+1,p} - m P_{m, p} \right]
+ \left[ (p + 1) P_{m, p+1} - p P_{m, p} \right] \right\}.
\end{aligned}
\label{eq_dF_dtau_complete}
\end{equation}

In order to make progress with this equation we note that we can distribute
the terms in parenthesis as
\begin{equation}
\sum_{m=0}^{\infty} \sum_{p=0}^{\infty} s^m z^p \left( P_{m+1, p} - P_{m, p}
\right) =
\sum_{m=0}^{\infty} \sum_{p=0}^{\infty} s^m z^p P_{m+1, p}
- \sum_{m=0}^{\infty} \sum_{p=0}^{\infty} s^m z^p P_{m, p}.
\label{eq_split_sum}
\end{equation}
We can now factorize $s^{-1}$ from the first term on the left hand side of
\eref[eq_split_sum] and redefine the variable to sum over.
\begin{equation}
\sum_{m=0}^{\infty} \sum_{p=0}^{\infty} s^m z^p \left( P_{m+1, p} - P_{m, p}
\right) =
s^{-1} \sum_{(m+1)=0}^{\infty} \sum_{p=0}^{\infty} s^{m+1} z^p P_{m+1, p}
- \sum_{m=0}^{\infty} \sum_{p=0}^{\infty} s^m z^p P_{m, p}.
\end{equation}
But since both sums on the left hand side are taken over the same ranges we can
write it as
\begin{equation}
\sum_{m=0}^{\infty} \sum_{p=0}^{\infty} s^m z^p \left( P_{m+1, p} - P_{m, p}
\right) =
\left( s^{-1} - 1 \right) \sum_{m=0}^{\infty} \sum_{p=0}^{\infty} s^m z^p P_{m,
p}
\end{equation}

With this identity we can rewrite \eref[eq_dF_dtau_complete] as
\begin{equation}
\begin{aligned}
\frac{\partial F}{\partial \tau} =
(s - 1) \sum_{m=0}^{\infty} \sum_{p=0}^{\infty} s^m z^p \left( a P_{m,p} \right)
+ (z - 1) \sum_{m=0}^{\infty} \sum_{p=0}^{\infty} s^m z^p b \gamma m P_{m,p}
\\
+ \left( s^{-1} - 1 \right) \sum_{m=0}^{\infty} \sum_{p=0}^{\infty} s^m z^p
\gamma m P_{m,p}
+ \left( z^{-1} - 1 \right) \sum_{m=0}^{\infty} \sum_{p=0}^{\infty} s^m z^p
\gamma p P_{m,p}.
\end{aligned}
\label{eq_dF_dtau_identity}
\end{equation}

Another useful identity can be derived if we note that
\begin{equation}
\sum_{m=0}^{\infty} \sum_{p=0}^{\infty} s^m z^p m P_{m, p} =
\sum_{m=0}^{\infty} \sum_{p=0}^{\infty} z^p s \frac{\partial s^m}{\partial s}
P_{m, p}.
\end{equation}
But since $z^p$ and $P_{m, p}$ do not depend on $s$ we can include these terms
inside the derivative, obtaining
\begin{equation}
\sum_{m=0}^{\infty} \sum_{p=0}^{\infty} s^m z^p m P_{m, p} =
s \frac{\partial}{\partial s}\left( \sum_{m=0}^{\infty} \sum_{p=0}^{\infty} s^m z^p P_{m, p} \right) =
s \frac{\partial}{\partial s}F.
\end{equation}

Using this identity in \eref[eq_dF_dtau_identity] allow us to remove all the
sums, obtaining a single PDE of the form
\begin{equation}
\frac{\partial F}{\partial \tau} =
(s - 1) a F + (z - 1) b \gamma s \frac{\partial F}{\partial s}
+ \left( s^{-1} - 1 \right) \gamma s \frac{\partial F}{\partial s}
+ \left( z^{-1} - 1 \right) z \frac{\partial F}{\partial Z}.
\end{equation}
Rearranging terms we obtain
\begin{equation}
\frac{\partial F}{\partial \tau} =
a (sF - s)
+ b \gamma \left( z s \frac{\partial F}{\partial s} - s \frac{\partial
F}{\partial s} \right)
+ \gamma \left( \frac{\partial F}{\partial s} - s \frac{\partial F}{\partial s}
\right)
+ \left( \frac{\partial F}{\partial z} - z \frac{\partial F}{\partial z}
\right).
\label{eq_df_dtau_PDE}
\end{equation}

With the generating function we were able to pass from an infinite system of PDE
for each mRNA and protein copy number to a single PDE. We now have to find a
solution for this equation and then infer back the probability distribution
given the definition of the generating function.

We now note a pattern in \eref[eq_df_dtau_PDE]. If we define $u \equiv s - 1$ and $v \equiv z - 1$, which satisfy
\begin{align}
  \frac{\partial}{\partial s} = \frac{\partial}{\partial u},\\
  \frac{\partial}{\partial z} = \frac{\partial}{\partial v},
\end{align}
we can rewrite \eref[eq_df_dtau_PDE] as
\begin{equation}
  \frac{\partial F}{\partial \tau} =
  a u F
  + b \gamma v \left( u + 1  \right) \frac{\partial F}{\partial u}
  - \gamma u \frac{\partial F}{\partial u}
  - v \frac{\partial F}{\partial v}.
\end{equation}

We now divide by $v$ and rearrange terms obtaining
\begin{equation}
  \frac{\partial F}{\partial v}
  - \gamma \left[ b (u + 1) - \frac{u}{v} \right] \frac{\partial F}{\partial u}
  + \frac{1}{v} \frac{\partial F}{\partial \tau}
  = a \frac{u}{v} F,
  \label{eq_swain_2}
\end{equation}
\manuelComment{which is Eq. (2) in \cite{Shahrezaei2008}.}

\eref[eq_swain_2] is a semi-linear PDE that can be solved using the method of
the characteristics (this method is described in the Appendix
\ref{seq_method_characteristics}). In order to apply this method we write
\eref[eq_swain_2] as
\begin{equation}
  A(v, u, \tau) \frac{\partial F}{\partial v}
  + B(v, u, \tau) \frac{\partial F}{\partial u}
  + C(v, u, \tau) \frac{\partial F}{\partial \tau}
  = f(u, v, \tau, F),
\end{equation}
where
\begin{equation}
  \begin{aligned}
  A &= 1,\\
  B &= - \gamma \left[ b (1 + u) - \frac{u}{v} \right],\\
  C &= \frac{1}{v},\\
  f &= a \frac{u}{v} F.
  \label{eq_coef_definition}
  \end{aligned}
\end{equation}

The method of characteristics requires us to solve the PDE along the
characteristic lines. To do so we can parametrize the characteristics with
parameter $r$. This allow us to write the equality
\begin{equation}
  {{dv \over dr} \over A(v, u, \tau)} =
  {{du \over dr} \over B(v, u, \tau)} =
  {{d\tau \over dr} \over C(v, u, \tau)} =
  {{dF \over dr} \over f(v, u, \tau, F)}.
  \label{eq_lagrange_charpit}
\end{equation}
Substituting \eref[eq_coef_definition] into \eref[eq_lagrange_charpit] gives
\begin{equation}
  {{dv \over dr} \over 1} =
  {{du \over dr} \over - \gamma \left[ b (1 + u) - \frac{u}{v} \right]} =
  {{d\tau \over dr} \over {1 \over v}} =
  {{dF \over dr} \over a \frac{u}{v} F}.
\end{equation}
Or, if a particular parametrization $r$ of the curves is fixed, then these
equations can be written as a system of ODEs as
\begin{align}
  \frac{\partial v}{\partial r} &= 1,
  \label{eq_ODE_1}\\
  \frac{\partial u}{\partial r} &= - \gamma \left[ b (1 + u) - \frac{u}{v}
  \right],
  \label{eq_ODE_2}\\
  \frac{\partial \tau}{\partial r} &= \frac{1}{v},
  \label{eq_ODE_3}\\
  \frac{\partial F}{\partial r} &= \frac{a u}{v} F.
  \label{eq_ODE_4}
\end{align}

\subsubsection{Solving system of ODEs}

Once we find the system of ODEs that integrate the variables along the
characteristic line we can try to solve this system in order to find the general
integral surface that solves the PDE. The initial conditions for the system are
given by $\tau = 0$, $u = u_0$, $v = v_0$. Let's start with \eref[eq_ODE_1].
For this equation we use separation of variables which simply gives
\begin{equation}
  {\partial v \over \partial r} = 1 \Rightarrow
  r = v + C_1,
  \label{eq_ODE_sol_1}
\end{equation}
where $C_1$ is an integration constant.

For \eref[eq_ODE_3] we can write
\begin{equation}
  {\partial \tau \over \partial r} = {1 \over v} \Rightarrow
  {\partial r \over \partial \tau} = v.
\end{equation}
If we now substitute \eref[eq_ODE_sol_1] we find
\begin{equation}
  {\partial r \over \partial \tau} = r - C_1.
\end{equation}
Using separation of variables on this equation we get
\begin{equation}
  \int {dr \over r - C_1} = \int d\tau,
\end{equation}
which when integrated gives
\begin{equation}
  \ln (r - C_1) = \tau + C_2'.
\end{equation}
where $C_2'$ is an integration constant. If we now solve for $r$ we find
\begin{equation}
  r = C_2e^{\tau} + C_1,
  \label{eq_ODE_sol_3}
\end{equation}
where $C_2 \equiv e^{C_2'}$.

If we substitute \eref[eq_ODE_sol_3] back on \eref[eq_ODE_sol_1] we obtain a
functional relationship between $v$ and $\tau$ of the form
\begin{equation}
  C_2 e^{\tau} + C_1 = v + C_1 \Rightarrow
  v = C_2 e^{tau}.
\end{equation}

Using the initial conditions $\tau = 0$, $v=v_0$ we find that $C_2 = v_0$,
therefore the functional relationship between $\tau$ and $v$ is of the form
\begin{equation}
  v = v_0 e^{\tau}.
  \label{eq_v_tau_relation}
\end{equation}

Using \eref[eq_ODE_1] and \eref[eq_ODE_2] we can compute
$\partial u / \partial v$ as
\begin{equation}
  {\partial u \over \partial v} =
  - \gamma \left[ b (1 + u) - {u \over v} \right].
\end{equation}
This is a first order linear ODE which we can rewrite as
\begin{equation}
  {du \over dv} = -\gamma b
  - \gamma b u
  + {\gamma u \over v} =
  -\gamma b
  + u \left( {\gamma \over v} - \gamma b \right).
\end{equation}
Grouping terms with respect to $u$ gives
\begin{equation}
  {du \over dv} + \left( \gamma b - {\gamma \over v} \right) u = - \gamma b.
  \label{eq_dudv}
\end{equation}
\eref[eq_dudv] can be solved by the integrating factor method. In this case
the integrating factor is given by
\begin{equation}
  e^{\int \gamma \left( b - {1 \over v} \right) dv} =
  e^{\gamma b v - \gamma \ln v} = v^{-\gamma} e^{\gamma b v}
  \label{eq_dudv_integ_factor}
\end{equation}

Multiplying both sides of \eref[eq_dudv] by \eref[eq_dudv_integ_factor] we find
that the left hand side is transformed into the derivative of the integrating
factor times $u$, i.e.
\begin{equation}
  {d \over dv}\left[ v^{-\gamma} e^{\gamma b v} \cdot u \right] =
  -\gamma b v^{-\gamma} e^{-\gamma b v}.
\end{equation}

We can now integrate both sides, obtaining
\begin{equation}
u v^{-\gamma} e^{\gamma b v} =
-\gamma b \int dv \; v^{-\gamma} e^{\gamma b v} + C,
\end{equation}
where $C$ is an integrating constant. Solving for $u$ we find
\begin{equation}
  u(v) = v^{\gamma} e^{-\gamma b v} \left[ C -
  \gamma b \int dv {e^{\gamma b v} \over v^{\gamma}} \right].
  \label{eq_implicit_sol_u}
\end{equation}
\manuelComment{Eq. (26)}

To solve the integral we expand the exponential as a Taylor series, i.e.
\begin{equation}
  e^{-\gamma b v} = \sum_{n=0}^{\infty} {\left( \gamma b v \right)^n \over n!}.
\end{equation}

Using this on the integral in \eref[eq_implicit_sol_u] gives
\begin{equation}
  \int dv {e^{\gamma b v} \over v^{\gamma}} =
  \int dv {1 \over v^{\gamma}}
  \sum_{n=0}^{\infty} {\left( \gamma b v \right)^n \over n!} =
  \sum_{n=0}^{\infty} {\left( \gamma b v \right)^n \over n!}
  \int dv \; v^{n - \gamma}
\end{equation}
We can take out the terms that do not depend on $v$ out of the integral to
obtain
\begin{equation}
\int dv {1 \over v^{\gamma}}
  \sum_{n=0}^{\infty} {\left( \gamma b v \right)^n \over n!} =
  \sum_{n=0}^{\infty} {\left( \gamma b \right)^n \over n!}
  \int dv \; v^{n - \gamma} =
  \sum_{n=0}^{\infty} {\left( \gamma b \right)^n \over n!}
  {v^{n - \gamma + 1} \over n - \gamma + 1}.
  \label{eq_integral_implicit_sol_u}
\end{equation}

Substituting \eref[eq_integral_implicit_sol_u] into \eref[eq_implicit_sol_u]
gives
\begin{equation}
  u(v) = v^\gamma e^{-\gamma b v}
  \left[ C - \gamma b \sum_{n=0}^\infty {\left( \gamma b \right)^n \over n!}
  {v^{n - \gamma + 1} \over n - \gamma + 1} \right].
\end{equation}
This can be simplified to obtain
\begin{equation}
  u(v) = e^{-\gamma b v} \left[ C v^{\gamma}
  - \sum_{n=0}^\infty {\left( \gamma b v \right)^{n+1} \over
  n! \left( n - \gamma + 1 \right)} \right].
\end{equation}
\manuelComment{Eq (27) in the SI.}
