\section{Derivation of the two stage promoter equation}

Shahreaei and Swain derive the full analytical protein distribution for a two
stage (i.e. an unregulated promoter) and a three stage (i.e. a promoter that
transitions between active and inactive) promoter \cite{Shahrezaei2008}. In
this section wi will follow the derivation augmenting the details at each step
for clarity.

First let us write the chemical master equation for this system. Let $p$ and $m$
be the protein and mRNA copy numbers, respectively, and $P_{m,p}(t)$ be the
probability of having $m$ mRNA and $p$ proteins at time t. Then we can use the
``spread the butter'' approach to write the discrete difference equation
\begin{equation}
\begin{aligned}
P_{m,p}(t + \Delta t) =
P_{m,p}(t)
+ \overbrace{r_m \Delta t \left[ P_{m-1,p}(t) - P_{m,p}(t) \right]}^\text{mRNA
production}
+ \overbrace{r_p m
\Delta t \left[ P_{m, p-1}(t) - P_{m, p}(t) \right]}^\text{protein
production}\\
+ \underbrace{\gamma_m \Delta t \left[ (m + 1) P_{m+1,p}(t) - m P_{m, p}(t)
\right]}_\text{mRNA degradation}
+ \underbrace{\gamma_p \Delta t \left[ (p + 1) P_{m, p+1}(t) - p P_{m, p}(t)
\right]}_\text{protein degradation},
\end{aligned}
\end{equation}
where $r_m$ and $r_p$ are the mRNA and protein production rates respectively,
$\gamma_m$ and $\gamma_p$ are the mRNA and protein degradation rates
respectively, and $\Delta t$ is a time interval small enough so that only one
event can take place.

We rearrange the terms, divide both sides by $\Delta t$ , obtaining
\begin{equation}
  \begin{aligned}
\frac{P_{m,p}(t + \Delta t) - P_{m,p}(t)}{\Delta t} = r_m \left[ P_{m-1,p}(t) -
P_{m,p}(t) \right] + r_p m \left[ P_{m, p-1}(t) - P_{m, p}(t) \right]\\
+ \gamma_m \left[ (m + 1) P_{m+1,p}(t) - m P_{m, p}(t) \right] + \gamma_p \left[
 (p + 1) P_{m, p+1}(t) - p P_{m, p}(t) \right].
  \end{aligned}
\end{equation}
We now take the limit $\Delta t \rightarrow 0$ to obtain the final form of the
chemical master equation
\begin{equation}
\begin{aligned}
\frac{\partial P_{m,p}}{\partial t} = r_m \left[ P_{m-1,p}(t) - P_{m,p}(t)
\right] + r_p m \left[ P_{m, p-1}(t) - P_{m, p}(t) \right]\\
+ \gamma_m \left[ (m + 1) P_{m+1,p}(t) - m P_{m, p}(t) \right] + \gamma_p \left[
(p + 1) p_{m, p+1}(t) - p p_{m, p}(t) \right].
  \end{aligned}
\end{equation}

Let us now divide by the slowest rate, i.e. $\gamma_r$
\begin{equation}
\begin{aligned}
\frac{1}{\gamma_p} \frac{\partial P_{m,p}}{\partial t} =
\frac{r_m}{\gamma_p} \left[ P_{m-1,p}(t) - P_{m,p}(t) \right]
+ \frac{r_p}{\gamma_p} m  \left[ P_{m, p-1}(t) - P_{m, p}(t) \right]\\
+ \frac{\gamma_m}{\gamma_p} \left[ (m + 1) P_{m+1,p}(t) - m P_{m, p}(t)
\right]
+ \left[ (p + 1) P_{m, p+1}(t) - p P_{m, p}(t) \right].
\end{aligned}
\label{eq_cme_over_gammap}
\end{equation}
We now introduce the following variables:
\begin{align}
  a \equiv \frac{r_m}{\gamma_p}\\
  b \equiv \frac{r_p}{\gamma_m}\\
  \gamma \equiv \frac{\gamma_m}{\gamma_p}\\
  \tau \equiv \gamma_p \cdot t
\end{align}
Substituting these variables into \eref[eq_cme_over_gammap] we obtain
\begin{equation}
\begin{aligned}
\frac{\partial P_{m,p}}{\partial \tau} =
a \left[ P_{m-1,p}(t) - P_{m,p}(t) \right]
+ b \gamma m  \left[ P_{m, p-1}(t) - P_{m, p}(t) \right]\\
+ \gamma \left[ (m + 1) P_{m+1,p}(t) - m P_{m, p}(t) \right]
+ \left[ (p + 1) P_{m, p+1}(t) - p P_{m, p}(t) \right].
\end{aligned}
\label{eq_cme_tau}
\end{equation}
Note that we used $\frac{1}{\gamma_p}\frac{\partial}{\partial t} =
\frac{\partial}{\partial \tau}$.

We now define the generating function
\begin{equation}
F\left[ s, z \right] = \sum_{m=0}^{\infty} \sum_{p=0}^{\infty} s^m z^p P_{m, p},
\end{equation}
where from now on we abbreviate $P_{m, p}(t)$ as $P_{m, p}$. The generating
function will allow us to write a single PDE rather than an infinite system of
PDEs for each mRNA and protein copy number. The generating function time
derivative is given by
\begin{equation}
\frac{\partial F}{\partial \tau} =
\frac{\partial}{\partial \tau} \sum_{m=0}^{\infty} \sum_{p=0}^{\infty} s^m z^p
P_{m, p} =
\sum_{m=0}^{\infty}
\sum_{p=0}^{\infty} s^m z^p \frac{\partial}{\partial \tau} P_{m, p}.
\label{eq_dF_dtau}
\end{equation}
We now substitute \eref[eq_cme_tau] into \eref[eq_dF_dtau]
\begin{equation}
\begin{aligned}
\frac{\partial F}{\partial \tau} =
\sum_{m=0}^{\infty} \sum_{p=0}^{\infty} s^m z^p a \left[ P_{m-1,p}(t) -
P_{m,p}(t) \right]
+ b \gamma m  \left[ P_{m, p-1}(t) - P_{m, p}(t) \right]\\
+ \gamma \left[ (m + 1) P_{m+1,p}(t) - m P_{m, p}(t) \right]
+ \left[ (p + 1) P_{m, p+1}(t) - p P_{m, p}(t) \right].
\end{aligned}
\label{eq_dF_dtau_complete}
\end{equation}

Note that
\begin{equation}
\sum_{m=0}^{\infty} \sum_{p=0}^{\infty} s^m z^p \left( P_{m+k, p} - P_{m, p}
\right) =
\sum_{m=0}^{\infty} \sum_{p=0}^{\infty} s^m z^p P_{m+k, p}
- \sum_{m=0}^{\infty} \sum_{p=0}^{\infty} s^m z^p P_{m, p}.
\end{equation}
We can now factorize $s^{-k}$ from the first term on the left hand side and
redefine the variable to sum over.
\begin{equation}
\sum_{m=0}^{\infty} \sum_{p=0}^{\infty} s^m z^p \left( P_{m+k, p} - P_{m, p}
\right) =
s^{-k} \sum_{(m+k)=0}^{\infty} \sum_{p=0}^{\infty} s^{m+k} z^p P_{m+k, p}
- \sum_{m=0}^{\infty} \sum_{p=0}^{\infty} s^m z^p P_{m, p}.
\end{equation}
But since both sums on the left hand side are taken over the same ranges we can
write it as
\begin{equation}
\sum_{m=0}^{\infty} \sum_{p=0}^{\infty} s^m z^p \left( P_{m+k, p} - P_{m, p}
\right) =
\left( s^{-k} - 1 \right) \sum_{m=0}^{\infty} \sum_{p=0}^{\infty} s^m z^p P_{m,
p}
\end{equation}

With this identity we can rewrite \eref[eq_dF_dtau_complete] as
\begin{equation}
\begin{aligned}
\frac{\partial F}{\partial \tau} =
(s - 1) \sum_{m=0}^{\infty} \sum_{p=0}^{\infty} s^m z^p \left( a P_{m,p} \right)
+ (z - 1) \sum_{m=0}^{\infty} \sum_{p=0}^{\infty} s^m z^p b \gamma m P_{m,p}
\\
+ \left( s^{-1} - 1 \right) \sum_{m=0}^{\infty} \sum_{p=0}^{\infty} s^m z^p
\gamma m P_{m,p}
+ \left( z^{-1} - 1 \right) \sum_{m=0}^{\infty} \sum_{p=0}^{\infty} s^m z^p
\gamma p P_{m,p}.
\end{aligned}
\label{eq_dF_dtau_identity}
\end{equation}

Another useful identity can be derived if we note that
\begin{equation}
\sum_{m=0}^{\infty} \sum_{p=0}^{\infty} s^m z^p m P_{m, p} =
\sum_{m=0}^{\infty} \sum_{p=0}^{\infty} z^p s \frac{\partial s^m}{\partial s}
P_{m, p}.
\end{equation}
But since $z^p$ and $P_{m, p}$ do not depend on $s$ we can write
\begin{equation}
\sum_{m=0}^{\infty} \sum_{p=0}^{\infty} s^m z^p m P_{m, p} =
s \frac{\partial}{\partial s}\left( \sum_{m=0}^{\infty} \sum_{p=0}^{\infty} s^m z^p P_{m, p} \right) =
s \frac{\partial}{\partial s}F.
\end{equation}

Using this identity in \eref[eq_dF_dtau_identity] allow us to remove all the
sums, obtaining a single PDE of the form
\begin{equation}
\frac{\partial F}{\partial \tau} =
(s - 1) a F + (z - 1) b \gamma s \frac{\partial F}{\partial s}
+ \left( s^{-1} - 1 \right) \gamma s \frac{\partial F}{\partial s}
+ \left( z^{-1} - 1 \right) z \frac{\partial F}{\partial Z}.
\end{equation}
Rearranging terms we obtain
\begin{equation}
\frac{\partial F}{\partial \tau} =
a (sF - s)
+ b \gamma \left( z s \frac{\partial F}{\partial s} - s \frac{\partial
F}{\partial s} \right)
+ \gamma \left( \frac{\partial F}{\partial s} - s \frac{\partial F}{\partial s}
\right)
+ \left( \frac{\partial F}{\partial z} - z \frac{\partial F}{\partial z}
\right).
\label{eq_df_dtau_PDE}
\end{equation}

With the generating function we were able to pass from an infinite system of PDE
for each mRNA and protein copy number to a single PDE. We now have to find a
solution for this equation and then infer back the probability distribution
given the definition of the generating function.

We now note a pattern in \eref[eq_df_dtau_PDE]. If we define $u \equiv s - 1$ and $v \equiv z - 1$, which satisfy
\begin{align}
  \frac{\partial}{\partial s} = \frac{\partial}{\partial u},\\
  \frac{\partial}{\partial z} = \frac{\partial}{\partial v},
\end{align}
we can rewrite \eref[eq_df_dtau_PDE] as
\begin{equation}
  \frac{\partial F}{\partial \tau} =
  a u F
  + b \gamma v \left( u + 1  \right) \frac{\partial F}{\partial u}
  - \gamma u \frac{\partial F}{\partial u}
  - v \frac{\partial F}{\partial v}.
\end{equation}

We now divide by $v$ and rearrange terms obtaining
\begin{equation}
  \frac{\partial F}{\partial v}
  - \gamma \left[ b (u + 1) - \frac{u}{v} \right] \frac{\partial F}{\partial u}
  + \frac{1}{v} \frac{\partial F}{\partial \tau}
  = a \frac{u}{v} F,
  \label{eq_swain_2}
\end{equation}
which is Eq. (2) in \cite{Shahrezaei2008}.

\eref[eq_swain_2] is a semi-linear PDE that can be solved using the method of
the characteristics (\manuelComment{write appendix about method}). For this we
write \eref[eq_swain_2] as
\begin{equation}
  A(v, u, \tau) \frac{\partial F}{\partial v}
  + B(v, u, \tau) \frac{\partial F}{\partial u}
  + C(v, u, \tau) \frac{\partial F}{\partial \tau}
  = f(u, v, \tau, F),
\end{equation}
where
\begin{align}
  A &= 1,\\
  B &= - \gamma \left[ b (1 + u) - \frac{u}{v} \right],\\
  C &= \frac{1}{v},\\
  f &= a \frac{u}{v} F.
\end{align}

If we parametrize the integral surface that solves this PDE with parameter $r$
we can write a system of four PDE \manuelComment{proof in the method
explanation}
\begin{align}
  \frac{\partial v}{\partial r} = 1,\\
  \frac{\partial u}{\partial r} = - \gamma \left[ b (1 + u) - \frac{u}{v}
  \right],\\
  \frac{\partial \tau}{\partial r} = \frac{1}{v},\\
  \frac{\partial F}{\partial r} = \frac{a u}{v} F.
\end{align}
