\section{Derivation of the two stage promoter equation}

Shahreaei and Swain derive the full analytical protein distribution for a two
stage (i.e. an unregulated promoter) and a three stage (i.e. a promoter that
transitions between active and inactive) promoter \cite{Shahrezaei2008}. In
this section wi will follow the derivation augmenting the details at each step
for clarity.

\subsection{From the master equation}

First let us write the chemical master equation for this system. Let $p$ and $m$
be the protein and mRNA copy numbers, respectively, and $P_{m,p}(t)$ be the
probability of having $m$ mRNA and $p$ proteins at time t. Then we can use the
``spread the butter'' approach to write the discrete difference equation
\begin{equation}
\begin{aligned}
P_{m,p}(t + \Delta t) =
P_{m,p}(t) +
\overbrace{r_m \Delta t
\left[ P_{m-1,p}(t) - P_{m,p}(t) \right]}^\text{mRNA production}
+ \overbrace{r_p m \Delta t
\left[ P_{m, p-1}(t) - P_{m, p}(t) \right]}^\text{protein production}\\
+ \underbrace{\gamma_m \Delta t
\left[ (m + 1) P_{m+1,p}(t) - m P_{m, p}(t) \right]}_\text{mRNA degradation}
+ \underbrace{\gamma_p \Delta t
\left[ (p + 1) P_{m, p+1}(t) - p P_{m, p}(t) \right]}_\text{protein
degradation},
\end{aligned}
\end{equation}
where $r_m$ and $r_p$ are the mRNA and protein production rates respectively,
$\gamma_m$ and $\gamma_p$ are the mRNA and protein degradation rates
respectively, and $\Delta t$ is a time interval small enough so that only one
event can take place.

We rearrange the terms, divide both sides by $\Delta t$ , obtaining
\begin{equation}
  \begin{aligned}
\frac{P_{m,p}(t + \Delta t) - P_{m,p}(t)}{\Delta t} =
r_m \left[ P_{m-1,p}(t) - P_{m,p}(t) \right]
+ r_p m \left[ P_{m, p-1}(t) - P_{m, p}(t) \right]\\
+ \gamma_m \left[ (m + 1) P_{m+1,p}(t) - m P_{m, p}(t) \right]
+ \gamma_p \left[ (p + 1) P_{m, p+1}(t) - p P_{m, p}(t) \right].
  \end{aligned}
\end{equation}
Note that the production of protein is taken per mRNA.
We now take the limit $\Delta t \rightarrow 0$ to obtain the final form of the
chemical master equation
\begin{equation}
  \begin{aligned}
\frac{\partial P_{m,p}}{\partial t} =
r_m \left[ P_{m-1,p}(t) - P_{m,p}(t) \right] +
r_p m \left[ P_{m, p-1}(t) - P_{m, p}(t) \right]\\
+ \gamma_m \left[ (m + 1) P_{m+1,p}(t)
- m P_{m, p}(t) \right]
+ \gamma_p \left[ (p + 1) P_{m, p+1}(t) - p P_{m, p}(t) \right].
  \end{aligned}
\end{equation}

\manuelComment{Eq. 1 of the paper.}

Let us now divide by the slowest rate, i.e. $\gamma_p$
\begin{equation}
\begin{aligned}
\frac{1}{\gamma_p} \frac{\partial P_{m,p}}{\partial t} =
\frac{r_m}{\gamma_p} \left[ P_{m-1,p}(t) - P_{m,p}(t) \right]
+ \frac{r_p}{\gamma_p} m  \left[ P_{m, p-1}(t) - P_{m, p}(t) \right]\\
+ \frac{\gamma_m}{\gamma_p} \left[ (m + 1) P_{m+1,p}(t) - m P_{m, p}(t) \right]
+ \left[ (p + 1) P_{m, p+1}(t) - p P_{m, p}(t) \right].
\end{aligned}
\label{eq_cme_over_gammap}
\end{equation}
We now introduce the following variables:
\begin{align}
  a \equiv \frac{r_m}{\gamma_p}\\
  b \equiv \frac{r_p}{\gamma_m}\\
  \gamma \equiv \frac{\gamma_m}{\gamma_p}\\
  \tau \equiv \gamma_p \cdot t
\end{align}
Substituting these variables into \eref[eq_cme_over_gammap] we obtain
\begin{equation}
\begin{aligned}
\frac{\partial P_{m,p}}{\partial \tau} =
a \left[ P_{m-1,p}(t) - P_{m,p}(t) \right]
+ b \gamma m  \left[ P_{m, p-1}(t) - P_{m, p}(t) \right]\\
+ \gamma \left[ (m + 1) P_{m+1,p}(t) - m P_{m, p}(t) \right]
+ \left[ (p + 1) P_{m, p+1}(t) - p P_{m, p}(t) \right].
\end{aligned}
\label{eq_cme_tau}
\end{equation}
Note that we used $\frac{1}{\gamma_p}\frac{\partial}{\partial t} =
\frac{\partial}{\partial \tau}$.

We now define the generating function
\begin{equation}
F\left[ s, z \right] = \sum_{m=0}^{\infty} \sum_{p=0}^{\infty} s^m z^p P_{m, p},
\label{eq_generating_function}
\end{equation}
where from now on we abbreviate $P_{m, p}(t)$ as $P_{m, p}$. The generating
function will allow us to write a single PDE rather than an infinite system of
PDEs for each mRNA and protein copy number. The generating function time
derivative is given by
\begin{equation}
\frac{\partial F}{\partial \tau} =
\frac{\partial}{\partial \tau} \sum_{m=0}^{\infty} \sum_{p=0}^{\infty} s^m z^p
P_{m, p} =
\sum_{m=0}^{\infty}
\sum_{p=0}^{\infty} s^m z^p \frac{\partial}{\partial \tau} P_{m, p}.
\label{eq_dF_dtau}
\end{equation}
We now substitute \eref[eq_cme_tau] into \eref[eq_dF_dtau]
\begin{equation}
\begin{aligned}
\frac{\partial F}{\partial \tau} =
\sum_{m=0}^{\infty} \sum_{p=0}^{\infty} s^m z^p
\left\{ a \left[ P_{m-1,p} - P_{m,p} \right]
+ b \gamma m  \left[ P_{m, p-1} - P_{m, p} \right] \right. \\
\left. + \gamma \left[ (m + 1) P_{m+1,p} - m P_{m, p} \right]
+ \left[ (p + 1) P_{m, p+1} - p P_{m, p} \right] \right\}.
\end{aligned}
\label{eq_dF_dtau_complete}
\end{equation}

In order to make progress with this equation we note that we can distribute
the terms in parenthesis as
\begin{equation}
\sum_{m=0}^{\infty} \sum_{p=0}^{\infty} s^m z^p \left( P_{m+1, p} - P_{m, p}
\right) =
\sum_{m=0}^{\infty} \sum_{p=0}^{\infty} s^m z^p P_{m+1, p}
- \sum_{m=0}^{\infty} \sum_{p=0}^{\infty} s^m z^p P_{m, p}.
\label{eq_split_sum}
\end{equation}
We can now factorize $s^{-1}$ from the first term on the left hand side of
\eref[eq_split_sum] and redefine the variable to sum over.
\begin{equation}
\sum_{m=0}^{\infty} \sum_{p=0}^{\infty} s^m z^p \left( P_{m+1, p} - P_{m, p}
\right) =
s^{-1} \sum_{(m+1)=0}^{\infty} \sum_{p=0}^{\infty} s^{m+1} z^p P_{m+1, p}
- \sum_{m=0}^{\infty} \sum_{p=0}^{\infty} s^m z^p P_{m, p}.
\end{equation}
But since both sums on the left hand side are taken over the same ranges we can
write it as
\begin{equation}
\sum_{m=0}^{\infty} \sum_{p=0}^{\infty} s^m z^p \left( P_{m+1, p} - P_{m, p}
\right) =
\left( s^{-1} - 1 \right) \sum_{m=0}^{\infty} \sum_{p=0}^{\infty} s^m z^p P_{m,
p}
\end{equation}

With this identity we can rewrite \eref[eq_dF_dtau_complete] as
\begin{equation}
\begin{aligned}
\frac{\partial F}{\partial \tau} =
(s - 1) \sum_{m=0}^{\infty} \sum_{p=0}^{\infty} s^m z^p \left( a P_{m,p} \right)
+ (z - 1) \sum_{m=0}^{\infty} \sum_{p=0}^{\infty} s^m z^p b \gamma m P_{m,p}
\\
+ \left( s^{-1} - 1 \right) \sum_{m=0}^{\infty} \sum_{p=0}^{\infty} s^m z^p
\gamma m P_{m,p}
+ \left( z^{-1} - 1 \right) \sum_{m=0}^{\infty} \sum_{p=0}^{\infty} s^m z^p
\gamma p P_{m,p}.
\end{aligned}
\label{eq_dF_dtau_identity}
\end{equation}

Another useful identity can be derived if we note that
\begin{equation}
\sum_{m=0}^{\infty} \sum_{p=0}^{\infty} s^m z^p m P_{m, p} =
\sum_{m=0}^{\infty} \sum_{p=0}^{\infty} z^p s \frac{\partial s^m}{\partial s}
P_{m, p}.
\end{equation}
But since $z^p$ and $P_{m, p}$ do not depend on $s$ we can include these terms
inside the derivative, obtaining
\begin{equation}
\sum_{m=0}^{\infty} \sum_{p=0}^{\infty} s^m z^p m P_{m, p} =
s \frac{\partial}{\partial s}\left( \sum_{m=0}^{\infty} \sum_{p=0}^{\infty} s^m z^p P_{m, p} \right) =
s \frac{\partial}{\partial s}F.
\end{equation}

Using this identity in \eref[eq_dF_dtau_identity] allow us to remove all the
sums, obtaining a single PDE of the form
\begin{equation}
\frac{\partial F}{\partial \tau} =
(s - 1) a F + (z - 1) b \gamma s \frac{\partial F}{\partial s}
+ \left( s^{-1} - 1 \right) \gamma s \frac{\partial F}{\partial s}
+ \left( z^{-1} - 1 \right) z \frac{\partial F}{\partial Z}.
\end{equation}
Rearranging terms we obtain
\begin{equation}
\frac{\partial F}{\partial \tau} =
a (sF - s)
+ b \gamma \left( z s \frac{\partial F}{\partial s} - s \frac{\partial
F}{\partial s} \right)
+ \gamma \left( \frac{\partial F}{\partial s} - s \frac{\partial F}{\partial s}
\right)
+ \left( \frac{\partial F}{\partial z} - z \frac{\partial F}{\partial z}
\right).
\label{eq_df_dtau_PDE}
\end{equation}

With the generating function we were able to pass from an infinite system of PDE
for each mRNA and protein copy number to a single PDE. We now have to find a
solution for this equation and then infer back the probability distribution
given the definition of the generating function.

We now note a pattern in \eref[eq_df_dtau_PDE]. If we define $u \equiv s - 1$ and $v \equiv z - 1$, which satisfy
\begin{align}
  \frac{\partial}{\partial s} = \frac{\partial}{\partial u},\\
  \frac{\partial}{\partial z} = \frac{\partial}{\partial v},
\end{align}
we can rewrite \eref[eq_df_dtau_PDE] as
\begin{equation}
  \frac{\partial F}{\partial \tau} =
  a u F
  + b \gamma v \left( u + 1  \right) \frac{\partial F}{\partial u}
  - \gamma u \frac{\partial F}{\partial u}
  - v \frac{\partial F}{\partial v}.
\end{equation}

We now divide by $v$ and rearrange terms obtaining
\begin{equation}
  \frac{\partial F}{\partial v}
  - \gamma \left[ b (u + 1) - \frac{u}{v} \right] \frac{\partial F}{\partial u}
  + \frac{1}{v} \frac{\partial F}{\partial \tau}
  = a \frac{u}{v} F,
  \label{eq_swain_2}
\end{equation}
\manuelComment{which is Eq. (2) in \cite{Shahrezaei2008}.}

\eref[eq_swain_2] is a semi-linear PDE that can be solved using the method of
the characteristics (this method is described in the Appendix
\ref{seq_method_characteristics}). In order to apply this method we write
\eref[eq_swain_2] as
\begin{equation}
  A(v, u, \tau) \frac{\partial F}{\partial v}
  + B(v, u, \tau) \frac{\partial F}{\partial u}
  + C(v, u, \tau) \frac{\partial F}{\partial \tau}
  = f(u, v, \tau, F),
\end{equation}
where
\begin{equation}
  \begin{aligned}
  A &= 1,\\
  B &= - \gamma \left[ b (1 + u) - \frac{u}{v} \right],\\
  C &= \frac{1}{v},\\
  f &= a \frac{u}{v} F.
  \label{eq_coef_definition}
  \end{aligned}
\end{equation}

The method of characteristics requires us to solve the PDE along the
characteristic lines. To do so we can parametrize the characteristics with
parameter $r$. This allow us to write the equality
\begin{equation}
  {{dv \over dr} \over A(v, u, \tau)} =
  {{du \over dr} \over B(v, u, \tau)} =
  {{d\tau \over dr} \over C(v, u, \tau)} =
  {{dF \over dr} \over f(v, u, \tau, F)}.
  \label{eq_lagrange_charpit}
\end{equation}
Substituting \eref[eq_coef_definition] into \eref[eq_lagrange_charpit] gives
\begin{equation}
  {{dv \over dr} \over 1} =
  {{du \over dr} \over - \gamma \left[ b (1 + u) - \frac{u}{v} \right]} =
  {{d\tau \over dr} \over {1 \over v}} =
  {{dF \over dr} \over a \frac{u}{v} F}.
\end{equation}
Or, if a particular parametrization $r$ of the curves is fixed, then these
equations can be written as a system of ODEs as
\begin{align}
  \frac{d v}{d r} &= 1,
  \label{eq_ODE_1}\\
  \frac{d u}{d r} &= - \gamma \left[ b (1 + u) - \frac{u}{v}
  \right],
  \label{eq_ODE_2}\\
  \frac{d \tau}{d r} &= \frac{1}{v},
  \label{eq_ODE_3}\\
  \frac{d F}{d r} &= \frac{a u}{v} F.
  \label{eq_ODE_4}
\end{align}

\subsubsection{Solving system of ODEs}

Once we find the system of ODEs that integrate the variables along the
characteristic line we can try to solve this system in order to find the general
integral surface that solves the PDE. The initial conditions for the system are
given by $\tau = 0$, $u = u_0$, $v = v_0$. Let's start with \eref[eq_ODE_1].
For this equation we use separation of variables which simply gives
\begin{equation}
  {\partial v \over \partial r} = 1 \Rightarrow
  r = v + C_1,
  \label{eq_ODE_sol_1}
\end{equation}
where $C_1$ is an integration constant.

For \eref[eq_ODE_3] we can write
\begin{equation}
  {\partial \tau \over \partial r} = {1 \over v} \Rightarrow
  {\partial r \over \partial \tau} = v.
\end{equation}
If we now substitute \eref[eq_ODE_sol_1] we find
\begin{equation}
  {\partial r \over \partial \tau} = r - C_1.
\end{equation}
Using separation of variables on this equation we get
\begin{equation}
  \int {dr \over r - C_1} = \int d\tau,
\end{equation}
which when integrated gives
\begin{equation}
  \ln (r - C_1) = \tau + C_2'.
\end{equation}
where $C_2'$ is an integration constant. If we now solve for $r$ we find
\begin{equation}
  r = C_2e^{\tau} + C_1,
  \label{eq_ODE_sol_3}
\end{equation}
where $C_2 \equiv e^{C_2'}$.

If we substitute \eref[eq_ODE_sol_3] back on \eref[eq_ODE_sol_1] we obtain a
functional relationship between $v$ and $\tau$ of the form
\begin{equation}
  C_2 e^{\tau} + C_1 = v + C_1 \Rightarrow
  v = C_2 e^{\tau}.
\end{equation}

Using the initial conditions $\tau = 0$, $v=v_0$ we find that $C_2 = v_0$,
therefore the functional relationship between $\tau$ and $v$ is of the form
\begin{equation}
  v = v_0 e^{\tau}.
  \label{eq_v_tau_relation}
\end{equation}

Using \eref[eq_ODE_1] and \eref[eq_ODE_2] we can compute
$\partial u / \partial v$ as
\begin{equation}
  {\partial u \over \partial v} =
  - \gamma \left[ b (1 + u) - {u \over v} \right].
\end{equation}
This is a first order linear ODE which we can rewrite as
\begin{equation}
  {du \over dv} = -\gamma b
  - \gamma b u
  + {\gamma u \over v} =
  -\gamma b
  + u \left( {\gamma \over v} - \gamma b \right).
\end{equation}
Grouping terms with respect to $u$ gives
\begin{equation}
  {du \over dv} + \left( \gamma b - {\gamma \over v} \right) u = - \gamma b.
  \label{eq_dudv}
\end{equation}
\eref[eq_dudv] can be solved by the integrating factor method. In this case
the integrating factor is given by
\begin{equation}
  e^{\int \gamma \left( b - {1 \over v} \right) dv} =
  e^{\gamma b v - \gamma \ln v} = v^{-\gamma} e^{\gamma b v}
  \label{eq_dudv_integ_factor}
\end{equation}

Multiplying both sides of \eref[eq_dudv] by \eref[eq_dudv_integ_factor] we find
that the left hand side is transformed into the derivative of the integrating
factor times $u$, i.e.
\begin{equation}
  {d \over dv}\left[ v^{-\gamma} e^{\gamma b v} \cdot u \right] =
  -\gamma b v^{-\gamma} e^{-\gamma b v}.
\end{equation}

We can now integrate both sides, obtaining
\begin{equation}
u v^{-\gamma} e^{\gamma b v} =
-\gamma b \int dv \; v^{-\gamma} e^{\gamma b v} + C,
\end{equation}
where $C$ is an integrating constant. Solving for $u$ we find
\begin{equation}
  u(v) = v^{\gamma} e^{-\gamma b v} \left[ C -
  \gamma b \int dv {e^{\gamma b v} \over v^{\gamma}} \right].
  \label{eq_implicit_sol_u}
\end{equation}
\manuelComment{Eq. (26)}

To solve the integral we expand the exponential as a Taylor series, i.e.
\begin{equation}
  e^{-\gamma b v} = \sum_{n=0}^{\infty} {\left( \gamma b v \right)^n \over n!}.
\end{equation}

Using this on the integral in \eref[eq_implicit_sol_u] gives
\begin{equation}
  \int dv {e^{\gamma b v} \over v^{\gamma}} =
  \int dv {1 \over v^{\gamma}}
  \sum_{n=0}^{\infty} {\left( \gamma b v \right)^n \over n!}
\end{equation}
We can take out the terms that do not depend on $v$ out of the integral to
obtain
\begin{equation}
\int dv {1 \over v^{\gamma}}
  \sum_{n=0}^{\infty} {\left( \gamma b v \right)^n \over n!} =
  \sum_{n=0}^{\infty} {\left( \gamma b \right)^n \over n!}
  \int dv \; v^{n - \gamma} =
  \sum_{n=0}^{\infty} {\left( \gamma b \right)^n \over n!}
  {v^{n - \gamma + 1} \over n - \gamma + 1}.
  \label{eq_integral_implicit_sol_u}
\end{equation}

Substituting \eref[eq_integral_implicit_sol_u] into \eref[eq_implicit_sol_u]
gives
\begin{equation}
  u(v) = v^\gamma e^{-\gamma b v}
  \left[ C - \gamma b \sum_{n=0}^\infty {\left( \gamma b \right)^n \over n!}
  {v^{n - \gamma + 1} \over n - \gamma + 1} \right].
\end{equation}
This can be simplified to obtain
\begin{equation}
  u(v) = e^{-\gamma b v} \left[ C v^{\gamma}
  - \sum_{n=0}^\infty {\left( \gamma b v \right)^{n+1} \over
  n! \left( n - \gamma + 1 \right)} \right].
  \label{eq_SI_27_uofv}
\end{equation}
\manuelComment{Eq (27) in the SI.}

The question now becomes how can we evaluate the sum
\begin{equation}
  S(\gamma b v) = \sum_{n=0}^\infty {\left( \gamma b v \right)^{n+1} \over
  n! \left( n - \gamma + 1 \right)}.
\end{equation}
Just as Shahrenzaei \& Swain we will follow Bender \& Orszag and use the
so-called Laplace's method for sums. The idea behind this method is to find the
leading behavior of the sum. What this means is that if there are some very
dominant terms in the sum we can just add those in the vincinity to get a decent
approximation of the sum.

To find the leading behavior of the sum we first need to identify the largest
terms in the series. Let us start by looking at the ratio of the
$(n+1)^{\text{th}}$ term to the $n^{\text{th}}$ term of the sum. This is given
by
\begin{equation}
{{(\gamma b v)^{n+2} \over (n+1)! (n - \gamma + 2)}
\over
{(\gamma b v)^{n+1} \over n! (n - \gamma + 1)}}
=
{\gamma b v \over n + 1} \cdot
{(n - \gamma + 1) \over (n - \gamma + 2)}.
\label{eq_ratio_series}
\end{equation}
From this ratio we can see that since we assumed that $\gamma \gg 1$ we can
discard the second term on the right hand side of \eref[eq_ratio_series] since
\begin{equation}
{(n - \gamma + 1) \over (n - \gamma + 2)} \approx 1.
\end{equation}

Taking this approximation we can see that the ratio is greater than 1 if
$n + 1 < \gamma b v$ and it is less than one if the opposite is true, i.e.
$n + 1 > \gamma b v$. That implies that terms in the series increase until they
reach $n \sim \gamma b v$, to then decrease with increasing $n$.

As $\gamma b v \rightarrow \infty$ the series become sharply peaked around
$n = \gamma b v$. Therefore we could approximate this sum by just summing in the
vicinity of $n = \gamma b v$. This would give us
\begin{align}
  S(\gamma b v) =
\sum_{n=0}^\infty {\left( \gamma b v \right)^{n+1} \over
  n! \left( n - \gamma + 1 \right)} \approx
\sum_{n=\gamma b v (1 - \epsilon)}
    ^{\gamma b v (1 + \epsilon)}
    {\left( \gamma b v \right)^{n+1} \over
  n! \left( n - \gamma + 1 \right)},
\end{align}
with errors increasingly smaller as $\gamma b v \rightarrow \infty$.

If this approximation is valid, that means we can use Stirling's formula to
approximate the factorial term remaining in the sum. If we let
$n \approx \gamma b v + s$, where $s$ is small compared to $\gamma b v$, then by
Stirling's approximation we have
\begin{equation}
  n! \approx \left( {n \over e} \right)^n \sqrt{2 \pi n}.
\end{equation}
Substituting our approximation for $n$ gives
\begin{equation}
  n! \approx (\gamma b v + s)^n e^{-n} \sqrt{2 \pi n}.
\end{equation}
For this equation we left the exponent with $n$ untouched. That violates our
approximation of $n \approx \gamma b v + s$, but we will come back to that
later. For now it is convenient to carry it as $n$ and substitute at the end.

We now factorize out the term $\gamma b v$ obtaining
\begin{equation}
  n! \approx \left[ \gamma b v \left( 1 + {s \over \gamma b v} \right) \right]^n
  e^{-n} \sqrt{2 \pi n}.
\end{equation}
We can rewrite the term in parenthesis in a convenient way as
\begin{equation}
  n! \approx (\gamma b v)^n
  \exp \left\{ n \ln \left( 1 + {s \over \gamma b v} \right) \right\}
  e^{-n} \sqrt{2 \pi n}.
\end{equation}
The term in the exponential can be Taylor expanded up to second order as
\begin{equation}
n \ln \left( 1 + {s \over \gamma b v} \right) \approx
n \left[ {s \over \gamma b v}
- {1 \over 2} \left( {s \over \gamma b v}  \right)^2 \right].
\end{equation}

Putting all these together we obtain an approximation of the form
\begin{equation}
  n! \approx (\gamma b v)^n
             e^{-n}
             e^{n {s \over \gamma b v}}
             e^{-{n \over 2} {s^2 \over (\gamma b v)^2}}
             \sqrt{2 \pi n}.
\end{equation}
Let's now substitute what we didn't do at the beginning for the exponents.
Let's replace $n = \gamma b v + s$ in the exponents obtaining
\begin{equation}
  e^{-n}
  e^{n {s \over \gamma b v}}
  e^{-{n \over 2} {s^2 \over (\gamma b v)^2}} =
  \exp \left\{ -\gamma b v - s
               + (\gamma b v + s){s \over \gamma b v}
               - {(\gamma b v + s) \over 2} {s^2 \over (\gamma b v)^2} \right\}.
\end{equation}
Expanding the terms in the exponent gives
\begin{equation}
\exp \left\{ -\gamma b v - s
             + (\gamma b v + s){s \over \gamma b v}
             - {(\gamma b v + s) \over 2} {s^2 \over (\gamma b v)^2} \right\} =
\exp \left\{ -\gamma b v - s
             + s
             + {s^2 \over \gamma b v}
             - {1 \over 2} {s^2 \over \gamma b v}
             - {1 \over s} {s^3 \over (\gamma b v)^2} \right\}
\end{equation}
Simplifying terms gives
\begin{equation}
  \exp \left\{ -\gamma b v
             - s
             + s
             + {s^2 \over \gamma b v}
             - {1 \over 2} {s^2 \over \gamma b v}
             - {1 \over s} {s^3 \over (\gamma b v)^2} \right\} =
\exp \left\{ -\gamma b v
             + {1 \over 2} {s^2 \over \gamma b v}
             - {1 \over s} {s^3 \over (\gamma b v)^2} \right\}
\end{equation}
Note that by assumption $s^3 \ll (\gamma b v)^2$, therefore we can remove the
third term in the exponent. All these steps finally give an approximation for
$n!$ of the form
\begin{equation}
  n! \approx (\gamma b v)^n e^{-\gamma b v} e^{s^2 \over 2 \gamma b v}
             \sqrt{2 \pi \gamma b v}, \;\;
             \text{for} \; (\gamma b v) \rightarrow \infty.
\end{equation}
\manuelComment{Eq (28) in the SI}

So far the Laplace approximation for sums allowed us to find the leading terms
of the sum to then have an approximation for the $n!$ term. We can now
substitute back this approximation into the sum
\begin{align}
  S(\gamma b v) &= \sum_{n = 0}^{\infty} {(\gamma b v)^{n + 1} \over
                                        n! (n - \gamma + 1)},\\
  &\approx \sum_{n = 0}^{\infty}
  {(\gamma b v)^{n + 1} \over (n - \gamma + 1)} \cdot
  \overbrace{
  { (\gamma b v)^{-n} e^{\gamma b v} e^{-s^2 \over 2 \gamma b v}
  \over
  \sqrt{2 \pi \gamma b v}
  }}^{1 \over n!}.
\end{align}
We will now substitute for all $n \approx \gamma b v + s$ and approximate the
sum as an integral on $s$ with an extended range from $-\infty$ to $\infty$.
This is valid since the terms around $n = \gamma b v$ dominate the sum, so the
extra terms would have a marginally small impact in the result. With this then
we have
\begin{equation}
  S(\gamma b v) \approx \int_{-\infty}^{\infty} ds {
  \gamma b v e^{\gamma b v} e^{-s^2 \over 2 \gamma b v}
  \over
  \left[ \gamma (b v - 1) + s + 1 \right] \sqrt{2 \pi \gamma b v}
  }.
\end{equation}

Now we factorize a term $\gamma (b v - 1)$ from the denominator, obtaining
\begin{equation}
  S(\gamma b v) \approx \int_{-\infty}^{\infty} ds {
  \gamma b v e^{\gamma b v} e^{-s^2 \over 2 \gamma b v}
  \over
  \gamma (b v - 1) \left[ 1 + {s + 1 \over \gamma (b v - 1) }\right]
  \sqrt{2 \pi \gamma b v}
  }.
\end{equation}
Since we assumed $\gamma \gg 1$ and $\gamma b v \gg s$ we can safely approximate
$\left[ 1 + {s + 1 \over \gamma (b v - 1) }\right] \approx 1$, obtaining
\begin{equation}
  S(\gamma b v) \approx {b v e^{\gamma b v} \over b v - 1}
  \int_{-\infty}^{\infty} ds
  {e^{-s^2 \over 2 \gamma b v}
  \over
  \sqrt{2 \pi \gamma b v}}.
\end{equation}
The term inside the integral integrates to 1 by the Gaussian integral, so we
have
\begin{equation}
  S(\gamma b v) \approx {b v e^{\gamma b v} \over b v - 1}.
\end{equation}
\manuelComment{Eq (29) in the SI.}

With this rather convoluted result we can go back to \eref[eq_SI_27_uofv] to
write
\begin{align}
  u(v) &= e^{-\gamma b v} \left[ C v^{\gamma}
  - \sum_{n=0}^\infty {\left( \gamma b v \right)^{n+1} \over
  n! \left( n - \gamma + 1 \right)} \right].\\
  &\approx e^{-\gamma b v}
  \left[ C v^{\gamma} - {b v e^{\gamma b v} \over b v - 1} \right]
\end{align}
Simplifying terms we have
\begin{equation}
  u(v) \approx C v^{\gamma} e^{-\gamma b v}
  - {b v \over b v - 1}
  \label{eq_SI_30_uofv}
\end{equation}
\manuelComment{Eq (30) in the SI.}

To obtain the value of the integration constant $C$ we use the initial
conditions $u(0) = u_o$ and $v(0) = v_o$. This gives us
\begin{equation}
  u_o = C v_o^{\gamma} e^{-\gamma b v_o}
  - {b v_o \over b v_o - 1}.
\end{equation}
Solving for $C$ gives
\begin{equation}
  C = \left[ u_o - {b v_o \over b v_o - 1} \right] v_o^{-\gamma} e^{\gamma b v}.
\end{equation}

That means that \eref[eq_SI_30_uofv] is given by
\begin{align}
  u(v) &= \left[ u_o - {b v_o \over b v_o - 1} \right]
  v_o^{-\gamma} e^{\gamma b v_o} v^{\gamma} e^{-\gamma b v}
  - {b v \over b v - 1}\\
  &= \left( u_o - {b v_o \over b v_o - 1}  \right)
  \left( {v \over v_o} \right)^{\gamma}
  e^{\gamma b v (v_o - v)}
  + {b v \over 1 - b v}.
  \label{eq_uofv_with_C}
\end{align}

Recall that \eref[eq_v_tau_relation] tells us the relationship between $v$ and
$\tau$, i.e. $v = v_o e^{\tau}$. This means that since $\tau > 0$ then
$v = v_o e^{\tau} > v_o$, therefore since the first term of
\eref[eq_uofv_with_C] is dominated by the exponential, and
$e^{\gamma b (v_o - v)}$ quickly converges to zero for large $\gamma$ we can
approximate \eref[eq_uofv_with_C] as
\begin{equation}
  u(v) \approx {b v \over 1 - b v}.
  \label{eq_SI_31_uofv}
\end{equation}
\manuelComment{Eq (30) in the SI and (5) in the main text.}
Interestingly since in the generating function \eref[eq_generating_function]
$u$ was indirectly related to mRNA copy number and $v$ to protein,
\eref[eq_SI_31_uofv] implies that since $u$ quickly converges to a fixed
function of $v$, for most of the protein lifetime the mRNA is at steady state.

\subsection{Finding the generating function}

With so much mathematical details so far it is easy to lose track of what is
what we are trying to achieve. Let's recall that the ODEs that define the
characteristics of the generating function are given byj
\begin{align}
  \frac{d v}{d r} &= 1,
  \\
  \frac{d u}{d r} &= - \gamma \left[ b (1 + u) - \frac{u}{v}
  \right],
  \\
  \frac{d \tau}{d r} &= \frac{1}{v},
  \\
  \frac{d F}{d r} &= \frac{a u}{v} F.
\end{align}
So far we found functional relationships between them given by
\begin{equation}
  v = v_o e^{\tau},
\end{equation}
and
\begin{equation}
  u = {b v \over 1 - b v}.
\end{equation}

Now we can combine ${dv \over dr}$ and ${dF \over dr}$ with these relationships
to obtain
\begin{align}
  {\partial F \over \partial v} &= a {u \over v} F, \\
  &= a {b v \over v (1 - b v)} F, \\
  &= {a b \over 1 - b v} F
\end{align}
This ODE can be solved by separation of variables as
\begin{equation}
  \int {dF \over F} = \int {a b \over 1 - b v} dv.
\end{equation}
Upon defining $x = 1 - b v, \; dx = - b\; dv$ we have
\begin{equation}
  \ln F = \int {a b \over x} \cdot {1 \over -b} dx = -a \ln x + C'.
\end{equation}
Therefore
\begin{equation}
  \ln F = -a \ln (1 - b v) + C',
  \label{eq_lnF_constant}
\end{equation}
whith $C'$ being an integration constant. To find the value of this constant
we must use the initial conditions. But we originally didn't set initial
conditions for $F$. If we start our birth and death process with $k$ proteins
at time $\tau = 0$, we would have that
\begin{align}
  F(\tau = 0) &= \sum_{p=0}^{\infty}P_p z^p\\
  &= \sum_{p=0}^{\infty} \delta (p - k) z^p = z^k.
\end{align}
This initial condition doesn't sum over mRNA, because as explained before for
\eref[eq_SI_31_uofv], for most of the protein lifetime the mRNA is in steady
state. Formally this initial condition is valid for times of order
${\gamma_p \over \gamma_m} = \gamma^{-1}$. Now $z = 1 + v$, therefore
$F(\tau = 0) = (1 - v_o)^k$. Using this initial condition we have
\begin{equation}
  \ln (1 + v_o)^k = -a \ln (1 - b v_o) + C'.
\end{equation}
Solving for $C'$ gives
\begin{equation}
  C' = \ln (1 + v_o)^k + a \ln (1 - b v_o).
\end{equation}
Therefore substituting this in \eref[eq_lnF_constant] gives
\begin{align}
  \ln F &= a \ln \left( {1 - b v_o \over 1 - b v} \right)^a
  + \ln (1 + v_o)^k,\\
  &= \ln \left[ \left( {1 - b v_o \over 1 - b v} \right)^a
  (1 + v_o)^k \right].
\end{align}
This gives us the generating function of the form
\begin{equation}
  F =   \left( {1 - b v_o \over 1 - b v} \right)^a
  (1 + v_o)^k .
  \label{eq_generating_function_onv}
\end{equation}

We would like to write the generating function as a function of the original
variable $z$ and $\tau$. This is done by recalling that we defined $z = 1 + v$,
in combination with \eref[eq_v_tau_relation] gives $v_o = (z - 1) e^{-\tau}$.
Using this on \eref[eq_generating_function_onv] gives
\begin{equation}
  F = \left( {1 - b (z - 1) e^{-\tau}
  \over
  1 - b (z - 1)
  } \right)^a
  \left[ 1 + (z - 1) e^{-\tau} \right]^k.
\end{equation}
\manuelComment{Eq (35) in the SI.}

If we set $k = 0$ we get
\begin{equation}
  F = \left( {1 - b (z - 1) e^{-\tau}
  \over
  1 - b (z - 1)
  } \right)^a.
  \label{eq_generating_function_onz}
\end{equation}
\manuelComment{Eq (7) in the main text.}

\subsection{Deriving the probability distribution for proteins}

Given that we know the generating function we can backtrack the entire
distribution. The difficulty resides in being able to write
\eref[eq_generating_function_onz] in the standard form
\begin{equation}
  F = \sum_{p} z^p P_p.
\end{equation}
By the definition of the generating function we can see that to compute any
probability we can compute
\begin{equation}
  P_p = {1 \over p!} \left. {\partial ^p F \over \partial z^p}.
  \right\rvert_{z=0}
  \label{eq_prob_from_generating}
\end{equation}
So let's rewrite \eref[eq_generating_function_onz] in a more convenient form
\begin{equation}
  F = \left( {1 + b (1 - z) e^{-\tau}
  \over
  1 + b (z - 1)
  }  \right)^a.
\end{equation}

Now we want to write $F$ as a product of two factors, one of them without any
$z$ term. This means we can write
\begin{align}
  F(z, \tau) &= \left( {1 + b e^{-\tau} \over 1 + b } \right)^a
  \left[
  \left( {1 + b} \over 1 + b (1 - z) \right)^a
  \left( {1 + b (1 - z) e^{-\tau} \over 1 + b e^{-\tau}}  \right)^a
  \right], \\
  &= \left( {1 + b e^{-\tau} \over 1 + b } \right)^a
  { \left( {1 + b (1 - z) \over 1 + b} \right)^{-a}
  \over
  \left( {1 + b (1 - z) e^{-\tau} \over 1 + b e^{-\tau}} \right)^{-a}
  }, \\
  &= \left( {1 + b e^{-\tau} \over 1 + b } \right)^a
  { \left( {1 + b - b z) \over 1 + b} \right)^{-a}
  \over
  \left( {1 + b e^{-\tau} - b z e^{-\tau} \over 1 + b e^{-\tau}} \right)^{-a}
  }, \\
  &= \left( {1 + b e^{-\tau} \over 1 + b } \right)^a
  { \left( 1 - {b z \over 1 + b} \right)^{-a}
  \over
  \left( 1 - {b z \over e^{\tau} + b} \right)^{-a}
  }. \label{eq_generating_final}
\end{align}
\manuelComment{Eq. (36) in the SI.}

There are two identities that Shahrezaei \& Swain quote for the following steps.
The first one is given by
\begin{equation}
    \left. {\partial ^n \over \partial z^n}
    \left[ 1 - q z \right]^{-a} \right\vert _{z=0} =
    {\Gamma(a + n) \over \Gamma(a)} q^n.
\end{equation}
To proof this relationship let's first take the first derivative of the function
\begin{align}
  {\partial \over \partial z}
  \left. \left[ 1 - q z \right]^{-a} \right\vert _{z=0}
  &= \left. -a \left[ 1 - q z \right]^{-(a + 1)} (-q) \right\vert _{z=0}\\
  &= a q.
\end{align}

Now let's take the second derivative
\begin{align}
  {\partial^2 \over \partial z^2}
  \left. \left[ 1 - q z \right]^{-a} \right\vert _{z=0}
  &= {\partial \over \partial z}
  \left. a q \left[ 1 - q z \right]^{-(a + 1)} \right\vert _{z=0}\\
  &= \left. -a (a + 1) q \left[ 1 - q z \right]^{-(a + 2)} (-q)
  \right\vert _{z=0} \\
  &= a(a + 1) q^2.
\end{align}

The pattern seems to be emerging now. For completeness let's take the 3rd
derivative
\begin{align}
  {\partial^3 \over \partial z^3}
  \left. \left[ 1 - q z \right]^{-a} \right\vert _{z=0}
  &= {\partial \over \partial z}
  \left. -a (a + 1) q \left[ 1 - q z \right]^{-(a + 2)} (-q)
  \right\vert _{z=0} \\
  &= \left. -a (a + 1) (a + 2) q^2
  \left[ 1 - q z \right]^{-(a + 3)} (-q) \right\vert _{z=0} \\
  &= a(a + 1) (a + 2) q^3.
\end{align}

We have that $a (a + 1) (a + 2) \cdots (a + n - 1) = {(a + n - 1)! \over (a -
1)!}$ for $n, a \in \mathbb{Z}$, or in general for any positive $n, a$ we have
$a (a + 1) (a + 2) \cdots (a + n - 1) = {\Gamma(a + n) \over \Gamma(a)}$.
Therefore for the $n^{\text{th}}$ derivative it must be true that
\begin{equation}
    \left. {\partial ^n \over \partial z^n}
    \left[ 1 - q z \right]^{-a} \right\vert _{z=0} =
    {\Gamma(a + n) \over \Gamma(a)} q^n. \blacksquare
    \label{eq_derivative_identity}
\end{equation}
\manuelComment{Eq (37) in the SI.}

The second identity the authors quote is more convoluted. It could be proof by
induction but it would be a really long proof. For convenience we will quote
the result. The identity tells us that the $n^{\text{th}}$ derivative of a ratio
of functions is given by
\begin{equation}
  {\partial ^n \over \partial z^n} {x(z) \over y(z)} =
  n! \sum_{k=0}^n {\partial ^{n-k} \over \partial z^{n-k}} x(z)
  \sum_{j = 0}^k (-1)^j (k + 1) y(z)^{-j-1}
  {\partial^k \over \partial z^k} y(z)^j
  \label{eq_derivative_ratio}
\end{equation}
\manuelComment{Eq (39) in SI.}

\eref[eq_prob_from_generating] gives us the recipe for how to compute any
probability from the generating function we derived in
\eref[eq_generating_final]. To use \eref[eq_derivative_ratio] we define from
\eref[eq_generating_final]
\begin{equation}
  x(z) = \left( 1 - {b z \over 1 + b} \right)^{-a},
\end{equation}
and
\begin{equation}
  y(z) = \left( 1 - {b z \over e^{\tau} + b} \right)^{-a}.
\end{equation}
With these definitions in hand we use \eref[eq_prob_from_generating] along with
\eref[eq_derivative_ratio] to compute
\begin{equation}
  \scriptscriptstyle
  P_p = \left. {1 \over p!} \left( {1 + b e^{-\tau} \over 1 + b} \right)^a
  \left[ p! \sum_{k=0}^p  {\partial^{p-k} \over \partial z^{p-k}}
  \left(1 - {b z \over 1 + b} \right) \cdot
  \sum_{j=0}^k {(-1)^j (k + 1) \over (j + 1)! (p - k)! (k - j)!} \cdot
  \left[ \left( 1 - {b z \over e^\tau + b} \right)^{-a} \right]^{-(j+1)} \cdot
  {\partial^k \over \partial z^k}
  \left[ \left( 1 - {b z \over e^\tau + b}  \right)^{-a} \right]^j
  \right]\right\vert _{z=0}.
\end{equation}
Using \eref[eq_derivative_identity] we can rewrite the derivatives as
\begin{equation}
  \scriptstyle
  P_p = \left( {1 + b e^{-\tau} \over 1 + b} \right)^a
  \sum_{k=0}^p {\Gamma(a + p - k) \over \Gamma(a)}
  \left( {b \over 1 + b} \right)^{p - k} \cdot
  \sum_{j=0}^k {(-1)^j (k + 1) \over (j + 1)! (p - k)! (k - j)!} \cdot
  {\Gamma(aj + k) \over \Gamma(aj)} \left( {b \over e^\tau + b} \right)^k
\end{equation}
\manuelComment{Eq (39) in SI.}

For the second sum the authors quote the following identity
\begin{equation}
  \sum_{j=0}^k {(-1)^j \Gamma(aj + k) \over (j + 1)! \Gamma(aj) (k - j)!} =
  {(-1)^k \Gamma(a + 1) \over \Gamma(a - k + 1) (k + 1)!}.
\end{equation}
\manuelComment{Eq (40) in SI. I tried really hard deriving this result, but I
couldn't figure it out. Nevertheless computationally it can be shown to be true
for several values of $k$.}

Using this we have
\begin{equation}
  \scriptstyle
  P_p = \left( {1 + b e^{-\tau} \over 1 + b} \right)^a
  \sum_{k=0}^p {\Gamma (a + p - k) \over \Gamma(a)}
  \left( {b \over 1 + b} \right)^{p - k}
  {(k + 1) \over (p - k)!} \left( b \over e^\tau + b \right)^k
  {(-1)^k \Gamma(a + 1) \over \Gamma(a - k + 1) (k + 1)!}.
\end{equation}
This can be further simplified to
\begin{equation}
  P_p = \left( {1 + b e^{-\tau} \over 1 + b} \right)^a
  \left( {b \over 1 + b} \right)^p
  \sum_{k=0}^p {(-1)^k \over k!}
  {a \Gamma(a + p - k) \over (p - k)! \Gamma(a - k + 1)}
  \left( {1 + b \over e^\tau + b} \right)^k.
  \label{eq_dist_nohyper}
\end{equation}
\manuelComment{Eq (41) in the SI. Shahrezaei and Swain missed one factor of
$a$ since $\Gamma(a + 1) / \Gamma(a) = a$, but the error doesn't propagate to
the other equations.}

\subsection{Gauss Hypergeometric Function}

An important function for the solution is the Hypergeometric function
$_2F_1(a; b; c; z)$. This funciton is defined as
\begin{equation}
  _2F_1(a; b; c; z) \equiv \sum_{k=0}^\infty
  {(a)_k (b)_k \over (c)_k} {z^k \over k!}.
\end{equation}

The series is indetermined if $c$ is a non-positive integer. The $(q)_k$ terms
are the so-called rising factorials or rising Pochhammer symbols defined as
\begin{equation}
  (q)_k = \begin{cases}
  1, & \text{for }k = 0.\\
  q (q + 1)\cdots (q + k - 1) & \text{for k > 0},
  \end{cases}
\end{equation}
or using gamma functions
\begin{equation}
  (q)_k = {\Gamma (q + k) \over \Gamma (q)}.
\end{equation}

If $a$ or $b$ are non-positive integers the series reduces to the polynomial
\begin{equation}
_2F_1(-n; b; c; z) = \sum_{k=0}^\infty (-1)^k {n \choose k}
                    {(b)_k \over (c)_k} z^k.
\end{equation}
Or for non-integer values we have
\begin{equation}
_2F_1(-n; b; c; z) = \sum_{k=0}^\infty (-1)^k
                    {\Gamma (n + 1) \over \Gamma (n - k + 1)}
                    {(b)_k \over (c)_k} {z^k \over k!}.
\end{equation}

From the definition of the Pochhammer symbols they must satisfy
\begin{equation}
  (-a)_k = {\Gamma (a + 1) \over \Gamma (a - k + 1)}.
\end{equation}
This relationship can be shown using the Euler reflection formula
\begin{equation}
  \Gamma (z) \Gamma (1 - z) = {\pi \over \sin (\pi z)}.
\end{equation}

If $z \equiv -a + k$ we have
\begin{equation}
  \Gamma(-a + k) \Gamma(1 - (-a + k)) = {\pi \over \sin(\pi (-a + k))}.
\end{equation}
On the other hand if $z \equiv -a$ we have
\begin{equation}
  \Gamma(-a) \Gamma(1 - (-a)) = {\pi \over \sin(\pi (-a))}.
\end{equation}

From the definition of $(-a)_k$ we have
\begin{equation}
  (-a)_k = {\Gamma(-a + k) \over \Gamma(-a)}
         = {{\pi \over \sin(\pi (-a + k)) \Gamma(1 - (-a + k))}
         \over
           {\pi \over \sin(\pi(-a)) \Gamma(1 - (-a))}}.
\end{equation}
Simplifying this we have
\begin{equation}
  (-a)_k = {\Gamma(1 - (-a)) \over \Gamma(1 - (-a + k))}\cdot
           {\sin(\pi (-a + k)) \over \sin(\pi (-a))}.
\end{equation}

For $\sin(\pi (-a + k))$ we can use the identity
\begin{equation}
  \sin(\pi (-a + k)) = \sin(\pi (-a)) \cos(\pi k) -
                       \cos(\pi (-a)) \sin(\pi k).
\end{equation}
Since for $k \in \mathbb{Z}$ we have that $\sin(\pi k) = 0$, and
$\cos(\pi k) = (-1)^k$, it must be true that for $a \notin \mathbb{N}$
\begin{align}
  (-a)_k &= {\Gamma(a + 1) \over \Gamma(a - k + 1)}
           {(-1)^k \sin(\pi (-a)) \over \sin(\pi (-a))},\\
         &= (-1)^k {\Gamma(a + 1) \over \Gamma(a - k + 1)}.
\end{align}
\manuelComment{Eq (43) in SI}.

From \eref[eq_dist_nohyper] we can rewrite the term inside the sum as a
hypergeometric function by rewriting $\Gamma(a - k + n) =
\Gamma(a + n - 1 - k + 1)$. This means that if we carefully look at the elements
of the sum we can rewrite it as a hypergeometric function like
\begin{align}
  \sum_{k=0}^p {(-1)^k \over k!}
  {a \Gamma(a + p - k) \over \Gamma(p - k + 1) \Gamma(a - k + 1)}
  \left( {1 + b \over e^\tau + b} \right)^k =
  a
  \overbrace{
  {\Gamma(p + a - 1 + 1) \over \Gamma(a + 1) \Gamma(p + 1)}
  }^{\text{terms to complete } {}_2F_1}
  \times \\
  \sum_{k=0}^p
  \overbrace{
  (-1)^k {\Gamma(p + 1) \over \Gamma(p - k + 1)}
  }^{(a)_k = (-p)_k}
  \overbrace{
  (-1)^k {\Gamma(a + 1) \over \Gamma(a - k + 1)}
  }^{(b)_k = (-a)_k}
  \underbrace{
  {1 \over (-1)^k }{\Gamma(p + a - 1 + k + 1) \over \Gamma(p + a - 1 + 1)}
  }_{{1 \over (c)_k} = {1 \over (-p - a + 1)_k}}
  \underbrace{
  {1 \over k!} \left( {1 + b \over e^\tau + b} \right)^k
  }_{z^k \over k!}.
\end{align}
Therefore the sum can be simply written as
\begin{equation}
\sum_{k=0}^p {(-1)^k \over k!}
  {a \Gamma(a + p - k) \over \Gamma(p - k + 1) \Gamma(a - k + 1)}
  \left( {1 + b \over e^\tau + b} \right)^k =
  {a \Gamma(p + a - 1 + 1) \over \Gamma(a + 1) \Gamma(p + 1)}
  {}_2F_1 \left( -p, -a, 1 - a - n, {1 + b \over e^\tau + b} \right)
\end{equation}

This relationship means that we can rewrite \eref[eq_dist_nohyper] as
\begin{equation}
  P_p = \left( {1 + b e^{-\tau} \over 1 + b} \right)^a
  \left( {b \over 1 + b} \right)^p
  {\Gamma(p + a) \over \Gamma(a) p!}
  {}_2F_1 \left( -p, -a, 1 - a - n, {1 + b \over e^\tau + b} \right),
\end{equation}
This is the final expression for the protein distribution valid for
$\Gamma \gg 1$ and $\tau \gg \Gamma^{-1}$.
\manuelComment{Eq (44) in SI and (8) in main text.}

\subsection{Steady State Distribution}
To compute the steady state distribution we take the limit when
$\tau \rightarrow \infty$. That gives ${}_2F_1(a, b, c, 0)$ which cancels all
terms in the sum except for the first one. Therefore ${}_2F_1(a, b, c, 0) = 1$.
This means that we have
\begin{equation}
  P_p^{ss} = {\Gamma(p + a) \over \Gamma(p + 1) \Gamma(a)}
  \left( 1 - {b \over 1 + b}  \right)^a
  \left( {b \over 1 + b} \right)^p.
\end{equation}
\manuelComment{Eq (9) in main text.}

This is a negative binomial distribution with mean
\begin{equation}
  \left\langle p \right\rangle = a \cdot b =
  {r_m \over \gamma_p}{r_p \over \gamma_m},
\end{equation}
and variance
\begin{equation}
  \sigma_p^2 = ab(1 + b) =
  {r_m \over \gamma_p}{r_p \over \gamma_m} \left(1 + {r_p \over \gamma_m}\right)
\end{equation}
