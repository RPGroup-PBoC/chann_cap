\section{Computing moments from the master equation}\label{supp_moments}

In this section we will compute the moment equations for the distribution  $P(m,
p)$. Without lost of generality here we will focus on the three-state
regulated promoter. The computation of the moments for the two-state promoter
follows the exact same procedure, changing only the definition of the matrices 
in the master equation.

\subsection{Computing moments of a distribution}

(Note: The Python code used for the calculations presented in this section can
be found in the
\href{https://www.rpgroup.caltech.edu//chann_cap/software/moment_dynamics_system.html}{following
link} as an annotated Jupyter notebook)

To compute any moment of our chemical master equation (\eref{eq_cme_matrix})
let us define a vector
\begin{equation}
  \ee{\bb{m^x p^y}} \equiv (\ee{m^x p^y}_A, \ee{m^x p^y}_I, \ee{m^x p^y}_R)^T,
\end{equation}
where $\ee{m^x p^y}_S$ is the expected value of $m^x p^y$ in state $S \in \{A,
I, R\}$ with $x, y \in \mathbb{N}$. In other words, just as we defined the
vector $\PP(m, p)$, here we define a vector to collect the expected value of
each of the promoter states. By definition, these moments $\ee{m^x p^y}_S$ are
computed as
\begin{equation}
  \ee{m^x p^y}_S \equiv \sum_{m=0}^\infty \sum_{p=0}^\infty m^x p^y P_S(m, p).
  \label{seq_mom_def}
\end{equation}
To simplify the notation, let $\sum_x \equiv \sum_{x=0}^\infty$. Since we are
working with a system of three ODEs, one for each state, let us define the
following operation:
\begin{equation}
  \ee{\bb{m^x p^y}} =
  \smp m^x p^y \PP(m, p) \equiv
  \begin{bmatrix}
    \smp m^x p^y P_A(m, p)\\
    \smp m^x p^y P_I(m, p)\\
    \smp m^x p^y P_R(m, p)\\
  \end{bmatrix}.
\end{equation}

With this in hand we can then apply this sum over $m$ and $p$ to
\eref{eq_cme_matrix}. For the left-hand side we have
\begin{equation}
  \smp m^x p^y \dt{\PP(m, p)} = \dt{}\left[ \smp m^x p^y \PP(m, p) \right],
  \label{seq_sum_mom}
\end{equation}
where we made use of the linearity property of the derivative to switch the
order between the sum and the derivative. Notice that the right-hand side of
\eref{seq_sum_mom} contains the definition of a moment from \eref{seq_mom_def}.
That means that we can rewrite it as
\begin{equation}
  \dt{}\left[ \smp m^x p^y \PP(m, p) \right] = \dt{\bb{\ee{m^x p^y}}}.
\end{equation}

Distributing the sum on the right-hand side of \eref{eq_cme_matrix} gives
\begin{equation}
  \begin{aligned}
    \dt{\bb{\ee{m^x p^y}}} &=
    \Km \smp m^x p^y \PP(m, p)\\
    &- \Rm \smp m^x p^y \PP(m, p) + \Rm \smp m^x p^y \PP(m-1, p)\\
    &- \Gm \smp (m) m^x p^y \PP(m, p) + \Gm \smp (m + 1) m^x p^y \PP(m + 1, p)\\
    &- \Rp \smp (m) m^x p^y \PP(m, p) + \Rp \smp (m) m^x p^y \PP(m, p - 1)\\
    &- \Gp \smp (p) m^x p^y \PP(m, p) + \Gp \smp (p + 1) m^x p^y \PP(m, p + 1).
  \end{aligned}
  \label{seq_master_sum}
\end{equation}

Let's look at each term on the right-hand side individually. For the terms in
\eref{seq_master_sum} involving $\PP(m, p)$ we can again use \eref{seq_mom_def}
to rewrite them in a more compact form. This means that we can rewrite the
state transition term as
\begin{equation}
  \Km \smp m^x p^y \PP(m, p) = \Km \bb{\ee{m^x p^y}}.
\end{equation}
The mRNA production term involving $\PP(m, p)$ can be rewritten as
\begin{equation}
  \Rm \smp m^x p^y \PP(m, p) = \Rm \bb{\ee{m^x p^y}}.
\end{equation}
In the same way the mRNA degradation term gives
\begin{equation}
  \Gm \smp (m) m^x p^y \PP(m, p) = \Gm \bb{\ee{m^{(x + 1)} p^y}}.
\end{equation}
For the protein production and degradation terms involving $\PP(m, p)$ we have
\begin{equation}
  \Rp \smp (m) m^x p^y \PP(m, p) = \Rp \bb{\ee{m^{(x + 1)} p^y}},
\end{equation}
and
\begin{equation}
  \Gp \smp (p) m^x p^y \PP(m, p) = \Gp \bb{\ee{m^x p^{(y + 1)}}},
\end{equation}
respectively.

For the sums terms in \eref{seq_master_sum} involving $\PP(m \pm 1, p)$ or
$\PP(m, p \pm 1)$ we can reindex the sum to work around this mismatch. To be
more specific let's again look at each term case by case. For the mRNA
production term involving $\PP(m-1, p)$ we define $m' \equiv m - 1$. Using this
we write
\begin{equation}
  \Rm \smp m^x p^y \PP(m-1, p) =
  \Rm \sum_{m' = -1}^\infty \sum_p (m' + 1)^x p^y \PP(m', p).
\end{equation}
Since having negative numbers of mRNA or protein doesn't make physical sense
we have that $\PP(-1, p) = 0$. Therefore we can rewrite the sum starting from 0
rather than from -1, obtaining
\begin{equation}
  \Rm \sum_{m' = -1}^\infty \sum_p (m' + 1)^x p^y \PP(m', p) =
  \Rm \sum_{m'=0}^\infty \sum_p (m' + 1)^x p^y \PP(m', p).
  \label{seq_reindex}
\end{equation}
Recall that our distribution $\PP(m, p)$ takes $m$ and $p$ as numerical inputs 
and returns a probability associated with such a molecule count.  Nevertheless,
$m$ and $p$ themselves are dimensionless quantities that serve as indices of how
many molecules are in the cell. The distribution is the same whether the
variable is called $m$ or $m'$; for a specific number, let's say $m = 5$, or $m'
= 5$, $\PP(5, p)$ will return the same result. This means that the variable name
is arbitrary, and the right-hand side of
\eref{seq_reindex} can be written as
\begin{equation}
  \Rm \sum_{m'=0}^\infty \sum_p (m' + 1)^x p^y \PP(m', p) =
  \Rm \bb{\ee{(m+1)^x p^y}},
\end{equation}
since the left-hand side corresponds to the definition of a moment.

For the mRNA degradation term involving $\PP(m + 1, p)$ we follow a similar
procedure in which we define $m' = m + 1$ to obtain
\begin{equation}
  \Gm \smp (m + 1) m^x p^y \PP(m + 1, p) =
  \Gm \sum_{m' = 1}^\infty \sum_p m' (m' - 1)^x p^y \PP(m', p).
\end{equation}
In this case since the term on the right-hand side of the equation is multiplied
by $m'$, starting the sum over $m'$ from 0 rather than from 1 will not affect
the result since this factor will not contribute to the total sum. Nevertheless
this is useful since our definition of a moment from \eref{seq_mom_def} requires
the sum to start at zero. This means that we can rewrite this term as
\begin{equation}
  \Gm \sum_{m' = 1}^\infty m' \sum_p (m' - 1)^x p^y \PP(m', p) =
  \Gm \sum_{m' = 0}^\infty m' \sum_p (m' - 1)^x p^y \PP(m', p).
\end{equation}
Here again we can change the arbitrary label $m'$ back to $m$ obtaining
\begin{equation}
  \Gm \sum_{m' = 0}^\infty m' \sum_p (m' - 1)^x p^y \PP(m', p) =
  \Gm \bb{\ee{m (m - 1)^x p^y}}.
\end{equation}

The protein production term involving $\PP(m, p - 1)$ can be reindexed by 
defining $p' \equiv p - 1$. This gives
\begin{equation}
  \Rp \smp (m) m^x p^y \PP(m, p - 1) =
  \Rp \sum_m \sum_{p'=-1}^\infty m^{(x + 1)} (p + 1)^y \PP(m, p').
\end{equation}
We again use the fact that negative molecule copy numbers are assigned with
probability zero to begin the sum from 0 rather than -1 and the arbitrary nature
of the label $p'$ to write
\begin{equation}
  \Rp \sum_m \sum_{p'=0}^\infty m^{(x + 1)} (p + 1)^y \PP(m, p') =
  \Rp \bb{\ee{m^{(x + 1)} (p + 1)^y}}.
\end{equation}
Finally, we take care of the protein degradation term involving $\PP(m, p + 1)$.
As before, we define $p' = p + 1$ and substitute this to obtain
\begin{equation}
  \Gp \smp (p + 1) m^x p^y \PP(m, p + 1) =
  \Gp \sum_m \sum_{p'=1}^\infty (p') m^x (p' - 1)^y \PP(m, p').
\end{equation}
Just as with the mRNA degradation term, having a term $p'$  inside the sum
allows us to start the sum over $p'$ from 0 rather than 1. We can therefore
write
\begin{equation}
  \Gp \sum_m \sum_{p'=0}^\infty (p') m^x (p' - 1)^y \PP(m, p') =
  \Gp \bb{\ee{m^x p (p - 1)^y}}.
\end{equation}

Putting all these terms together we can write the general moment ODE. This is
of the form
\begin{equation}
  \begin{aligned}
    \dt{\bb{\ee{m^x p^y}}} &=
    \Km \bb{\ee{m^x p^y}}
    \text{  (promoter state transition)}\\
    &- \Rm \bb{\ee{m^x p^y}} + \Rm \bb{\ee{(m+1)^x p^y}}
    \text{  (mRNA production)}\\
    &- \Gm \bb{\ee{m^{(x + 1)} p^y}} + \Gm \bb{\ee{m (m - 1)^x p^y}}
    \text{  (mRNA degradation)}\\
    &- \Rp \bb{\ee{m^{(x + 1)} p^y}} + \Rp \bb{\ee{m^{(x + 1)} (p + 1)^y}}
    \text{  (protein production)}\\
    &- \Gp \bb{\ee{m^x p^{(y + 1)}}} + \Gp \bb{\ee{m^x p (p - 1)^y}}
    \text{  (protein degradation)}.
  \end{aligned}
  \label{seq_mom_ode}
\end{equation}

\subsection{Moment closure of the simple-repression distribution}

A very interesting and useful feature of \eref{seq_mom_ode} is that for a given
value of $x$ and $y$ the moment $\bb{\ee{m^x p^y}}$ is only a function of lower
moments. Specifically $\bb{\ee{m^x p^y}}$ is a function of moments
$\bb{\ee{m^{x'} p^{y'}}}$ that satisfy two conditions:
\begin{equation}
  \begin{aligned}
    &1) y' \leq y,\\
  &2) x' + y' \leq x + y.
  \end{aligned}
  \label{seq_mom_conditions}
\end{equation}

To prove this we rewrite \eref{seq_mom_ode} as
\begin{equation}
  \begin{aligned}
    \dt{\bb{\ee{m^x p^y}}} &=
    \Km \bb{\ee{m^x p^y}}\\
    &+ \Rm \bb{\ee{p^y \left[ (m + 1)^x -m^x \right]}}\\
    &+ \Gm \bb{\ee{m p^y \left[ (m - 1)^x - m^x \right]}}\\
    &+ \Rp \bb{\ee{m^{(x + 1)} \left[ (p + 1)^y - p^y \right]}}\\
    &+ \Gp \bb{\ee{m^x p \left[ (p - 1)^y - p^y \right]}},
    \label{seq_mom_ode_factorized}
  \end{aligned}
\end{equation}
where the factorization is valid given the linearity of expected values. Now the
objective is to find the highest moment for each term once the relevant
binomial, such as $(m-1)^x$, is expanded. Take, for example, a simple case in
which we want to find the second moment of the mRNA distribution. We then set $x
= 2$ and $y = 0$. \eref{seq_mom_ode_factorized} then becomes
\begin{equation}
  \begin{aligned}
    \dt{\bb{\ee{m^2 p^0}}} &=
    \Km \bb{\ee{m^2 p^0}}\\
    &+ \Rm \bb{\ee{p^0 \left[ (m + 1)^2 - m^2 \right]}}\\
    &+ \Gm \bb{\ee{m p^0 \left[ (m - 1)^2 - m^2 \right]}}\\
    &+ \Rp \bb{\ee{m^{(2 + 1)} \left[ (p + 1)^0 - p^0 \right]}}\\
    &+ \Gp \bb{\ee{m^2 p \left[ (p - 1)^0 - p^0 \right]}}.
  \end{aligned}
\end{equation}
Simplifying this equation gives
\begin{equation}
    \dt{\bb{\ee{m^2}}} =
    \Km \bb{\ee{m^2}}
    + \Rm \bb{\ee{\left[ 2m + 1 \right]}}
    + \Gm \bb{\ee{\left[- 2m^2 + m \right]}}.
    \label{seq_second_mom_mRNA}
\end{equation}

\eref{seq_second_mom_mRNA} satisfies both of our conditions. Since we set $y$ to
be zero, none of the terms depend on any moment that involves the protein number,
therefore $y' \leq y$ is satisfied. Also the highest moment in
\eref{seq_second_mom_mRNA} also satisfies $x' + y' \leq x + y$ since the second
moment of mRNA doesn't depend on any moment higher than $\bb{\ee{m^2}}$. To
demonstrate that this is true for any $x$ and $y$ we now rewrite
\eref{seq_mom_ode_factorized}, making use of the binomial expansion
\begin{equation}
  (z \pm 1)^n = \sum_{k=0}^n {n \choose k} (\pm 1)^{k} z^{n-k}.
\end{equation}
Just as before let's look at each term individually. For the mRNA production
term we have
\begin{equation}
  \Rm \bb{\ee{p^y \left[ (m + 1)^x -m^x \right]}} =
  \Rm \bb{\ee{p^y \left[ \sum_{k=0}^x {x \choose k} m^{x-k} - m^x \right]}}.
\end{equation}
When $k = 0$, the term inside the sum on the right-hand side cancels with the
other $m^x$, so we can simplify to
\begin{equation}
  \Rm \bb{\ee{p^y \left[ (m + 1)^x -m^x \right]}} =
  \Rm \bb{\ee{p^y \left[ \sum_{k=1}^x {x \choose k} m^{x-k} \right]}}.
\end{equation}
Once the sum is expanded we can see that the highest moment in this sum is given
by $\bb{\ee{m^{(x-1)} p^y}}$ which satisfies both of the conditions on
\eref{seq_mom_conditions}.

For the mRNA degradation term we similarly have
\begin{equation}
  \Gm \bb{\ee{m p^y \left[ (m - 1)^x - m^x \right]}} =
  \Gm \bb{\ee{m p^y \left[ \sum_{k=0}^x {x \choose k}(-1)^k m^{x-k} -
                          m^x \right]}}.
\end{equation}
Simplifying terms we obtain
\begin{equation}
  \Gm \bb{\ee{m p^y \left[ \sum_{k=0}^x {x \choose k}(-1)^k m^{x-k} -
                          m^x \right]}} =
  \Gm \bb{\ee{p^y \left[ \sum_{k=1}^x {x \choose k}(-1)^k m^{x+1-k} \right]}}.
\end{equation}
The largest moment in this case is $\bb{\ee{m^x p^y }}$, which again satisfies
the conditions on \eref{seq_mom_conditions}.

The protein production term gives
\begin{equation}
  \Rp \bb{\ee{m^{(x + 1)} \left[ (p + 1)^y - p^y \right]}} =
  \Rp \bb{\ee{m^{(x + 1)} \left[ \sum_{k=0}^y {y \choose k} (-1)^k p^{y-k}
                                - p^y \right]}}.
\end{equation}
Upon simplification we obtain
\begin{equation}
  \Rp \bb{\ee{m^{(x + 1)} \left[ \sum_{k=0}^y {y \choose k} (-1)^k p^{y-k}
                                - p^y \right]}} =
  \Rp \bb{\ee{m^{(x + 1)} \left[ \sum_{k=1}^y {y \choose k}(-1)^k p^{y-k}
  \right]}}.
\end{equation}
Here the largest moment is given by $\bb{\ee{m^{x+1} p^{y-1}}}$, that again
satisfies both of our conditions. For the last term, for protein degradation we
have
\begin{equation}
  Rp \bb{\ee{m^{(x + 1)} \left[ (p + 1)^y - p^y \right]}} =
  Rp \bb{\ee{m^{(x + 1)} \left[ \sum_{k=1}^y {y \choose k} (-1^k) p^{y - k}
  \right]}}.
\end{equation}
The largest moment involved in this term is therefore $\bb{\ee{m^x p^{y-1}}}$.
With this, we show that the four terms involved in our general moment equation
depend only on lower moments that satisfy \eref{seq_mom_conditions}.

As a reminder, what we showed in this section is that the kinetic model
introduced in \fref{fig2_minimal_model}(A) has no moment-closure problem. In
other words, moments of the joint mRNA and protein distribution can be computed
just from knowledge of lower moments. This allows us to cleanly integrate the
moments of the distribution dynamics as cells progress through the cell cycle.

\subsection{Computing single promoter steady-state moments}

(Note: The Python code used for the calculations presented in this section can
be found in the
\href{https://www.rpgroup.caltech.edu//chann_cap/software/chemical_master_steady_state_moments_general.html}{following
link} as an annotated Jupyter notebook)

As discussed in \secref{sec_cell_cycle}, one of the main factors contributing to
cell-to-cell variability in gene expression is the change in gene copy number
during the cell cycle as cells replicate their genome before cell division. Our
minimal model accounts for this variability by considering the time trajectory
of the distribution moments as given by \eref{seq_mom_ode_factorized}. These
predictions will be contrasted with the predictions from a kinetic model that
doesn't account for changes in gene copy number during the cell cycle in
\siref{supp_multi_gene}.

If we do not account for change in gene copy number during the cell cycle or for
the partition of proteins during division, the dynamics of the moments of the
distribution described in this section will reach a steady state. In order to
compute the steady-state moments of the kinetic model with a single gene across
the cell cycle, we use the moment closure property of our master equation. By
equating \eref{seq_mom_ode_factorized} to zero for a given $\bb{x}$ and
$\bb{y}$, we can solve the resulting linear system and obtain a solution for
$\bb{\ee{m^x p^y}}$ at steady state as a function of moments $\bb{\ee{m^{x'}
p^{y'}}}$ that satisfy \eref{seq_mom_conditions}. Then, by solving for the
zero$\th$ moment $\bb{\ee{m^0 p^0}}$ subject to the constraint that the
probability of the promoter being in any state should add up to one, we can
substitute back all of the solutions in terms of moments $\bb{\ee{m^{x'}
p^{y'}}}$ with solutions in terms of the rates shown in
\fref{fig2_minimal_model}. In other words, through an iterative process, we can
get at the value of any moment of the distribution. We start by solving for the
zero$\th$ moment. Since all higher moments, depend on lower moments we can use
the solution of the zero$\th$ moment to compute the first mRNA moment. This
solution is then used for higher moments in a hierarchical iterative process.
