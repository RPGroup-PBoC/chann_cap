\section{Computing moments from the master equation}\label{supp_moments}

In this section we will compute the moment equations for the distribution  $P(m,
p; t)$. Without lost of generality here we will focus on the three-state
regulated promoter. The computation of the moments for the two-state promoter
follows the exact same procedure, having only the definition of the matrices
change.

\subsection{Computing a distribution moment}

To compute any moment of our chemical master equation (\eref{eq_cme_matrix})
let us define a vector
\begin{equation}
  \ee{\bb{m^x p^y}} \equiv (\ee{m^x p^y}_E, \ee{m^x p^y}_B, \ee{m^x p^y}_R)^T,
\end{equation}
where $\ee{m^x p^y}_S$ is the expected value of $m^x p^y$ in state $S \in \{E,
P, R\}$ with $x, y \in \mathbb{N}$. In other words, just as we defined the
vector $\PP(m, p)$, here we define a vector to collect the expected value of
each of the promoter states. By definition any of these moments $\ee{m^x p^y}_S$
is computed as
\begin{equation}
  \ee{m^x p^y}_S \equiv \sum_{m=0}^\infty \sum_{p=0}^\infty m^x p^y P_S(m, p).
  \label{seq_mom_def}
\end{equation}
To simplify the notation let $\sum_x \equiv \sum_{x=0}^\infty$. Since we are
working with a system of three ODE, one for each state, let us define the
following operation:
\begin{equation}
  \ee{\bb{m^x p^y}} =
  \smp m^x p^y \PP(m, p) \equiv
  \begin{bmatrix}
    \smp m^x p^y P_E(m, p)\\
    \smp m^x p^y P_B(m, p)\\
    \smp m^x p^y P_R(m, p)\\
  \end{bmatrix}.
\end{equation}

With this in hand we can then apply this sum over $m$ and $p$ to
\eref{eq_cme_matrix}. For the left-hand side we have
\begin{equation}
  \smp m^x p^y \dt{\PP(m, p)} = \dt{}\left[ \smp m^x p^y \PP(m, p) \right],
  \label{seq_sum_mom}
\end{equation}
where we made use of the linearity property of the derivative to switch the
order between the sum and the derivative. Notice that the right-hand side of
\eref{seq_sum_mom} contains the definition of a moment from \eref{seq_mom_def}.
That means that we can rewrite it as
\begin{equation}
  \dt{}\left[ \smp m^x p^y \PP(m, p) \right] = \dt{\bb{\ee{m^x p^y}}}.
\end{equation}

Distributing the sum on the right-hand side of \eref{eq_cme_matrix} gives
\begin{equation}
  \begin{aligned}
    \dt{\bb{\ee{m^x p^y}}} &=
    \Km \smp m^x p^y \PP(m, p)\\
    &- \Rm \smp m^x p^y \PP(m, p) + \Rm \smp m^x p^y \PP(m-1, p)\\
    &- \Gm \smp (m) m^x p^y \PP(m, p) + \Gm \smp (m + 1) m^x p^y \PP(m + 1, p)\\
    &- \Rp \smp (m) m^x p^y \PP(m, p) + \Rp \smp (m) m^x p^y \PP(m, p - 1)\\
    &- \Gp \smp (p) m^x p^y \PP(m, p) + \Gp \smp (p + 1) m^x p^y \PP(m, p + 1).
  \end{aligned}
  \label{seq_master_sum}
\end{equation}

Let's look at each term on the right-hand side individually. For the terms in
\eref{seq_master_sum} involving $\PP(m, p)$ we can again use \eref{seq_mom_def}
to rewrite them in a more compact form. This means that we can rewrite the
state transition term as
\begin{equation}
  \Km \smp m^x p^y \PP(m, p) = \Km \bb{\ee{m^x p^y}}.
\end{equation}
The mRNA production term involving $\PP(m, p)$ can be rewritten as
\begin{equation}
  \Rm \smp m^x p^y \PP(m, p) = \Rm \bb{\ee{m^x p^y}}.
\end{equation}
In the same way the mRNA degradation term gives
\begin{equation}
  \Gm \smp (m) m^x p^y \PP(m, p) = \Gm \bb{\ee{m^{(x + 1)} p^y}}.
\end{equation}
As for the protein production and degradation terms involving $\PP(m, p)$ we
have
\begin{equation}
  \Rp \smp (m) m^x p^y \PP(m, p) = \Rp \bb{\ee{m^{(x + 1)} p^y}},
\end{equation}
and
\begin{equation}
  \Gp \smp (p) m^x p^y \PP(m, p) = \Gp \bb{\ee{m^x p^{(y + 1)}}},
\end{equation}
respectively.

For the sums terms in \eref{seq_master_sum} involving $\PP(m \pm 1, p)$ or
$\PP(m, p \pm 1)$ we can reindex the sum to work around this mismatch. To be
more specific let's again look at each term case by case. For the mRNA
production term involving $\PP(m-1, p)$ we define $m' \equiv m - 1$. Using this
we write
\begin{equation}
  \Rm \smp m^x p^y \PP(m-1, p) =
  \Rm \sum_{m' = -1}^\infty \sum_p (m' + 1)^x p^y \PP(m', p).
\end{equation}
Since having negative numbers of mRNA of protein doesn't make physical sense
we have that $\PP(-1, p) = 0$. Therefore we can rewrite the sum starting from 0
rather than from -1, obtaining
\begin{equation}
  \Rm \sum_{m' = -1}^\infty \sum_p (m' + 1)^x p^y \PP(m', p) =
  \Rm \sum_{m'=0}^\infty \sum_p (m' + 1)^x p^y \PP(m', p).
  \label{seq_reindex}
\end{equation}
\mrm{In my experience a lot of people have a problem with simply substituting
back $m'$ with $m$ for the final step. PBoC doesn't give a detailed
justification for this either, so I've tried explaining this in a clear way in
what follows, but it is far from perfect.}.
Recall that our distribution $\PP(m, p)$ takes as numerical inputs $m$ and $p$
and returns a probability associated with such a molecule count.  Nevertheless
$m$ and $p$ themselves are dimensionless quantities that serve as indexes of how
many molecules are in the cell. So our distribution doesn't care if the variable
is called $m$ or $m'$; for a specific number, let's say $m = 5$, or $m' = 5$,
$\PP(5, p)$ will return the same result. What that means is that the variable
name is somehow arbitrary, and the right-hand side of \eref{seq_reindex} can be
written as
\begin{equation}
  \Rm \sum_{m'=0}^\infty \sum_p (m' + 1)^x p^y \PP(m', p) =
  \Rm \bb{\ee{(m+1)^x p^y}},
\end{equation}
since the left-hand side corresponds to the definition of a moment.

For the mRNA degradation term involving $\PP(m + 1, p)$ we follow a similar
procedure in which we define $m' = m + 1$ to obtain
\begin{equation}
  \Gm \smp (m + 1) m^x p^y \PP(m + 1, p) =
  \Gm \sum_{m' = 1}^\infty \sum_p m' (m' - 1)^x p^y \PP(m', p).
\end{equation}
In this case since the term on the right-hand side of the equation is multiplied
by $m'$, starting the sum over $m'$ from 0 rather than from 1 will not affect
the result since this factor will not contribute to the total sum. Nevertheless
this is useful since our definition of a moment from \eref{seq_mom_def} requires
the sum to start at zero. This means that we can rewrite this term as
\begin{equation}
  \Gm \sum_{m' = 1}^\infty m' \sum_p (m' - 1)^x p^y \PP(m', p) =
  \Gm \sum_{m' = 0}^\infty m' \sum_p (m' - 1)^x p^y \PP(m', p).
\end{equation}
Here again we can change the arbitrary label $m'$ back to $m$ obtaining
\begin{equation}
  \Gm \sum_{m' = 0}^\infty m' \sum_p (m' - 1)^x p^y \PP(m', p) =
  \Gm \bb{\ee{m (m - 1)^x p^y}}.
\end{equation}

The protein production term involving $\PP(m, p - 1)$ can be reindexed defining
$p' \equiv p - 1$. This gives
\begin{equation}
  \Rp \smp (m) m^x p^y \PP(m, p - 1) =
  \Rp \sum_m \sum_{p'=-1}^\infty m^{(x + 1)} (p + 1)^y \PP(m, p').
\end{equation}
We use again the fact that negative molecule copy numbers are assigned with
probability zero to begin the sum from 0 rather than -1 and the arbitrary nature
of the label $p'$ to write
\begin{equation}
  \Rp \sum_m \sum_{p'=0}^\infty m^{(x + 1)} (p + 1)^y \PP(m, p') =
  \Rp \bb{\ee{m^{(x + 1)} (p + 1)^y}}.
\end{equation}

Finally we take care of the protein degradation term involving $\PP(m, p + 1)$.
As before we define $p' = p + 1$ and substitute this to obtain
\begin{equation}
  \Gp \smp (p + 1) m^x p^y \PP(m, p + 1) =
  \Gp \sum_m \sum_{p'=1}^\infty (p') m^x (p' - 1)^y \PP(m, p').
\end{equation}
Just as with the mRNA degradation term having a term $p'$  inside the sum allow
us to start the sum over $p'$ from 0 rather than 1. We can therefore write
\begin{equation}
  \Gp \sum_m \sum_{p'=0}^\infty (p') m^x (p' - 1)^y \PP(m, p') =
  \Gp \bb{\ee{m^x p (p - 1)^y}}.
\end{equation}

Putting all these terms together we can write the general moment PDE. This is
of the form
\begin{equation}
  \begin{aligned}
    \dt{\bb{\ee{m^x p^y}}} &=
    \Km \bb{\ee{m^x p^y}}\\
    &- \Rm \bb{\ee{m^x p^y}} + \Rm \bb{\ee{(m+1)^x p^y}}\\
    &- \Gm \bb{\ee{m^{(x + 1)} p^y}} + \Gm \bb{\ee{m (m - 1)^x p^y}}\\
    &- \Rp \bb{\ee{m^{(x + 1)} p^y}} + \Rp \bb{\ee{m^{(x + 1)} (p + 1)^y}}\\
    &- \Gp \bb{\ee{m^x p^{(y + 1)}}} + \Gp \bb{\ee{m^x p (p - 1)^y}}.
  \end{aligned}
  \label{seq_mom_pde}
\end{equation}

\subsection{Moment closure of the simple-repression distribution}

A very interesting and useful feature of \eref{seq_mom_pde} is that for a given
value of $x$ and $y$ the moments $\bb{\ee{m^x p^y}}$ is only a function of lower
moments. Specifically $\bb{\ee{m^x p^y}}$ is a function of moments
$\bb{\ee{m^{x'} p^{y'}}}$ that satisfy two conditions:
\begin{equation}
  \begin{aligned}
    &1) y' \leq y,\\
  &2) x' + y' \leq x + y.
  \end{aligned}
  \label{seq_mom_conditions}
\end{equation}

To prove this we rewrite \eref{seq_mom_pde} as
\begin{equation}
  \begin{aligned}
    \dt{\bb{\ee{m^x p^y}}} &=
    \Km \bb{\ee{m^x p^y}}\\
    &+ \Rm \bb{\ee{p^y \left[ (m + 1)^x -m^x \right]}}\\
    &+ \Gm \bb{\ee{m p^y \left[ (m - 1)^x - m^x \right]}}\\
    &+ \Rp \bb{\ee{m^{(x + 1)} \left[ (p + 1)^y - p^y \right]}}\\
    &+ \Gp \bb{\ee{m^x p \left[ (p - 1)^y - p^y \right]}},
    \label{seq_mom_pde_factorized}
  \end{aligned}
\end{equation}
where the factorization is valid given the linearity of expected values. The
objective now is to find for each term which is the highest moment once the
binomial is expanded. Take for example a simple case in which we want to find
the second moment of the mRNA. We then set $x = 2$ and $y = 0$.
\eref{seq_mom_pde_factorized} then becomes
\begin{equation}
  \begin{aligned}
    \dt{\bb{\ee{m^2 p^0}}} &=
    \Km \bb{\ee{m^2 p^0}}\\
    &+ \Rm \bb{\ee{p^0 \left[ (m + 1)^2 - m^2 \right]}}\\
    &+ \Gm \bb{\ee{m p^0 \left[ (m - 1)^2 - m^2 \right]}}\\
    &+ \Rp \bb{\ee{m^{(2 + 1)} \left[ (p + 1)^0 - p^0 \right]}}\\
    &+ \Gp \bb{\ee{m^2 p \left[ (p - 1)^0 - p^0 \right]}}.
  \end{aligned}
\end{equation}
Simplifying this equation gives
\begin{equation}
    \dt{\bb{\ee{m^2}}} =
    \Km \bb{\ee{m^2}}
    + \Rm \bb{\ee{\left[ 2m + 1 \right]}}
    + \Gm \bb{\ee{\left[- 2m^2 + m \right]}}.
    \label{seq_second_mom_mRNA}
\end{equation}

\eref{seq_second_mom_mRNA} satisfies both of our conditions. Since we set $y$ to
be zero, non of the terms depend on any moment that involves the protein number,
therefore $y' \leq y$ is satisfied. Also the highest moment in
\eref{seq_second_mom_mRNA}  satisfies $x' + y' \leq x + y$ since the second
moment of mRNA doesn't depend on any moment higher than $\bb{\ee{m^2}}$. To
demonstrate this is true for any $x$ and $y$ we now simplify
\eref{seq_mom_pde_factorized} making use of the binomial expansion
\begin{equation}
  (z \pm 1)^n = \sum_{k=0}^n {n \choose k} (\pm 1)^{k} z^{n-k}.
\end{equation}
Just as before let's look at each term individually. For the mRNA production
term we have
\begin{equation}
  \Rm \bb{\ee{p^y \left[ (m + 1)^x -m^x \right]}} =
  \Rm \bb{\ee{p^y \left[ \sum_{k=0}^x {x \choose k} m^{x-k} - m^x \right]}}.
\end{equation}
Since on the right-hand side when $k=0$ the term inside the sum is cancelled with the other $\bb{m^x}$ we can simplify
this to
\begin{equation}
  \Rm \bb{\ee{p^y \left[ (m + 1)^x -m^x \right]}} =
  \Rm \bb{\ee{p^y \left[ \sum_{k=1}^x {x \choose k} m^{x-k} \right]}}.
\end{equation}
Once the sum is expanded we can see that the highest moment in this sum is given
by $\bb{\ee{m^{(x-1)} p^y}}$ which satisfies both of the conditions on
\eref{seq_mom_conditions}.

For the mRNA degradation term we similarly have
\begin{equation}
  \Gm \bb{\ee{m p^y \left[ (m - 1)^x - m^x \right]}} =
  \Gm \bb{\ee{m p^y \left[ \sum_{k=0}^x {x \choose k}(-1)^k m^{x-k} -
                          m^x \right]}}.
\end{equation}
Simplifying terms we obtain
\begin{equation}
  \Gm \bb{\ee{m p^y \left[ \sum_{k=0}^x {x \choose k}(-1)^k m^{x-k} -
                          m^x \right]}} =
  \Gm \bb{\ee{p^y \left[ \sum_{k=1}^x {x \choose k}(-1)^k m^{x+1-k} \right]}}.
\end{equation}
For this case we have that the largest moment is therefore $\bb{\ee{m^x p^y }}$
which again satisfies the conditions on \eref{seq_mom_conditions}.

The protein production term gives
\begin{equation}
  \Rp \bb{\ee{m^{(x + 1)} \left[ (p + 1)^y - p^y \right]}} =
  \Rp \bb{\ee{m^{(x + 1)} \left[ \sum_{k=0}^y {y \choose k} (-1)^k p^{y-k}
                                - p^y \right]}}.
\end{equation}
Upon simplification we obtain
\begin{equation}
  \Rp \bb{\ee{m^{(x + 1)} \left[ \sum_{k=0}^y {y \choose k} (-1)^k p^{y-k}
                                - p^y \right]}} =
  \Rp \bb{\ee{m^{(x + 1)} \left[ \sum_{k=1}^y {y \choose k}(-1)^k p^{y-k}
  \right]}}.
\end{equation}
Here the largest moment is given by $\bb{\ee{m^{x+1} p^{y-1}}}$, that again
satisfies both of our conditions. For the last term having to do with protein
degradation we have
\begin{equation}
  Rp \bb{\ee{m^{(x + 1)} \left[ (p + 1)^y - p^y \right]}} =
  Rp \bb{\ee{m^{(x + 1)} \left[ \sum_{k=1}^y {y \choose k} (-1^k) p^{y - k}
  \right]}}.
\end{equation}
The largest moment involved in this term is therefore $\bb{\ee{m^x p^{y-1}}}$.
Having showed that the four terms involved in our general moment equation depend
on lower moments that satisfy \eref{seq_mom_conditions}.
