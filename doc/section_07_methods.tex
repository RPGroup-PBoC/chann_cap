\section{Methods}

\subsection{Growth conditions}

For all experiments cultures were initiated from a 50\% glycerol frozen stock at
-80$^\circ$C. Three strains - autofluorescence ($auto$), $\Delta lacI$
($\Delta$), and a strain with a known binding site and repressor copy number
($R$) - were inoculated into individual tubes with 2 mL of Lysogeny Broth (LB
Miller Powder, BD Medical) with 20 $\mu$g/mL of chloramphenicol and 30 $\mu$g/mL
of kanamycin. These cultures were grown overnight at 37$^\circ$C and rapid
agitation to reach saturation. The saturated cultures were diluted 1:1000 into
500 $\mu$L of M9 minimal media (M9 5X Salts, Sigma-Aldrich M6030; 2 mM magnesium
sulfate, Mallinckrodt Chemicals 6066-04; 100 mM calcium chloride, Fisher
Chemicals C79-500) supplemented with 0.5\% (w/v) glucose on a 2 mL 96-deep-well
plate. The $R$ strain was diluted into 12 different wells with minimal media,
each with a different IPTG  concentration (0 $\mu$M, 0.1 $\mu$M, 5 $\mu$M, 10
$\mu$M, 25 $\mu$M, 50 $\mu$M, 75 $\mu$M, 100 $\mu$M, 250 $\mu$M, 500 $\mu M$,
1000 $\mu$M, 5000 $\mu$M) while  the $auto$ and $\delta$ strains were diluted
into two wells (0 $\mu$M, 5000 $\mu$M). Each of the IPTG concentration came from
a single preparation stock kept in 100-fold concentrated aliquots. The 96 well
plate was then incubated at 37$^\circ$C with rapid agitation for 8 hours before
imaging.

\subsection{Microscopy imaging procedure}

The microscopy pipeline used for this work followed exactly the steps from
\cite{Razo-Mejia2018}. Briefly, twelve 2\% agarose (Life Technologies UltraPure
Agarose, Cat.No. 16500100) gels were made out of M9 media (or PBS buffer) with
the corresponding IPTG concentration (See growth conditions) and placed between
two glass coverslips for them to solidify after microwaving.

After the 8 hour incubation in minimal media 1 $\mu$L of a 1:10 dilution of the
cultures was placed into small squares (roughly 10 mm $\times$ 10 mm) of the
different agarose gels. A total of 16 agarose squares - 12 concentrations of
IPTG for the $R$ strain, 2 concentrations for the $\Delta$ and 2 for the $auto$
strain - were mounted into a single glass-bottom dish (Ted Pella Wilco Dish,
Cat. No. 14027-20) that was sealed with parafilm.

All imaging was done on an inverted fluorescent microscope (Nikon Ti-Eclipse)
with custom-built laser illumination system. The YFP fluorescence (quantitative
reporter) was imaged with a CrystaLaser 514 nm excitation laser coupled with a
laser-optimized (Semrock Cat. No. LF514-C-000) emission filter. All strains,
including the $auto$ strain included a constitutively expressed mCherry protein
to aid for the segmentation. Therefore for each image 3 channels YFP, mCherry,
and phase contrast were acquired.

On average 30 images with roughly 20 cells per condition were taken. 25 images
of a fluorescent slide and 25 images of the camera background noise were taken
every time in order to flatten the illumination. The image  processing pipeline
for this work is exactly the same as \cite{Razo-Mejia2018}.
