\subsection*{Maximum Entropy approximation}

Having numerically computed the moments of the mRNA and protein distributions as
cells progress through the cell cycle we now proceed to make an approximating
reconstruction of the full distributions given this limited information. As
hinted in \secref{sec_moments} the maximum entropy (MaxEnt) principle, first
proposed by E.T. Jaynes in 1957, approximates the entire distribution by
maximizing the Shannon entropy subject to constraints given by, among other
quantities, the values of the moments of the distribution \cite{Jaynes1957}.
This procedure leads to a Boltzmann distribution of the form \mrm{See appendix
XX for full derivation.}
\begin{equation}
  P_H(m, p) = {1 \over \mathcal{Z}}
              \exp \left( - \sum_{(x,y)} \lambda_{(x,y)} m^x p^y \right),
  \label{eq_maxEnt_joint}
\end{equation}
where $\lambda_{(x,y)}$ is the Lagrange multiplier associated with the
constraint set by the moment $\ee{m^x p^y}$, and $\mathcal{Z}$ is a
normalization constant. The more moments $\ee{m^x p^y}$ included as constraints,
the more accurate the approximation of \eref{eq_maxEnt_joint} becomes.

The computational challenge then becomes a minimization routine in which the
values for the Lagrange multipliers $\lambda_{(x,y)}$ that are consistent with
the constraints set by the moments values $\ee{m^x p^y}$ need to be found.
\mrm{Appendix XX} details our implementation of a robust algorithm to find such
values. \fref{fig4_maxent} shows example predicted protein distributions
reconstructed using six moments of the protein distribution for a suite of
different biophysical parameters and environmental inducer concentrations. As
repressor-DNA binding affinity (columns in \fref{fig4_maxent}) and repressor
copy number (rows in \fref{fig4_maxent}) are varied, the responses to different
signals (i.e. inducer concentrations) overlap with different degrees. For
example the upper right corner frame with a weak binding site ($\eR = -9.7 \;
k_BT$) and a low repressor copy number (22 repressors per cell) has virtually
identical distributions regardless of the input inducer concentration. This
means that cells with this set of parameters cannot resolve a difference in any
concentration of the signal. As the number of repressors is increased along the
rows, the degree of overlap between distributions changes, allowing cells to
better resolve the value of the signal input. On the opposite extreme the lower
left panel shows a strong binding site ($\eR = -15.3 \; k_BT$) and a high
repressor copy number (1740 repressors per cell) shows overlap between
distributions due to the lack of ability of the system to respond to the inducer
given the high degree of repression, giving again little ability for the cells
to resolve the inputs. In the following section we formalize the notion of how
well can cells resolve different inputs from an information theoretic
perspective via the channel capacity.

\begin{figure}[h!]
	\centering \includegraphics
  {./fig/main/PMF_grid_joyplot_protein.pdf}
	\caption{\textbf{MaxEnt protein distributions for varying physical
	parameters.} Protein distributions under different inducer (IPTG)
	concentrations for different combinations of repressor-DNA affinities
	(columns) and repressor copy numbers (rows). The first six moments of the
	protein distribution used to constrain the MaxEnt approximation were computed
	by integrating \eref{eq_gral_mom} as cells progressed through the cell cycle
	as described in \secref{sec_cell_cycle}.}
  \label{fig4_maxent}
\end{figure}
