\subsection{Accounting for cell-cycle dependent variability in gene dosage}
\label{sec_cell_cycle}

As cells progress through the cell cycle, the genome has to be replicated to
guarantee that each daughter cell receives a copy of the genetic material.
This replication of the genome implies that cells spend part of the cell cycle
with multiple copies of each gene depending on the cellular growth rate and the
relative position of the gene with respect to the replication origin
\cite{Bremer1996}. Genes closer to the replication origin spend a larger
fraction of the cell cycle with multiple copies compared to genes closer to the
replication termination site \cite{Bremer1996}. \fref{fig3_cell_cycle}(A)
depicts a schematic of this process where the replication origin ({\it oriC})
and the relevant locus for our experimental measurements ({\it galK}) are
highlighted.
\note{GC: I think that your argument would be strengthened with an estimate.
Show that the fraction spent with two copies is $\gg r_m, \gm$}

Since this change in gene copy number has been shown to have an effect on
cell-to-cell variability in gene expression \cite{Jones2014a, Peterson2015}, we
now extend our minimal model to account for these changes in gene copy number
during the cell cycle.  We reason that the only difference between the
single-copy state and the two-copies states of the promoter is a doubling of the
mRNA production rate $r_m$. In particular the RNAP rates $\kpon$ and $\kpoff$
and the mRNA production rate $r_m$ inferred in \secref{sec_model}
assume that cells spend a fraction $f$ of the cell cycle  with one copy of the
promoter (mRNA production rate $r_m$) and a fraction $(1-f)$ of the cell cycle
with two copies of the promoter (mRNA production rate $2 r_m$). This inference
was performed considering that at each cell state the mRNA level immediately
reaches the steady state value for the corresponding mRNA production rate. The
steady state assumption is justified since the timescale to reach this steady
state depends only on the degradation rate $\gm$, which for the mRNA  is much
shorter ($\approx 3$ min) than the length of the cell cycle (100 min for our
experimental conditions).
\mrm{Cite for the mRNA degradation rate.}
\siref{supp_param_inference} shows that a model
accounting for this gene copy number variability is able to capture the
experimental data from single molecule mRNA counts of an unregulated promoter.

Given that the protein degradation rate $\gp$ in our model is set by the cell
division time, we cannot consider that the protein count reaches the
corresponding steady state value for each stage in the cell cycle. In other
words, cells do not spend long enough with two copies of the promoter for the
protein to reach the steady state value corresponding to a transcription rate of
$2 r_m$. We therefore use the dynamical equations developed in
\secref{sec_moments} to numerically integrate the moments of the distribution
dynamics with the corresponding parameters for each phase of the cell cycle.
\fref{fig3_cell_cycle}(B) shows an example corresponding to the mean mRNA level
(upper panel) and the mean protein level (lower panel) for the case of the
unregulated promoter. Since we inferred the RNAP rate parameters considering
that mRNA reaches steady state at each stage, we see that the numerical
integration of the equations is consistent with the assumption of having the
mRNA reach a stable value at each stage (See \fref{fig3_cell_cycle}(B) upper
panel). On the other hand, the mean protein level does not reach a steady state
at each of the cellular stages. Nevertheless it is interesting to observe that
after a couple of cell cycles the trajectory from cycle to cycle follows a
repetitive pattern (See \fref{fig3_cell_cycle}(B) lower panel).

To test the effects of including this gene copy number variability in our model
we now compare the predictions of the model with experimental data. Specifically
as detailed in \mrm{Methods} we obtained single-cell fluorescence values of
different strains under twelve different inducer concentrations. The strains
imaged spanned three orders of magnitude in repressor copy number and variable
repressor-DNA affinities. Since growth was asynchronous, we reason that cells
were sampled at any stage of the cell cycle. Therefore when computing statistics
from the data such as the mean fluorescence value, in reality we are averaging
over the cell cycle. In other words, as depicted in \fref{fig3_cell_cycle}(B)
quantities such as the mean protein copy number change over time, i.e. $\ee{p}
\equiv \ee{p(t)}$. This means that computing the mean of a population of
unsynchronized cells is equivalent to averaging this mean protein copy number
over the span of the cell cycle. Mathematically this is expressed as
\begin{equation}
	\ee{p}_c = \int_{t_o}^{t_d} \ee{p(t)} P(t) dt,
	\label{eq_time_avg}
\end{equation}
where $\ee{p}_c$ represents the average protein copy number over a cell cycle,
$t_o$ represents the start of the cell cycle, $t_d$ represents the time of cell
division, and $P(t)$ represents the probability of any cell being at time $t \in
[t_o, t_d]$ of their cell cycle. We do not consider cells uniformly distributed
along the cell cycle since it is known that cells follow an exponential
distribution, having more younger than older cells at any time point
\cite{Powell1956}. All computations hereafter are therefore done by applying an
averaging like the one in \eref{eq_time_avg} for the span of a cell cycle.

\fref{fig3_cell_cycle}(C) compares zero-parameter fit predictions (solid lines)
with experimentally determined quantities. The upper row shows the
non-dimensional quantity known as the fold-change in gene expression
\cite{Garcia2011c}. This fold-change is defined as the relative mean expression
level with respect to an unregulated promoter. For protein this is
\begin{equation}
	\foldchange = {\ee{p(R \neq 0)}_c \over \ee{p(R = 0)}_c},
\end{equation}
where $\ee{p(R \neq 0)}_c$ represents the mean protein count for cells with
non-zero repressor copy number $R$ count over the entire cell cycle, and
$\ee{p(R = 0)}_c$ represents the equivalent for a strain with no repressors
present. The experimental points were determined from the fluorescent
intensities of cells with varying repressor copy number and a $\Delta lacI$
strain with no repressor gene present \mrm{See image analysis appendix for
further details}. The fold-change in gene expression has served as a metric to
test the validity of equlibrium-based models \cite{Phillips2015}. We note that
the curves shown in the upper panel of \fref{fig3_cell_cycle}(C) are consistent
with the predictions from equilibrium models \cite{Razo-Mejia2018} despite being
generated from a clearly non-equilibrium process as shown in
\fref{fig3_cell_cycle}(B). The kinetic model from \fref{fig2_minimal_model}(A)
goes beyond the equilibrium picture to generate predictions for moments of the
distribution other than the mean mRNA or mean protein count. To test this
extended predictive power the lower row of \fref{fig3_cell_cycle}(C) shows the
noise in gene expression defined as the standard deviation over the mean protein
count. The good correspondence between the theoretical predictions and the
experimental data is only achieved when considering the gene copy number
variability treated in this section. (See \siref{supp_multi_gene} for
comparison with single-copy predictions).

\begin{figure}[h!]
	\centering \includegraphics
  {./fig/main/cell_cycle_moments_v01.pdf}
	\caption{\textbf{Accounting for gene copy number variability during the cell
	cycle.} (A) Schematic of a replicating genome. As cells progress through the
	cell cycle the genome is replicated, duplicating gene copies for a fraction of
	the cell cycle. {\it oriC} indicates the replication origin, and {\it galK}
	indicates the locus at which the reporter construct was integrated. (B) mean
	mRNA (upper panel) and mean protein (lower panel) dynamics. Cells spend a
	fraction of the cell cycle with a single copy of the promoter (light brown)
	and the rest of the cell cycle with two copies (light yellow). Black arrows
	indicate time of cell division. (C) Zero parameter-fit predictions (lines) and
	experimental data (circles) of the fold-change (upper row) and noise (lower
	row) for different repressor binding sites with different affinities and
	different repressor copy numbers per cell. Dotted line indicate linear scale
	while solid line indicate logarithmic scale. White dots on the lower row are
	plotted on a different scale for visual clarity.}
  \label{fig3_cell_cycle}
\end{figure}
