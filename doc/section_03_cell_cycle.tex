\subsection{Accounting for cell-cycle dependent variability in gene dosage}
\label{sec_cell_cycle}

As cells progress through the cell cycle, the genome has to be replicated to
guarantee that each daughter cell receives a copy of the genetic material. As
replication of the genome can take longer than the total cell cycle, this
implies that cells spend part of the cell cycle with multiple copies of each
gene depending on the cellular growth rate and the relative position of the
gene with respect to the replication origin \cite{Bremer1996}. Genes closer to
the replication origin spend a larger fraction of the cell cycle with multiple
copies compared to genes closer to the replication termination site
\cite{Bremer1996}. \fref{fig3_cell_cycle}(A) depicts a schematic of this
process where the replication origin ({\it oriC}) and the relevant locus for
our experimental measurements ({\it galK}) are highlighted.

Since this change in gene copy number has been shown to have an effect on
cell-to-cell variability in gene expression \cite{Jones2014a, Peterson2015}, we
now extend our minimal model to account for these changes in gene copy number
during the cell cycle.  We reason that the only difference between the
single-copy state and the two-copy state of the promoter is a doubling of the
mRNA production rate $r_m$. In particular, the promoter activation and
inactivation rates $\kpon$ and $\kpoff$ and the mRNA production rate $r_m$
inferred in \secref{sec_model} assume that cells spend a fraction $f$ of the
cell cycle  with one copy of the promoter (mRNA production rate $r_m$) and a
fraction $(1-f)$ of the cell cycle with two copies of the promoter (mRNA
production rate $2 r_m$). This inference was performed considering that at each
cell state the mRNA level immediately reaches the steady state value for the
corresponding mRNA production rate. This assumption is justified since the
timescale to reach this steady state depends only on the degradation rate
$\gm$, which for the mRNA  is much shorter ($\approx 3$ min) than the length of
the cell cycle ($\approx$ 60 min for our experimental conditions)
\cite{Dong1995}. \siref{supp_param_inference} shows that a model accounting for
this gene copy number variability is able to capture data from single molecule
mRNA counts of an unregulated (constitutively expressed) promoter.

Given that the protein degradation rate $\gp$ in our model is set by the cell
division time, we do not expect that the protein count will reach the
corresponding steady state value for each stage in the cell cycle. In other
words, cells do not spend long enough with two copies of the promoter for the
protein level to reach the steady state value corresponding to a transcription
rate of $2 r_m$. We therefore use the dynamical equations developed in
\secref{sec_moments} to numerically integrate the time trajectory of the
moments of the distribution with the corresponding parameters for each phase of
the cell cycle. \fref{fig3_cell_cycle}(B) shows an example corresponding to the
mean mRNA level (upper panel) and the mean protein level (lower panel) for the
case of the unregulated promoter. Given that we inferred the promoter rates
parameters considering that mRNA reaches steady state at each stage, we see
that the numerical integration of the equations is consistent with the
assumption of having the mRNA reach a stable value at each stage (See
\fref{fig3_cell_cycle}(B) upper panel). On the other hand, the mean protein
level does not reach a steady state at either of the cellular stages.
Nevertheless it is notable that after a couple of cell cycles the trajectory
from cycle to cycle follows a repetitive pattern (See \fref{fig3_cell_cycle}(B)
lower panel). Previously we have experimentally observed this repetitive
pattern by tracking the expression level over time with video microscopy as
observed in Fig. 18 of \cite{Phillips2019}.

To test the effects of including this gene copy number variability in our model
we now compare the predictions of the model with experimental data.
Specifically as detailed in the Methods section we obtained single-cell
fluorescence values of different {\it E. coli} strains carrying a YFP protein
under the control of the $lacI$ repressor. Each strain was exposed to twelve
different inducer concentrations until reaching steady-state expression. The
strains imaged spanned three orders of magnitude in repressor copy number and
three distinct repressor-DNA affinities. Since growth was asynchronous, we
reason that cells were randomly sampled at all stages of the cell cycle.
Therefore when computing statistics from the data such as the mean fluorescence
value, in reality we are averaging over the cell cycle. In other words, as
depicted in \fref{fig3_cell_cycle}(B) quantities such as the mean protein copy
number change over time, i.e. $\ee{p} \equiv \ee{p(t)}$. This means that
computing the mean of a population of unsynchronized cells is equivalent to
averaging this time dependent mean protein copy number over the span of the
cell cycle. Mathematically this is expressed as
\begin{equation}
	\ee{p}_c = \int_{t_o}^{t_d} \ee{p(t)} P(t) dt,
	\label{eq_time_avg}
\end{equation}
where $\ee{p(t)}$ represents the first moment of the protein distribution as
computed from \eref{eq_gral_mom}, $\ee{p}_c$ represents the average protein
copy number over a cell cycle, $t_o$ represents the start of the cell cycle,
$t_d$ represents the time of cell division, and $P(t)$ represents the
probability of any cell being at time $t \in [t_o, t_d]$ of their cell cycle.
We do not consider cells uniformly distributed along the cell cycle since it is
known that cells follow an exponential distribution, having more younger than
older cells at any time point \cite{Powell1956} (See \siref{supp_cell_age_dist}
for further details). All computations hereafter are therefore done by applying
an average like that in \eref{eq_time_avg} for the span of a cell cycle. We
remind the reader that these time averages are done under a fixed environmental
state. It is the trajectory of cells over cell cycles under a constant
environment what we need to account for.
\sm{From an information theory point of view this is interesting because you’re
breaking stationarity.  Would be good to mention here how you get around this
in capacity calculations by doing this averaging.  It’s exactly how you turn a
periodic process into something “stationary”.}

\fref{fig3_cell_cycle}(C) compares zero-parameter fit predictions (lines) with
experimentally determined quantities (points). The upper row shows the
non-dimensional quantity known as the fold-change in gene expression
\cite{Garcia2011c}. This fold-change is defined as the relative mean gene
expression level with respect to an unregulated promoter. For protein this is
\begin{equation}
	\foldchange = {\ee{p(R > 0)}_c \over \ee{p(R = 0)}_c},
\end{equation}
where $\ee{p(R \neq 0)}_c$ represents the mean protein count for cells with
non-zero repressor copy number count $R$ over the entire cell cycle, and
$\ee{p(R = 0)}_c$ represents the equivalent for a strain with no repressors
present. The experimental points were determined from the fluorescent
intensities of cells with varying repressor copy number and a $\Delta lacI$
strain with no repressor gene present (See Methods for further details). The
fold-change in gene expression has previously served as a metric to test the
validity of equilibrium-based models \cite{Phillips2015}. We note that the
curves shown in the upper panel of \fref{fig3_cell_cycle}(C) are consistent
with the predictions from equilibrium models \cite{Razo-Mejia2018} despite
being generated from a clearly non-equilibrium process as shown in
\fref{fig3_cell_cycle}(B). The kinetic model from \fref{fig2_minimal_model}(A)
goes beyond the equilibrium picture to generate predictions for moments of the
distribution other than the mean mRNA or mean protein count. To test this
extended predictive power the lower row of \fref{fig3_cell_cycle}(C) shows the
noise in gene expression defined as the standard deviation over the mean
protein count accounting for the changes in gene dosage during the cell cycle.
\siref{supp_multi_gene} shows the equivalent results when this gene duplication
is not taken into account. For the noise our model tracks the experimental data
up to a systematic multiplicative factor (See \siref{supp_multi_gene} for
details). Possible origins could be the intrinsic cell-to-cell variability of
rate parameters given the variability in the molecular components of the
central dogma machinery \cite{Jones2014a}. Nevertheless we remind the reader
that all predictions presented in this paper are zero parameter fits; making
the model still highly informative about the physical nature of how cells
regulate their gene expression. This lack of exact numerical agreement between
theory and data open an opportunity to gain new insights into the biophysical
origin of cell-to-cell variability. 

\begin{figure}[h!]
	\centering \includegraphics
  {./fig/main/fig3_cell_cycle_moments.pdf}
	\caption{\textbf{Accounting for gene copy number variability during the
	cell cycle.} (A) Schematic of a replicating bacterial genome. As cells
	progress through the cell cycle the genome is replicated, duplicating gene
	copies for a fraction of the cell cycle. {\it oriC} indicates the
	replication origin, and {\it galK} indicates the locus at which the
	reporter construct was integrated.(B) mean (solid line) $\pm$ standard
	deviation (shaded region) for the mRNA (upper panel) and protein (lower
	panel) dynamics. Cells spend a fraction of the cell cycle with a single
	copy of the promoter (light brown) and the rest of the cell cycle with two
	copies (light yellow). Black arrows indicate time of cell division. (C)
	Zero parameter-fit predictions (lines) and experimental data (circles) of
	the gene expression fold-change (upper row) and noise (lower row) for
	repressor binding sites with different affinities (different columns) and
	different repressor copy numbers per cell (different lines on each panel).
	Error bars in data represent the 95\% confidence interval on the quantities
	as computed from 10,000 bootstrap estimates generated from > 500
	single-cell fluorescence measurements. In the theory curves dotted lines
	indicate plot in linear scale to include zero while solid lines indicate
	logarithmic scale. For visual clarity, data points in the noise panel with
	really large values were plot on a separate panel with a log scale.}
	\label{fig3_cell_cycle}
\end{figure}
