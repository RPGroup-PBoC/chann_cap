\label{sec_multi_gene}
\section{Accounting for the variability in gene copy number during the cell
cycle}

When growing in rich media, bacteria can double every $\approx$ 20 minutes. With
a replication fork that travels at $\approx$ 1000 bp per second, and a genome of
$\approx$ 5 Mbp for {\it E. coli}\cite{Moran2010}, a cell would need $\approx$
80 minutes to replicate its genome. The apparent paradox is solved by the fact
that cells have multiple replisomes, i.e. molecular machines that replicate the
genome running in parallel. Cells can have up to 8 copies of the genome being
replicated at all time depending on the growth rate \cite{Bremer1996}.

That means that during the cell cycle gene copy number varies. This variation
depends on the relative position of the gene with respect to the replication
origin, having genes close to the origin spending more time with multiple copies
compare to genes closer to the termination. This change in gene dosage has a
direct effect on the cell-to-cell variability in gene expression
\cite{Jones2014a, Peterson2015}. Furthermore, since the time to reach steady
state is determined by the degradation rate, mRNA and protein will experience
experience these changes in gene copy number differently from each other.

\subsection{Numerical integration of moment equations}

For our specific locus ({\it galK}) and a doubling time of $\approx$ 100 min,
cells have on average 1.4 copies of the reporter gene during the cell cycle.
What this means is that cells spend 60\% of the time having one copy of the gene
and 40\% of the time with two copies. Our model needs to account for this
variability in gene copy number along the cell cycle. To do so we numerically
integrate the moment equations derived in \mrm{reference moment derivation
section} for a time $t = [0, t_s]$ with an mRNA production rate $r_m$, where
$t_s$ is the time point at which the replication fork reaches our specific
locus. For the remaining time before the cell division $t = [t_s, t_d]$ that the
cell spends with two promoters, we assume that the only parameter that changes
is the mRNA production rate from $r_m$ to $2 r_m$. This simplifying assumption
ignores potential changes in protein translation rate $r_p$ or changes in the
repressor copy number that would be reflected in changes on the repressor on
rate $\kron$.

\subsubsection{Initial conditions for numerical integration}

In order to define the initial conditions for the numerical integration for each
of the cell cycle stages we follow a simple procedure:
\begin{enumerate}
  \item Initialize the zeroth moment, i.e. the probability of the promoter of
  being on each of the two (unregulated) or three (regulated) states at any
  value, constrained to the fact that the sum of all zeroth moments should add
  up to one.
  \item Integrate moment equations for a long time using parameters
  corresponding to a single promoter. This will set the initial conditions to
  be used for the single promoter case
  \item Integrate the moment equations for a time $t = [0, t_s]$ using the
  initial conditions determined in the previous step.
  \item Update parameters to the ones corresponding to two promoter copies.
  \item Integrate the moment equations for a time $t = [t_s, t_d]$ using the
  last time point of the single promoter time as initial conditions.
  \item Divide the mRNA first moment $\ee{\bb{m}}$ and the protein first moment
  $\ee{\bb{p}}$ by two to represent the cell division.
  \item Integrate the moment equations for a long time with the constraint that
  the zeroth moment, the mRNA and protein first moments remain fixed as defined
  by halving the values from the previous step.
  \item Repeat process from step 3, using as initial conditions the last point
  of the integration in step 7.
\end{enumerate}

Step 7 in our procedure to determine initial conditions serves as a way to
determine how higher moments change as the cells divide. Only the first moments
of both mRNA and protein can simply be halved after cell division, but it is
less clear how higher moments of the distribution change as cells half their
content. To visualize why this is not a trivial step let's consider a binomially
distributed variable $X \sim \text{Bin}(N, p)$. For the first moment, given by
$\ee{X} = Np$ is easy to see that upon halving the number of trials $N$, the new
first moment would simply be $\ee{X}_{\text{new}} = {N \over 2} p$. But for the
second moment given by $\ee{X^2} = Np - Np^2 + N^2p^2$ it is obvious that
halving the number of attempts does not translate to halving the second moment.
Given this non-linear transformation between moments, we concluded that the
easiest way to converge to the value of higher moments after cell division was
to simply integrate the moment equations keeping the first moments fixed until
higher moments relaxed to a steady state given these constrained conditions.

\fref{fig_first_mom_cycles} shows how the first moment of both mRNA and protein
changes over several cell cycles. The mRNA seems to quickly relax to the steady
state corresponding to the parameters for both a single and two promoter copies.
This is expected since the parameters for the mRNA production were determined
in the first place under this assumption (See \mrm{cite mRNA MCMC section}). On
the other hand given that this relaxation time is determined by the degradation
rate the protein doesn't reach such steady state for either case. Interestingly
Once a couple of cell cycles have passed the cells seem to have a reproducible
trajectory over cell cycles. We call this a dynamic steady-state for the
protein.

\begin{figure}[h!]
	\centering \includegraphics
  {../fig/moment_dynamics_numeric/first_mom_cycles.pdf}
	\caption{\textbf{First moment dynamics over cell the cell cycle.}(A) mean
	mRNA and (B) mean protein copy number as the cell cycle progresses. The light
	shaded region delimits the fraction of the cell cycle that cells spend with  a
	single copy of the promoter. The dark shaded region delimits the fraction of
	the cell cycle that cells spend with two copies of the promoter. For a 100
  min doubling time at the {\it galK} locus cells spend 60\% of the time with
  one copy of the promoter and the rest with two copies.}
  \label{fig_first_mom_cycles}
\end{figure}

Since our experiments were not performed with synchronized cells, in principle
we sampled cells over the entirety of the cell cycle, so the moments that we
determined experimentally correspond to an average over the cell cycle.  In
order to compute these averages in the following section we discuss how to
account for the fact that cells are not uniformly distributed along the cell
cycle

\subsection{Exponentially distributed ages}

As first mentioned in Section \mrm{ref to mRNA MCMC section}, cells in a log
phase have exponentially distributed ages along the cell cycle, having more
young cells compared to old ones. Specifically the probability of a cell of
being at any time point in the cell cycle is given by \cite{Powell1956}
\begin{equation}
  P(a) = (\ln 2) \cdot 2^{1 - a},
  \label{eq_age_prob}
\end{equation}
where $a \in [0, 1]$ is the stage of the cell cycle, with $a = 0$ being the
start of the cycle and $a = 1$ being the division.

Our numerical integration of the moment equations gave us a time evolution of
the moments along the cell cycle. Without loss of generality let's focus on the
first mRNA moment $\ee{m(t)}$ (the same can be applied to all other moments).
In order to calculate the first moment along the entire cell cycle we must
average each time point by the corresponding probability that a cell is found
in such time point. This translates to computing the integral
\begin{equation}
  \ee{m} = \int_{\text{beginning cell cycle}}^{\text{end cell cycle}}
                       \ee{m(t)} P(t) dt.
\end{equation}

If we map each time point in the cell cycle into a fraction we can use
\eref{eq_age_prob} and compute instead
\begin{equation}
  \ee{m} = \int_0^1 \ee{m(a)} P(a) da.
  \label{eq_moment_avg}
\end{equation}
We perform this integral numerically for all moments using Simpson's rule.

\subsection{Reproducing the equilibrium picture}

Given the large variability of the first moments depicted in
\fref{fig_first_mom_cycles} it is worth considering why a simplistic equilibrium
picture has shown to be very successful in predicting the mean expression level
under diverse conditions \cite{Garcia2011c, Brewster2014, Barnes2018,
Razo-Mejia2018}. In this section we compare the simple repression thermodynamic
model with this dynamical picture of the cell cycle. But first it is worth
recapping the assumptions that go into the equilibrium model.

\subsubsection{Steady state under the equilibrium picture}

Since our equilibrium model only work for the thermodynamic limit of large
number of particles, using this theoretical framework we are only allowed to
describe the dynamics of the first moment \cite{Phillips2015}. Again without
loss of generality let's only focus on the mRNA first moment $\ee{m}$. The same
principles apply if we think about the protein first moment. We can write a
dynamical system of the form
\begin{equation}
  {d \ee{m} \over dt} = r_m \cdot \pbound - \gm \ee{m},
\end{equation}
where as before $r_m$ and $\gm$ are the mRNA production and degradation rates
respectively, and $\pbound$ is the probability of finding the RNAP
bound to the promoter \cite{Bintu2005a}. This dynamical system is predicted to
have a single stable fixed point that we can find by computing the steady state.
When we solve for $\ee{m}$ at steady state we find
\begin{equation}
  \ee{m}_{ss} = {r_m \over \gm} \pbound.
\end{equation}

Since we assume that the only effect that the repressor has over the regulation
of the promoter is to block the RNAP from binding to the promoter, we assume
that only $\pbound$ depends on the repressor copy number $R$. Therefore when
computing the fold-change in gene expression we  are left with
\begin{equation}
  \foldchange = {\ee{m}_{ss}(R \neq 0) \over \ee{m}_{ss}(R = 0)}
              = {\pbound (R \neq 0) \over \pbound (R = 0)}.
\end{equation}
And as derived elsewhere this is of the form \cite{Garcia2011c}
\begin{equation}
  \foldchange = \left(1 + {R \over \Nns}e^{-\beta \eR}  \right)^{-1},
  \label{eq_fold_change_thermo}
\end{equation}
where $\beta \equiv k_BT^{-1}$, $\eR$ is the repress-DNA binding energy, and
$\Nns$ is the number of non-specific binding sites.

To arrive to \eref{eq_fold_change_thermo} considering both mRNA or protein we
ignore the physiological changes that happen along the cell cycle. One of the
most important being the variability in gene copy number that we are exploring
in this section. It is therefore worth thinking about whether or not the dynamic
picture exemplified in \fref{fig_first_mom_cycles} can be reconciled with the
predictions done by \eref{eq_fold_change_thermo}.

\subsubsection{Steady state under the kinetic picture}

For our kinetic picture with changes over the cell cycle the mRNA (or protein)
moment is computed as the integral defined in \eref{eq_moment_avg}.
\fref{fig_first_mom_cycles}(A) suggests that for the mRNA at least we could
approximate the integral as the sum of the steady state value of the
single-promoter state and the steady state value of the two-promoters state
weighted by their respective probabilities. This is because by construction the
relaxation to steady state happens at a faster time scale compared with the
length of each of the stages along the cell cycle. In other words if $\ee{m}_i$
is the steady state value for the state with $i$ promoters, then we can
approximate the average mRNA copy number as
\begin{equation}
  \ee{m} \approx \ee{m}_1 \int_0^{f} P(a)da +
                 \ee{m}_2 \int_f^{1} P(a)da,
\end{equation}
where $f$ is the fraction of the cell cycle that cells spend with a single
promoter. Since the steady state values $\ee{m}_i$ do not depend on the stage
of the cell cycle, we were able to take them out of the integral. We can further
simplify this equation by pretending that $P(a)$ is uniform, obtaining
\begin{equation}
  \ee{m} \approx f \cdot \ee{m}_1 + (1 - f) \ee{m}_2.
  \label{eq_mRNA_mean_simplified}
\end{equation}
These simplifications will generate discrepancies between the equilibrium and
kinetic results, but as we will show bellow, these errors are within our
experimental resolution.

For both, the unregulated and the regulated promoter it can be shown that
$\ee{m}_2 = 2 \ee{m}_1$. Therefore \eref{eq_mRNA_mean_simplified} can be written
as
\begin{equation}
  \ee{m} \approx (2 - f) \ee{m}_1.
\end{equation}
Therefore the fold-change is given by
\begin{equation}
  \foldchange = {(2 - f) \ee{m}_1(R \neq 0) \over (2 - f) \ee{m}_1(R = 0)}.
\end{equation}
This equation implicitly ignores the changes in the repressor copy number as
cells progress through the cell cycle. This again is another simplifying
assumption that will generate discrepancies between the thermodynamic and the
kinetic results.

Having the dependence on the cell cycle cancel out we arrive to the same
expression as the one defined in \mrm{ref section on parameters}
\begin{equation}
  \foldchange = \left[ 1 + {\kron \over \kroff}
                \left( {\kpoff \over \kpon + \kpoff} \right)\right]^{-1}.
\end{equation}
Therefore we can constraint the ratio ${\kron \over \kroff}$ given the
parameters determined for the equilibrium picture.

\fref{fig_lacI_titration} compares the predictions of both theoretical
frameworks for varying repressor copy numbers and repressor-DNA affinities. The
solid lines are directly computed from \eref{eq_fold_change_thermo}, while the
hollow triangles and the solid circles, representing the fold-change in mRNA and
protein respectively were computed by numerically integrating the moment
equations for both the two- and the three-state promoter (See
\fref{fig_first_mom_cycles} for the unregulated case) and then averaging the
time series accounting for the probability of cells being at each stage of the
cell cycle as defined in \eref{eq_moment_avg}. The systematic deviations between
both models are a consequence of the previously mentioned assumptions. But these
differences are within what we can resolve experimentally with this system
\cite{Garcia2011c, Brewster2014, Barnes2018, Razo-Mejia2018}.

For completeness \fref{fig_IPTG_titration} compares the kinetic and equilibrium
models for the extended model of \cite{Razo-Mejia2018} in which the inducer
concentration enters to the equation. The solid line is directly computed from
Eq. 5 of \cite{Razo-Mejia2018}. The hollow triangles and solid points follow the
same procedure as for \fref{fig_lacI_titration}, where the only effect that the
inducer is assume to have in the kinetics is an effective change in the number
of active repressors, affecting therefore $\kron$. Again the systematic
deviations due to our simplified assumptions fall within the experimental
resolution of the system.

\begin{figure}[h!]
	\centering \includegraphics
  {../fig/moment_dynamics_numeric/lacI_titration.pdf}
	\caption{\textbf{Comparison of the equilibrium and kinetic reressor titration
	predictions.} The equilibrium model (solid lines) and the kinetic model with
	variation over the cell cycle (solid circles and white triangles) predictions
	are compared for varying repressor copy numbers and operator binding energy.
	The equilibrium model is directly computed from \eref{eq_fold_change_thermo}
	while the kinetic model is computed by numerically integrating the moment
	equations over several cell cycles, and then averaging over the extend of the
	cell cycle as defined in \eref{eq_moment_avg}.}
  \label{fig_lacI_titration}
\end{figure}


\begin{figure}[h!]
	\centering \includegraphics
  {../fig/moment_dynamics_numeric/IPTG_titration.pdf}
	\caption{\textbf{Comparison of the equilibrium and kinetic inducer titration
	predictions.} The equilibrium model (solid lines) and the kinetic model with
	variation over the cell cycle (solid circles and white triangles) predictions
	are compared for varying repressor copy numbers and inducer concentrations.
	The equilibrium model is directly computed as Eq. 5 of reference
	\cite{Razo-Mejia2018} while the kinetic model is computed by numerically
	integrating the moment equations over several cell cycles, and then averaging
	over the extend of the cell cycle as defined in \eref{eq_moment_avg}.}
  \label{fig_IPTG_titration}
\end{figure}
