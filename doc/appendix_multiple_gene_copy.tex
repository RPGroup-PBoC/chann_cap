\label{sec_multi_gene}
\section{Accounting for the variability in gene copy number during the cell
cycle}

When growing in rich media, bacteria can double every $\approx$ 20 minutes. With
a replication fork that travels $\approx$ 1000 bp per second, and a genome of
$\approx$ 5 Mbp for {\it E. coli}\cite{Moran2010}, a cell would need $\approx$
80 minutes to replicate its genome. The apparent paradox is solved by the fact
that cells have multiple replisomes, i.e. molecular machines that replicate the
genome running in parallel. Cells can have up to 8 copies of the genome being
replicated at all time depending on the growth rate.

That means that during the cell cycle gene copy number varies. This variation
depends on the relative position of the gene with respect to the replication
origin, having genes close to the origin spending more time with multiple copies
compare to genes closer to the termination. This change in gene dosage has a
direct effect on the cell-to-cell variability in gene expression
\cite{Peterson2015, Jones2014a}. Furthermore, since the time to reach steady
state is determined by the degradation rate, mRNA and protein will experience
experience these changes in gene copy number differently from each other.

\subsection{Numerical integration of moment equations}

Our model needs to account for this variability in gene copy number along the
cell cycle. To do so we numerically integrate the moment equations derived in
\mrm{reference moment derivation section} for a time $0 < t_$
