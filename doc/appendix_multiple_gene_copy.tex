\label{sec_multi_gene}
\section{Accounting for the variability in gene copy number during the cell
cycle}

When growing in rich media, bacteria can double every $\approx$ 20 minutes. With
a replication fork that travels at $\approx$ 1000 bp per second, and a genome of
$\approx$ 5 Mbp for {\it E. coli}\cite{Moran2010}, a cell would need $\approx$
80 minutes to replicate its genome. The apparent paradox is solved by the fact
that cells have multiple replisomes, i.e. molecular machines that replicate the
genome running in parallel. Cells can have up to 8 copies of the genome being
replicated at all time depending on the growth rate \cite{Bremer1996}.

That means that during the cell cycle gene copy number varies. This variation
depends on the relative position of the gene with respect to the replication
origin, having genes close to the origin spending more time with multiple copies
compare to genes closer to the termination. This change in gene dosage has a
direct effect on the cell-to-cell variability in gene expression
\cite{Jones2014a, Peterson2015}. Furthermore, since the time to reach steady
state is determined by the degradation rate, mRNA and protein will experience
experience these changes in gene copy number differently from each other.

\subsection{Numerical integration of moment equations}

For our specific locus ({\it galK}) and a doubling time of $\approx$ 100 min,
cells have on average 1.4 copies of the reporter gene during the cell cycle.
What this means is that cells spend 60\% of the time having one copy of the gene
and 40\% of the time with two copies. Our model needs to account for this
variability in gene copy number along the cell cycle. To do so we numerically
integrate the moment equations derived in \mrm{reference moment derivation
section} for a time $t = [0, t_s]$ with an mRNA production rate $r_m$, where
$t_s$ is the time point at which the replication fork reaches our specific
locus. For the remaining time before the cell division $t = [t_s, t_d]$ that the
cell spends with two promoters, we assume that the only parameter that changes
is the mRNA production rate from $r_m$ to $2 r_m$. This simplifying assumption
ignores potential changes in protein translation rate $r_p$ or changes in the
repressor copy number that would be reflected in changes on the repressor on
rate $\kron$.

\subsubsection{Initial conditions for numerical integration}

In order to define the initial conditions for the numerical integration for each
of the cell cycle stages we follow a simple procedure:
\begin{enumerate}
  \item Initialize the zeroth moment, i.e. the probability of the promoter of
  being on each of the two (unregulated) or three (regulated) states at any
  value, constrained to the fact that the sum of all zeroth moments should add
  up to one.
  \item Integrate moment equations for a long time using parameters
  corresponding to a single promoter. This will set the initial conditions to
  be used for the single promoter case
  \item Integrate the moment equations for a time $t = [0, t_s]$ using the
  initial conditions determined in the previous step.
  \item Update parameters to the ones corresponding to two promoter copies.
  \item Integrate the moment equations for a time $t = [t_s, t_d]$ using the
  last time point of the single promoter time as initial conditions.
  \item Divide the mRNA first moment $\ee{\bb{m}}$ and the protein first moment
  $\ee{\bb{p}}$ by two to represent the cell division.
  \item Integrate the moment equations for a long time with the constraint that
  the zeroth moment, the mRNA and protein first moments remain fixed as defined
  by halving the values from the previous step.
  \item Repeat process from step 3, using as initial conditions the last point
  of the integration in step 7.
\end{enumerate}

Step 7 in our procedure to determine initial conditions serves as a way to
determine how higher moments change as the cells divide. Only the first moments
of both mRNA and protein can simply be halved after cell division, but it is
less clear how higher moments of the distribution change as cells half their
content. To visualize why this is not a trivial step let's consider a binomially
distributed variable $X \sim \text{Bin}(N, p)$. For the first moment, given by
$\ee{X} = Np$ is easy to see that upon halving the number of trials $N$, the new
first moment would simply be $\ee{X}_{\text{new}} = {N \over 2} p$. But for the
second moment given by $\ee{X^2} = Np - Np^2 + N^2p^2$ it is obvious that
halving the number of attempts does not translate to halving the second moment.
Given this non-linear transformation between moments, we concluded that the
easiest way to converge to the value of higher moments after cell division was
to simply integrate the moment equations keeping the first moments fixed until
higher moments relaxed to a steady state given this constrained conditions.

\fref{fig_first_mom_cycles} shows how the first moment of both mRNA and protein
changes over several cell cycles. The mRNA seems to quickly relax to the steady
state corresponding to the parameters for both a single and two promoter copies.
This is expected since the parameters for the mRNA production were determined
in the first place under this assumption (See \mrm{cite mRNA MCMC section}). On
the other hand given that this relaxation time is determined by the degradation
rate the protein doesn't reach such steady state for either case. Interestingly
Once a couple of cell cycles have passed the cells seem to have a reproducible
trajectory over cell cycles. We call this a dynamic steady-state for the
protein.

\begin{figure}[h!]
	\centering \includegraphics
  {../fig/moment_dynamics_numeric/first_mom_cycles.pdf}
	\caption{\textbf{First moment dynamics over cell the cell cycle.}(A) mean
	mRNA and (B) mean protein copy number as the cell cycle progresses. The light
	shaded region delimits the fraction of the cell cycle that cells spend with  a
	single copy of the promoter. The dark shaded region delimits the fraction of
	the cell cycle that cells spend with two copies of the promoter. For a 100
  min doubling time at the {\it galK} locus cells spend 60\% of the time with
  one copy of the promoter and the rest with two copies.}
  \label{fig_first_mom_cycles}
\end{figure}

Since our experiments were not performed with synchronized cells, in principle
we sampled cells over the entirety of the cell cycle, so the moments that we
determined experimentally correspond to an average over the cell cycle.  In
order to compute these averages in the following section we discuss how to
account for the fact that cells are not uniformly distributed along the cell
cycle

\subsection{Exponentially distributed ages}

As first mentioned in Section \mrm{ref to mRNA MCMC section}, cells in a log
phase have exponentially distributed ages along the cell cycle, having more
young cells compared to old ones. Specifically the probability of a cell of
being at any time point in the cell cycle is given by \cite{Powell1956}
\begin{equation}
  P(a) = (\ln 2) \cdot 2^{1 - a},
  \label{eq_age_prob}
\end{equation}
where $a \in [0, 1]$ is the stage of the cell cycle, with $a = 0$ being the
start of the cycle and $a = 1$ being the division.

Our numerical integration of the moment equations gave us a time evolution of
the moments along the cell cycle. Without loss of generality let's focus on the
first mRNA moment $\ee{m(t)}$ (the same can be applied to all other moments).
In order to calculate the first moment along the entire cell cycle we must
average each time point by the corresponding probability that a cell is found
in such time point. This translates to computing the integral
\begin{equation}
  \ee{m} = \int_{\text{beginning cell cycle}}^{\text{end cell cycle}}
                       \ee{m(t)} P(t) dt.
\end{equation}

If we map each time point in the cell cycle into a fraction we can use
\eref{eq_age_prob} and compute instead
\begin{equation}
  \ee{m} = \int_0^1 \ee{m(a)} P(a) da.
  \label{eq_moment_avg}
\end{equation}
We perform this integral numerically for all moments using Simpson's rule.
