\section{Models}

\textcolor{blue}{Introduce Jane's paper as a means of mRNA variability}


\begin{figure}[h!]
	\centering \includegraphics[scale=0.1]{Figure2.jpg} \caption{Channel capacity of the Lac repressor. Experimental data with \textbf{theory perfectly overlaid on top}, omg bbq roflcopter!}
	\label{figChannelCapacity}
\end{figure}

\begin{figure}[h!]
	\centering \includegraphics[scale=0.2]{Figure3v2.jpg} \caption{Experimental results of gene expression overlaid with theoretical prediction. \talComment{Show all results for O2, say that other plots are in SI}}
	\label{figFullProfile}
\end{figure}

\talComment{Potential other figures:
	\begin{itemize}
		\item Varying cross all $R$ and $k_{off}$ values, what is the maximal channel capacity you can achieve?
		\item Potentially look at Mitch Lewis's strains which will have different $K_{DNA}$ or $K_{A}$ and $K_{I}$ parameters and redo the analysis there, showing that the theory is amazingly good!
	\end{itemize}
	}

\talComment{Other things that would be very interesting to explore:
\begin{itemize}
	\item Explore the channel capacity analytically in the limit of small noise (discuss whether this is a valid assumption from our data)
	\item Be theorists, and consider how other variables affect the channel capacity. For example, the (active) repressor-DNA binding energy, or $K_A/K_I$, or the rates in the problem. Would be really cool to basically look at the effects of all of the variables that we have on channel capacity, and consider which yield the coolest results.
\end{itemize}
}