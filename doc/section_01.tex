\section{Models}

\subsection{Mutual Information and Channel Capacity}

The mutual information
between an environmental state $s$ and the gene expression level $g$ can be
written as
\begin{equation}
I(s;g) = \sum_s \sum_g P(s) P(g \mid s) \log_2 \frac{P(g \mid s)}{P(g)}.
\end{equation}
$P(g \mid s)$, the probability of a gene expression level $g$
given an inducer concentration $s$ (the input-output function), is set by the
physics of the system and can be analytically derived using the chemical master
equation approach \cite{Shahrezaei2008, Swain2016} in combination with the
theoretical framework explained in aim 1. $P(g)$, the distribution of gene
expression regardless of the inducer concentration, can be computed from the
input-output function by averaging over all inputs. Mathematically this is
written as
\begin{equation}
P(g) = \sum_s P(g \mid s) P(s).
\end{equation}
$P(s)$, the distribution of environments/inputs, which is not a property of the
communication channel, in principle cannot be determined -- In aim 3 we will
come back to this point where as experimentalists we get to choose different
distributions for the environments that the cells are exposed to. A useful
metric to compute then, is one of the key results from Shannon's work known as
the channel capacity \cite{Rhee2012a}. This quantity is defined as
\begin{equation}
C(s; g) \equiv \max_{P(s)} I(s; g),
\end{equation}
which represents the maximum amount of information that can be sent through a
channel optimized over all possible distributions of the signal $s$.
This quantity has previously been used as a metric for the regulatory power that
cells have over their gene expression profiles \cite{Tkacik2008a, Rieckh2014}.

\subsection{Full Gene Expression Profile}

\talComment{Peter Swain's paper. Not yet finalized on exactly which model we are doing}

\begin{figure}[h!]
	\centering \includegraphics[scale=0.1]{Figure2.jpg} \caption{Channel capacity of the Lac repressor. Experimental data with \textbf{theory perfectly overlaid on top}, omg bbq roflcopter!}
	\label{figChannelCapacity}
\end{figure}

\begin{figure}[h!]
	\centering \includegraphics[scale=0.2]{Figure3v2.jpg} \caption{Experimental results of gene expression overlaid with theoretical prediction. \talComment{Show all results for O2, say that other plots are in SI}}
	\label{figFullProfile}
\end{figure}